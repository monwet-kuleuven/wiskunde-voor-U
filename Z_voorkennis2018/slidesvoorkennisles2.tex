%Latex->ps->pdf
\documentclass{beamer}
%\documentclass[handout]{beamer}
\newcommand{\ds}{\displaystyle}
\newcommand{\N}{\ensuremath{\mathbb{N}}}
\newcommand{\Nnul}{\ensuremath{\mathbb{N}_0}}
\newcommand{\R}{\ensuremath{\mathbb{R}}}
\newcommand{\perdef}{\overset{\mathrm{def}}{=}}

\usepackage{amsmath,amssymb}
\usepackage[latin1]{inputenc}
\usepackage{fancybox}
\usepackage{epic,overpic}

\usepackage{etex}
\usepackage[dutch]{babel}
\usepackage{hyphenat}
\usepackage{amsmath}
\usepackage{amssymb}
\usepackage{amsfonts}
\usepackage{graphicx}

\usepackage{color}
%\usepackage[thmmarks,framed,amsmath]{ntheorem}
\usepackage{answers}

\usepackage{framed}

\usepackage{pst-all}
\usepackage{epic}
\usepackage{eepic}
\usepackage{array}
\usepackage{booktabs}
\usepackage{multirow}
\usepackage{color}


\title{Voorkennisles 2}

\author{Annouk Van Vlierden}
\date{2018-2019}
%\usetheme{Montpellier}
\usetheme{CambridgeUS}
\usecolortheme{seahorse}

\begin{document}

\begin{frame}
  \titlepage
\end{frame}



\section{Limieten}
%\section{Inleiding}

\begin{frame}{Idee: Waar gaat $f(x)$ naar toe als $x$ naar $a$ gaat?}
\begin{exampleblock}{Voorbeeld}
Waar gaat $f(x)=\frac{1}{x}$ naar toe als $x$ naar $a=2$ gaat?
\begin{itemize}
	\item naar $\ds \frac{1}{2}$
	\item notatie:   $\displaystyle{\lim_{x \rightarrow 2}} \frac{1}{x} = \frac{1}{2} $
\end{itemize}


\end{exampleblock}

\end{frame}

\begin{frame}{Linker- en rechterlimiet}
\begin{exampleblock}{Voorbeeld}
$$\lim_{x \to 7} (-2 + \sqrt{x-7})$$ bestaat niet omdat $x$ alleen langs rechts $(x>7)$ naar 7 kan gaan.

De RECHTERlimiet $$\displaystyle \lim_{x  \to 7^+} (-2 + \sqrt{x-7}) $$ bestaat wel.  
$$\displaystyle \lim_{x  \to 7^+} (-2 + \sqrt{x-7})=-2 $$
\end{exampleblock}
\end{frame}

\begin{frame}{Let op notatie limiet}

\begin{itemize}
\item $\displaystyle{\lim_{x \rightarrow a}} f(x) = L $ betekent: 
\begin{itemize}
	\item[] Als $x$
naar $a$ gaat, dan gaat de functiewaarde $f(x)$ naar $L$.
\end{itemize}

\item $\displaystyle{\lim_{x  \rightarrow a^+}} f(x) = L $ betekent: 
\begin{itemize}
	\item[] Als $x$ langs rechts
naar $a$ gaat ($x>a$), dan gaat de functiewaarde $f(x)$ naar $L$.
\item[] rechterlimiet
\end{itemize}

\item $\displaystyle{\lim_{x  \rightarrow a^-}} f(x) = L $ betekent:
\begin{itemize}
	\item[] Als $x$ langs links
naar $a$ gaat ($x<a$), dan gaat de functiewaarde $f(x)$ naar $L$.
\item[] linkerlimiet
\end{itemize}
 
\end{itemize}
\end{frame}

\begin{frame}

$$\displaystyle{\lim_{x \rightarrow 2}} \frac{1}{x} = \frac12 = f(2) $$
want $f(x)= \frac{1}{x}$ is continue functie in 2. 

Zo eenvoudig is het niet altijd.


We bestuderen verder enkel limieten 
\begin{itemize}
	\item in $a = \pm \infty$
	\item in nulpunten van de noemer
\end{itemize}
en nemen als inleidend voorbeeld de rationale functie $f : x \mapsto \frac 1x$
\end{frame}

\begin{frame}{Waar gaat $f(x)=\frac{1}{x}$ naar toe als $x$ naar $0$ gaat?}

\begin{tabular}{r|llcl}
x & $\frac{1}{2}$ & $\frac{1}{3}$ & \dots & $ \to  0$\\
\hline \\
f(x) & 2&3& \dots & $\rightarrow + \infty$
\end{tabular}
\centerline{\scalebox{.4}{\input{Afbeeldingen/18.pst}}}
\end{frame}
\begin{frame}{Waar gaat $f(x)=\frac{1}{x}$ naar toe als $x$ naar $0$ gaat?}

\[\begin{array}{cccccc}
 \text{ als } &  x \to  0^+ & \text{ dan } & \frac1x \rightarrow + \infty &
 \text{ notatie: } & \displaystyle{\lim_{x \to  0^+} \frac1x = +
\infty} \\
 \\
 \text { als } & x \to
 0^- & \text{ dan } & \frac 1 x \rightarrow -\infty &
 \text{ notatie: } & \displaystyle{\lim_{x \to  0^-} \frac 1x = - \infty}\\
\end{array}
\]

De rechte $x = 0$ (de Y-as) noemen we een \textit{ verticale
asymptoot} van (de grafiek van) $f$.

\begin{alertblock}{Opsplitsen in linker- en rechterlimiet}
Merk op dat $\displaystyle{\lim_{x \to  0}} \frac 1x$ niet bestaat
omdat de linker- en rechterlimiet verschillend zijn.
\end{alertblock}
\end{frame}

\begin{frame}{Waar gaat $f(x)=\frac{1}{x}$ naar toe als $x$ naar $+ \infty$ of $- \infty$ gaat?}
\[\begin{array}{cccccc}
 \text{ als } &  x \rightarrow + \infty & \text{ dan } & \frac1x \rightarrow 0 &
 \text{ notatie: } & \displaystyle{\lim_{x \rightarrow + \infty} \frac1x = 0} \\
 \\
 \text { als } & x \rightarrow - \infty & \text{ dan } & \frac 1 x \rightarrow 0 &
 \text{ notatie: } & \displaystyle{\lim_{x \rightarrow - \infty} \frac 1x = 0}\\
\end{array}
\]
\\
De rechte $y=0$ (de $X$-as) noemen we een \textit{ horizontale
asymptoot} van (de grafiek van) $f$.
\centerline{\scalebox{.3}{\input{Afbeeldingen/18.pst}}}

\end{frame}

\begin{frame}
Ook voor de exponenti\"{e}le en de logaritmische functie kunnen we de
limieten aflezen op de grafiek:
\begin{center}
\scalebox{0.5}{\includegraphics{Afbeeldingen/Logaritme.eps}}
\end{center}
\end{frame}

\begin{frame}{Oefening}
Vind de limieten met de hulp van voorgaande grafieken:
\begin{itemize}

\item $\ds \lim_{x \rightarrow + \infty} e^x = \pause + \infty$
\item $\ds \lim_{x \rightarrow - \infty} e^x = \pause 0$

\item $\displaystyle{\lim_{x \to  0^+}} (e^{\frac 1x}) = \pause +\infty$
\item $\displaystyle{\lim_{x  \to  0^-}} (e^{\frac 1x}) = \pause 0$
\item $\ds \lim_{x \rightarrow + \infty} \ln x = \pause +\infty$
\item $\displaystyle{\lim_{x \to  0^+}} \ln x = \pause -\infty$
\item $\ds \lim_{x \rightarrow - \infty} \ln x = \pause \mbox{bestaat niet}$
\item $\ds \lim_{x \rightarrow + \infty} \ln(1+ \frac 1x) = \pause \ln 1 = 0$
\end{itemize}

\end{frame}



\section{Rekenregels voor limieten }
\begin{frame}{Rekenregels voor limieten}


\begin{itemize}
	\item De limiet
van een som is de som van de limieten
\item De limiet van een product is
het product van de limieten
\item De limiet van een quoti\"ent is het
quoti\"ent van de limieten
\end{itemize}
TENZIJ je daardoor een onbepaalde vorm
krijgt.

Deze rekenregels gelden voor $x \rightarrow a \text{ met } {a \in
\mathbb{R} \text{ of } a = \pm \infty}$.
\end{frame}

\begin{frame}{Voorbeeld rekenregels zonder onbepaalde vormen}
\begin{itemize}
\item $\ds \lim_{x \rightarrow - \infty} 3 x^2 = \pause 3(- \infty)(- \infty) = + \infty$
\item $\ds \lim_{x \rightarrow + \infty} (-4x)= \pause (-4)(+ \infty)= - \infty$
\item $\ds \lim_{x \rightarrow + \infty} \frac{-4}{x} = \pause \frac{-4}{+ \infty}=0$
\item $\ds \lim_{x \rightarrow + \infty} e^{-x} = \pause \lim_{x \rightarrow + \infty} \frac{1}{e^x} = \frac{1}{+ \infty} = 0$
\end{itemize}
\end{frame}

\begin{frame}{Rekenregels ZONDER onbepaalde vormen}
\begin{itemize}
\item voor de som
\[\aligned
& (+ \infty) + (+ \infty) = + \infty \\
& (- \infty) + (-\infty) = - \infty \\
& a + (+ \infty) = (+ \infty) + a = + \infty \quad \text { voor elke
} \quad a
\in \mathbb{R} \\
& a + (- \infty) = (-\infty) + a = - \infty \quad \text { voor elke } \quad a
\in \mathbb{R} \\
\endaligned
\]
\end{itemize}
\end{frame}
\begin{frame}{Overzicht rekenregels ZONDER onbepaalde vormen}
\begin{itemize}
\item voor het product
\[\aligned
& (+ \infty) \cdot (+ \infty) = + \infty\\
& (+ \infty) \cdot (-\infty) = (-\infty) \cdot (+ \infty) = - \infty \\
& (-\infty) \cdot (-\infty) = + \infty   \\
& a \cdot (+ \infty) = (+ \infty) \cdot a = + \infty \quad \text {
als } \quad a
> 0 \\
& a \cdot (+ \infty) = (+ \infty) \cdot a = - \infty \quad \text { als } \quad a
< 0 \\
& a \cdot (- \infty) = (- \infty) \cdot a = - \infty \quad \text { als } \quad a
> 0 \\
& a \cdot (- \infty) = (- \infty) \cdot a = + \infty \quad \text { als } \quad a
< 0 \\
\endaligned
\]

\end{itemize}
\end{frame}
\begin{frame}{Overzicht rekenregels ZONDER onbepaalde vormen}
\begin{itemize}
	\item voor het quoti\"ent
	$$\frac{a}{+ \infty} = \frac{a}{-\infty} = 0 \quad \text { voor elke
} \quad a \in \mathbb{R}$$
\end{itemize}
\end{frame}


\begin{frame}
\begin{alertblock}{ONBEPAALDE VORMEN}
\begin{itemize}
	\item voor de som
	\[\aligned
	& (+ \infty) + (-\infty) \\
& (- \infty) + (+ \infty) \\
\endaligned
\]
\item voor het product
\[\aligned
& 0 \cdot (+ \infty) \quad \text { en } \quad (+ \infty) \cdot 0 \\
& 0 \cdot (- \infty) \quad \text { en } \quad (- \infty) \cdot 0
\\
\endaligned
\]
\item voor het quoti\"ent
\[\aligned
& \frac{0}{0} \quad \text { en} \quad \frac{\infty}{\infty}
\endaligned
\]

\end{itemize}
\end{alertblock}
\end{frame}

\begin{frame}{Vanwaar de naam ONBEPAALDE VORM?}
\begin{exampleblock}{We nemen als voorbeeld de onbepaalde vorm $0 \cdot (+
\infty)$ }
\begin{itemize}
\item[(a)]
$\displaystyle{\lim_{x \to 0^+}} (x^2 \cdot \frac{1}{x}) =
\displaystyle{\lim_{x \to 0^+}} x = 0$ 
\item[(b)]
$\displaystyle{\lim_{x \to 0^+}} (x \cdot \frac{1}{x})=
\displaystyle{\lim_{x \to 0^+}} 1 = 1$ 
\item[(c)]
$\displaystyle{\lim_{x \to 0^+}} (x \cdot \frac{1}{x^2}) =
\displaystyle{\lim_{x \to 0^+}} \frac{1}{x} = + \infty$



\end{itemize}
\end{exampleblock}
\end{frame}

\begin{frame}
Bij een onbepaalde vorm kan de rekenregel
\begin{itemize}
	\item De limiet
van een som is de som van de limieten
\item De limiet van een product is
het product van de limieten
\item De limiet van een quoti\"ent is het
quoti\"ent van de limieten
\end{itemize}
NIET gebruikt worden.

\end{frame}
\section{Limiet in $a = \pm \infty$}
\begin{frame}{Limiet in $a = \pm \infty$}
Hoe berekenen we
$$\lim_{x \rightarrow + \infty} (3x^2 - 4x + 2) ?$$
\begin{block}{FOUT}
$$\lim_{x \rightarrow + \infty} (3x^2 - 4x + 2) = \lim_{x \rightarrow + \infty}
3x^2 + \lim_{x \rightarrow + \infty} (-4x) + \lim_{x \rightarrow +
\infty} 2$$
want dan  onbepaalde vorm $(+ \infty) + (-\infty)$.
\end{block}
\end{frame}
\begin{frame}
\begin{block}{JUIST}

\[\aligned
\lim_{x \rightarrow + \infty} (3x^2 - 4x + 2) & = \lim_{x
\rightarrow + \infty}
[x^2 (3 - \frac{4}{x} + \frac{2}{x^2})] \\
& = \lim_{x \rightarrow + \infty} x^2 \cdot \lim_{x \rightarrow +
\infty} (3 -
\frac{4}{x} + \frac {2}{x^2}) \\
& = (+ \infty) \cdot 3 \\
& = + \infty
\endaligned
\]
\end{block}

\end{frame}
\begin{frame}{Voorbeelden}
De hoogste macht van $x$ buiten haken zetten om van een som over te gaan op een product.
\begin{exampleblock}{Veelterm (oef 4(a))}
$\displaystyle{\lim_{x \rightarrow + \infty}} (2x^3 - 5x^2)= \pause \displaystyle{\lim_{x \rightarrow + \infty}} x^3 (2 - \frac{5}{x})= \pause (+ \infty) . 2 = \pause + \infty$
\end{exampleblock}
\pause
\begin{exampleblock}{Veeltermbreuk}
$\displaystyle{\lim_{x \rightarrow - \infty}} \frac{3x^2 - 6x
+ 1}{2x^2 + 11x - 8}
= \displaystyle{\lim_{x \rightarrow - \infty}} \frac{x^2 (3- \frac{6}{x}
+ \frac{1}{x^2})}{x^2 (2 + \frac{11}{x} - \frac{8}{x^2})}
= \displaystyle{\lim_{x \rightarrow - \infty}} \frac{3- \frac{6}{x}
+ \frac{1}{x^2}}{2 + \frac{11}{x} - \frac{8}{x^2}}
=\frac{3}{2}$

\end{exampleblock}
\end{frame}
\begin{frame}
\begin{exampleblock}{Irrationale functies (oef 4(c))}
$\displaystyle{\lim_{x \rightarrow - \infty}} (\sqrt{4x^2+7x} + 3x)= \pause \displaystyle{\lim_{x \rightarrow - \infty}} (\sqrt{x^2(4 + \frac{7}{x})} + 3x)= \pause \displaystyle{\lim_{x \rightarrow - \infty}} (|x| \sqrt{4 + \frac{7}{x}} + 3x)= \pause \displaystyle{\lim_{x \rightarrow - \infty}} (-x \sqrt{4 + \frac{7}{x}} + 3x)= \pause \displaystyle{\lim_{x \rightarrow - \infty}} x (-\sqrt{4 + \frac{7}{x}} + 3)= \pause (- \infty). (- \sqrt4 + 3) = \pause - \infty$ \pause

\end{exampleblock}
\begin{alertblock}{Let op}
 
\begin{itemize}
	\item $\ds \lim_{x \rightarrow + \infty} \sqrt{x^2} = \lim_{x \rightarrow + \infty} x$
	\item $\ds \lim_{x \rightarrow - \infty} \sqrt{x^2} = \lim_{x \rightarrow - \infty} -x$
\end{itemize}
want $\sqrt{x^2} = |x| = -x$ in het geval dat $x<0$
\end{alertblock}
\end{frame}
\begin{frame}
\begin{alertblock}{De worteltruc (oef 4(d))}
Voor $\displaystyle{\lim_{x \rightarrow - \infty}} (\sqrt{4x^2+7x} + 2x)$ werkt voorgaande methode niet, hier gebruiken we de worteltruc: 
$$\lim_{x \rightarrow - \infty} (\sqrt{4x^2+7x} + 2x)= \lim_{x \rightarrow - \infty} \frac{(\sqrt{4x^2+7x} + 2x)(\sqrt{4x^2+7x} - 2x)}{(\sqrt{4x^2+7x} - 2x)}$$
\end{alertblock}
\pause
\begin{exampleblock}{Met een parameter}
$\displaystyle{\lim_{x \rightarrow + \infty}} \frac{\sqrt{x^2+1} + ax}{\sqrt{x^2+2}}
=\displaystyle{\lim_{x \rightarrow + \infty}} \frac{|x| \sqrt{1+\frac{1}{x^2}} + ax}{|x| \sqrt{1+ \frac{2}{x^2}}}
=\displaystyle{\lim_{x \rightarrow + \infty}} \frac{x \sqrt{1+\frac{1}{x^2}} + ax}{x \sqrt{1+ \frac{2}{x^2}}}
=\displaystyle{\lim_{x \rightarrow + \infty}} \frac{ \sqrt{1+\frac{1}{x^2}} + a}{ \sqrt{1+ \frac{2}{x^2}}}
=\displaystyle{\lim_{x \rightarrow + \infty}} \frac{1+a}{1} = 1+a$
\end{exampleblock}
\end{frame}

\section{Limiet in een nulpunt van de noemer}
\begin{frame}{Limiet in een nulpunt van de noemer}
\begin{block}{Herhaling  inleidend voorbeeld}
\begin{itemize}
	\item $\displaystyle{\lim_{x  \to  0^+}} \frac 1x = +
\infty$
\item $\displaystyle{\lim_{x  \to  0^-}} \frac 1x = -
\infty$
\item $\displaystyle{\lim_{x \to  0}} \frac 1x$ bestaat niet
\end{itemize}
\end{block}
\end{frame}
\begin{frame}{Geval $\displaystyle{\lim_{x \rightarrow a}} \frac{1}{f(x)}$
met $f(a) = 0$, $a \in \mathbb{R}$}


Steeds $+ \infty$ of $- \infty$, 

tenzij
linker- en rechterlimiet verschillen, dan bestaat
$\displaystyle{\lim_{x \rightarrow a}} \frac 1{f(x)}$ niet.

Om te weten of het $+ \infty$ of $- \infty$ is moeten we het tekenverloop van
$f(x)$ kennen in buurt van $a$.
\end{frame}

\begin{frame}
\begin{exampleblock}{Voorbeeld (oef 2(a))}
$$\displaystyle{\lim_{x \rightarrow 1}} \frac {1}{(x-1)^2}= \pause + \infty$$
\end{exampleblock}
\end{frame}

\begin{frame}{Geval: $\displaystyle{\lim_{x \rightarrow a}}
\frac{g(x)}{f(x)}$ met $f(a) = 0$, $a \in \mathbb{R}$}
We onderscheiden we 2 gevallen:
\begin{itemize}
\item $a$ is geen nulpunt van de teller
\item $a$ is een nulpunt van teller en noemer
\end{itemize}
\begin{block}{$a$ is geen nulpunt van de teller}
We schrijven $\displaystyle{\lim_{x \rightarrow a}}
\frac{g(x)}{f(x)}$ als $\displaystyle{\lim_{x \rightarrow a}} g(x) .
\displaystyle{\lim_{x \rightarrow a}} \frac{1}{f(x)} = g(a) .
\displaystyle{\lim_{x \rightarrow a}} \frac{1}{f(x)}$.
\\Dit is geen onbepaalde vorm vermits $g(a) \neq 0$.
\end{block}
\end{frame}
\begin{frame}
\begin{exampleblock}{Voorbeeld (oef2(b))}

$$\aligned & \lim_{x  \to 1^+} \frac{x^2 + x - 1}{x-1} \\ & = \lim_{x  \to 1^+} (x^2+x-1) . \lim_{x  \to 1^+} \frac{1}{x-1} \\ & = ((1^2+1-1). (+ \infty) = + \infty \endaligned $$

\begin{center}
\begin{tabular}{r||c|c|c}$x$& &$ 1$& \\ \hline $x-1$ &
 $-$&$0$ &  $+$
\end{tabular}\end{center}
\end{exampleblock}
\end{frame}

\begin{frame}
\begin{block}{$a$ is een nulpunt van teller en noemer}
We krijgen dan de
onbepaalde vorm $\frac{0}{0}$.
\begin{itemize}


\item Bij rationale functies: teller en noemer
ontbinden in factoren en de factor $x-a$ wegdelen in teller en
noemer. 

\item Bij irrationale functies: worteltruc
\end{itemize}


\end{block}
\begin{alertblock}{Opmerking}
Limieten die de onbepaalde vorm $\frac{0}{0}$ geven kunnen ook berekend kunnen worden met de regels van de l'H\^{o}pital. 
\end{alertblock}
\end{frame}
\begin{frame}
\begin{exampleblock}{Voorbeelden}
\begin{itemize}
	\item 
$\displaystyle{\lim_{x \to 2}} \frac{x^2 - 4 }{x-2}
= \pause \displaystyle{\lim_{x \to 2}} \frac{(x - 2)(x + 2) }{x-2} = \pause
\displaystyle{\lim_{x \to 2}} (x + 2) = 4 $ (oef 3(a))
\item 
$\ds \lim_{x \rightarrow 2} \frac {\sqrt{x+2} -2}{x-2} = \pause \lim_{x \rightarrow 2} \frac {(\sqrt{x+2} -2)(\sqrt{x+2} +2)}{(x-2)(\sqrt{x+2} +2)}= \lim_{x \rightarrow 2} \frac {x+2-4}{(x-2)(\sqrt{x+2} +2)}= \pause\lim_{x \rightarrow 2} \frac {1}{\sqrt{x+2} +2}=\frac14$
\end{itemize}
\end{exampleblock}
\end{frame}

\begin{frame}{Overzicht limieten in nulpunten van de noemer}
\begin{itemize}
	\item Geval $\displaystyle{\lim_{x \rightarrow a}} \frac{1}{f(x)}$
met $f(a) = 0$: 
\begin{itemize}
	\item[] $+ \infty$ of $- \infty$, tekenverloop van noemer in de buurt van $a$. 
\end{itemize}
\item Geval: $\displaystyle{\lim_{x \rightarrow a}}:
\frac{g(x)}{f(x)}$ met $f(a) = 0$
\begin{itemize}
	\item $a$ is geen nulpunt van de teller
	
	$\displaystyle{\lim_{x \rightarrow a}}
\frac{g(x)}{f(x)}$ = $\displaystyle{\lim_{x \rightarrow a}} g(x) .
\displaystyle{\lim_{x \rightarrow a}} \frac{1}{f(x)} = g(a) .
\displaystyle{\lim_{x \rightarrow a}} \frac{1}{f(x)}$.
\item $a$ is een nulpunt van teller en noemer: $\ds \frac00$

	 l'H\^{o}pital
	 OF 
	
	\begin{itemize}
		\item Bij rationale functies: teller en noemer
ontbinden in factoren en de factor $x-a$ wegdelen in teller en
noemer.
\item Bij irrationale functies: worteltruc
	
\end{itemize}
	
\end{itemize}
\end{itemize}

\end{frame}

\end{document} 
