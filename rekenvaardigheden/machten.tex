\documentclass{ximera}
%
% copied from https://github.com/mooculus/calculus
%
\usepackage[utf8]{inputenc}


\graphicspath{
	{./}
	{goniometrie/}
}


%\usepackage{todonotes}
%\usepackage{mathtools} %% Required for wide table Curl and Greens
%\usepackage{cuted} %% Required for wide table Curl and Greens
\newcommand{\todo}{}

% Font niet (correct?) geinstalleerd in MikTeX?
%\usepackage{esint} % for \oiint
%\ifxake%%https://math.meta.stackexchange.com/questions/9973/how-do-you-render-a-closed-surface-double-integral
%\renewcommand{\oiint}{{\large\bigcirc}\kern-1.56em\iint}
%\fi


\newcommand{\mooculus}{\textsf{\textbf{MOOC}\textnormal{\textsf{ULUS}}}}

\usepackage{tkz-euclide}\usepackage{tikz}
\usepackage{tikz-cd}
\usetikzlibrary{arrows}
\tikzset{>=stealth,commutative diagrams/.cd,
  arrow style=tikz,diagrams={>=stealth}} %% cool arrow head
\tikzset{shorten <>/.style={ shorten >=#1, shorten <=#1 } } %% allows shorter vectors

\usetikzlibrary{backgrounds} %% for boxes around graphs
\usetikzlibrary{shapes,positioning}  %% Clouds and stars
\usetikzlibrary{matrix} %% for matrix
\usepgfplotslibrary{polar} %% for polar plots
\usepgfplotslibrary{fillbetween} %% to shade area between curves in TikZ
\usetkzobj{all}
\usepackage[makeroom]{cancel} %% for strike outs
%\usepackage{mathtools} %% for pretty underbrace % Breaks Ximera
%\usepackage{multicol}
\usepackage{pgffor} %% required for integral for loops



%% http://tex.stackexchange.com/questions/66490/drawing-a-tikz-arc-specifying-the-center
%% Draws beach ball
\tikzset{pics/carc/.style args={#1:#2:#3}{code={\draw[pic actions] (#1:#3) arc(#1:#2:#3);}}}



\usepackage{array}
\setlength{\extrarowheight}{+.1cm}
\newdimen\digitwidth
\settowidth\digitwidth{9}
\def\divrule#1#2{
\noalign{\moveright#1\digitwidth
\vbox{\hrule width#2\digitwidth}}}





\newcommand{\RR}{\mathbb R}
\newcommand{\R}{\mathbb R}
\newcommand{\N}{\mathbb N}
\newcommand{\Z}{\mathbb Z}

\newcommand{\sagemath}{\textsf{SageMath}}


%\renewcommand{\d}{\,d\!}
\renewcommand{\d}{\mathop{}\!d}
\newcommand{\dd}[2][]{\frac{\d #1}{\d #2}}
\newcommand{\pp}[2][]{\frac{\partial #1}{\partial #2}}
\renewcommand{\l}{\ell}
\newcommand{\ddx}{\frac{d}{\d x}}

\newcommand{\zeroOverZero}{\ensuremath{\boldsymbol{\tfrac{0}{0}}}}
\newcommand{\inftyOverInfty}{\ensuremath{\boldsymbol{\tfrac{\infty}{\infty}}}}
\newcommand{\zeroOverInfty}{\ensuremath{\boldsymbol{\tfrac{0}{\infty}}}}
\newcommand{\zeroTimesInfty}{\ensuremath{\small\boldsymbol{0\cdot \infty}}}
\newcommand{\inftyMinusInfty}{\ensuremath{\small\boldsymbol{\infty - \infty}}}
\newcommand{\oneToInfty}{\ensuremath{\boldsymbol{1^\infty}}}
\newcommand{\zeroToZero}{\ensuremath{\boldsymbol{0^0}}}
\newcommand{\inftyToZero}{\ensuremath{\boldsymbol{\infty^0}}}



\newcommand{\numOverZero}{\ensuremath{\boldsymbol{\tfrac{\#}{0}}}}
\newcommand{\dfn}{\textbf}
%\newcommand{\unit}{\,\mathrm}
\newcommand{\unit}{\mathop{}\!\mathrm}
\newcommand{\eval}[1]{\bigg[ #1 \bigg]}
\newcommand{\seq}[1]{\left( #1 \right)}
\renewcommand{\epsilon}{\varepsilon}
\renewcommand{\phi}{\varphi}


\renewcommand{\iff}{\Leftrightarrow}

\DeclareMathOperator{\arccot}{arccot}
\DeclareMathOperator{\arcsec}{arcsec}
\DeclareMathOperator{\arccsc}{arccsc}
\DeclareMathOperator{\si}{Si}
\DeclareMathOperator{\scal}{scal}
\DeclareMathOperator{\sign}{sign}


%% \newcommand{\tightoverset}[2]{% for arrow vec
%%   \mathop{#2}\limits^{\vbox to -.5ex{\kern-0.75ex\hbox{$#1$}\vss}}}
\newcommand{\arrowvec}[1]{{\overset{\rightharpoonup}{#1}}}
%\renewcommand{\vec}[1]{\arrowvec{\mathbf{#1}}}
\renewcommand{\vec}[1]{{\overset{\boldsymbol{\rightharpoonup}}{\mathbf{#1}}}\hspace{0in}}

\newcommand{\point}[1]{\left(#1\right)} %this allows \vector{ to be changed to \vector{ with a quick find and replace
\newcommand{\pt}[1]{\mathbf{#1}} %this allows \vec{ to be changed to \vec{ with a quick find and replace
\newcommand{\Lim}[2]{\lim_{\point{#1} \to \point{#2}}} %Bart, I changed this to point since I want to use it.  It runs through both of the exercise and exerciseE files in limits section, which is why it was in each document to start with.

\DeclareMathOperator{\proj}{\mathbf{proj}}
\newcommand{\veci}{{\boldsymbol{\hat{\imath}}}}
\newcommand{\vecj}{{\boldsymbol{\hat{\jmath}}}}
\newcommand{\veck}{{\boldsymbol{\hat{k}}}}
\newcommand{\vecl}{\vec{\boldsymbol{\l}}}
\newcommand{\uvec}[1]{\mathbf{\hat{#1}}}
\newcommand{\utan}{\mathbf{\hat{t}}}
\newcommand{\unormal}{\mathbf{\hat{n}}}
\newcommand{\ubinormal}{\mathbf{\hat{b}}}

\newcommand{\dotp}{\bullet}
\newcommand{\cross}{\boldsymbol\times}
\newcommand{\grad}{\boldsymbol\nabla}
\newcommand{\divergence}{\grad\dotp}
\newcommand{\curl}{\grad\cross}
%\DeclareMathOperator{\divergence}{divergence}
%\DeclareMathOperator{\curl}[1]{\grad\cross #1}
\newcommand{\lto}{\mathop{\longrightarrow\,}\limits}

\renewcommand{\bar}{\overline}

\colorlet{textColor}{black}
\colorlet{background}{white}
\colorlet{penColor}{blue!50!black} % Color of a curve in a plot
\colorlet{penColor2}{red!50!black}% Color of a curve in a plot
\colorlet{penColor3}{red!50!blue} % Color of a curve in a plot
\colorlet{penColor4}{green!50!black} % Color of a curve in a plot
\colorlet{penColor5}{orange!80!black} % Color of a curve in a plot
\colorlet{penColor6}{yellow!70!black} % Color of a curve in a plot
\colorlet{fill1}{penColor!20} % Color of fill in a plot
\colorlet{fill2}{penColor2!20} % Color of fill in a plot
\colorlet{fillp}{fill1} % Color of positive area
\colorlet{filln}{penColor2!20} % Color of negative area
\colorlet{fill3}{penColor3!20} % Fill
\colorlet{fill4}{penColor4!20} % Fill
\colorlet{fill5}{penColor5!20} % Fill
\colorlet{gridColor}{gray!50} % Color of grid in a plot

\newcommand{\surfaceColor}{violet}
\newcommand{\surfaceColorTwo}{redyellow}
\newcommand{\sliceColor}{greenyellow}




\pgfmathdeclarefunction{gauss}{2}{% gives gaussian
  \pgfmathparse{1/(#2*sqrt(2*pi))*exp(-((x-#1)^2)/(2*#2^2))}%
}


%%%%%%%%%%%%%
%% Vectors
%%%%%%%%%%%%%

%% Simple horiz vectors
\renewcommand{\vector}[1]{\left\langle #1\right\rangle}


%% %% Complex Horiz Vectors with angle brackets
%% \makeatletter
%% \renewcommand{\vector}[2][ , ]{\left\langle%
%%   \def\nextitem{\def\nextitem{#1}}%
%%   \@for \el:=#2\do{\nextitem\el}\right\rangle%
%% }
%% \makeatother

%% %% Vertical Vectors
%% \def\vector#1{\begin{bmatrix}\vecListA#1,,\end{bmatrix}}
%% \def\vecListA#1,{\if,#1,\else #1\cr \expandafter \vecListA \fi}

%%%%%%%%%%%%%
%% End of vectors
%%%%%%%%%%%%%

%\newcommand{\fullwidth}{}
%\newcommand{\normalwidth}{}



%% makes a snazzy t-chart for evaluating functions
%\newenvironment{tchart}{\rowcolors{2}{}{background!90!textColor}\array}{\endarray}

%%This is to help with formatting on future title pages.
\newenvironment{sectionOutcomes}{}{}



%% Flowchart stuff
%\tikzstyle{startstop} = [rectangle, rounded corners, minimum width=3cm, minimum height=1cm,text centered, draw=black]
%\tikzstyle{question} = [rectangle, minimum width=3cm, minimum height=1cm, text centered, draw=black]
%\tikzstyle{decision} = [trapezium, trapezium left angle=70, trapezium right angle=110, minimum width=3cm, minimum height=1cm, text centered, draw=black]
%\tikzstyle{question} = [rectangle, rounded corners, minimum width=3cm, minimum height=1cm,text centered, draw=black]
%\tikzstyle{process} = [rectangle, minimum width=3cm, minimum height=1cm, text centered, draw=black]
%\tikzstyle{decision} = [trapezium, trapezium left angle=70, trapezium right angle=110, minimum width=3cm, minimum height=1cm, text centered, draw=black]

\usepackage{booktabs}
\usepackage{multirow}

\author{Zomercursus KU Leuven}
\outcome{Algebraïsch kunnen rekenen met breuken.}


\title[Rekenvaardigheden:]{Machten}

\begin{document}
\begin{abstract}
	
\end{abstract}
\maketitle

\subsection{De machtsverheffing: definitie en rekenregels}
\subsection{Machtsverheffing met een re\"eel grondtal en een gehele exponent}
\begin{definition}(Machtsverheffing met re\"eel grondtal en gehele exponent)
	Zij $a\in\R$ en $n\in\N$. De \emph{macht} $a^n$ (spreek uit: $a$
	tot de macht $n$ of $a$ tot de $n$-de) met \emph{grondtal} $a$ en
	\emph{exponent} $n$ wordt als volgt gedefinieerd:
	\[a^n\perdef
	\left\{
	\begin{array}{cl}
	1 & \textrm{als }n=0 \\
	\underbrace{a\cdot a\cdots a}_{n\mathrm{\ factoren}} & \textrm{als }n\in\Nnul. \\
	\end{array}
	\right.\]
	
	Zij $a\in\Rnul$ en $n\in\N$. De macht $a^{-n}$ met grondtal $a$ en
	exponent $-n$ wordt als volgt gedefinieerd:
	\[a^{-n}\perdef\frac{1}{a^n}.\]
\end{definition}
%\newpage
\begin{example}
	\begin{itemize}
		\item $a^0=1$ voor alle $a\in\Rnul$.
		\item $a^1=a$ voor alle $a\in\R$
		\item $\ds a^{-1}=\frac{1}{a}$ voor alle $a\in\Rnul$
		\item $0^n=0$ voor alle $n\in\Nnul$.
		\item $0^z$ is niet gedefinieerd voor $z\in\Zmin$.
		\item $1^z=1$ voor alle $z\in\Z$.
	\end{itemize}
\end{example}
\begin{proposition}
	Zij $x,y\in\R$ en $m,n\in\N$. Er geldt:
	\[
	\begin{array}{r@{\ =\ }l@{\qquad}l}
	\specialrule{.1em}{0ex}{1ex}
	\ds{x^{m}x^{n}}&\ds{x^{m+n}}&\\
	\midrule
	\ds{\frac{x^{m}}{x^{n}}}&\ds{x^{m-n}}&\textrm{als }x\neq0\\
	\midrule
	\ds{(xy)^n}&\ds{x^ny^n}&\\
	\midrule
	\ds{\left(\frac{x}{y}\right)^{n}}&\ds{\frac{x^{n}}{y^{n}}}&\textrm{als }y\neq0\\
	\midrule
	\ds{\left(x^{m}\right)^{n}}&\ds{x^{mn}}&\\
	\specialrule{.1em}{1ex}{0ex}
	\end{array}
	\]
	Zij $x,y\in\Rnul$ en $m,n\in\Z$. Er geldt:
	\[
	\begin{array}{r@{\ =\ }l}
	\specialrule{.1em}{0ex}{1ex}
	\ds{x^{m}x^{n}}&\ds{x^{m+n}}\\
	\midrule
	\ds{\frac{x^{m}}{x^{n}}}&\ds{x^{m-n}}\\
	\midrule
	\ds{(xy)^n}&\ds{x^ny^n}\\
	\midrule
	\ds{\left(\frac{x}{y}\right)^{n}}&\ds{\frac{x^{n}}{y^{n}}}\\
	\midrule
	\ds{\left(x^{m}\right)^{n}}&\ds{x^{mn}}\\
	\specialrule{.1em}{1ex}{0ex}
	\end{array}
	\]
\end{proposition}


\newpage
\subsection{Machtsverheffing met een strikt positief re\"eel grondtal en een re\"ele exponent}
\begin{definition}(Machtsverheffing met grondtal in \Rplus\ en exponent in \Q)
	
	%%%%%%\rule{0ex}{0ex}\\[-3ex]
	
	Zij $a\in\Rplus$ en $n\in\Nnul$. De macht $\ds
	a^{\frac{1}{n}}$ met grondtal $a$ en exponent $\ds\frac{1}{n}$ wordt
	gedefinieerd als het uniek positief re\"eel getal waarvan de $n$-de
	macht gelijk is aan $a$:
	\[\left\{
	\begin{array}{l}
	\ds a^{\frac{1}{n}}\geq 0\textrm{\ \ \ en}\\
	\ds\left(a^{\frac{1}{n}}\right)^n=a.\\
	\end{array}
	\right.\] De macht $\ds a^{\frac{1}{n}}$ wordt ook wel genoteerd
	met $\ds\sqrt[n]{a}$.
	
	Zij $a\in\Rnulplus$, $m\in\Z$ en $n\in\Nnul$. De macht $\ds
	a^{\frac{m}{n}}$ met grondtal $a$ en exponent $\ds\frac{m}{n}$
	wordt als volgt gedefinieerd:
	\[a^{\frac{m}{n}}\perdef \left(a^{\frac{1}{n}}\right)^m.\]
\end{definition}
\begin{example}
	\begin{itemize}
		\item $\ds a^{\frac{1}{n}}=\sqrt[n]{a}$ voor alle $a\in\Rnulplus$ en alle
		$n\in\Nnul$, i.h.b.~hebben we
		\item $\ds a^{\frac{1}{2}}=\sqrt{a}$ voor alle $a\in\Rnulplus$.
		\item $\ds a^{-\frac{1}{n}}=\frac{1}{\sqrt[n]{a}}=\sqrt[n]{\frac{1}{a}}$ voor alle $a\in\Rnulplus$ en alle $n\in\Nnul$
		\item $\ds a^{\frac{m}{n}}=\left(\sqrt[n]{a}\right)^m=\sqrt[n]{a^m}$ voor alle $a\in\Rnulplus$, alle $m\in\Z$ en alle $n\in\Nnul$
		\item $1^q=1$ voor alle $q\in\Q$.
	\end{itemize}
\end{example}
\subsubsection*{Machtsverheffing met een strikt positief re\"eel grondtal en een re\"ele exponent}
We wensen tenslotte ook $a^r$ te defini\"eren met $a\in\Rnulplus$
en $r\in\R$. Dit is niet evident en we zullen de definitie hier
ook niet in detail geven. Hiervoor verwijzen we naar de cursus
wiskunde uit het eerste jaar waar dit zeker nog aan bod komt. Wel
schets ik hier kort \'{e}\'{e}n mogelijke manier om zin te geven
aan de uitdrukking $a^r$ met $a\in\Rnulplus$ en $r\in\R$.

Neem dus $a\in\Rnulplus$ en $r\in\R$. We weten reeds dat $a^q$
goed gedefinieerd is voor elk rationaal getal $q$. Ook weet je
wellicht nog dat je elke re\"eel getal $r$ \lq oneindig goed\rq\
kan benaderen d.m.v.~rationale getallen. Concreet wil dit zeggen
dat gegeven een re\"eel getal $r$ er een oneindige rij
$q_1,q_2,q_3,\ldots$ bestaat van rationale getallen die $r$ met
steeds hogere nauwkeurigheid benaderen, zodat uiteindelijk elke
gewenste nauwkeurigheid vanaf een bepaald getal in de rij bereikt
wordt. Men zegt in dit geval dat de rij $q_1,q_2,q_3,\ldots$ naar
$r$ \emph{convergeert} en men noteert dit met \lq $q_n\to r$ als
$n\to\infty$\rq\ of met $\lim_{n\to\infty}q_n=r$.

Stel nu dat $q_1,q_2,q_3,\ldots$ zo'n rij is die naar $r$
convergeert. We kunnen dan voor elke $q_n$ in die rij de macht
$\ds a^{q_n}$ beschouwen. Op die manier bekomen we een nieuwe rij
$a^{q_1},a^{q_2},a^{q_3},\ldots$ Nu blijkt dat ook deze rij steeds
zal convergeren naar een zeker re\"eel getal $s$ en dat deze
limiet niet afhangt van de keuze van de rij $q_1,q_2,q_3,\ldots$
die we gebruikt hebben om $r$ te benaderen. Dit laat ons dan toe
om de macht $a^r$ te defini\"eren als dat getal $s$ dat je op die
manier bekomt en dat enkel afhangt van $a$ en $r$.

Zelfs voor machten met re\"ele exponenten (en dus i.h.b.~voor
machten met rationale exponenten) blijven de gebruikelijke
rekenregels (zoals we die reeds zagen voor machten met gehele
exponenten) gelden. We sommen ze hieronder nog eens op,
(natuurlijk) zonder bewijs.
\begin{proposition}
	Zij $x,y\in\Rnulplus$ en $r,s\in\R$. Er geldt:
	\[
	\begin{array}{r@{\ =\ }l}
	\specialrule{.1em}{0ex}{1ex}
	\ds{x^{r}x^{s}}&\ds{x^{r+s}}\\
	\midrule
	\ds{\frac{x^{r}}{x^{s}}}&\ds{x^{r-s}}\\
	\midrule
	\ds{(xy)^r}&\ds{x^ry^r}\\
	\midrule
	\ds{\left(\frac{x}{y}\right)^{r}}&\ds{\frac{x^{r}}{y^{r}}}\\
	\midrule
	\ds{\left(x^{r}\right)^{s}}&\ds{x^{rs}}\\
	\specialrule{.1em}{1ex}{0ex}
	\end{array}
	\]
\end{proposition}

\subsection{Voorbeeldoefeningen}
\begin{enumerate}
	\item Numerieke voorbeelden
	\begin{enumerate}
		\item $\ds{\left(\left(\frac{1}{2}\right)^{2}\right)^{-3}=\left(\frac{1}{2}\right)^{-6}=2^{6}=64}$
		\item $\ds{\left(\frac{\frac{1}{2}}{\frac{1}{3}}\right)^{2}=\frac{\left(\frac{1}{2}\right)^{2}}{\left(\frac{1}{3}\right)^{2}}=
			\frac{\frac{1}{4}}{\frac{1}{9}}=\frac{1}{4} \cdot 9=\frac{9}{4}}$
		\item $\ds{\frac{4}{\sqrt{8}}=\frac{2^{2}}{8^{\frac{1}{2}}}=\frac{2^{2}}{(2^{3})^{\frac{1}{2}}}=2^{2-\frac{3}{2}}=
			2^{\frac{1}{2}}=\sqrt{2}}$
		\item
		$\ds{\sqrt[3]{8} \cdot \sqrt{64}=8^{\frac{1}{3}}\cdot \sqrt{8^{2}}=8^{\frac{1}{3}}\cdot8=8^{\frac{1}{3}+1}=8^{\frac{4}{3}}=(\sqrt[3]{8})^4}=2^4=16$
	\end{enumerate}
	\pagebreak
	\item Stel $x,y \in \Rnul$ dan geldt:
	\begin{enumerate}
		\item
		$\ds{\frac{\sqrt[4]{x^{3}}}{x}=x^{\frac{3}{4}-1}=x^{\frac{-1}{4}}=\frac{1}{\sqrt[4]{x}}}$
		\item
		$\ds{\left(x^{3}\cdot \sqrt[4]{x}\right)^{2}=x^{6}\cdot \left(\sqrt[4]{x}\right)^{2}=x^{6}\cdot \left(x^{\frac{1}{4}}\right)^{2}=x^{6}\cdot x^{\frac{2}{4}}=
			x^{6+\frac{1}{2}}=x^{\frac{13}{2}}=\sqrt{x^{13}}}$
		\item
		$\ds{y^{2} \cdot \left(\frac{\sqrt{x^{2}y}}{xy}\right)^{4}=y^{2} \cdot \frac{\left(x^{2}y\right)^{\frac{4}{2}}}{x^{4}y^{4}}=y^{2}\cdot \frac{x^{4}y^{2}}{x^{4}y^{4}}
			=y^{2} \cdot \frac{1}{y^{2}}=1}$
		\item
		$\ds{\sqrt{48 x^{9}y^{4}}=\sqrt{(16 \cdot 3)(x^{8} \cdot
				x)y^{4}}=\sqrt{16 \cdot 3\cdot (x^{4})^{2}\cdot x\cdot
				(y^{2})^{2}}=4 x^{4}y^{2}\sqrt{3x}}$
		
	\end{enumerate}
\end{enumerate}



\subsection{Rekenregels machten}\label{machtenrekenregels}

\begin{proposition}
	
\subsection{Machten}
Zij $x,y>0$ en $m,n\in\RR$. Dan geldt:
\begin{align*}
	x^{m}x^{n}          &= x^{m+n}     & \text{(product van machten met zelfde grondtal)}\\
	\sqrt[n] x          &= x^{\frac1n}  & \text{(notatie)}\\
	\frac{x^{m}}{x^{n}} &= x^{m-n}      & \text{(quotiënt van machten)} \\
	(xy)^n              &= x^ny^n      & \text{(product van machten met zelfde exponent)}\\
	\left(x^{m}\right)^{n}&= x^{mn}    & \text{(macht van een macht)}\\	
\end{align*}
\end{proposition}


\end{document}
