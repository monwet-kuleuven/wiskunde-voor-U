\documentclass{ximera}
%\documentclass[wordchoicegiven]{ximera}

%
% copied from https://github.com/mooculus/calculus
%
\usepackage[utf8]{inputenc}


\graphicspath{
	{./}
	{goniometrie/}
}


%\usepackage{todonotes}
%\usepackage{mathtools} %% Required for wide table Curl and Greens
%\usepackage{cuted} %% Required for wide table Curl and Greens
\newcommand{\todo}{}

% Font niet (correct?) geinstalleerd in MikTeX?
%\usepackage{esint} % for \oiint
%\ifxake%%https://math.meta.stackexchange.com/questions/9973/how-do-you-render-a-closed-surface-double-integral
%\renewcommand{\oiint}{{\large\bigcirc}\kern-1.56em\iint}
%\fi


\newcommand{\mooculus}{\textsf{\textbf{MOOC}\textnormal{\textsf{ULUS}}}}

\usepackage{tkz-euclide}\usepackage{tikz}
\usepackage{tikz-cd}
\usetikzlibrary{arrows}
\tikzset{>=stealth,commutative diagrams/.cd,
  arrow style=tikz,diagrams={>=stealth}} %% cool arrow head
\tikzset{shorten <>/.style={ shorten >=#1, shorten <=#1 } } %% allows shorter vectors

\usetikzlibrary{backgrounds} %% for boxes around graphs
\usetikzlibrary{shapes,positioning}  %% Clouds and stars
\usetikzlibrary{matrix} %% for matrix
\usepgfplotslibrary{polar} %% for polar plots
\usepgfplotslibrary{fillbetween} %% to shade area between curves in TikZ
\usetkzobj{all}
\usepackage[makeroom]{cancel} %% for strike outs
%\usepackage{mathtools} %% for pretty underbrace % Breaks Ximera
%\usepackage{multicol}
\usepackage{pgffor} %% required for integral for loops



%% http://tex.stackexchange.com/questions/66490/drawing-a-tikz-arc-specifying-the-center
%% Draws beach ball
\tikzset{pics/carc/.style args={#1:#2:#3}{code={\draw[pic actions] (#1:#3) arc(#1:#2:#3);}}}



\usepackage{array}
\setlength{\extrarowheight}{+.1cm}
\newdimen\digitwidth
\settowidth\digitwidth{9}
\def\divrule#1#2{
\noalign{\moveright#1\digitwidth
\vbox{\hrule width#2\digitwidth}}}





\newcommand{\RR}{\mathbb R}
\newcommand{\R}{\mathbb R}
\newcommand{\N}{\mathbb N}
\newcommand{\Z}{\mathbb Z}

\newcommand{\sagemath}{\textsf{SageMath}}


%\renewcommand{\d}{\,d\!}
\renewcommand{\d}{\mathop{}\!d}
\newcommand{\dd}[2][]{\frac{\d #1}{\d #2}}
\newcommand{\pp}[2][]{\frac{\partial #1}{\partial #2}}
\renewcommand{\l}{\ell}
\newcommand{\ddx}{\frac{d}{\d x}}

\newcommand{\zeroOverZero}{\ensuremath{\boldsymbol{\tfrac{0}{0}}}}
\newcommand{\inftyOverInfty}{\ensuremath{\boldsymbol{\tfrac{\infty}{\infty}}}}
\newcommand{\zeroOverInfty}{\ensuremath{\boldsymbol{\tfrac{0}{\infty}}}}
\newcommand{\zeroTimesInfty}{\ensuremath{\small\boldsymbol{0\cdot \infty}}}
\newcommand{\inftyMinusInfty}{\ensuremath{\small\boldsymbol{\infty - \infty}}}
\newcommand{\oneToInfty}{\ensuremath{\boldsymbol{1^\infty}}}
\newcommand{\zeroToZero}{\ensuremath{\boldsymbol{0^0}}}
\newcommand{\inftyToZero}{\ensuremath{\boldsymbol{\infty^0}}}



\newcommand{\numOverZero}{\ensuremath{\boldsymbol{\tfrac{\#}{0}}}}
\newcommand{\dfn}{\textbf}
%\newcommand{\unit}{\,\mathrm}
\newcommand{\unit}{\mathop{}\!\mathrm}
\newcommand{\eval}[1]{\bigg[ #1 \bigg]}
\newcommand{\seq}[1]{\left( #1 \right)}
\renewcommand{\epsilon}{\varepsilon}
\renewcommand{\phi}{\varphi}


\renewcommand{\iff}{\Leftrightarrow}

\DeclareMathOperator{\arccot}{arccot}
\DeclareMathOperator{\arcsec}{arcsec}
\DeclareMathOperator{\arccsc}{arccsc}
\DeclareMathOperator{\si}{Si}
\DeclareMathOperator{\scal}{scal}
\DeclareMathOperator{\sign}{sign}


%% \newcommand{\tightoverset}[2]{% for arrow vec
%%   \mathop{#2}\limits^{\vbox to -.5ex{\kern-0.75ex\hbox{$#1$}\vss}}}
\newcommand{\arrowvec}[1]{{\overset{\rightharpoonup}{#1}}}
%\renewcommand{\vec}[1]{\arrowvec{\mathbf{#1}}}
\renewcommand{\vec}[1]{{\overset{\boldsymbol{\rightharpoonup}}{\mathbf{#1}}}\hspace{0in}}

\newcommand{\point}[1]{\left(#1\right)} %this allows \vector{ to be changed to \vector{ with a quick find and replace
\newcommand{\pt}[1]{\mathbf{#1}} %this allows \vec{ to be changed to \vec{ with a quick find and replace
\newcommand{\Lim}[2]{\lim_{\point{#1} \to \point{#2}}} %Bart, I changed this to point since I want to use it.  It runs through both of the exercise and exerciseE files in limits section, which is why it was in each document to start with.

\DeclareMathOperator{\proj}{\mathbf{proj}}
\newcommand{\veci}{{\boldsymbol{\hat{\imath}}}}
\newcommand{\vecj}{{\boldsymbol{\hat{\jmath}}}}
\newcommand{\veck}{{\boldsymbol{\hat{k}}}}
\newcommand{\vecl}{\vec{\boldsymbol{\l}}}
\newcommand{\uvec}[1]{\mathbf{\hat{#1}}}
\newcommand{\utan}{\mathbf{\hat{t}}}
\newcommand{\unormal}{\mathbf{\hat{n}}}
\newcommand{\ubinormal}{\mathbf{\hat{b}}}

\newcommand{\dotp}{\bullet}
\newcommand{\cross}{\boldsymbol\times}
\newcommand{\grad}{\boldsymbol\nabla}
\newcommand{\divergence}{\grad\dotp}
\newcommand{\curl}{\grad\cross}
%\DeclareMathOperator{\divergence}{divergence}
%\DeclareMathOperator{\curl}[1]{\grad\cross #1}
\newcommand{\lto}{\mathop{\longrightarrow\,}\limits}

\renewcommand{\bar}{\overline}

\colorlet{textColor}{black}
\colorlet{background}{white}
\colorlet{penColor}{blue!50!black} % Color of a curve in a plot
\colorlet{penColor2}{red!50!black}% Color of a curve in a plot
\colorlet{penColor3}{red!50!blue} % Color of a curve in a plot
\colorlet{penColor4}{green!50!black} % Color of a curve in a plot
\colorlet{penColor5}{orange!80!black} % Color of a curve in a plot
\colorlet{penColor6}{yellow!70!black} % Color of a curve in a plot
\colorlet{fill1}{penColor!20} % Color of fill in a plot
\colorlet{fill2}{penColor2!20} % Color of fill in a plot
\colorlet{fillp}{fill1} % Color of positive area
\colorlet{filln}{penColor2!20} % Color of negative area
\colorlet{fill3}{penColor3!20} % Fill
\colorlet{fill4}{penColor4!20} % Fill
\colorlet{fill5}{penColor5!20} % Fill
\colorlet{gridColor}{gray!50} % Color of grid in a plot

\newcommand{\surfaceColor}{violet}
\newcommand{\surfaceColorTwo}{redyellow}
\newcommand{\sliceColor}{greenyellow}




\pgfmathdeclarefunction{gauss}{2}{% gives gaussian
  \pgfmathparse{1/(#2*sqrt(2*pi))*exp(-((x-#1)^2)/(2*#2^2))}%
}


%%%%%%%%%%%%%
%% Vectors
%%%%%%%%%%%%%

%% Simple horiz vectors
\renewcommand{\vector}[1]{\left\langle #1\right\rangle}


%% %% Complex Horiz Vectors with angle brackets
%% \makeatletter
%% \renewcommand{\vector}[2][ , ]{\left\langle%
%%   \def\nextitem{\def\nextitem{#1}}%
%%   \@for \el:=#2\do{\nextitem\el}\right\rangle%
%% }
%% \makeatother

%% %% Vertical Vectors
%% \def\vector#1{\begin{bmatrix}\vecListA#1,,\end{bmatrix}}
%% \def\vecListA#1,{\if,#1,\else #1\cr \expandafter \vecListA \fi}

%%%%%%%%%%%%%
%% End of vectors
%%%%%%%%%%%%%

%\newcommand{\fullwidth}{}
%\newcommand{\normalwidth}{}



%% makes a snazzy t-chart for evaluating functions
%\newenvironment{tchart}{\rowcolors{2}{}{background!90!textColor}\array}{\endarray}

%%This is to help with formatting on future title pages.
\newenvironment{sectionOutcomes}{}{}



%% Flowchart stuff
%\tikzstyle{startstop} = [rectangle, rounded corners, minimum width=3cm, minimum height=1cm,text centered, draw=black]
%\tikzstyle{question} = [rectangle, minimum width=3cm, minimum height=1cm, text centered, draw=black]
%\tikzstyle{decision} = [trapezium, trapezium left angle=70, trapezium right angle=110, minimum width=3cm, minimum height=1cm, text centered, draw=black]
%\tikzstyle{question} = [rectangle, rounded corners, minimum width=3cm, minimum height=1cm,text centered, draw=black]
%\tikzstyle{process} = [rectangle, minimum width=3cm, minimum height=1cm, text centered, draw=black]
%\tikzstyle{decision} = [trapezium, trapezium left angle=70, trapezium right angle=110, minimum width=3cm, minimum height=1cm, text centered, draw=black]


\author{Zomercursus KU Leuven}
\outcome{Algebraïsch kunnen rekenen met machten.}


\title[Rekenvaardigheden:]{Machten}

\begin{document}
\begin{abstract}
	$(M8)^2 = \sqrt{666}$
	% Macht in het kwadraat is de wortel van het kwaad ...; needs some explanation !
\end{abstract}
\maketitle

\subsection{Machten met een re\"eel grondtal en een gehele exponent}

Eén manier om de vermenigvuldiging te definiëren is via 'vermenigvuldigen is herhaald optellen': $3\times 2 \perdef 2 + 2+ 2$, en $n\times a = a+a+\dots +a$ ($n$ keer). 

Op dezelfde manier geldt: 'machtsverheffen is herhaald vermenigvuldigen': $2^3 \perdef 2\times2\times2$. 

Het blijkt handig om dit begrip direct uit te breiden tot negatieve exponenten. 

\begin{definition}(Machtsverheffing met re\"eel grondtal en gehele exponent) \ 
	
	Zij $a\in\Rnul$ en $n\in\N$. 
	
	We definiëren de \emph{$n$-de macht}  van $a$ en noteren $a^n$:
	% (spreek uit: $a$	tot de macht $n$ of $a$ tot de $n$-de) met \emph{grondtal} $a$ en 	\emph{exponent} $n$ wordt als volgt gedefinieerd:
	\[a^n\perdef
	\left\{
	\begin{array}{cl}
	1 & \textrm{als }n=0 \\
	\underbrace{a\cdot a\cdots a}_{n\mathrm{\ factoren}} & \textrm{als }n\in\Nnul. \\
	\end{array}
	\right.\]
	
	We noemen $a$ het \textit{grondtal} en $n$ de \textit{exponent}.
		
	We definiëren ook machten met negatieve exponenten:
	
	\[a^{-n}\perdef\frac{1}{a^n}.\]
	
	Voor $a=0$ en $n>0$ definiëren we $0^n = 0$. 
	
	De uitdrukkingen $0^0$ en $0^{-n}$ zijn \textit{onbepaald} en hebben dus \textit{geen betekenis}.
	
\end{definition}

%\renewcommand\grapje[1]{}

\begin{remark} \ 
	
	\begin{enumerate}
	\item Onmiddellijke gevolgen van de definities:
			\begin{itemize}
			\item $a^0=$\wordChoice{\choice[correct]{$1$}\choice{$0$}\choice{$a$}} voor alle $a\in\Rnul$.
			\item $a^1=$\wordChoice{\choice{$1$}\choice{$0$}\choice[correct]{$a$}} voor alle $a\in\R$
			\item $a^{-1}=\frac{1}{a}$ voor alle $a\in$\wordChoice{\choice{$\R$}\choice{$\Rplus$}\choice[correct]{$\Rnul$}}
			%		\item $0^n=0$ voor alle $n\in\Nnul$.
			%		\item $0^z$ is niet gedefinieerd voor $z\in\Zmin$.
			\item $1^n=1$ voor alle $n\in$\wordChoice{\choice{$\N$}\choice{$\R$}\choice[correct]{$\Z$}}.
		\end{itemize}
	\grapje{\item (Uitweiding) Er is natuurlijk geen enkele reden om 'Y is herhaald X'en' niet verder te herhalen. Zie \link[Wikipedia]{https://nl.wikipedia.org/wiki/Knuths_pijlomhoognotatie} voor meer info.
	}
\end{enumerate} 
\end{remark}

\begin{proposition}
	Zij $x,y\in\R$ en $m,n\in\N$. Er geldt:
{
	\savebox\strutbox{$\vphantom{\dfrac11^n}$}   % hack; doesn't work in html
\begin{align*}
	x^{m}x^{n}  & = x^{m+n}            & \text{(product van gelijksoortige machten)} \\
	(xy)^n      & = x^ny^n             & \text{(macht van product)}\\
	\left(x^{m}\right)^{n} & = x^{mn}  & \text{(macht van een macht)} \\
	\\
	\frac{x^{m}}{x^{n}}    & = x^{m-n} & \text{als }x\neq0\\
	\\
	\left(\frac{x}{y}\right)^{n} & = \frac{x^{n}}{y^{n}} & \text{als }y\neq0\\
\end{align*}

	Dezelfde regels gelden voor gehele $m,n\in\Z$ voor zover ze betekenis hebben 
	\\ (dus: grondtal niet nul als de exponent negatief is).
	
	Dezelfde regels gelden ook voor reële $m,n\in\R$ voor zover ze betekenis hebben
	\\ (zie verder paragraaf \ref{machten-r}).
}
\end{proposition}

\begin{proof} We geven enkel het bewijs voor $m,n\in\N$, en dat bestaat telkens uit een eenvoudige toepassing van de definitie en de rekenregels voor producten (dus: commutativiteit en associativiteit):
	\begin{align*}
		x^{m}x^{n}  
			& = \underbrace{(x\cdot x\cdot\ldots \cdot x)}_{n \text{ factoren}}\cdot \underbrace{(x\cdot x\cdot\ldots \cdot x)}_{m \text{ factoren}} 
			\underset{(\cdot \text{ is assoc.})}{=} \underbrace{(x\cdot x\cdot\ldots \cdot x)}_{n+m \text{ factoren}}
			=  x^{m+n} \\
		(xy)^{n}  
		  & = \underbrace{(xy)\cdot (xy)\cdot\ldots \cdot (xy)}_{n \text{ factoren}} 
			\underset{(\cdot \text{ is assoc. en comm.})}{=} \underbrace{(x\cdot x\cdot\ldots \cdot x )}_{n \text{ factoren}} \cdot \underbrace{(y\cdot y\cdot\ldots \cdot y)}_{n\text{ factoren}} =  x^ny^n  \\
		(x^m)^{n}  
		  & = \underbrace{x^m\cdot x^m \cdot\ldots \cdot x^m}_{n \text{ factoren}} 
		    = \underbrace{\underset{m \text{ factoren}}{(x\cdot x\cdot\ldots \cdot x)}\cdot \underset{m \text{ factoren}}{(x\cdot x\cdot\ldots \cdot x)}\cdot\ldots \cdot \underset{m \text{ factoren}}{(x\cdot x\cdot\ldots \cdot x)}}_{n \text{ factoren}} 
		    = \underbrace{x\cdot x\cdot\ldots \cdot x}_{n\cdot m\text{ factoren}} 
		    =  x^{mn}  \\
		 & \text{ Men kan bovenstaande uitdrukkingen ook bewijzen voor $m,n \in\Z$} \\
	 	\frac{x^{m}}{x^{n}}    
	 	  & \underset{(\text{def. $x^{-n}$})}{=} x^m \cdot x^{-n}  
	 	  \underset{\text{(zie boven!)}}{=} x^{m-n} \\
	 	\left(\frac{x}{y}\right)^n    
	 	  & = (x\cdot y^{-1})^n = x^n\cdot y^{-n} = \frac{x^n}{y^n}   
	\end{align*}
\end{proof}

\begin{example} 
	$\left(\left(\frac{1}{2}\right)^{2}\right)^{-3}=\answer[given]{64}$
	\begin{feedback} (want $\left(\left(\frac{1}{2}\right)^{2}\right)^{-3} = \left(\left(2^{-1}\right)^{2}\right)^{-3} = 2^{(-1)\cdot2\cdot6} = 2^6 = 64$ )
	\end{feedback}
\end{example}	
\begin{example} $\left(\frac{\frac{1}{2}}{\frac{1}{3}}\right)^{2}= \answer[given]{\frac{9}{4}}$
	\begin{feedback}
     (want $\left(\frac{\frac{1}{2}}{\frac{1}{3}}\right)^{2} = \frac{\left(2^{-1}\right)^{2}}{\left(3^{-1}\right)^{2}}= 2^{-2}\cdot3^2= \frac{9}{4}$)
	\end{feedback}
\end{example}	

\subsection{Machten met een strikt positief re\"eel grondtal en een rationale exponent}

Als we ons beperken tot positieve grondtallen, kunnen we de machtsverheffing tamelijk eenvoudig nog verder veralgemenen: we definiëren eerst $a^{\frac1n}$, dan $a^{\frac mn} = a^q, q\in\Q$, en tenslotte zelfs $a^r, r\in\R$.
 

\begin{definition}(Machtsverheffing met grondtal in \Rplus\ en exponent in \Q)
	
	Zij $a,b\in\Rplus$ en $n\in\Nnul$, $m\in\Z$. 
	
	De macht $a^{\frac{1}{n}}$ met grondtal $a$ en exponent $\ds\frac{1}{n}$ wordt
	gedefinieerd als het uniek positief re\"eel getal $b$ waarvan de $n$-de
	macht gelijk is aan $a$:
	\[b = a^\frac 1n \iff b^n = a\]

	
	We noemen $a^{\frac{1}{n}}$ de \textit{$n$-de machtswortel }uit $a$, of de $n$-de wortel uit $a$ en noteren
	\[
		\sqrt[n]{a} \perdef a^\frac 1n
	\]
	We definiëren ook
	\[a^{\frac{m}{n}}\perdef \left(a^{\frac{1}{n}}\right)^m.\]
\end{definition}

\begin{remark} \ 
\begin{enumerate}
	\item Voor $n=2$ schrijven we gewoon $\sqrt{a}$ in plaats van $\sqrt[2]{a}$.
	\item De bovenstaande definitie is enkel zinvol als er inderdaad een dergelijk uniek positief reëel getal $b$ bestaat. Precies om dat te garanderen, hebben we ons beperkt tot \textsc{positieve} $a$ en $b$. Voor negatieve grondtallen is de situatie subtieler. We bespreken het kort in volgende uitweiding:
	\begin{uitweiding}
		\begin{expandable}
	Voor negatieve grondtallen is de definitie van machten met rationale exponenten wat subtieler, en we zullen er hier niet verder op ingaan. Het kan volstaan op te merken dat $-8$ geen vierkantswortel heeft, maar wel een (negatieve!) derdemachtswortel (namelijk $-2$, want $(-2)^3 = -8$). Maar, als we negatieve wortels toelaten, dan heeft $4$ weer twee mogelijke wortels, namelijk $2$ en $-2$.  
			
	De definitie gebruikt een veel voorkomend wiskundig paradigma: we willen een nieuw begrip $X$  definiëren door een voorwaarde te geven waaraan $X$ moet voldoen. Als we ten eerste kunnen bewijzen dat er inderdaad zo'n $X$ bestaat (existentie), en ten tweede dat zo'n $X$ uniek is (uniciteit), dan hebben we een goede definitie (we zeggen: 'begrip $X$ is goed gedefinieerd').
	
	Voorbeelden:
	\begin{enumerate}
		\item Definitie van \textit{negatieve getallen}: als $a\in\N$, dan definiëren we $-a$ als het unieke getal $b$ zodat $a + b = 0$. 
		\item Definitie van \textit{logaritme}: als $a,x\in\Rplus$, dan definiëren we ${}^a\log(x)$ als het unieke getal $b$ zodat $a^b = x$.  
		\item Niet-voorbeeld voor \textit{vierkantswortel}: als $a\in\R$ dan zouden we naiefweg de vierkantswortel $\sqrt{a}$ kunnen proberen definëren als het unieke reëel getal $b\in\R$ zodat $b^2 = a$. Deze definitie is fout,  ten eerste omdat voor negatieve $a$ een dergelijke $b$ niet bestaat, en ten tweede omdat er voor positieve $a$ steeds twee dergelijke $b$'s zijn (namelijk een positieve en een negatieve 'vierkantswortel'). In bovenstaande definitie is dit probleem opgelost door zowel $a$ als $b$ te beperken tot \textit{positieve} reële getallen: in dat geval hebben we zowel existentie als uniciteit. Maar, de oplossing voldoet niet helemaal, want we zouden graag vierkantswortles kunne trekken uit negatieve getallen. Dat kan, als we complexe getallen toelaten. Maar, dan wordt de uniciteit problematisch: voor complexe getallen heeft het begrip '\textit{positieve} wortel' dan weer geen betekenis meer. Miserie miserie. \todo{referentie wortels van complexe getallen toevoegen ?}
	\end{enumerate}
\end{expandable}
\end{uitweiding}

 \item Onmiddellijke gevolgen van de definities:
	\begin{itemize}
%		\item $\ds a^{\frac{1}{n}}=\sqrt[n]{a}$ voor alle $a\in\Rnulplus$ en alle 	$n\in\Nnul$, i.h.b.~hebben we
%		\item $\ds a^{\frac{1}{2}}=\sqrt{a}$ voor alle $a\in\Rnulplus$.
		\item $\ds a^{-\frac{1}{n}}=\frac{1}{\sqrt[n]{a}}=\sqrt[n]{\frac{1}{a}}$ voor alle $a\in\Rnulplus$ en alle $n\in\Nnul$
		\item $\ds a^{\frac{m}{n}}=\left(\sqrt[n]{a}\right)^m=\sqrt[n]{a^m}$ voor alle $a\in\Rnulplus$, alle $m\in\Z$ en alle $n\in\Nnul$
		\item $1^q=1$ voor alle $q\in\Q$.
	\end{itemize}
	\end{enumerate}


\end{remark}

\begin{example} Bereken
 $\frac{4}{\sqrt{8}}= \answer[given]{\sqrt{2}}$
 	\begin{feedback}
 		(want $\frac{2^{2}}{8^{\frac{1}{2}}}=2^{2}(2^{-3})^{\frac{1}{2}}=2^{2-\frac{3}{2}}=
	2^{\frac{1}{2}}=\sqrt{2}$)
\end{feedback}
\end{example}
\begin{example} Bereken $\sqrt[3]{8} \cdot \sqrt{64}= \answer[given]{16}$
	\begin{feedback}	(want $\sqrt[3]{8} \cdot \sqrt{64} = 8^{\frac{1}{3}}\cdot \sqrt{8^{2}}=8^{\frac{1}{3}}\cdot8=8^{\frac{1}{3}+1}=8^{\frac{4}{3}}=(\sqrt[3]{8})^4=2^4=16$)
	\end{feedback}
\end{example}
%	Stel $x,y \in \Rnul$ dan geldt:

\begin{example} Bereken 
	$\frac{\sqrt[4]{x^{3}}}{x}=\answer[given]{x^{-\frac14}}$
	\begin{feedback} ( want $\frac{\sqrt[4]{x^{3}}}{x} = x^{\frac{3}{4}-1}=x^{\frac{-1}{4}}=\frac{1}{\sqrt[4]{x}}$ )
	\end{feedback}
\end{example}
\begin{example} Bereken 
	$\left(x^{3}\cdot \sqrt[4]{x}\right)^{2}= \answer[given]{\sqrt{x^13}}$
	\begin{feedback} ( want $\left(x^{3}\cdot \sqrt[4]{x}\right)^{2} = x^{6}\cdot \left(\sqrt[4]{x}\right)^{2}=x^{6}\cdot \left(x^{\frac{1}{4}}\right)^{2}=x^{6}\cdot x^{\frac{2}{4}}=
		x^{6+\frac{1}{2}}=x^{\frac{13}{2}}=\sqrt{x^{13}}$ )
	\end{feedback}
\end{example}
\begin{example} Bereken 
	$y^{2} \cdot \left(\frac{\sqrt{x^{2}y}}{xy}\right)^{4}=\answer[given]{1}$ \begin{feedback} (want $y^{2} \cdot \left(\frac{\sqrt{x^{2}y}}{xy}\right)^{4} = y^{2} \cdot \frac{\left(x^{2}y\right)^{\frac{4}{2}}}{x^{4}y^{4}}=y^{2}\cdot \frac{x^{4}y^{2}}{x^{4}y^{4}}
		=y^{2} \cdot \frac{1}{y^{2}}=1$ )
	\end{feedback}
	\end{example}
\begin{example} Bereken
	$\sqrt{48 x^{9}y^{4}}= \answer[given]{4 x^{4}y^{2}\sqrt{3x}}$
		\begin{feedback} (want $\sqrt{48 x^{9}y^{4}} = 
		\sqrt{(16 \cdot 3)(x^{8} \cdot
			x)y^{4}}=\sqrt{16 \cdot 3\cdot (x^{4})^{2}\cdot x\cdot
			(y^{2})^{2}}=4 x^{4}y^{2}\sqrt{3x}$ )
		\end{feedback}
\end{example}
\todo{Voorbeeld enigszins dubieus zonder verdere uitleg?}

\subsection*{Machten met een strikt positief re\"eel grondtal en een re\"ele exponent}\label{machten-r}
Men kan ook $a^r$ definiëren met $a\in\Rnulplus$
en $r\in\R$ (dus uitdrukkingen zoals $2^{\sqrt{3}}$, $\pi^{\sqrt{5}}$ en $\sqrt{7}^{\sqrt{\pi}}$). Dit is technisch enigszins subtiel en we zullen de definitie hier
ook niet in detail geven. 

Wie toch nieuwsgierig is kan volgende uitweiding lezen:


\begin{uitweiding}
	\begin{expandable}
Details vind je in wat meer gevorderde cursussen wiskunde, maar hier schetsen we kort \'{e}\'{e}n mogelijke manier om een precieze betekenis te geven
aan de uitdrukking $a^r$ met $a\in\Rnulplus$ en $r\in\R$.

Neem dus $a\in\Rnulplus$ en $r\in\R$. We weten reeds dat $a^q$
goed gedefinieerd is voor elk rationaal getal $q$. Ook weet je
wellicht nog dat je elke re\"eel getal $r$ \lq oneindig goed\rq\
kan benaderen d.m.v.~rationale getallen. Concreet wil dit zeggen
dat gegeven een re\"eel getal $r$ er een oneindige rij
$q_1,q_2,q_3,\ldots$ bestaat van rationale getallen die $r$ met
steeds hogere nauwkeurigheid benaderen, zodat uiteindelijk elke
gewenste nauwkeurigheid vanaf een bepaald getal in de rij bereikt
wordt. Men zegt in dit geval dat de rij $q_1,q_2,q_3,\ldots$ naar
$r$ \emph{convergeert} en men noteert dit met \lq $q_n\to r$ als
$n\to\infty$\rq\ of met $\lim_{n\to\infty}q_n=r$.

Stel nu dat $q_1,q_2,q_3,\ldots$ zo'n rij is die naar $r$
convergeert. We kunnen dan voor elke $q_n$ in die rij de macht
$\ds a^{q_n}$ beschouwen. Op die manier bekomen we een nieuwe rij
$a^{q_1},a^{q_2},a^{q_3},\ldots$ 	
%%%%%%\rule{0ex}{0ex}\\[-3ex]
Men kan bewijzen dat ook deze rij steeds
zal convergeren naar een zeker re\"eel getal $s$ en dat deze
limiet niet afhangt van de keuze van de rij $q_1,q_2,q_3,\ldots$
die we gebruikt hebben  om $r$ te benaderen. Dit laat ons dan toe
om de macht $a^r$ te defini\"eren als dat getal $s$ dat je op die
manier bekomt en dat enkel afhangt van $a$ en $r$.
\end{expandable}
\end{uitweiding}

Ook voor machten met re\"ele exponenten (en dus i.h.b.~voor
machten met rationale exponenten) blijven de gebruikelijke
rekenregels  gelden:

\begin{proposition}
	Zij $x,y\in\Rnulplus$ en $r,s\in\R$. Er geldt:
	\[
	\arraycolsep=1.4pt\def\arraystretch{2.2}
	\begin{array}{r@{\ =\ }l}
	\ds{x^{r}x^{s}}&\ds{x^{r+s}}\\
	\ds{(xy)^r}&\ds{x^ry^r}\\
	\ds{\left(x^{r}\right)^{s}}&\ds{x^{rs}}\\
	x^{-r} & \frac{1}{x^r} \\
	\end{array}
	\]
\end{proposition}



\subsection{Herhaling / Onthoud }\label{machtenrekenregels}

\begin{proposition}
Zij $x,y>0$ en $m,n\in\R$. Dan geldt:
\begin{align*}
	x^{m}x^{n}          &= x^{m+n}     & \text{(product van machten met zelfde grondtal)}\\
	(xy)^n              &= x^ny^n      & \text{(product van machten met zelfde exponent)}\\
	\left(x^{m}\right)^{n}&= x^{mn}    & \text{(macht van een macht)}\\	
	x^{\frac1n} &= \sqrt[n] x          & \text{(notatie/definitie)}\\
	x^{-n}          &= \frac{1}{x^n}  & \text{(notatie/definitie)}\\
	\frac{x^{m}}{x^{n}} &= x^{m-n}      & \text{(quotiënt van machten)} \\
	x^0&=1, \quad x^1=x, \quad x^{-1}=\frac1x \\
	\sqrt[m]{x^n} & = x^{\frac{n}{m}} = (\sqrt[m]{x})^n
\end{align*}
\end{proposition}


\end{document}
