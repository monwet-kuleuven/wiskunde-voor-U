\documentclass{beamer}
%\documentclass[handout]{beamer}

%
% copied from https://github.com/mooculus/calculus
%
\usepackage[utf8]{inputenc}


\graphicspath{
	{./}
	{goniometrie/}
}


%\usepackage{todonotes}
%\usepackage{mathtools} %% Required for wide table Curl and Greens
%\usepackage{cuted} %% Required for wide table Curl and Greens
\newcommand{\todo}{}

% Font niet (correct?) geinstalleerd in MikTeX?
%\usepackage{esint} % for \oiint
%\ifxake%%https://math.meta.stackexchange.com/questions/9973/how-do-you-render-a-closed-surface-double-integral
%\renewcommand{\oiint}{{\large\bigcirc}\kern-1.56em\iint}
%\fi


\newcommand{\mooculus}{\textsf{\textbf{MOOC}\textnormal{\textsf{ULUS}}}}

\usepackage{tkz-euclide}\usepackage{tikz}
\usepackage{tikz-cd}
\usetikzlibrary{arrows}
\tikzset{>=stealth,commutative diagrams/.cd,
  arrow style=tikz,diagrams={>=stealth}} %% cool arrow head
\tikzset{shorten <>/.style={ shorten >=#1, shorten <=#1 } } %% allows shorter vectors

\usetikzlibrary{backgrounds} %% for boxes around graphs
\usetikzlibrary{shapes,positioning}  %% Clouds and stars
\usetikzlibrary{matrix} %% for matrix
\usepgfplotslibrary{polar} %% for polar plots
\usepgfplotslibrary{fillbetween} %% to shade area between curves in TikZ
\usetkzobj{all}
\usepackage[makeroom]{cancel} %% for strike outs
%\usepackage{mathtools} %% for pretty underbrace % Breaks Ximera
%\usepackage{multicol}
\usepackage{pgffor} %% required for integral for loops



%% http://tex.stackexchange.com/questions/66490/drawing-a-tikz-arc-specifying-the-center
%% Draws beach ball
\tikzset{pics/carc/.style args={#1:#2:#3}{code={\draw[pic actions] (#1:#3) arc(#1:#2:#3);}}}



\usepackage{array}
\setlength{\extrarowheight}{+.1cm}
\newdimen\digitwidth
\settowidth\digitwidth{9}
\def\divrule#1#2{
\noalign{\moveright#1\digitwidth
\vbox{\hrule width#2\digitwidth}}}





\newcommand{\RR}{\mathbb R}
\newcommand{\R}{\mathbb R}
\newcommand{\N}{\mathbb N}
\newcommand{\Z}{\mathbb Z}

\newcommand{\sagemath}{\textsf{SageMath}}


%\renewcommand{\d}{\,d\!}
\renewcommand{\d}{\mathop{}\!d}
\newcommand{\dd}[2][]{\frac{\d #1}{\d #2}}
\newcommand{\pp}[2][]{\frac{\partial #1}{\partial #2}}
\renewcommand{\l}{\ell}
\newcommand{\ddx}{\frac{d}{\d x}}

\newcommand{\zeroOverZero}{\ensuremath{\boldsymbol{\tfrac{0}{0}}}}
\newcommand{\inftyOverInfty}{\ensuremath{\boldsymbol{\tfrac{\infty}{\infty}}}}
\newcommand{\zeroOverInfty}{\ensuremath{\boldsymbol{\tfrac{0}{\infty}}}}
\newcommand{\zeroTimesInfty}{\ensuremath{\small\boldsymbol{0\cdot \infty}}}
\newcommand{\inftyMinusInfty}{\ensuremath{\small\boldsymbol{\infty - \infty}}}
\newcommand{\oneToInfty}{\ensuremath{\boldsymbol{1^\infty}}}
\newcommand{\zeroToZero}{\ensuremath{\boldsymbol{0^0}}}
\newcommand{\inftyToZero}{\ensuremath{\boldsymbol{\infty^0}}}



\newcommand{\numOverZero}{\ensuremath{\boldsymbol{\tfrac{\#}{0}}}}
\newcommand{\dfn}{\textbf}
%\newcommand{\unit}{\,\mathrm}
\newcommand{\unit}{\mathop{}\!\mathrm}
\newcommand{\eval}[1]{\bigg[ #1 \bigg]}
\newcommand{\seq}[1]{\left( #1 \right)}
\renewcommand{\epsilon}{\varepsilon}
\renewcommand{\phi}{\varphi}


\renewcommand{\iff}{\Leftrightarrow}

\DeclareMathOperator{\arccot}{arccot}
\DeclareMathOperator{\arcsec}{arcsec}
\DeclareMathOperator{\arccsc}{arccsc}
\DeclareMathOperator{\si}{Si}
\DeclareMathOperator{\scal}{scal}
\DeclareMathOperator{\sign}{sign}


%% \newcommand{\tightoverset}[2]{% for arrow vec
%%   \mathop{#2}\limits^{\vbox to -.5ex{\kern-0.75ex\hbox{$#1$}\vss}}}
\newcommand{\arrowvec}[1]{{\overset{\rightharpoonup}{#1}}}
%\renewcommand{\vec}[1]{\arrowvec{\mathbf{#1}}}
\renewcommand{\vec}[1]{{\overset{\boldsymbol{\rightharpoonup}}{\mathbf{#1}}}\hspace{0in}}

\newcommand{\point}[1]{\left(#1\right)} %this allows \vector{ to be changed to \vector{ with a quick find and replace
\newcommand{\pt}[1]{\mathbf{#1}} %this allows \vec{ to be changed to \vec{ with a quick find and replace
\newcommand{\Lim}[2]{\lim_{\point{#1} \to \point{#2}}} %Bart, I changed this to point since I want to use it.  It runs through both of the exercise and exerciseE files in limits section, which is why it was in each document to start with.

\DeclareMathOperator{\proj}{\mathbf{proj}}
\newcommand{\veci}{{\boldsymbol{\hat{\imath}}}}
\newcommand{\vecj}{{\boldsymbol{\hat{\jmath}}}}
\newcommand{\veck}{{\boldsymbol{\hat{k}}}}
\newcommand{\vecl}{\vec{\boldsymbol{\l}}}
\newcommand{\uvec}[1]{\mathbf{\hat{#1}}}
\newcommand{\utan}{\mathbf{\hat{t}}}
\newcommand{\unormal}{\mathbf{\hat{n}}}
\newcommand{\ubinormal}{\mathbf{\hat{b}}}

\newcommand{\dotp}{\bullet}
\newcommand{\cross}{\boldsymbol\times}
\newcommand{\grad}{\boldsymbol\nabla}
\newcommand{\divergence}{\grad\dotp}
\newcommand{\curl}{\grad\cross}
%\DeclareMathOperator{\divergence}{divergence}
%\DeclareMathOperator{\curl}[1]{\grad\cross #1}
\newcommand{\lto}{\mathop{\longrightarrow\,}\limits}

\renewcommand{\bar}{\overline}

\colorlet{textColor}{black}
\colorlet{background}{white}
\colorlet{penColor}{blue!50!black} % Color of a curve in a plot
\colorlet{penColor2}{red!50!black}% Color of a curve in a plot
\colorlet{penColor3}{red!50!blue} % Color of a curve in a plot
\colorlet{penColor4}{green!50!black} % Color of a curve in a plot
\colorlet{penColor5}{orange!80!black} % Color of a curve in a plot
\colorlet{penColor6}{yellow!70!black} % Color of a curve in a plot
\colorlet{fill1}{penColor!20} % Color of fill in a plot
\colorlet{fill2}{penColor2!20} % Color of fill in a plot
\colorlet{fillp}{fill1} % Color of positive area
\colorlet{filln}{penColor2!20} % Color of negative area
\colorlet{fill3}{penColor3!20} % Fill
\colorlet{fill4}{penColor4!20} % Fill
\colorlet{fill5}{penColor5!20} % Fill
\colorlet{gridColor}{gray!50} % Color of grid in a plot

\newcommand{\surfaceColor}{violet}
\newcommand{\surfaceColorTwo}{redyellow}
\newcommand{\sliceColor}{greenyellow}




\pgfmathdeclarefunction{gauss}{2}{% gives gaussian
  \pgfmathparse{1/(#2*sqrt(2*pi))*exp(-((x-#1)^2)/(2*#2^2))}%
}


%%%%%%%%%%%%%
%% Vectors
%%%%%%%%%%%%%

%% Simple horiz vectors
\renewcommand{\vector}[1]{\left\langle #1\right\rangle}


%% %% Complex Horiz Vectors with angle brackets
%% \makeatletter
%% \renewcommand{\vector}[2][ , ]{\left\langle%
%%   \def\nextitem{\def\nextitem{#1}}%
%%   \@for \el:=#2\do{\nextitem\el}\right\rangle%
%% }
%% \makeatother

%% %% Vertical Vectors
%% \def\vector#1{\begin{bmatrix}\vecListA#1,,\end{bmatrix}}
%% \def\vecListA#1,{\if,#1,\else #1\cr \expandafter \vecListA \fi}

%%%%%%%%%%%%%
%% End of vectors
%%%%%%%%%%%%%

%\newcommand{\fullwidth}{}
%\newcommand{\normalwidth}{}



%% makes a snazzy t-chart for evaluating functions
%\newenvironment{tchart}{\rowcolors{2}{}{background!90!textColor}\array}{\endarray}

%%This is to help with formatting on future title pages.
\newenvironment{sectionOutcomes}{}{}



%% Flowchart stuff
%\tikzstyle{startstop} = [rectangle, rounded corners, minimum width=3cm, minimum height=1cm,text centered, draw=black]
%\tikzstyle{question} = [rectangle, minimum width=3cm, minimum height=1cm, text centered, draw=black]
%\tikzstyle{decision} = [trapezium, trapezium left angle=70, trapezium right angle=110, minimum width=3cm, minimum height=1cm, text centered, draw=black]
%\tikzstyle{question} = [rectangle, rounded corners, minimum width=3cm, minimum height=1cm,text centered, draw=black]
%\tikzstyle{process} = [rectangle, minimum width=3cm, minimum height=1cm, text centered, draw=black]
%\tikzstyle{decision} = [trapezium, trapezium left angle=70, trapezium right angle=110, minimum width=3cm, minimum height=1cm, text centered, draw=black]


\usepackage{pgfplots}

%\newcommand{\ds}{\displaystyle}
%\newcommand{\N}{\ensuremath{\mathbb{N}}}
%\newcommand{\Nnul}{\ensuremath{\mathbb{N}_0}}
%\newcommand{\RR}{\ensuremath{\mathbb{R}}}
%\newcommand{\perdef}{\overset{\mathrm{def}}{=}}
%\usepackage{amsmath,amssymb}
%%\usepackage[latin1]{inputenc}
%\usepackage{fancybox}
%\usepackage{epic,overpic}
%
%\usepackage{etex}
%\usepackage[dutch]{babel}
%\usepackage{hyphenat}
%\usepackage{amsmath}
%\usepackage{amssymb}Voorbeelden: $|5|=5,\; |-\pi|=\pi$ en $|1-\sqrt{2}|=\sqrt{2}-1$
%\usepackage{amsfonts}
%\usepackage{graphicx}
%
%\usepackage{color}
%%\usepackage[thmmarks,framed,amsmath]{ntheorem}
%\usepackage{answers}
%
%\usepackage{framed}
%
%\usepackage{pst-all}
%\usepackage{epic}
%\usepackage{eepic}
%\usepackage{array}
%\usepackage{booktabs}
%\usepackage{multirow}
%\usepackage{color}
%
%
%
%%\usepackage{color}
%%\usepackage[thmmarks,framed,amsmath]{ntheorem}


\title{Voorkennisles 1: Goniometrie en Rekenvaardigheden}

\author{Annouk Van Vlierden \& Wim Obbels}
\date{2019-2020}
\usetheme{Madrid}
%\usetheme{Montpellier}
%\usetheme{CambridgeUS}
%\usecolortheme{seahorse}

\usepackage{cancel}
\providecommand{\p}{\pause}

\begin{document}
\frame{\titlepage}

\begin{frame}{Absolute waarde: definitie}
\p
\begin{definition}
Voor een reëel getal $a\in \R$ definiëren we de \emph{absolute waarde} van $a$ als 
$$
|a|\perdef \left\{
\begin{array}{rll  } 
   a & \mbox{als}  & a \geqslant 0, \\
  -a & \mbox{als}  & a<0.\\
\end{array}\right.
$$
\end{definition}
\vfill\p


Voorbeelden: $|5|=5,\; |-\pi|=\pi$ en $|1-\sqrt{2}|=\sqrt{2}-1$
\vfill\p

Pas op voor $\cancel{|-a| = a}$ want dat is fout voor $a=-7$ \\
Inderdaad:  $|-a| = |-(-7)| = 7 \neq a $  ( want $a= -7$ !) 
\end{frame}

\begin{frame}{Absolute waarde: eigenschappen}  
\p
\begin{proposition}
     Als $x \in \R$ en $r>0 $, dan geldt: 
     \begin{align}
        |x|<r\; & \Leftrightarrow \; -r<x<r \\
        |x-y|<r\; & \Leftrightarrow \; -r<x-y<r \\
                  & \Leftrightarrow \; \text{de afstand tussen $x$ en $y$ is kleiner dan $r$}
      \end{align}
      
     
\end{proposition}
\p
Voorbeeld: $|x| < 7 \iff -7 < x < 7$
\end{frame}

\begin{frame}{Abolute waarde: oefeningen}


    Onderzoek welke $x\in \R$ voldoen aan $|3x-2| < 6$.
        
        \begin{enumerate} 
        \p\item Oplossing 1: \p 
$$
        \begin{array}{cccc}
        |3x - 2| < 6 & \iff& 6   < 3x - 2 < 6 \\
                  \p & \iff& -6+2<3x<6+2 \\
                  \p & \iff& -4  <3x<8 \\
                  \p & \iff& \frac{-4}{3}<x<\frac{8}{3}
        \end{array}
$$        
        \p\item Oplossing 2: \p 
        
        De afstand tussen $3x$ en $2$ moet kleiner zijn dan $6$
        
        \p Dus moet de afstand tussen $x$ en $\frac{2}{3}$ kleiner zijn  dan $2$.
        
        \p Dus moet 
        $$ 
             - 2 < x- \frac{2}{3} < +2 \text{ of }
        $$
        
        $$
            \frac{-4}{3}<x<\frac{8}{3}
        $$
        
 %       \item TODO: grafische voorstelling
\end{enumerate}

UOVT (Uitstekende Oefening Voor Thuis):  idem voor $|3-2x| < 1$ 

\end{frame}

\begin{frame}{Ongelijkheden: eigenschappen}
\begin{proposition}
    
Voor elke $x,y,r \in\R $ geldt
\begin{align}
 x< y                  & \iff \;x+r< y+r \\
(r>0 \text{ en }  x<y) & \implies \; r\cdot x< r\cdot y \\
(r<0 \text{ en } x<y)  & \implies \; r\cdot x> r\cdot y \text{ (of ook } r\cdot y < r\cdot x)
\end{align}

\end{proposition}

\vfill
\p Dus: vermenigvuldigen met een \textit{negatief} getal \textit{keert de ongelijkheid om}.

\vfill
\p PAS OP: \textit{iedereen} laat zich daar wel eens aan vangen. 

\p {\color{red}U bent gewaarschuwd: \textit{iedereen}, dus ook \textit{u}.\p{ Jawel: ook \textit{U} daar}}. 


\end{frame}

\begin{frame}{Ongelijkheden: oefeningen}
       
           Onderzoek welke $x\in \R$ voldoen aan
        
\begin{enumerate}      
\item $|\sqrt{x}+3|<9$
\item $\ds \left|  \frac{9 - \sqrt{1+20x}}{10} \right| < 1$
\item (UOVT) $\ds \left|5-\frac{2}{x}\right|>1$
\end{enumerate}

\end{frame}

\begin{frame}{Een functie toepassen op beide leden van een ongelijkheid}

Zij $x,y\in\R$ en $f$ een reële functie.

\begin{problem}
    Probleem: als $x<y$, kunnen we dan iets zeggen over $f(x)$ en $f(y)$?
\end{problem}

\p
In De Ideale Wereld: als $x<y$, dan \p $f(x)\overset{?}{<}f(y)$.

\p
Met kwadrateren: als $\phantom{-}x<\phantom{-}y$, dan  $x^2\overset{?}{<}y^2$ ? 
\\
\p\ \phantom{Met kwadrateren: }als $\phantom{-}2<\phantom{-}4$, dan $2^2\overset{?}{<}4^2$ ? 
\\
\p\ \phantom{Met kwadrateren: }als $-4<-2$, dan $(-4)^2\overset{?}{<}(-2)^2$ ? 
\\
\p\ \phantom{Met kwadrateren: }als $-4<\phantom{-}2$, dan $(-4)^2\overset{?}{<}(2)^2$ ? 


\begin{proposition}
\setlength{\abovedisplayskip}{-5pt}
\begin{align}
 0\leq x<y \implies x^2<y^2 \\
 x<y\leq 0 \implies x^2>y^2 %\\
%\mbox{als } a<0<b  &\mbox{ dan kan je niet weten of }\;& a^2<b^2 \; \mbox{of} \;a^2>b^2 
\end{align}
\end{proposition}
\end{frame}


\begin{frame}{Ongelijkheden en kwadraten}

\begin{figure}[H]
	\begin{tikzpicture}[scale=1]
\begin{axis}
[
%xtick={-4,-3,-2-1,0,1,2,3,4},
%xticklabels = {$4,},
ytick={4,9,16},
yticklabels = {$4$, $9$, $16$},
%axis equal,
ymax=20, %ymin=-5,
%samples=200,
axis lines=center,
%extra y ticks={0},
%restrict y to domain=-10:10,
%legend style={legend pos=south west,font=\tiny},
]
\addplot[domain=-5:5,color=blue] {x^2};
\node at (axis cs:-2,4) [circle, scale=0.3, draw=black!80,fill=black!80] {};
\end{axis}
\end{tikzpicture}	
\end{figure}
\end{frame}

\begin{frame} {Ontbinden in factoren}
\begin{problem}
Ontbind een derdegraadsveelterm $f(x) = b x^3+ c x^2+ d x+ e$ in
factoren.% van zo laag mogelijke graad: 
\end{problem}
\begin{itemize} 
    \item Stap 1: Zoek een nulpunt $a$ van $f(x)$. Dan kunnen we $f(x)$ schrijven als
$f(x) = (x-a) Q(x)$ met $Q(x)$ een tweedegraadsveelterm.
    \item Stap 2: Bereken $\ds Q(x) = \frac{f(x)}{x-a}$
    \item Stap 3: Ontbind $Q(x)$ verder in factoren indien mogelijk
\end{itemize}
Eig:  

Als $b x^3+ c x^2+ d x+ e$, met $b$,$c$, $d$, $e$ gehele
getallen, deelbaar is door $(x-a)$, met $a$ geheel, dan moet $a$ een
deler zijn van de constante term $e$. 

Indien $f(x)$ geen enkel geheel nulpunt heeft dan moeten we een benaderingsmethode gebruiken 

(Cursus Hoofdstuk 5).
\end{frame}


\end{document}

