\documentclass{beamer}
%\documentclass[handout]{beamer}


\usepackage[a4paper]{geometry}

%\usepackage[utf8]{inputenc}
\usepackage{multicol}


\graphicspath{
	{./}
	{goniometrie/}
}

% we willen (bijna) altijd \geqslant ipv \geq ...!
\newcommand{\geqnoslant}{\geq}
\renewcommand{\geq}{\geqslant}
\newcommand{\leqnoslant}{\leq}
\renewcommand{\leq}{\leqslant}

%overkill? Gebuikt in module limieten
\newcommand{\naar}{\rightarrow}

% Shortcuts voor limieten
% MERK OP: hier kan dus ook de notatie voor linker/rechterlimiet worden gekozen !!!
% Usage: \limx geeft lim voor x-> 0;  \limx[a^2]  geeft lim voor x-> a^2 en \limxi geeft lim voor x -> \infy \limxmi -> -\infty 
% Mmm, zonder de \ifblank lijkt het niet te werken in htlatex ...?
\newcommand{\limx}[1][]{\lim_{x \rightarrow \ifblank{#1}{0}{#1}}}
\newcommand{\llimx}[1][]{\lim_{x \underset < \rightarrow \ifblank{#1}{0}{#1}}}
\newcommand{\rlimx}[1][]{\lim_{x \underset > \rightarrow \ifblank{#1}{0}{#1}}}

\newcommand{\limxi}{\limx[+\infty]}  % I voor \Infty
\newcommand{\limxmi}{\limx[-\infty]} % MI voor Min \Infty

% bestaan niet ...!
%\newcommand{\rlimxi}{\rlimx[+\infty]}
%\newcommand{\rlimxmi}{\rlimx[-\infty]}
%
%\newcommand{\llimxi}{\llimx[+\infty]}
%\newcommand{\llimxmi}{\llimx[-\infty]}

%
% Poging tot aanpassen layout
%
\usepackage{mdframed}
\usepackage{tcolorbox}
\tcbuselibrary{theorems}

% Herdefinieer enkele omgevingen (PAS OP: enkel voor PDF, voor html: zie css..!!!)
% remove italics def
\makeatletter   % because of the @ below: make @ a (normal) letter!!
\let\definition\relax
\let\c@definition\relax
\let\enddefinition\relax
\theoremstyle{definition}
%\newtheorem*{definition}{Definitie}
\newmdtheoremenv{definition}{Definitie}
%\newtcbtheorem[number within=section]{definition}{Definitie}{colback=blue!5,colframe=blue!35!black,fonttitle=\bfseries}{th}
%\newtcbtheorem{definition}{Definitie}{colback=blue!5,colframe=blue!35!black,fonttitle=\bfseries}{th}



% remove italics def
\let\example\relax
\let\c@example\relax
\let\endexample\relax
\theoremstyle{definition}
\newtheorem{example}{Voorbeeld}

% remove italics def
\let\remark\relax
\let\c@remark\relax
\let\endremark\relax
\theoremstyle{definition}
\newtheorem{remark}{Opmerking}

% remove italics def
\let\proposition\relax
\let\c@proposition\relax
\let\endproposition\relax
%\theoremstyle{proposition}
\newmdtheoremenv{proposition}{Eigenschap}

% remove italics def
\let\problem\relax
\let\c@problem\relax
\let\endproblem\relax
%\theoremstyle{problem}
\newtheorem{problem}{Voorbeeld oefening}

% remove italics def
\let\exercise\relax
\let\c@exercise\relax
\let\endexercise\relax
%\theoremstyle{problem}
\newtheorem{exercise}{Oef.}

\newtheorem*{oplossing}{Oplossing}
%\newtheorem{oplossing}[definition]{Oplossing}

\makeatother




%definities nieuwe commando's - afkortingen veel gebruikte symbolen
\newcommand{\ds}{\displaystyle}
\newcommand{\R}{\ensuremath{\mathbb{R}}}
\newcommand{\Rnul}{\ensuremath{\mathbb{R}_0}}
\newcommand{\Reen}{\ensuremath{\mathbb{R}\setminus\{1\}}}
\newcommand{\Rnuleen}{\ensuremath{\mathbb{R}\setminus\{0,1\}}}
\newcommand{\Rplus}{\ensuremath{\mathbb{R}^+}}
\newcommand{\Rmin}{\ensuremath{\mathbb{R}^-}}
\newcommand{\Rnulplus}{\ensuremath{\mathbb{R}_0^+}}
\newcommand{\Rnulmin}{\ensuremath{\mathbb{R}_0^-}}
\newcommand{\Rnuleenplus}{\ensuremath{\mathbb{R}^+\setminus\{0,1\}}}
\newcommand{\N}{\ensuremath{\mathbb{N}}}
\newcommand{\Nnul}{\ensuremath{\mathbb{N}_0}}
\newcommand{\Z}{\ensuremath{\mathbb{Z}}}
\newcommand{\Znul}{\ensuremath{\mathbb{Z}_0}}
\newcommand{\Zplus}{\ensuremath{\mathbb{Z}^+}}
\newcommand{\Zmin}{\ensuremath{\mathbb{Z}^-}}
\newcommand{\Znulplus}{\ensuremath{\mathbb{Z}_0^+}}
\newcommand{\Znulmin}{\ensuremath{\mathbb{Z}_0^-}}
\newcommand{\C}{\ensuremath{\mathbb{C}}}
\newcommand{\Cnul}{\ensuremath{\mathbb{C}_0}}
\newcommand{\Cplus}{\ensuremath{\mathbb{C}^+}}
\newcommand{\Cmin}{\ensuremath{\mathbb{C}^-}}
\newcommand{\Cnulplus}{\ensuremath{\mathbb{C}_0^+}}
\newcommand{\Cnulmin}{\ensuremath{\mathbb{C}_0^-}}
\newcommand{\Q}{\ensuremath{\mathbb{Q}}}
\newcommand{\Qnul}{\ensuremath{\mathbb{Q}_0}}
\newcommand{\Qplus}{\ensuremath{\mathbb{Q}^+}}
\newcommand{\Qmin}{\ensuremath{\mathbb{Q}^-}}
\newcommand{\Qnulplus}{\ensuremath{\mathbb{Q}_0^+}}
\newcommand{\Qnulmin}{\ensuremath{\mathbb{Q}_0^-}}
\newcommand{\perdef}{\overset{\mathrm{def}}{=}}
\newcommand{\pernot}{\overset{\mathrm{not}}{=}}
\newcommand{\bgsin}{\mathrm{bgsin}\,}
\newcommand{\bgcos}{\mathrm{bgcos}\,}
\newcommand{\bgtan}{\mathrm{bgtan}\,}
\newcommand{\bgcot}{\mathrm{bgcot}\,}
\newcommand{\bgsinh}{\mathrm{bgsinh}\,}
\newcommand{\bgcosh}{\mathrm{bgcosh}\,}
\newcommand{\bgtanh}{\mathrm{bgtanh}\,}
\newcommand{\bgcoth}{\mathrm{bgcoth}\,}
\newcommand{\Bgsin}{\mathrm{Bgsin}\,}
\newcommand{\Bgcos}{\mathrm{Bgcos}\,}
\newcommand{\Bgtan}{\mathrm{Bgtan}\,}
\newcommand{\Bgcot}{\mathrm{Bgcot}\,}
\newcommand{\Bgsinh}{\mathrm{Bgsinh}\,}
\newcommand{\Bgcosh}{\mathrm{Bgcosh}\,}
\newcommand{\Bgtanh}{\mathrm{Bgtanh}\,}
\newcommand{\Bgcoth}{\mathrm{Bgcoth}\,}
\newcommand{\cosec}{\mathrm{cosec}\,}
\newcommand{\dom}{\mathrm{dom}\,}
\newcommand{\bld}{\mathrm{bld}\,}
\newcommand{\graf}{\mathrm{graf}\,}
\newcommand{\rc}{\mathrm{rc}\,}
\newcommand{\co}{\mathrm{co}\,}
\newcommand{\oefverwijzing}[1]{\ensuremath{\hookrightarrow}\ \textsl{#1}}
\newcommand{\startletternummering}{\renewcommand{\labelenumi}{(\alph{enumi})}}
\newcommand{\eindeletternummering}{\renewcommand{\labelenumi}{\arabic{enumi}.}}
\newcommand{\bron}[1]{\begin{scriptsize} \emph{#1} \end{scriptsize}} 


%
% copied from https://github.com/mooculus/calculus
%

%\usepackage{todonotes}
%\usepackage{mathtools} %% Required for wide table Curl and Greens
%\usepackage{cuted} %% Required for wide table Curl and Greens
\newcommand{\todo}{}

% Font niet (correct?) geinstalleerd in MikTeX?
%\usepackage{esint} % for \oiint
%\ifxake%%https://math.meta.stackexchange.com/questions/9973/how-do-you-render-a-closed-surface-double-integral
%\renewcommand{\oiint}{{\large\bigcirc}\kern-1.56em\iint}
%\fi


\newcommand{\mooculus}{\textsf{\textbf{MOOC}\textnormal{\textsf{ULUS}}}}

\usepackage{tkz-euclide}\usepackage{tikz}
\usepackage{tikz-cd}
\usetikzlibrary{arrows}
\tikzset{>=stealth,commutative diagrams/.cd,
  arrow style=tikz,diagrams={>=stealth}} %% cool arrow head
\tikzset{shorten <>/.style={ shorten >=#1, shorten <=#1 } } %% allows shorter vectors

\usetikzlibrary{backgrounds} %% for boxes around graphs
\usetikzlibrary{shapes,positioning}  %% Clouds and stars
\usetikzlibrary{matrix} %% for matrix
\usepgfplotslibrary{polar} %% for polar plots
\usepgfplotslibrary{fillbetween} %% to shade area between curves in TikZ
\usetkzobj{all}
\usepackage[makeroom]{cancel} %% for strike outs
%\usepackage{mathtools} %% for pretty underbrace % Breaks Ximera
%\usepackage{multicol}
\usepackage{pgffor} %% required for integral for loops



%% http://tex.stackexchange.com/questions/66490/drawing-a-tikz-arc-specifying-the-center
%% Draws beach ball
\tikzset{pics/carc/.style args={#1:#2:#3}{code={\draw[pic actions] (#1:#3) arc(#1:#2:#3);}}}



\usepackage{array}
\setlength{\extrarowheight}{+.1cm}
\newdimen\digitwidth
\settowidth\digitwidth{9}
\def\divrule#1#2{
\noalign{\moveright#1\digitwidth
\vbox{\hrule width#2\digitwidth}}}





\newcommand{\RR}{\mathbb R}
%\newcommand{\R}{\mathbb R}
%\newcommand{\N}{\mathbb N}
%\newcommand{\Z}{\mathbb Z}

\newcommand{\sagemath}{\textsf{SageMath}}


%\renewcommand{\d}{\,d\!}
\renewcommand{\d}{\mathop{}\!d}
\newcommand{\dd}[2][]{\frac{\d #1}{\d #2}}
\newcommand{\pp}[2][]{\frac{\partial #1}{\partial #2}}
\renewcommand{\l}{\ell}
\newcommand{\ddx}{\frac{d}{\d x}}

\newcommand{\zeroOverZero}{\ensuremath{\boldsymbol{\tfrac{0}{0}}}}
\newcommand{\inftyOverInfty}{\ensuremath{\boldsymbol{\tfrac{\infty}{\infty}}}}
\newcommand{\zeroOverInfty}{\ensuremath{\boldsymbol{\tfrac{0}{\infty}}}}
\newcommand{\zeroTimesInfty}{\ensuremath{\small\boldsymbol{0\cdot \infty}}}
\newcommand{\inftyMinusInfty}{\ensuremath{\small\boldsymbol{\infty - \infty}}}
\newcommand{\oneToInfty}{\ensuremath{\boldsymbol{1^\infty}}}
\newcommand{\zeroToZero}{\ensuremath{\boldsymbol{0^0}}}
\newcommand{\inftyToZero}{\ensuremath{\boldsymbol{\infty^0}}}



\newcommand{\numOverZero}{\ensuremath{\boldsymbol{\tfrac{\#}{0}}}}
\newcommand{\dfn}{\textbf}
%\newcommand{\unit}{\,\mathrm}
\newcommand{\unit}{\mathop{}\!\mathrm}
\newcommand{\eval}[1]{\bigg[ #1 \bigg]}
\newcommand{\seq}[1]{\left( #1 \right)}
\renewcommand{\epsilon}{\varepsilon}
\renewcommand{\phi}{\varphi}


\renewcommand{\iff}{\Leftrightarrow}

\DeclareMathOperator{\arccot}{arccot}
\DeclareMathOperator{\arcsec}{arcsec}
\DeclareMathOperator{\arccsc}{arccsc}
\DeclareMathOperator{\si}{Si}
\DeclareMathOperator{\scal}{scal}
\DeclareMathOperator{\sign}{sign}


%% \newcommand{\tightoverset}[2]{% for arrow vec
%%   \mathop{#2}\limits^{\vbox to -.5ex{\kern-0.75ex\hbox{$#1$}\vss}}}
\newcommand{\arrowvec}[1]{{\overset{\rightharpoonup}{#1}}}
%\renewcommand{\vec}[1]{\arrowvec{\mathbf{#1}}}
\renewcommand{\vec}[1]{{\overset{\boldsymbol{\rightharpoonup}}{\mathbf{#1}}}\hspace{0in}}

\newcommand{\point}[1]{\left(#1\right)} %this allows \vector{ to be changed to \vector{ with a quick find and replace
\newcommand{\pt}[1]{\mathbf{#1}} %this allows \vec{ to be changed to \vec{ with a quick find and replace
\newcommand{\Lim}[2]{\lim_{\point{#1} \to \point{#2}}} %Bart, I changed this to point since I want to use it.  It runs through both of the exercise and exerciseE files in limits section, which is why it was in each document to start with.

\DeclareMathOperator{\proj}{\mathbf{proj}}
\newcommand{\veci}{{\boldsymbol{\hat{\imath}}}}
\newcommand{\vecj}{{\boldsymbol{\hat{\jmath}}}}
\newcommand{\veck}{{\boldsymbol{\hat{k}}}}
\newcommand{\vecl}{\vec{\boldsymbol{\l}}}
\newcommand{\uvec}[1]{\mathbf{\hat{#1}}}
\newcommand{\utan}{\mathbf{\hat{t}}}
\newcommand{\unormal}{\mathbf{\hat{n}}}
\newcommand{\ubinormal}{\mathbf{\hat{b}}}

\newcommand{\dotp}{\bullet}
\newcommand{\cross}{\boldsymbol\times}
\newcommand{\grad}{\boldsymbol\nabla}
\newcommand{\divergence}{\grad\dotp}
\newcommand{\curl}{\grad\cross}
%\DeclareMathOperator{\divergence}{divergence}
%\DeclareMathOperator{\curl}[1]{\grad\cross #1}
\newcommand{\lto}{\mathop{\longrightarrow\,}\limits}

\renewcommand{\bar}{\overline}

\colorlet{textColor}{black}
\colorlet{background}{white}
\colorlet{penColor}{blue!50!black} % Color of a curve in a plot
\colorlet{penColor2}{red!50!black}% Color of a curve in a plot
\colorlet{penColor3}{red!50!blue} % Color of a curve in a plot
\colorlet{penColor4}{green!50!black} % Color of a curve in a plot
\colorlet{penColor5}{orange!80!black} % Color of a curve in a plot
\colorlet{penColor6}{yellow!70!black} % Color of a curve in a plot
\colorlet{fill1}{penColor!20} % Color of fill in a plot
\colorlet{fill2}{penColor2!20} % Color of fill in a plot
\colorlet{fillp}{fill1} % Color of positive area
\colorlet{filln}{penColor2!20} % Color of negative area
\colorlet{fill3}{penColor3!20} % Fill
\colorlet{fill4}{penColor4!20} % Fill
\colorlet{fill5}{penColor5!20} % Fill
\colorlet{gridColor}{gray!50} % Color of grid in a plot

\newcommand{\surfaceColor}{violet}
\newcommand{\surfaceColorTwo}{redyellow}
\newcommand{\sliceColor}{greenyellow}




\pgfmathdeclarefunction{gauss}{2}{% gives gaussian
  \pgfmathparse{1/(#2*sqrt(2*pi))*exp(-((x-#1)^2)/(2*#2^2))}%
}


%%%%%%%%%%%%%
%% Vectors
%%%%%%%%%%%%%

%% Simple horiz vectors
%\renewcommand{\vector}[1]{\left\langle #1\right\rangle}


%% %% Complex Horiz Vectors with angle brackets
%% \makeatletter
%% \renewcommand{\vector}[2][ , ]{\left\langle%
%%   \def\nextitem{\def\nextitem{#1}}%
%%   \@for \el:=#2\do{\nextitem\el}\right\rangle%
%% }
%% \makeatother

%% %% Vertical Vectors
%% \def\vector#1{\begin{bmatrix}\vecListA#1,,\end{bmatrix}}
%% \def\vecListA#1,{\if,#1,\else #1\cr \expandafter \vecListA \fi}

%%%%%%%%%%%%%
%% End of vectors
%%%%%%%%%%%%%

%\newcommand{\fullwidth}{}
%\newcommand{\normalwidth}{}



%% makes a snazzy t-chart for evaluating functions
%\newenvironment{tchart}{\rowcolors{2}{}{background!90!textColor}\array}{\endarray}

%%This is to help with formatting on future title pages.
\newenvironment{sectionOutcomes}{}{}



%% Flowchart stuff
%\tikzstyle{startstop} = [rectangle, rounded corners, minimum width=3cm, minimum height=1cm,text centered, draw=black]
%\tikzstyle{question} = [rectangle, minimum width=3cm, minimum height=1cm, text centered, draw=black]
%\tikzstyle{decision} = [trapezium, trapezium left angle=70, trapezium right angle=110, minimum width=3cm, minimum height=1cm, text centered, draw=black]
%\tikzstyle{question} = [rectangle, rounded corners, minimum width=3cm, minimum height=1cm,text centered, draw=black]
%\tikzstyle{process} = [rectangle, minimum width=3cm, minimum height=1cm, text centered, draw=black]
%\tikzstyle{decision} = [trapezium, trapezium left angle=70, trapezium right angle=110, minimum width=3cm, minimum height=1cm, text centered, draw=black]


\newcommand{\degree}{^\circ}
\newcommand{\m}{\phantom{$-$}}


\usepackage{pgfplots}    % Mmm, niet in preamble, because already in ximera?
\usepackage{cancel}
\renewcommand{\CancelColor}{\color{red}}

% to be done properly in preamble!
\makeatletter
\let\proposition\relax
\let\c@proposition\relax
\let\endproposition\relax
\theoremstyle{eigenschap}
\newtheorem{proposition}{Eigenschap}
\makeatother

% hack: make \pause work in \align ...
\makeatletter
\let\save@measuring@true\measuring@true
\def\measuring@true{%
	\save@measuring@true
	\def\beamer@sortzero##1{\beamer@ifnextcharospec{\beamer@sortzeroread{##1}}{}}%
	\def\beamer@sortzeroread##1<##2>{}%
	\def\beamer@finalnospec{}%
}
\makeatother

\title{Voorkennisles 1: Goniometrie en Rekenvaardigheden}

\author{Annouk Van Vlierden \& Wim Obbels}
\date{2019-2020}
\usetheme{Madrid}
%\usetheme{Montpellier}
%\usetheme{CambridgeUS}
%\usecolortheme{seahorse}


\providecommand{\p}{\pause}
%\renewcommand{\p}{}
\providecommand{\redgt}{\mathbin{\pmb{\color{red}>}}}
\providecommand{\redlt}{\mathbin{\pmb{\color{red}<}}}

\usetikzlibrary{math,calc} 

\begin{document}
\frame{\titlepage}

\begin{frame}{Agenda: \textit{wat} gaan we doen?}
\begin{itemize}
	\p\item Wat \textit{goniometrie}:  \p hoeken $\alpha$ en $\beta$, sinus en cosinus, \ldots
	\p\item Wat \textit{rekenwerk}: \p $|x^2+1|$, $2x^2 < 4x -1$ , $(x^m)^n$, veeltermen, \ldots
	\p\item Wat \textit{exponentiële} functies: \p $a^x$, $2^{2x+3}$, \ldots
	\p\item Wat \textit{logaritmische} functies: \p $\log 10^{25}$, $\ln x$, \ldots
	\p\item Wat \textit{limieten}: \p $+\infty$ en $-\infty$, \ldots
	\vspace{1cm}
	\p\item Maar eerst: \p een verhaaltje ...
\end{itemize}
\end{frame}


%
% Adhoc commando's om in Tikz verhaaltjes te vertellen
%
\newcommand<>{\mynext}[1]{\only<+>{\draw[color=red] (-2.5,1.4) node[right] {\Huge #1};}}
\newcommand<>{\myuitleg}[1]{\onslide<.>{\draw[color=black] (-2.5,-1.4) node[right] {\LARGE #1};}}
\newcommand<>{\myadd}[1]{\onslide<.->{#1}}
%\newcommand<>{\mytemp}[1]{\onslide<.>{#1}}

\begin{frame}{De goniometrische cirkel}

{\centering
	\vfill
		
	\begin{tikzpicture}[scale=5,cap=round,transform canvas={scale=0.5}]
	
	\tikzmath{\hoek = 30; \myc = cos(\hoek); \mys = sin(\hoek); 
	\hoekb = 70;}
	
	\mynext{In den beginne}
	\mynext{In den beginne was er niets ...}
	
	\mynext{Toen schiep God een hoek}
	\onslide<.-10>{
		\draw[thick,color=blue,  shift={(-1,-1)}, rotate= 30]  (1,0) -- (0,0) -- ++(\hoek:1);
	}
	\mynext{God noemde de hoek $\alpha$ (lees: alpha)}
	\onslide<.-10>{
		\draw[thick,->, color=blue, shift={(-1,-1)}, rotate= 30] (0.5,0) arc (0:\hoek:0.5cm) node [midway,right] {$\alpha$};   
	}
	
	\mynext{Vervolgens schiep God nog een hoek $\beta$ (lees: beta)}
	\onslide<.-10>{
		\draw[thick,color=red, shift={( 0.9, -0.5)}, rotate= 145]  (1,0) -- (0,0) -- ++(\hoekb:1);
		\draw[thick,->,   color=red, shift={( 0.9, -0.5)}, rotate= 145]  (0.5,0) arc (0:\hoekb:0.5cm) node [midway,right] {$\beta$};   
		
	}
	
	\mynext{En even later waren er meer hoeken...}
	\onslide<.-10>{
		\draw[thick,color=blue, shift={( 1.1, 0.1)}, rotate=-20]  (1,0) -- (0,0) -- ++(\hoek:1);
		\draw[thick,color=blue, shift={( 0.9, 1.5)}, rotate=-40]  (1,0) -- (0,0) -- ++(\hoek:1);
		\draw[thick,color=red, shift={( -1, -1.1)}, rotate=130]  (1,0) -- (0,0) -- ++(\hoekb:1);
	}

	\mynext{En even later nog meer ...}
	\onslide<.-10>{
		\draw[thick,color=gray, shift={( -1.2, 0.2)}, rotate=45]  (1,0) -- (0,0) -- ++({\hoek+\hoekb}:1);
		\draw[thick,color=gray, shift={( 1.2, -1.5)}, rotate=45]  (1,0) -- (0,0) -- ++({\hoek+\hoekb}:1);

	}
	\mynext{Toen schiep God de cirkel}
	\myadd{
		\draw (0,0) circle (1cm);
	}
	
	
	%    \p\only<2>{\draw[color=red] (-1,1.1) node {Een cirkel met straal $1$};}
	\mynext{Een cirkel met middelpunt in de oorsprong en met straal $1$}
	\myadd{
		\draw[->] (-1.2,0) -- (1.2,0);% node[right] {$x$};
		\draw[->] (0,-1.2) -- (0,1.2);% node[above] {$y$};
		\draw  ( 1, 0) node [below right] {$ 1$};
		\draw  (-1, 0) node [below left]  {$-1$};	
		\draw  ( 0, 1) node [above left]  {$ 1$};		
		\draw  ( 0,-1) node [below left]  {$-1$};
	}
	%	
	%	\draw[thick] (0,0) -- (15:1);
	%	\node at (15:1) [circle,fill,inner sep=0pt,scale=0.5]{A};
	
	%	\foreach \x in {75,105,165,-15,-75,-105,-165}
	%	\node at (\x:1) [circle,fill,inner sep=0pt,scale=0.3]{A};
	%	
	\mynext{En God verschoof de hoeken $\alpha$ en $\beta$ naar de oorsprong}
	%    \p\only<3>{\draw[color=red] (-1,1.1) node {Een cirkel met een hoek $\alpha$};}
	\onslide<.>{
		\draw[color=blue,ultra thick,rotate=30] 
		(1.5,0) -- (0:0)  -- (\hoek:1.5)  node [circle,fill,inner sep=0pt,scale=0.3, above right,align=left] {}; % {$\alpha$};
		\draw[->, color=blue,rotate=30] (0.3,0) arc (0:\hoek:0.3cm) node [midway, right] {$\alpha$};    
		\draw[ultra thick,color=red, rotate= 145]  (1,0) -- (0,0) -- ++(\hoekb:1);
		\draw[ultra thick,->,   color=red, rotate= 145]  (0.5,0) arc (0:\hoekb:0.5cm) node [midway,right] {$\beta$};   

	}

	\mynext{En draaide ze met hun beginbeen op de $x$-as}
	\onslide<.>{
		\draw[color=blue,ultra thick] 
			(1.5,0) -- (0:0)  -- (\hoek:1.5)  node [circle,fill,inner sep=0pt,scale=0.3, above right,align=left] {}; % {$\alpha$};
		\draw[->, color=blue,] (0.3,0) arc (0:\hoek:0.3cm) node [midway, right] {$\alpha$};    	
		\draw[ultra thick,color=red]  (1,0) -- (0,0) -- ++(\hoekb:1);
		\draw[ultra thick,->,color=red]  (0.5,0) arc (0:\hoekb:0.5cm) node [midway,right] {$\beta$};   
	
}
	
	\mynext{God plaatste ALLE hoeken op die manier in de cirkel}
	\onslide<.>{
		\draw[color=blue,ultra thick] 
		(1.5,0) -- (0:0)  -- (\hoek:1.5)  node [circle,fill,inner sep=0pt,scale=0.3, above right,align=left] {}; % {$\alpha$};
		\draw[thick, ->, color=blue] (0.5,0) arc (0:\hoek:0.5cm) node [midway, right] {\Large$\alpha$};    
	}
	
%	\mynext{TODO: God vergat iets te zeggen over de som van hoeken}
	
	
	\mynext{God noemde ook het snijpunt van de hoek en de cirkel $\alpha$}
	\myadd{
		\draw[color=blue] (0,0) -- (\hoek:1) 
			node [color=black, circle,fill,inner sep=2pt,scale=0.5] {A}
			node [color=blue,below right] {\Huge$\alpha$} ;    
		\draw[thick, ->, color=blue] (0.5,0) arc (0:\hoek:0.5cm) node [midway, right] {\Large$\alpha$};    

	}
	
	
	\mynext{Daarna schiep God de rechthoekige driehoek}
	\myadd{
		\draw[color=red, thick]  {(\myc,0)} --  node [right] {$a$} (\myc,\mys) ;	
		\draw[color=red, thick]  (0,0) --  node [below] {$b$} (\myc,0) ;
		\draw[color=red, thick]  (0,0) --  node [above] {$c=1$} (\myc,\mys) ;
		
	}
	
	\mynext{God vroeg zich af: is er een verband tussen \textit{hoek} $\alpha$ en \textit{zijden} $a$ en $b$?}
	
	\mynext{En God sprak: er zij een volgende slide ....}
	
	\end{tikzpicture}
	\vfill
}
\end{frame}


\begin{frame}{Hoeken meten}
%
We gebruiken twee manieren om hoeken te meten:

\begin{definition}
	Een rechte hoek meet $90\degree = \dfrac{\pi}{2}$ rad (=radialen).
\end{definition}
%

\begin{proposition}
	\begin{tabular}{cc}
		$360\degree = 2\pi $ & $180\degree = \p\pi $ \\
		$ 60\degree =\p \frac{\pi}{3}$ & $\frac{2\pi}{3} = \p 120\degree$
	\end{tabular}
\end{proposition}
%
Merk op: je kan het symbool $\degree$ dus begrijpen als een afkorting voor $\cdot\dfrac{2\pi}{360}$. Radialen worden soms met het symbool rad aangeduid, maar meestal wordt er \textit{geen} symbool gebruikt voor de eenheid (Voor fanatici: het is een 'dimensielose grootheid', zie bv \url{https://nl.wikipedia.org/wiki/Radiaal\_(wiskunde)}).  

\end{frame}



\begin{frame}{Sinus, cosinus en tangens}
\p

{\centering
	\vspace{1cm}
%	\resizebox{1cm}{!}{
	\begin{tikzpicture}[scale=4, transform canvas={scale=0.7}]
	%\draw (0,0) -- node [below] {$\cos\alpha$} (0.87,0) -- node [right] {$\sin\alpha$} (0.87,0.5) --  node [above left] {$1$} (0,0); 
	
	\draw (0,0) -- node [below] {$b$} (0.87,0) -- node [right] {$a$} (0.87,0.5) --  node [above left] {$c$} (0,0); 
	\draw[->, color=blue] (0.3,0) arc (0:30:0.3cm) node [below right] {$\alpha$};    
\only<8->{	
	\draw (1,0) -- node [below] {$b=\cos\alpha$} (1.87,0) -- node [right] {$a=\sin\alpha$} (1.87,0.5) --  node [above left] {$c=1$} (1,0); 	
	\draw[->, color=blue] (1.3,0) arc (0:30:0.3cm) node [below right] {$\alpha$};    
}
\only<9->{
	\draw (3,0) -- node [below] {$b=c\cdot\cos\alpha$} (3.87,0) -- node [right] {$a=c\cdot\sin\alpha$} (3.87,0.5) --  node [above left] {$c$} (3,0); 	
	\draw[->, color=blue] (3.3,0) arc (0:30:0.3cm) node [below right] {$\alpha$};    
}		
	\end{tikzpicture}
%	}
	\vspace{1cm}
}

\p
Met een scherpe hoek $\alpha$ associëren we volgende \textbf{goniometrische getallen}:
\begin{align*}
\p \sin\alpha &= \dfrac{\text{overstaande zijde}}{\text{schuine zijde}} = \dfrac{a}{c} && a = \sin\alpha \cdot c \tag{SOS} \\
\p \cos\alpha &= \dfrac{\text{aanliggende zijde}}{\text{schuine zijde}} = \dfrac{b}{c} && b = \cos\alpha \cdot c\tag{CAS}  \\
\p \tan\alpha &= \dfrac{\text{overstaande zijde}}{\text{aanliggende zijde}} = \dfrac{a}{b} = \dfrac{\sin\alpha}{\cos\alpha}  && a = \tan\alpha \cdot b\tag{TOA}  \\
%\csc\alpha &= \dfrac{1}{\sin\alpha} \tag{cosecans} \\
%\sec\alpha &= \dfrac{1}{\cos\alpha} \tag{secans} \\
\p \cot\alpha &= \dfrac{1}{\tan\alpha} = \dfrac{\cos\alpha}{\sin\alpha} \tag{cotangens} 
\end{align*}
%Merk op: $\csc90\degree, \tan90\degree, \sec0\degree$ en $\cot0\degree$ bestaan niet (wegens delen door nul!)

\p Dus: $\sin\alpha, \cos\alpha \in [-1,1]; \tan\alpha, \cot\alpha\in\mathbb{R}$ ;  $\tan\frac{\pi}{2}$  en $\cot0$ bestaan niet. 

\end{frame}


\begin{frame}{Eigenschappen sinus en cosinus}

\begin{proposition}
	Het punt $P_\alpha$ dat op de goniometrische cirkel overeenkomt met de hoek $\alpha$ heeft coördinaten $(\cos\alpha,\sin\alpha)$.
	
	Uit de Stelling van Pythagoras volgt direct dat 
	\begin{equation}
	\cos^2\alpha + \sin^2 \alpha = 1 \tag{Hoofdformule goniometrie}
	\end{equation}
	
\end{proposition}

%
\begin{itemize}
	\item Lees volgende waarden af op de goniometrische cirkel:
	\begin{itemize}
		\item[] $\sin 0= $ \hspace{2cm} $\sin \frac{\pi}{2}=$ \hspace{2cm} $\sin \pi=$ \hspace{2cm} $\sin \frac{ 3 \pi}{2}=$ \hspace{2cm} $\sin 2 \pi=$
		
	\end{itemize}
	\item Schets de grafiek van $f(x) = \sin x$
	\newpage
	\item Geef alle oplossingen van volgende  vergelijkingen
	\begin{enumerate}
		\item $\sin x = 0$
		\vspace{3mm}
		\item $\sin x  = 1$
		\vspace{3mm}
		\item $\ds \sin ( \frac{\pi}{2} x) = 0$
	\end{enumerate}
\end{itemize}
%
\end{frame}

\begin{frame}{Oefening}
%Zie cursustekst Wiskunde I p 54. 
Let op de gebruikte notaties: 
$$
(\cos \alpha)^2 \neq \cos (\alpha^2)
$$ 
Wel geldt:
$$(\cos \alpha)^2 = \cos^2 (\alpha)= \cos^2 \alpha$$
Onthoud ook dat $\cos 2 \alpha \neq 2 \cos \alpha$. 
( Waaraan is $\cos 2\alpha$ wel gelijk? )
\begin{itemize}
	\item Toon aan dat $\displaystyle 1 + \tan^2 \alpha = \frac{1}{\cos^2 \alpha}$ met behulp van de hoofdformule $\sin^2 \alpha + \cos^2
	\alpha = 1$
\end{itemize}
\end{frame}

\begin{frame}{Goniometrische getallen van standaardhoeken}

{\centering	
		
	\renewcommand{\arraystretch}{3}
	\begin{tabular}{|c || c | c | c | c | c |}
		\hline
		$\alpha$  & \quad$0$ \quad & \quad$\dfrac{\pi}{6}$ \quad&\quad $\dfrac{\pi}{4}$\quad &\quad $\dfrac{\pi}{3}$ \quad& \quad $\dfrac{\pi}{2}$ \quad \\ 
		          & \quad$0\degree$ \quad & \quad$30\degree$ \quad&\quad $45\degree$\quad &\quad $60\degree$ \quad& \quad $90\degree$ \quad \\ \hline

		$\sin\alpha$ & $0$ & $\dfrac{1}{2}$  & $\dfrac{\sqrt{2}}{2}$ & $\dfrac{\sqrt{3}}{2}$  & $1$ \\ \hline
		$\cos\alpha$ & $1$ & $\dfrac{\sqrt{3}}{2}$  & $\dfrac{\sqrt{2}}{2}$  & $\dfrac{1}{2}$ & $0$ \\ \hline
		$\tan\alpha$ & $0$ & $\dfrac{1}{\sqrt{3}}$ & $1$ & $\sqrt{3}$& $/$\\ \hline
		\hline
	\end{tabular} 
}
\end{frame}

\newcommand{\mydot}{node [color=black,circle,fill,inner sep=2pt,scale=0.5] {}}

\begin{frame}{Verwante hoeken}

{\centering
	\vfill
	
	\begin{tikzpicture}[scale=5,cap=round,transform canvas={scale=0.5}]


	\tikzmath{
			\hoek = 22.5; 
		    \myc = cos(\hoek); \mys = sin(\hoek); 
	        \hoekc = 90 -\hoek;
            \hoekac = 90 + \hoek;
          }

    \mynext{Het tweede scheppingsverhaal}

    \mynext{God sprak: Er zij een rechthoekige driehoek}
    \myadd{
    	\draw[color=blue,ultra thick] 
	    (0:0)  -- (\hoek:1)  node [circle,fill,inner sep=0pt,scale=0.3] {} 
    								% node [above right,align=left] {\Large$\alpha$}
    		;
	}
	\onslide<.>{
   		\draw[->, thick, color=blue] (0.5,0) arc (0:\hoek:0.5cm) node [below right] {\Huge$\alpha$};    
		\draw[color=red, ultra thick, dashed]  {(\myc,0)} --  node [right] {\Huge$a$} (\myc,\mys) ;	
		\draw[color=red, ultra thick, dashed]  (0,0) --  node [below] {\Huge$b$} (\myc,0) ;
	}


	\mynext{God sprak: Er zij ook een goniometrische cirkel}
	\onslide<.>{
		\draw[color=gray] (0,0) circle (1cm);
		\draw[->] (-1.2,0) -- (1.2,0);% node[right] {$x$};
		\draw[->] (0,-1.2) -- (0,1.2);% node[above] {$y$};
		\draw  (1,0)  node [below right] {$ 1$};
		\draw  (-1,0) node [below left]  {$-1$};	
		\draw  (0,1)  node [above left]  {$ 1$};		
		\draw  (0,-1) node [below left]  {$-1$};
		\draw[->, thick, color=blue] (0.5,0) arc (0:\hoek:0.5cm) node [below right] {\Huge$\alpha$};    
		\draw[color=red, ultra thick, dashed]  {(\myc,0)} --  node [right] {\Huge$a$} (\myc,\mys) ;	
		\draw[color=red, ultra thick, dashed]  (0,0) --  node [below] {\Huge$b$} (\myc,0) ;
	}

	\mynext{Hij verplaatste de symbolen ...}
	\onslide<.>{   % omdat we verder cos/sin willen toevoegen; wat geknoei
			\draw[color=gray] (0,0) circle (1cm);
			\draw[->,color=gray] (-1.2,0) -- (1.2,0);% node[right] {$x$};
			\draw[->,color=gray] (0,-1.2) -- (0,1.2);% node[above] {$y$};
		
			\draw[color=blue]  (\hoek:1) \mydot node [above right] {\huge$\alpha$};

			\draw[dotted,color=red] (\myc,0) -- (\hoek:1);
			\draw[dotted,color=red] (0,\mys) -- (\hoek:1);
			
			\draw[color=red]  ( \myc,0)  node [below]      {\Large $b$};
			\draw[color=red]  ( 0,\mys)  node [above left] {\Large $a$};

	}
	
	\mynext{... en definieerde de sinus en cosinus}
	\myadd{
		\draw[color=gray] (0,0) circle (1cm);
		\draw[->,color=gray] (-1.2,0) -- (1.2,0);% node[right] {$x$};
		\draw[->,color=gray] (0,-1.2) -- (0,1.2);% node[above] {$y$};
		
		\draw[dotted,color=red] (\myc,0) \mydot  -- (\hoek:1);
		\draw[dotted,color=red] (0,\mys) \mydot -- (\hoek:1);
		\draw[color=blue]  (\hoek:1) node [above right] {\huge$\alpha$};

		\draw[color=red]  ( \myc,0)  node [below]      {\Large $b=\cos\alpha$};
		\draw[color=red]  ( 0,\mys)  node [above left] {\Large $a=\sin\alpha$};
	}


	\mynext{De tweede dag: God schiep de verwante hoeken}

	\mynext{Tegengestelde hoeken}
	\onslide<.,15>{
		\draw[color=blue] (0,0) -- (-\hoek:1)  node [below right,align=left]  {$- \alpha$\\tegengesteld};
		\draw[color=red,dotted]  ( \myc,0) -- (-\hoek:1) -- ( 0,-\mys);  
	}
	\myuitleg{Tegengestelde hoeken: spiegelen over de $x$-as  ($y \longleftrightarrow -y$)}

	\mynext{Tegengestelde hoeken en hun sinus en cosinus}
	\onslide<.> {
		\draw[color=blue] (0,0) -- (-\hoek:1)  node [below right,align=left]  {$- \alpha$\\tegengesteld};		
		\draw[color=red] (\myc,0) node [below]  {};%\large \ \\$ = \cos(-\alpha)$};
		\draw[color=red] (0,-\mys) node [left] {\large$-a=\sin(-\alpha)$};
		\draw[color=red,dotted]  ( \myc,0) \mydot -- (-\hoek:1) \mydot -- ( 0,-\mys) \mydot;  
	}
	\myuitleg{Tegengestelde hoeken: \fbox{$\sin(-\alpha) = - \sin(\alpha)$ en $\cos(-\alpha) = \cos(\alpha)$}}


	\mynext{Complementaire hoeken}
	\onslide<.,15>{
		\draw[color=blue] (0,0) -- (\hoekc:1)  node [above right,align=left]  {$90\degree - \alpha$\\complementair};
		\draw[color=red,dotted]  ( \mys,0) -- (\hoekc:1) -- ( 0,\myc);  
	}
	\myuitleg{Complementaire hoeken: spiegelen over de eerste bissectrice ($x \longleftrightarrow y$)}
	\onslide<.>{
		\draw[color=blue,dotted]  ( 0,0) -- (45:1);  
	}

	\mynext{Complementaire hoeken en hun sinus en cosinus}
	\onslide<.> {
		\draw[color=blue] (0,0) -- (\hoekc:1)  node [above right,align=left]  {$90\degree - \alpha$\\complementair};
		\draw[color=red] (\mys,0) node [below]  {\large$a=\cos(90\degree-\alpha)$};
		\draw[color=red] (0,\myc) node [left] {\large$b=\sin(90\degree-\alpha)$};
		\draw[color=red,dotted]  ( \mys,0) \mydot -- (\hoekc:1) \mydot-- ( 0,\myc) \mydot;  
	}
	\myuitleg{Complementaire hoeken: \fbox{$\sin(90\degree-\alpha) =  \cos(\alpha)$ en $\cos(90\degree-\alpha) = \sin(\alpha)$}}

	
	\mynext{Supplementaire hoeken}
	\onslide<.,15>{
		\draw[color=blue] (0,0) -- (-\hoek:-1) node [above left, align=left] {$180\degree - \alpha$\\supplementair};
		\draw[color=red,dotted]  ( -\myc,0) -- (-\hoek:-1) -- ( 0,\mys);  	

	}
	\myuitleg{Supplementaire hoeken: spiegelen over de $y$-as ($x \longleftrightarrow -x$)}
	
	\mynext{Supplementaire hoeken en hun sinus en cosinus}
	\onslide<.>{		
		\draw[color=blue] (0,0) -- (-\hoek:-1)  node [above left,align=left]  {$180\degree - \alpha$\\supplementair};
		\draw[color=red]  ( -\myc,0)  node [below]      {\large $b=\cos(180\degree-\alpha)$};		\draw[color=red]  ( 0,\mys)   node [below left] {\large $a=\sin(180\degree-\alpha)$};		
		\draw[color=red,dotted]  ( -\myc,0) \mydot -- (-\hoek:-1) \mydot -- ( 0,\mys) \mydot ;  	
	}
	\myuitleg{Supplementaire hoeken: \fbox{$\sin(180\degree-\alpha) =  \sin(\alpha)$ en $\cos(180\degree-\alpha) = -\cos(\alpha)$}}



	\mynext{Anti-complementaire hoeken}
	\onslide<.,15>{
		\draw[color=blue] (0,0) -- (\hoekac:1)  node [above left,align=left]  {$90\degree + \alpha$\\anti-complementair};
		\draw[color=red,dotted]  ( -\mys,0) -- (\hoekac:1) -- ( 0,\myc);  
	}
	\myuitleg{Anti-complementaire hoeken: $90\degree$ roteren ($x \rightarrow y, y\rightarrow -x$)}


	\mynext{Anti-complementaire hoeken en hun sinus en cosinus}
	\onslide<.> {
		\draw[color=blue] (0,0) -- (\hoekac:1)  node [above left,align=left]  {$90\degree + \alpha$\\anti-complementair};
		\draw[color=red] (-\mys,0) node [below]  {\large$-a=\cos(90\degree+\alpha)$};
		\draw[color=red] (0,\myc) node [right] {\large$b=\sin(90\degree+\alpha)$};
		\draw[color=red,dotted]  ( -\mys,0) \mydot -- (\hoekac:1) \mydot -- ( 0,\myc) \mydot ;  		
	}
	\myuitleg{Anti-complementaire hoeken: \fbox{$\sin(90\degree+\alpha) =  \cos(\alpha)$ en $\cos(90\degree+\alpha) = -\sin(\alpha)$}}

	
	\mynext{En al de rest...}
	\myadd{
		\draw[color=blue] (0,0) -- (-\hoekac:1)  node [below left,align=left]  {$270\degree - \alpha$};
		\draw[color=blue] (0,0) -- (-\hoekc:1)  node [below right,align=right]  {$270\degree + \alpha$};		
		
	\draw[color=blue,thick] 
	(\hoek:-1) node [below left, align=right] {$180\degree + \alpha$\\$\alpha - 180\degree$\\anti-supplementair} --
	(\hoek:1)  node [above right,align=left]  {$\alpha+360\degree$\\gelijk\\};

	\draw[color=red,dotted]  (\hoekc:1) -- (\hoekac:1) -- (-\hoekac:1) -- (-\hoekc:1) --cycle;  
	\draw[color=red,dotted]  (\hoek:1) -- (-\hoek:-1) -- (\hoek:-1) -- (-\hoek:1) --cycle;  

	
	\foreach  \x in {-\myc,-\mys,\mys,\myc} \node at (\x,0) [circle,fill,inner sep=0pt,scale=0.2]{A};
	\foreach  \x in {-\myc,-\mys,\mys,\myc} \node at (0,\x) [circle,fill,inner sep=0pt,scale=0.2]{A};
	
	}

	\end{tikzpicture}
	\vfill
	
}
\end{frame}

\begin{frame}{Verwante hoeken: formules}

\small
We zien op de vorige tekening dat
\[
\begin{tabular}{|l c l c r| }
\hline
\text{Voor de cosinus} && \text{(dus op de $x$-as)} && \\
\hline
$ \cos\alpha$             & = &$\cos(-\alpha)$ & = &b \\
$ \cos(180\degree - \alpha)$&= &$\cos(180\degree +  \alpha)$ & = &-b \\
$ \cos(90\degree-\alpha)$ & = &$\cos(-90\degree+\alpha)$ &= & a\\
$ \cos(90\degree+\alpha)$& = &$\cos(-90\degree-\alpha)$ & =& -a\\
\hline
\end{tabular}
\]

\[
\begin{tabular}{|l c l c r|}
\hline
\text{Voor de sinus} && \text{(dus op de $y$-as)} && \\
\hline
$ \sin(90\degree-\alpha)$ & =& $\sin(90\degree+\alpha)$ &=& b \\   
$ \sin(-90\degree-\alpha)$ & =& $\sin(-90\degree+\alpha)$ &=& -b\\
$ \sin\alpha$             & = & $\sin(180\degree - \alpha)$ &=& a \\
$ \sin(-\alpha)$         & =& $\sin(180\degree +  \alpha)$ &=& -a \\
\hline
\end{tabular}
\]

en dus 

\end{frame}

\begin{frame}{Verwante hoeken: overzicht}

\small
Zij $\alpha$  en $\beta$ twee hoeken (en veronderstel dat $\alpha \leq \beta$). 

Dan definiëren we volgende mogelijke verwantschappen:
\vfill

\resizebox{\textwidth}{!}{
\renewcommand{\arraystretch}{1.8}
{\footnotesize
	%\begin{table}
	\begin{tabular}{|r | c @{} c @{} c @{} l |  c @{} c @{} c @{} l | l | l @{} l |l @{} l |}
		%	\setlength{\tabcolsep}{1pt}
		\hline
		\textit{naam} & \multicolumn{4}{|c|}{$\beta$ is .... aan $\alpha$ asa}
		& \multicolumn{4}{|c|}{$\alpha$ en $\beta$ zijn .... asa}
		& meetkundige operatie
		& sinus & &  cosinus &   
		\\ \hline \hline \hline
		\textbf{gelijk}        & $\beta =$ & $\alpha$ &     & $+k\cdot 360\degree $ 
		& $\alpha - \beta$&$ = $&$ 0 $&$ + k \cdot 360\degree$  
		& zelfde beeldpunt 
		& $\sin(\alpha+k\cdot360\degree)  $&$=\hphantom{-}\sin(\alpha)$
		& $\cos(\alpha+k\cdot360 \degree) $&$=\hphantom{-}\cos(\alpha)$
		\\ \hline
		\textbf{tegengesteld}  & $\beta = $&$-\alpha$ &     & $+k\cdot 360\degree$ 
		& $\alpha + \beta $&$=$&$ 0 $&$ + k \cdot 360\degree$  
		& spiegel t.o.v. $x$-as
		& $\sin(-\alpha) $&$= -\sin(\alpha)$
		& $\cos(-\alpha) $&$= \hphantom{-}\cos(\alpha)$
		\\ \hline \hline
		\textbf{supplementair} & $\beta =$ & $-\alpha$ & $+ 180 \degree$ & $+ k\cdot 360\degree $ 
		& $\alpha + \beta $&$= $&$180 \degree $&$ + k\cdot 360\degree $ 
		& spiegel t.o.v. $y$-as 
		& $\sin(180\degree - \alpha) $&$= \hphantom{-}\sin(\alpha)$
		& $\cos(180\degree - \alpha) $&$= -\cos(\alpha)$
		\\ \hline
		\textbf{complementair} & $\beta =$ & $-\alpha$ & $+ 90 \degree$ & $+ k\cdot 360\degree$ 
		& $\alpha + \beta $&$= $&$90 \degree $&$ + k\cdot 360\degree$ 
		& spiegel t.o.v. $y=x$ 
		& $\sin(\hphantom{1}90\degree - \alpha) $&$=\hphantom{-} \cos(\alpha)$
		& $\cos(\hphantom{1}90\degree - \alpha) $&$=\hphantom{-} \sin(\alpha)$
		\\ \hline \hline
		{\footnotesize anti-supplementair} & $\beta =$ & $\alpha$ & $+ 180 \degree$ & $+ k\cdot 360\degree $ 
		& $\alpha - \beta $&$= $&$180 \degree $&$ + k\cdot 360\degree $ 
		& puntspiegel t.o.v. $O$ 
		& $\sin(180\degree + \alpha) $&$= -\sin(\alpha)$
		& $\cos(180\degree + \alpha) $&$= -\cos(\alpha)$
		\\ \hline				  
		{\footnotesize anti-complementair} & $\beta =$ & $\alpha$ & $+ 90 \degree$ & $+ k\cdot 360\degree$ 
		& $-\alpha + \beta $&$= $&$90 \degree $&$ + k\cdot 360\degree$ 
		& hoekbenen loodrecht 
		& $\sin(\hphantom{1}90\degree + \alpha) $&$=\hphantom{-} \cos(\alpha)$
		& $\cos(\hphantom{1}90\degree + \alpha) $&$= -\sin(\alpha)$
		\\ \hline
		
	\end{tabular}
}
}
\end{frame}

\begin{frame}{Absolute waarde: definitie}
\p
\begin{definition}
Voor een reëel getal $a\in \R$ definiëren we de \emph{absolute waarde} van $a$ als 
$$
|a|\perdef \left\{
\begin{array}{rll  } 
   a & \mbox{als}  & a \geqslant 0, \\
  -a & \mbox{als}  & a<0.\\
\end{array}\right.
$$
\end{definition}
\vfill\p


Voorbeelden: $|-42|=|42|=42,\; |-\pi|=\pi$ en $|1-\sqrt{2}|=\sqrt{2}-1$
\vfill\p

Pas op voor $\xcancel{|-a| = a}$ want dat is fout voor bv $a=-42$ \\
Inderdaad:  $|-a|$ is dan  $|-(-42)| = 42 \p \neq a $  \p ( want $a= -42$ !) 

\end{frame}

\begin{frame}{Absolute waarde: eigenschappen}  
\p
\begin{proposition}
     Als $x \in \R$ en $r>0 $, dan geldt: 
     \begin{align}
         |x|<r   & \Leftrightarrow \; -r<x<r \\
\p       |x-y|<r & \Leftrightarrow \; -r<x-y<r \\
\p               & \Leftrightarrow \; \text{de afstand tussen $x$ en $y$ is kleiner dan $r$}
      \end{align}
      
     
\end{proposition}
\p\vfill

Voorbeeld: $|x| < 7 \iff -7 < x < 7$
\end{frame}

\begin{frame}{Absolute waarde: oefeningen}


    Onderzoek welke $x\in \R$ voldoen aan $|3x-2| < 6$.
        
        \begin{enumerate} 
        \p\item Oplossing 1: \p 
$$
        \begin{array}{cccc}
        |3x - 2| < 6 & \iff& 6   < 3x - 2 < 6 \\
                  \p & \iff& -6+2<3x<6+2 \\
                  \p & \iff& -4  <3x<8 \\
                  \p & \iff& \frac{-4}{3}<x<\frac{8}{3}
        \end{array}
$$        
        \p\item Oplossing 2: \p 
        
        De afstand tussen $3x$ en $2$ moet kleiner zijn dan $6$
        
        \p Dus moet de afstand tussen $x$ en $\frac{2}{3}$ kleiner zijn  dan $2$.
        
        \p Dus moet 
        $$ 
             - 2 < x- \frac{2}{3} < +2 \text{ of }
        $$
        
        $$
            \frac{-4}{3}<x<\frac{8}{3}
        $$
        
 %       \item TODO: grafische voorstelling
\end{enumerate}

UOVT (Uitstekende Oefening Voor Thuis):  idem voor $|3-2x| < 1$ 

\end{frame}

\begin{frame}{Ongelijkheden: eigenschappen}
\begin{proposition}
    
Voor elke $x,y,r \in\R $ geldt
\begin{align}
 x< y                  & \iff \;x+r< y+r \\
(r>0 \text{ en }  x<y) & \implies \; r\cdot x< r\cdot y \\
(r<0 \text{ en } x<y)  & \implies \; r\cdot x> r\cdot y \text{ (of ook } r\cdot y < r\cdot x)
\end{align}

\end{proposition}

\vfill
\p Dus: vermenigvuldigen met een \textit{negatief} getal \textit{keert de ongelijkheid om}.

\vfill
\p PAS OP: \textit{iedereen} laat zich daar wel eens aan vangen. 

\p {\color{red}U bent gewaarschuwd: \textit{iedereen}, dus ook \textit{u}.\p{ Jawel: ook \textit{U} daar}}. 


\end{frame}

\begin{frame}{Ongelijkheden: oefeningen}
       
Onderzoek welke $x\in \R$ voldoen aan
        
\begin{enumerate}      
\item $|\sqrt{x}+3|<9$
\item $\ds \left|  \frac{9 - \sqrt{1+20x}}{10} \right| < 1$
\item (UOVT) $\ds \left|5-\frac{2}{x}\right|>1$
\end{enumerate}

\end{frame}

\begin{frame}{Een functie toepassen op beide leden van een ongelijkheid}

Zij $x,y\in\R$ en $f$ een reële functie.

\begin{problem}
     als $x<y$, kunnen we dan iets zinnigs zeggen over $f(x)$ en $f(y)$?
\end{problem}

\begin{tabular}{rrclccc}
%\p In De Ideale Wereld:& als & $x<y$   &dan &  
%            $f(x)$ & $\only<.>{?}\only<+->{<}$ & $f(y)$.\\
\p In De Ideale Wereld:& als & $x<y$   &dan &  
           $f(x)$ & $\onslide<-+>{?}\onslide<+->{\overset{?}{<}}$ & $f(y)$.\\
\p Met kwadrateren:    & als & $x<y$   &dan & 
            $x^2$ & $\onslide<-+>{?}\onslide<+->{\overset{?}{<}}$ & $y^2$ \\
\p                     & als & $2<4$   &dan &  
            $2^2$ & $\onslide<-+>{?}\onslide<+->{<}$ & $4^2$ \\
\p                     & als & $-4<-2$ &dan &  
            $(-4)^2$ & $\onslide<-+>{?}\onslide<+->{\redgt}$ & $(-2)^2$ \\
\p                     & als & $-4<2$  &dan &  
            $(-4)^2$ & $\onslide<-+>{?}\onslide<+->{\redgt}$ & $(2)^2$ \\
\p                     & als & $-2<4$  &dan &  
            $(-2)^2$ & $\onslide<-+>{?}\onslide<+->{\redlt}$ & $(4)^2$
\end{tabular}

\p
\begin{proposition}
\setlength{\abovedisplayskip}{-5pt}
\begin{align}
 0\leq &\;x<y         &&\implies &x^2<y^2 \\
       &\;x<y\;\leq 0 &&\implies &x^2>y^2 %\\
%\mbox{als } a<0<b  &\mbox{ dan kan je niet weten of }\;& a^2<b^2 \; \mbox{of} \;a^2>b^2 
\end{align}
\end{proposition}
\end{frame}


\begin{frame}{Ongelijkheden en kwadraten}

\begin{figure}[H]
	\begin{tikzpicture}[scale=0.5]
\begin{axis}
[
%xtick={-4,-3,-2-1,0,1,2,3,4},
%xticklabels = {$4,},
ytick={4,16},
yticklabels = {$4$, $16$},
%axis equal,
ymax=20, %ymin=-5,
%samples=200,
axis lines=center,
%extra y ticks={0},
%restrict y to domain=-10:10,
%legend style={legend pos=south west,font=\tiny},
]
\addplot[domain=-5:5,color=blue] {x^2};
\node at (axis cs:-2, 4) [circle, scale=0.3,draw=black, fill=black] {};
\node at (axis cs:-4,16) [circle, scale=0.3,draw=black, fill=black] {};
\node at (axis cs: 4,16) [circle, scale=0.3,draw=black, fill=black] {};
\node at (axis cs: 2, 4) [circle, scale=0.3,draw=black, fill=black] {};
\end{axis}
\onslide<1->
\end{tikzpicture}	
\end{figure}
Dus: 

%\renewcommand{\p}{\action<+->}
\begin{align*}
-4 &< -2 &&\text { en } \p & (-4)^2 &\mathbin{\pmb{\color{red}>}} (-2)^2 \\
 2 &<  4 &&\text { en } \p &    2^2 & < 4^2 \\
-2 &<  4 &&\text { en } \p & (-2)^2 & < 4^2 \\
-4 &<  2 &&\text { en } \p & (-4)^2 &\mathbin{\pmb{\color{red}>}} 2^2 
\end{align*}
 
\end{frame}

\begin{frame}{Een functie toepassen op beide leden van een ongelijkheid}

Zij $x,y\in\R$ en $f$ een reële functie.

\begin{proposition}
%	\setlength{\abovedisplayskip}{-5pt}
	\begin{tabular}{ccc}
	$f$ is stijgend & $\iff$   & ( $x>y \implies f(x) > f(y)$ ) \\
	$f$ is dalend   & $\iff$   & ( $x>y \implies f(x) < f(y)$ ) 
	\end{tabular}
\end{proposition}
\end{frame}


\begin{frame}{Eerstegraadsveeltermen (p.25)}
%\subsection{Eerste - en tweedegraads veeltermen p25}
De grafiek van de functie $f$ met voorschrift $ f(x)= ax+b$ is een rechte met $a$ als richtingsco\"effici\"ent en b als snijpunt met de $y$-as.
	
	De vergelijking van de rechte door $(x_1,y_1)$ en $(x_2,y_2)$ is:
	
	$$
	y-y_1 = \frac{y_2-y_1}{x_2-x_1} (x-x_1)
	$$

\end{frame}	
\begin{frame}{Tweedegraadsveeltermen (p.25)}	
 De grafiek van de functie $f$ met voorschrift $ f(x)= ax^2+bx+c$ met
	$a\neq 0$ is een parabool.
	
	We hebben een bergparabool als $a<0$ en een dalparabool als $a>0$.
	\\
	
	$D=b^2-4ac$ noemt men de \emph{discriminant}
	\begin{itemize}
		\item Als $D > 0$ heeft $f$ 2 nulpunten:
		$\displaystyle{\frac{-b\pm \sqrt{D}}{2a}}$
		\item Als $D = 0$ heeft $f$ 1 nulpunt: $\displaystyle{-\frac{b}{2a}}$
		\item Als $D<0$ heeft $f$ geen nulpunten.
	\end{itemize}
	Als $x_1$ en $x_2$ nulpunten zijn dan is ${\color{red}a}x^2+bx+c = {\color{red}a}
	(x-x_1)(x-x_2)$.


\end{frame}

\begin{frame} {Ontbinden in factoren: wat is dat?}

\begin{definitie}
	Per definitie betekent \textit{iets ontbinden in factoren}  hetzelfde als \\
	\textit{iets schrijven
	als een product} (van \textit{zo veel mogelijk} (niet triviale) \textit{factoren}).
\end{definitie}

\p
\textbf{Voorbeelden:} Ontbind volgende \textit{ietsen} in factoren:
\begin{enumerate}
	\item   $12 = \p 2\cdot2\cdot3 = 2^2\cdot3$ \p (maar niet $4\cdot3$, of $1^{42}\cdot2^2\cdot3$ ) 
	\item\p $32 = \p 2^5$
	\item\p $x^2+2x+1 = \p (x+1)^2 \quad (\; =(x+1)(x+1) \;)$
\end{enumerate}

\p \textit{Meneer, waarom moeten wij dat kunnen ?}

\p  {\small Welnu\p,  sommige dingen zijn \textit{veel} gemakkelijker voor \textit{producten} dan \textit{sommen}. Een som kunnen schrijven als een product is dan \textit{zeer} handig.}

\p\textit{ Meneer, kan je daar een voorbeeld van geven?}

\p  {\small Zeer zeker: \p zoek maar eens de nulpunten van de  veeltermen  }\\
$f(x)=x^{42}+x^{13}+7$ \qquad (een som!) en \\
$g(x)=(x-4)^{2019}(x+42)^2(x-\pi)^{42}$ \qquad (een product!).

\end{frame}

\begin{frame} {Ontbinden in factoren}
\begin{problem}
Ontbind een veelterm $f(x)$ in factoren.
\end{problem}
\begin{itemize} 
    \item Stap 1: Zoek een nulpunt $a$ van $f(x)$. 
    \\ {\small (Eigenschap: dan kunnen we $f(x)$ schrijven als
$f(x) = (x-a) g(x)$ met $g(x)$ een veelterm met lagere graad.)}
    \item Stap 2: Bereken dus het quotiënt $\ds g(x) = \frac{f(x)}{x-a}$
    \item Stap 3: Ontbind (indien mogelijk) ook $g(x)$ verder in factoren.
\end{itemize}
Eig:  

Als $b x^3+ c x^2+ d x+ e$, met $b$,$c$, $d$, $e$ gehele
getallen, deelbaar is door $(x-a)$, met $a$ geheel, dan moet $a$ een
deler zijn van de constante term $e$. 

Indien $f(x)$ geen enkel geheel nulpunt heeft dan moeten we een benaderingsmethode gebruiken 

(Cursus Hoofdstuk 5).
\end{frame}


\begin{frame}{Faculteit en binomiaalco\"effici\"ent}
\begin{definition}
\begin{itemize}
	\item[] $n! = n.(n-1).(n-2) \ldots  3.2.1$
	\item[] $\displaystyle \left( \begin{array}{c} n \\ k \end{array} \right) = \frac{n!}{k! (n-k)!}$
\end{itemize}
\end{definition}
\end{frame}

\begin{frame}{Rekenen met breuken}
Zij $a,b,c,d\in \R$. Er geldt:
\begin{proposition}
\begin{itemize}
	\item $\ds{\frac{a}{b}+\frac{c}{d}}=\ds{\frac{ad+cb}{bd}}$ \hspace{3mm}   indien $b,d \neq 0$
	\item $\ds{\frac{\frac{a}{b}}{\ c\ }}=\ds{\frac{a}{bc}}$ \hspace{3mm}   indien $b,c \neq 0$
	\item $\ds{\frac{a}{\ \frac{b}{c}\ }}=\ds{\frac{ac}{b}}$ \hspace{3mm}   indien $b,c \neq 0$
	%\item $\ds{\frac{\frac{a}{b}}{\ \frac{c}{d}\ }}=\ds{\frac{ad}{bc}}$ \hspace{3mm}   indien $b,c,d \neq 0$
\end{itemize}
\end{proposition}
\end{frame}

\begin{frame}{Machten p. 6}
Zij $x,y>0$ en $m,n\in\R$. Er geldt:

\begin{proposition}
\begin{itemize}
	\item $\ds{x^{m}x^{n}}= \ds{x^{m+n}}$
	\item $\ds{\sqrt[n] x = x^{\frac1n}}$
	\item $\ds{\frac{x^{m}}{x^{n}}}= \ds{x^{m-n}}$
	\item $\ds{(xy)^n}=\ds{x^ny^n}$
	\item $\ds{\left(x^{m}\right)^{n}}= \ds{x^{mn}}$
\end{itemize}
\end{proposition}
\end{frame}

\end{document}

