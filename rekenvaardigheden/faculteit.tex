%%
% ximera activiteit
% Copyrigth
%
\documentclass{ximera}
%
% Opties (enkel te gebruiken voor lokale testen; NOOIT committen met opties)
%
%\pdfOnly{\providecommand\showtodonotes{}}
%\newcommand\xmnouitweiding{}


%
% copied from https://github.com/mooculus/calculus
%
\usepackage[utf8]{inputenc}


\graphicspath{
	{./}
	{goniometrie/}
}


%\usepackage{todonotes}
%\usepackage{mathtools} %% Required for wide table Curl and Greens
%\usepackage{cuted} %% Required for wide table Curl and Greens
\newcommand{\todo}{}

% Font niet (correct?) geinstalleerd in MikTeX?
%\usepackage{esint} % for \oiint
%\ifxake%%https://math.meta.stackexchange.com/questions/9973/how-do-you-render-a-closed-surface-double-integral
%\renewcommand{\oiint}{{\large\bigcirc}\kern-1.56em\iint}
%\fi


\newcommand{\mooculus}{\textsf{\textbf{MOOC}\textnormal{\textsf{ULUS}}}}

\usepackage{tkz-euclide}\usepackage{tikz}
\usepackage{tikz-cd}
\usetikzlibrary{arrows}
\tikzset{>=stealth,commutative diagrams/.cd,
  arrow style=tikz,diagrams={>=stealth}} %% cool arrow head
\tikzset{shorten <>/.style={ shorten >=#1, shorten <=#1 } } %% allows shorter vectors

\usetikzlibrary{backgrounds} %% for boxes around graphs
\usetikzlibrary{shapes,positioning}  %% Clouds and stars
\usetikzlibrary{matrix} %% for matrix
\usepgfplotslibrary{polar} %% for polar plots
\usepgfplotslibrary{fillbetween} %% to shade area between curves in TikZ
\usetkzobj{all}
\usepackage[makeroom]{cancel} %% for strike outs
%\usepackage{mathtools} %% for pretty underbrace % Breaks Ximera
%\usepackage{multicol}
\usepackage{pgffor} %% required for integral for loops



%% http://tex.stackexchange.com/questions/66490/drawing-a-tikz-arc-specifying-the-center
%% Draws beach ball
\tikzset{pics/carc/.style args={#1:#2:#3}{code={\draw[pic actions] (#1:#3) arc(#1:#2:#3);}}}



\usepackage{array}
\setlength{\extrarowheight}{+.1cm}
\newdimen\digitwidth
\settowidth\digitwidth{9}
\def\divrule#1#2{
\noalign{\moveright#1\digitwidth
\vbox{\hrule width#2\digitwidth}}}





\newcommand{\RR}{\mathbb R}
\newcommand{\R}{\mathbb R}
\newcommand{\N}{\mathbb N}
\newcommand{\Z}{\mathbb Z}

\newcommand{\sagemath}{\textsf{SageMath}}


%\renewcommand{\d}{\,d\!}
\renewcommand{\d}{\mathop{}\!d}
\newcommand{\dd}[2][]{\frac{\d #1}{\d #2}}
\newcommand{\pp}[2][]{\frac{\partial #1}{\partial #2}}
\renewcommand{\l}{\ell}
\newcommand{\ddx}{\frac{d}{\d x}}

\newcommand{\zeroOverZero}{\ensuremath{\boldsymbol{\tfrac{0}{0}}}}
\newcommand{\inftyOverInfty}{\ensuremath{\boldsymbol{\tfrac{\infty}{\infty}}}}
\newcommand{\zeroOverInfty}{\ensuremath{\boldsymbol{\tfrac{0}{\infty}}}}
\newcommand{\zeroTimesInfty}{\ensuremath{\small\boldsymbol{0\cdot \infty}}}
\newcommand{\inftyMinusInfty}{\ensuremath{\small\boldsymbol{\infty - \infty}}}
\newcommand{\oneToInfty}{\ensuremath{\boldsymbol{1^\infty}}}
\newcommand{\zeroToZero}{\ensuremath{\boldsymbol{0^0}}}
\newcommand{\inftyToZero}{\ensuremath{\boldsymbol{\infty^0}}}



\newcommand{\numOverZero}{\ensuremath{\boldsymbol{\tfrac{\#}{0}}}}
\newcommand{\dfn}{\textbf}
%\newcommand{\unit}{\,\mathrm}
\newcommand{\unit}{\mathop{}\!\mathrm}
\newcommand{\eval}[1]{\bigg[ #1 \bigg]}
\newcommand{\seq}[1]{\left( #1 \right)}
\renewcommand{\epsilon}{\varepsilon}
\renewcommand{\phi}{\varphi}


\renewcommand{\iff}{\Leftrightarrow}

\DeclareMathOperator{\arccot}{arccot}
\DeclareMathOperator{\arcsec}{arcsec}
\DeclareMathOperator{\arccsc}{arccsc}
\DeclareMathOperator{\si}{Si}
\DeclareMathOperator{\scal}{scal}
\DeclareMathOperator{\sign}{sign}


%% \newcommand{\tightoverset}[2]{% for arrow vec
%%   \mathop{#2}\limits^{\vbox to -.5ex{\kern-0.75ex\hbox{$#1$}\vss}}}
\newcommand{\arrowvec}[1]{{\overset{\rightharpoonup}{#1}}}
%\renewcommand{\vec}[1]{\arrowvec{\mathbf{#1}}}
\renewcommand{\vec}[1]{{\overset{\boldsymbol{\rightharpoonup}}{\mathbf{#1}}}\hspace{0in}}

\newcommand{\point}[1]{\left(#1\right)} %this allows \vector{ to be changed to \vector{ with a quick find and replace
\newcommand{\pt}[1]{\mathbf{#1}} %this allows \vec{ to be changed to \vec{ with a quick find and replace
\newcommand{\Lim}[2]{\lim_{\point{#1} \to \point{#2}}} %Bart, I changed this to point since I want to use it.  It runs through both of the exercise and exerciseE files in limits section, which is why it was in each document to start with.

\DeclareMathOperator{\proj}{\mathbf{proj}}
\newcommand{\veci}{{\boldsymbol{\hat{\imath}}}}
\newcommand{\vecj}{{\boldsymbol{\hat{\jmath}}}}
\newcommand{\veck}{{\boldsymbol{\hat{k}}}}
\newcommand{\vecl}{\vec{\boldsymbol{\l}}}
\newcommand{\uvec}[1]{\mathbf{\hat{#1}}}
\newcommand{\utan}{\mathbf{\hat{t}}}
\newcommand{\unormal}{\mathbf{\hat{n}}}
\newcommand{\ubinormal}{\mathbf{\hat{b}}}

\newcommand{\dotp}{\bullet}
\newcommand{\cross}{\boldsymbol\times}
\newcommand{\grad}{\boldsymbol\nabla}
\newcommand{\divergence}{\grad\dotp}
\newcommand{\curl}{\grad\cross}
%\DeclareMathOperator{\divergence}{divergence}
%\DeclareMathOperator{\curl}[1]{\grad\cross #1}
\newcommand{\lto}{\mathop{\longrightarrow\,}\limits}

\renewcommand{\bar}{\overline}

\colorlet{textColor}{black}
\colorlet{background}{white}
\colorlet{penColor}{blue!50!black} % Color of a curve in a plot
\colorlet{penColor2}{red!50!black}% Color of a curve in a plot
\colorlet{penColor3}{red!50!blue} % Color of a curve in a plot
\colorlet{penColor4}{green!50!black} % Color of a curve in a plot
\colorlet{penColor5}{orange!80!black} % Color of a curve in a plot
\colorlet{penColor6}{yellow!70!black} % Color of a curve in a plot
\colorlet{fill1}{penColor!20} % Color of fill in a plot
\colorlet{fill2}{penColor2!20} % Color of fill in a plot
\colorlet{fillp}{fill1} % Color of positive area
\colorlet{filln}{penColor2!20} % Color of negative area
\colorlet{fill3}{penColor3!20} % Fill
\colorlet{fill4}{penColor4!20} % Fill
\colorlet{fill5}{penColor5!20} % Fill
\colorlet{gridColor}{gray!50} % Color of grid in a plot

\newcommand{\surfaceColor}{violet}
\newcommand{\surfaceColorTwo}{redyellow}
\newcommand{\sliceColor}{greenyellow}




\pgfmathdeclarefunction{gauss}{2}{% gives gaussian
  \pgfmathparse{1/(#2*sqrt(2*pi))*exp(-((x-#1)^2)/(2*#2^2))}%
}


%%%%%%%%%%%%%
%% Vectors
%%%%%%%%%%%%%

%% Simple horiz vectors
\renewcommand{\vector}[1]{\left\langle #1\right\rangle}


%% %% Complex Horiz Vectors with angle brackets
%% \makeatletter
%% \renewcommand{\vector}[2][ , ]{\left\langle%
%%   \def\nextitem{\def\nextitem{#1}}%
%%   \@for \el:=#2\do{\nextitem\el}\right\rangle%
%% }
%% \makeatother

%% %% Vertical Vectors
%% \def\vector#1{\begin{bmatrix}\vecListA#1,,\end{bmatrix}}
%% \def\vecListA#1,{\if,#1,\else #1\cr \expandafter \vecListA \fi}

%%%%%%%%%%%%%
%% End of vectors
%%%%%%%%%%%%%

%\newcommand{\fullwidth}{}
%\newcommand{\normalwidth}{}



%% makes a snazzy t-chart for evaluating functions
%\newenvironment{tchart}{\rowcolors{2}{}{background!90!textColor}\array}{\endarray}

%%This is to help with formatting on future title pages.
\newenvironment{sectionOutcomes}{}{}



%% Flowchart stuff
%\tikzstyle{startstop} = [rectangle, rounded corners, minimum width=3cm, minimum height=1cm,text centered, draw=black]
%\tikzstyle{question} = [rectangle, minimum width=3cm, minimum height=1cm, text centered, draw=black]
%\tikzstyle{decision} = [trapezium, trapezium left angle=70, trapezium right angle=110, minimum width=3cm, minimum height=1cm, text centered, draw=black]
%\tikzstyle{question} = [rectangle, rounded corners, minimum width=3cm, minimum height=1cm,text centered, draw=black]
%\tikzstyle{process} = [rectangle, minimum width=3cm, minimum height=1cm, text centered, draw=black]
%\tikzstyle{decision} = [trapezium, trapezium left angle=70, trapezium right angle=110, minimum width=3cm, minimum height=1cm, text centered, draw=black]


\begin{document}
    % Start specifieke settings:    
    \author{Zomercursus KU Leuven}
%    \outcome{}
    \xmtitle{De faculteit}{}
    % Start inhoud ximera 


%\title {Het sommatieteken en de faculteit\\{\large(met als belangrijke toepassing binomiaalco\"effici\"enten)}}
%{\qrslidesA{https://set.kuleuven.be/zomercursussen/wiskunde/lesmateriaal/slides-sommatieteken-faculteit.pdf}}{}

\section*{Inleiding}
In deze module zullen we het gebruik van het
sommatieteken en de faculteitsoperatie herhalen en bespreken
a.d.h.v.~voorbeeldoefeningen en enkele toepassingen.

Het \emph{sommatieteken} \lq$\sum$\rq\ wordt veel gebruikt in de
context van rijen en reeksen. Ook in de statistiek, de
kansrekening en de discrete wiskunde is het sommatieteken niet weg
te denken. In deze takken van de wiskunde wordt veel gewerkt met
lange sommen van termen die \lq op een gelijkaardige manier zijn
opgebouwd\rq\footnote{Dit wordt verder uitgelegd.}\ en is er
behoefte aan een elegante schrijfwijze voor deze sommen en een
manier om er gemakkelijk mee te kunnen rekenen. Beschouw
bijvoorbeeld $100$ re\"ele getallen die we aanduiden met
$a_1,a_2,\ldots,a_{100}$ en stel dat we de som van deze getallen
nodig hebben. Deze som kunnen we aanduiden
met\[a_1+a_2+a_3+\cdots+a_{99}+a_{100},\]maar als we deze som vaak
nodig hebben en ermee willen rekenen, is dit geen handige notatie.
Het sommatieteken biedt een uitweg, met behulp van dit
sommatieteken wordt bovenstaande som verkort geschreven
als\[a_1+a_2+a_3+\cdots+a_{99}+a_{100}\
\overset{\textrm{notatie}}{=}\ \sum_{k=1}^{100}a_k.\]We lezen dit
als \lq de som van de termen $a_k$ waarbij (de index) $k$ loopt
van $1$ tot $100$\rq. Dit is een korte en duidelijke (zonder de
vage \lq$\ldots$\rq) schrijfwijze die alle noodzakelijke
informatie bevat. Bovendien laat deze schrijfwijze toe gemakkelijk
met de sommen te rekenen. Hiervoor bestaan de nodige rekenregels
(zie Sectie~\ref{M02_rekenregels}). Een som schrijven m.b.v.~het
sommatieteken werkt wel enkel als de verschillende termen een
gemeenschappelijke vorm hebben (in het voorbeeld hierboven zijn
alle termen van de vorm \lq$a_k$\rq). Deze gemeenschappelijke vorm
kan van velerlei aard zijn, zoals zal blijken in de voorbeelden en
oefeningen. De som $5+283-7/3+\pi/2+18-205-11$ kunnen we
bijvoorbeeld niet korter opschrijven, tenzij natuurlijk dat we de
termen een naam geven: $a_1=5$, $a_2=283$, $a_3=-7/3$, \ldots

De \emph{faculteitsoperatie} is een operatie op natuurlijke
getallen die vaak opduikt in telproblemen, kansberekeningen en
statistiek, maar ook in de analyse (taylorreeksen,
differentiaal- en integraalrekening). Als inleidend voorbeeld
bekijken we volgend telprobleem. Stel er wordt een renwedstrijd
gehouden met twaalf paarden. Bij de \emph{tierc\'{e}} kunnen
gokkers dan voorspellen in welke volgorde de paarden over de
eindstreep komen. Hoeveel mogelijke voorspellingen zijn er? Voor
het winnende paard zijn er precies $12$ mogelijkheden. Eens het
winnende paard vastligt, blijven er nog $11$ paarden over die als
tweede kunnen eindigen, er zijn dus nog $11$ mogelijkheden om het
tweede paard te \lq kiezen\rq. Voor het paard dat als derde over
de eindmeet zal komen, blijven er dan nog $10$ mogelijkheden over
enzovoort\ldots\ Het aantal mogelijke pronostieken voor de
tierc\'e is bijgevolg gelijk aan
$12\cdot11\cdot10\cdot9\cdot8\cdot7\cdot6\cdot5\cdot4\cdot3\cdot2\cdot1=479001600$.
Veralgemenen we dit concept tot een paardenwedstrijd met $n$
paarden ($n\in\Nnul$), bekomen we als aantal mogelijke
pronostieken
\begin{formula}\label{M02_nfac}
\[n(n-1)(n-2)(n-3)\cdots3\cdot2\cdot1.\]
\end{formula}
Een uitdrukking van de vorm (\ref{M02_nfac}) komt heel vaak voor in de
wiskunde en i.h.b.~in de hogergenoemde disciplines. Een verkorte
schrijfwijze is dus aangewezen: het product (\ref{M02_nfac}) noteren
we met $n!$ (lees: $n$-faculteit).

In de toepassingen bij dit pakket behandelen we  nog \emph{de
driehoek van Pascal} en het \emph{binomium van Newton}.
%
%Voor meer oefenmateriaal kan je ook terecht op
%\texttt{www.usolv-it.be} bij de topic \textcolor{red}{(?)}. Meer
%uitleg kan je eveneens vinden in volgende handboeken:
%\begin{itemize}
%\item Delta 5/6, Kansrekenen (6/8u)
%\item Van basis tot limiet 3/4, Statistiek
%\item Van basis tot limiet 5/6, Combinatoriek en Kansrekenen.
%\end{itemize}


\section{$n$-Faculteit}
\subsection{Definitie}
In de inleiding, toen we het hadden over paardenrennen, hebben we
reeds een definitie gegeven van de faculteitsoperatie \lq$!$\rq,
dit ging als volgt:
\begin{definition}[faculteitsoperatie{\normalfont\ ---\ 1$^{\textrm{e}}$ versie}]
Zij $n\in\Nnul$. We defini\"eren het natuurlijk getal $n!$ (lees:
$n$-faculteit) als
\[n!\ \perdef\ n(n-1)(n-2)(n-3)\cdots3\cdot2\cdot1.\]
\end{definition}

Dit getal, $n!$, kwam overeen met het aantal mogelijke uitslagen
van een renwedstrijd met $n$ paarden of -- in het algemeen -- het
aantal mogelijke manieren om $n$ verschillende objecten te
rangschikken (op een rij te zetten). Bijvoorbeeld, op hoeveel
manieren kan je $22$ leerlingen van een klas achter elkaar in een
rij zetten? Antwoord: op $22!=1124000727777607680000$ manieren.

Ook het getal $0!$ is gedefinieerd: $0!\perdef1$. Deze definitie
van $0$-faculteit is in overeenstemming met de eerder gegeven
interpretatie van de faculteit als het aantal mogelijke manieren
om verschillende objecten te rangschikken: op hoeveel manieren kan
men $0$ objecten rangschikken? Antwoord: er valt niets te
rangschikken, dit kan op juist $1$ manier.

Meestal echter wordt de faculteitsoperatie niet gedefinieerd zoals
wij hier net gedaan hebben, maar wel op een zogenaamde
\emph{recursieve}\footnote{Een definitie heet recursief wanneer
men in de definitie datgene wat men definieert, reeds gebruikt.
Dit moet natuurlijk met de nodige omzichtigheid gebeuren, de hier
gegeven definitie van de faculteitsoperatie is een goed
voorbeeld.} manier. Dit gaat als volgt:
\begin{definition}[faculteitsoperatie{\normalfont\ ---\ 2$^{\textrm{e}}$ versie}]
De faculteitsfunctie $!:\N\to\N:n\mapsto n!$ is de unieke functie
van \N\ naar \N\ die voldoet aan volgende twee voorwaarden:
\begin{enumerate}
\item[(i)] $0!=1$ en
\item[(ii)] $n!=n\cdot(n-1)!$ voor elke $n\in\Nnul$.
\end{enumerate}
\end{definition}

Uit deze definitie halen we dat

\noindent
\begin{tabular}{cl@{\;}c@{\;}c@{\;}c@{\;}rl}
(0) & $0!$ & & & $=$ & $1$ & uit (i)\\
(1) & $1!$ & $=$ & $1\cdot(1-1)!=1\cdot0!=1\cdot1$ & $=$ & $1$ & uit (ii) en (0)\\
(2) & $2!$ & $=$ & $2\cdot(2-1)!=2\cdot1!=2\cdot1$ & $=$ & $2$ & uit (ii) en (1)\\
(3) & $3!$ & $=$ & $3\cdot(3-1)!=3\cdot2!=3\cdot2$ & $=$ & $6$ & uit (ii) en (2)\\
(4) & $4!$ & $=$ & $4\cdot(4-1)!=4\cdot3!=4\cdot6$ & $=$ & $24$ & uit (ii) en (3)\\
$\vdots$ & & & & & &
\end{tabular}

In deze definitie staat meteen ook de belangrijkste rekenregel
voor de faculteitsoperatie:
\begin{kaderrekenregel}
\mbox{\ }\hfill
\begin{tabular}[t]{lr@{\;}c@{\;}ll}
& $n!$ & $=$ & $n(n-1)!$ & voor alle $n\in\Nnul$ \\
of nog & $(n+1)!$ & $=$ & $(n+1)n!$ & voor alle $n\in\N$
\end{tabular}
\hfill\mbox{\ }
\end{kaderrekenregel}

\newpage
\startletternummering
\subsection{Voorbeeldoefeningen}
\begin{voorbeeldoefening}
Vul aan. ($k,l\in\N$, $l\geqslant3$)
\begin{enumerate}
\item $\ds{(k+1)!+k!=(\ldots)k!}$
\begin{oplossing}
\begin{eqnarray*}
(k+1)!+k!&=&(k+1)k!+k!\\
&=&\left((k+1)+1\right)k!\\
&=&(k+2)k!
\end{eqnarray*}
\end{oplossing}
\item $\ds{3(l-2)!+2(l-3)!=(\ldots)(l-3)!}$
\begin{oplossing}
\begin{eqnarray*}
3(l-2)!+2(l-3)!&=&3(l-2)(l-3)!+2(l-3)!\\
&=&\left(3(l-2)+2\right)(l-3)!\\
&=&(3l-4)(l-3)!
\end{eqnarray*}
\end{oplossing}~
\end{enumerate}
\end{voorbeeldoefening}
\begin{voorbeeldoefening}
Vereenvoudig. ($k,n\in\N$, $k+n\geqslant2$)
\begin{enumerate}
\item $\ds{\frac{(k+1)!}{k!}=\frac{(k+1)k!}{k!}=k+1}$
\item $\ds{\frac{(k+3)!}{k!}=\frac{(k+3)(k+2)(k+1)k!}{k!}=(k+3)(k+2)(k+1)}$
\item $\ds{\frac{(k+n)!}{k+n-1}=\frac{(k+n)(k+n-1)(k+n-2)!}{k+n-1}=(k+n)(k+n-2)!}$
\end{enumerate}
\end{voorbeeldoefening}
\oefverwijzing{Probeer nu zelf op analoge wijze Oefening 3 van
Sectie~\ref{M02_oef} op te lossen.}
\eindeletternummering

\subsection{Toepassingen}
\subsubsection{Telproblemen: variaties}\label{M02_pr}
We hernemen het voorbeeld van de paardenrennen uit de inleiding.
We houden een wedstrijd met $12$ paarden. Opnieuw moeten gokkers
de uitslag van de wedstrijd voorspellen, maar deze keer niet de
volledige uitslag. Ze moeten enkel een pronostiek maken van welke
drie paarden eerst zullen eindigen en in welke volgorde. Op
hoeveel manieren is dit mogelijk? Als we dezelfde redenering
volgen als in de inleiding, vinden we dat er
$12\cdot11\cdot10=1320$ verschillende pronostieken mogelijk zijn.

Wiskundig kan men zo'n pronostiek zien als een geordend drietal
van elementen afkomstig uit een verzameling van $12$ elementen.
Men spreekt in deze context van een \emph{variatie} van $3$
elementen uit $12$ elementen. Het aantal mogelijke variaties kan
men berekenen m.b.v.~de faculteitsoperatie. Zij namelijk
$p,n\in\N$ met $p\leqslant n$, dan is het aantal variaties van $p$
elementen uit $n$ elementen gegeven door
\begin{eqnarray*}
  V^p_n &\overset{\textrm{notatie}}{=}& \underbrace{n(n-1)(n-2)\cdots(n-p+1)}_{p\
\textrm{factoren}}\\
&=&
\frac{n(n-1)(n-2)\cdots(n-p+1)(n-p)(n-p-1)(n-p-2)\cdots3\cdot2\cdot1}{(n-p)(n-p-1)(n-p-2)\cdots3\cdot2\cdot1}\\
&=& \frac{n!}{(n-p)!}.
\end{eqnarray*}

\subsubsection{Combinaties, de driehoek van Pascal,
het binomium van Newton} We hernemen nogmaals het voorbeeld van de
paardenrennen uit vorige paragraaf~(\ref{M02_pr}). Ditmaal moeten de
gokkers voorspellen welke drie paarden op de eerste drie plaatsen
zullen eindigen, de volgorde heeft geen enkel belang. Wiskundig
vertaalt zich een dergelijke pronostiek in een deelverzameling van
$3$ elementen afkomstig uit een verzameling van $12$ elementen (er
nemen $12$ paarden deel aan de wedstrijd). Als $p,n\in\N$ en
$p\leqslant n$, noemen we een deelverzameling van $p$ elementen
uit een verzameling van $n$ elementen een \emph{combinatie} van
$p$ elementen uit $n$ elementen.

Hoeveel mogelijke pronostieken bestaan er onder deze nieuwe
voorwaarden, m.a.w.~hoeveel combinaties zijn er van $3$ uit $12$?
Omdat de volgorde nu geen rol meer speelt, bestaan er in elk geval
minder combinaties dan variaties. Daar je een verzameling van $3$
elementen op precies $3!=6$ manieren kan ordenen, bestaan er voor
elke combinatie $\{a,b,c\}$ van $3$ elementen precies $6$
verschillende variaties $(a,b,c)$, $(a,c,b)$, $(b,a,c)$,
$(b,c,a)$, $(c,a,b)$ en $(c,b,a)$, opgebouwd uit diezelfde $3$
elementen $a$, $b$ en $c$. M.a.w.~er bestaan precies zes keer meer
variaties van $3$ uit $12$ dan er combinaties bestaan van $3$ uit
$12$. Algemeen hebben we dan dat voor $p,n\in\N$ met $p\leqslant
n$ er $p!$ keer meer variaties zijn van $p$ elementen uit $n$
elementen dan er combinaties zijn van $p$ uit $n$, zodat het
totaal aantal combinaties van $p$ elementen uit $n$ elementen
gegeven is door onderstaande formule:
\[C^p_n\overset{\textrm{notatie }}{=}{n\choose p}\overset{\textrm{notatie}}{=}\frac{V^p_n}{p!}=\frac{n!}{p!(n-p)!}.\]
\begin{example}\ \vspace{0ex}
\begin{enumerate}
\item[(a)] $\ds{C^3_5={5\choose3}=\frac{5!}{3!(5-3)!}=\frac{5!}{3!2!}=\frac{5\cdot4\cdot3\cdot2\cdot1}{(3\cdot2\cdot1)(2\cdot1)}=10}$
\item[(b)] $\ds{C^2_2={2\choose2}=\frac{2!}{2!0!}=\frac{2\cdot1}{(2\cdot1)\cdot1}=1}$
\item[(c)] $\ds{C^0_4={4\choose0}=\frac{4!}{0!4!}=\frac{4\cdot3\cdot2\cdot1}{1\cdot(4\cdot3\cdot2\cdot1)}=1}$
\item[(d)]
$\ds{C^1_4={4\choose1}=\frac{4!}{1!3!}=\frac{4\cdot3\cdot2\cdot1}{1\cdot(3\cdot2\cdot1)}=4}$.
\end{enumerate}
\end{example}

Natuurlijke getallen van de vorm ${n\choose p}$ met $p,n\in\N$ en
$p\leqslant n$, worden ook wel \emph{binomiaalgetallen} genoemd.
De herkomst van deze naam wordt verder duidelijk. De
binomiaalgetallen ${n\choose p}$ worden vaak voorgesteld in een
driehoekig schema, de zgn.~\emph{driehoek van
Pascal}\footnotemark\ (zie Figuur~\ref{M02_driehoek}).
\footnotetext{De Fransman Blaise Pascal was filosoof, theoloog,
wis- en natuurkundige. Hij werd geboren in 1623 in Clermont
Ferrand en stierf in Parijs in 1662. Zijn belangrijkste prestaties
situeren zich in het gebied van de waarschijnlijkheidsrekening, de
combinatieleer, de projectieve meetkunde, de hydrostatica (de wet
van Pascal voor de druk in een vloeistof) en de hydrodynamica.}

\begin{figure}
\addtolength{\parindent}{-1.85cm}
\begin{tabular}{c|cccccccc}
$n$ & ${n\choose0}$ & ${n\choose1}$ & ${n\choose2}$ & ${n\choose3}$ & ${n\choose4}$ & ${n\choose5}$ & ${n\choose6}$ & $\cdots$ \\\\[-2.1ex]
\hline\\[-2.1ex]
$0$ & $1$ &  &  &  &  & & & \\
$1$ & $1$ & $1$ &  &  &  &  & &  \\
$2$ & $1$ & $2$ & $1$ &  &  &  & & \\
$3$ & $1$ & $3$ & $3$ & $1$ &  &  & & \\
$4$ & $1$ & $4$ & $6$ & $4$ & $1$ &  & & \\
$5$ & $1$ & $5$ & $10$ & $10$ & $5$ & $1$ & & \\
$6$ & $1$ & $6$ & $15$ & $20$ & $15$ & $6$ & $1$ & \\
$\vdots$ & $\vdots$ & $\vdots$ & $\vdots$ & $\vdots$ & $\vdots$ & $\vdots$ & $\vdots$ & $\ddots$
\end{tabular}
\ \ of\ \
\begin{tabular}{ccccccccccccc}
&&&&&& $1$ &&&&& \\
&&&&& $1$ && $1$ &&&&& \\
&&&& $1$ && $2$ && $1$ &&&& \\
&&& $1$ && $3$ && $3$ && $1$ &&& \\
&& $1$ && $4$ && $6$ && $4$ && $1$ && \\
& $1$ && $5$ && $10$ && $10$ && $5$  && $1$ & \\
$1$ && $6$ && $15$ && $20$ && $15$ && $6$ && $1$
\end{tabular}
\addtolength{\parindent}{+1.85cm} \caption{De driehoek van
Pascal}\label{M02_driehoek}
\end{figure}

Over de driehoek van Pascal valt heel veel te vertellen. We zullen
hier enkele eigenschappen en toepassingen vermelden zonder ze echt
te verklaren of te bewijzen. Dit zou ons nu te ver
leiden en vrijwel zeker kom je al deze dingen nog wel tegen in een
of andere cursus gedurende je opleiding.

Vooreerst is de driehoek van Pascal een driehoek, omdat voor een
zekere $n\in\N$ het binomiaalgetal ${n\choose p}$ enkel
gedefinieerd is voor $p\in\N$ met $p\leqslant n$. Ten tweede is de
driehoek symmetrisch, d.w.z. \[{n\choose p}={n\choose
n-p}\quad\textrm{voor alle}\ n,p\in\N\ \textrm{met}\ p\leqslant
n.\] Bovendien begint en eindigt elke rij met een $1$, terwijl het
tweede en het voorlaatste getal van de $n$-de rij steeds $n$ is,
in symbolen:
\[{n\choose 0}={n\choose
n}=1\quad\textrm{voor alle}\
n\in\N\qquad\textrm{en}\qquad{n\choose 1}={n\choose
n-1}=n\quad\textrm{voor alle}\ n\in\Nnul.\] Een andere
opmerkelijke eigenschap is de volgende. Bekijk de rechtse driehoek
in bovenstaande figuur, kies een rij en tel twee naburige
binomiaalgetallen op, je bekomt het binomiaalgetal in het midden
onder deze twee getallen of nog \[{n\choose p-1}+{n\choose
p}={n+1\choose p}\quad\textrm{voor alle}\ n,p\in\Nnul\
\textrm{met}\ p\leqslant n.\] De laatste erg belangrijke
eigenschap van de driehoek van Pascal die we hier zullen
vermelden, staat bekend onder \lq \emph{het binomium van
Newton}\rq\ en is meteen ook de verklaring voor de herkomst van de
naam \lq binomiaalgetallen\rq. Bekijk aandachtig onderstaande
lijst uitwerkingen van machten van de tweeterm of \emph{binoom}
$(a+b)$.

\begin{figure}[h]
\addtolength{\parindent}{-2.5cm}
\begin{tabular}{ccccccccccccccc}
$\ds{(a+b)^{\mathbf{0}}}$&$=$&&&&&&& $\mathbf{1}$ &&&&& \\
$\ds{(a+b)^{\mathbf{1}}}$&$=$&&&&&& $\mathbf{1}a$ &$+$& $\mathbf{1}b$ &&&&& \\
$\ds{(a+b)^{\mathbf{2}}}$&$=$&&&&& $\mathbf{1}a^2$ &$+$& $\mathbf{2}ab$ &$+$& $\mathbf{1}b^2$ &&&& \\
$\ds{(a+b)^{\mathbf{3}}}$&$=$&&&& $\mathbf{1}a^3$ &$+$& $\mathbf{3}a^2b$ &$+$& $\mathbf{3}ab^2$ &$+$& $\mathbf{1}b^3$ &&& \\
$\ds{(a+b)^{\mathbf{4}}}$&$=$&&& $\mathbf{1}a^4$ &$+$& $\mathbf{4}a^3b$ &$+$& $\mathbf{6}a^2b^2$ &$+$& $\mathbf{4}ab^3$ &$+$& $\mathbf{1}b^4$ && \\
$\ds{(a+b)^{\mathbf{5}}}$&$=$&& $\mathbf{1}a^5$ &$+$& $\mathbf{5}a^4b$ &$+$& $\mathbf{10}a^3b^2$ &$+$& $\mathbf{10}a^2b^3$ &$+$& $\mathbf{5}ab^4$  &$+$& $\mathbf{1}b^5$ & \\
$\ds{(a+b)^{\mathbf{6}}}$&$=$&$\mathbf{1}a^6$ &$+$& $\mathbf{6}a^5b$ &$+$& $\mathbf{15}a^4b^2$ &$+$& $\mathbf{20}a^3b^3$ &$+$& $\mathbf{15}a^2b^4$ &$+$& $\mathbf{6}ab^5$ &$+$& $\mathbf{1}b^6$ \\
$\vdots$&&&&&&&&&&&&&&
\end{tabular}
\addtolength{\parindent}{+2.5cm} \caption{Het binomium van
Newton}\label{M02_figuurbinomium}
\end{figure}

Wanneer we de uitwerkingen van de machten $(a+b)^n$ verticaal
rangschikken volgens stijgende $n$ en horizontaal rangschikken
volgens dalende machten van $a$ (en stijgende machten van $b$),
vormen de co\"effici\"enten (vet gedrukt) precies de driehoek van
Pascal. Dit resultaat kunnen we als volgt formuleren:
\newpage\begin{observation}[binomium van
Newton]\normalfont\footnotemark Zij $a,b\in\R$ en $n\in\N$, dan
geldt \[(a+b)^n=\sum^n_{p=0}{n\choose
p}a^{n-p}b^p=\sum^n_{p=0}{n\choose p}a^{p}b^{n-p}.\]
\end{observation}\footnotetext{Sir Isaac Newton was een Brits
natuurkundige, filosoof, wiskundige en alchemist. Hij werd geboren
in 1642 in Lincolnshire in Engeland en stierf in 1727 in Londen.
Newton was de leidende figuur in de wetenschappelijke revolutie van
de 17$^{\textrm{e}}$ eeuw en wordt dan ook algemeen erkend als
\'{e}\'{e}n der zeer groten in de wetenschap.}

\section{Oefeningen}\label{M02_oef}
\Opensolutionfile{ans}[ansM02]
\renewcommand{\labelenumi}{(\alph{enumi})}

\begin{oefening2}
Bereken volgende sommen:

\noindent
\begin{minipage}[t]{.24\textwidth}
\begin{enumerate}
\item $\ds{\sum^4_{k=2}2^k}$
\item $\ds{\sum^3_{k=1}3}$
\item $\ds{\sum^5_{a=0}(2a+1)}$
\end{enumerate}
\end{minipage}
\begin{minipage}[t]{.24\textwidth}
\begin{enumerate}\setcounter{enumi}{3}
\item $\ds{\sum^4_{l=1}(5l-1)}$
\item $\ds{\sum^3_{k=1}k^2}$
\item $\ds{\sum^{m+5}_{i=m}k}$\\$(m\in\N,\ k\in\R)$
\end{enumerate}
\end{minipage}
\begin{minipage}[t]{.27\textwidth}
\begin{enumerate}\setcounter{enumi}{6}
\item $\ds{\sum^6_{i=1}\frac{60}{i}}$
\item $\ds{\sum^3_{k=0}\sin^2\left(k\frac{\pi}{2}\right)}$
\item $\ds{\sum^5_{k=1}(n-k)-7}$\\$(n\in\R)$
\end{enumerate}
\end{minipage}
\begin{minipage}[t]{.25\textwidth}
\begin{enumerate}\setcounter{enumi}{9}
\item $\ds{\sum^8_{p=0}10^{5-p}}$
\item $\ds{\sum^5_{n=2}\left(\sum^n_{k=1}n\right)}$
\item $\ds{3\sum^3_{j=0}k+j}$\\$(j,k\in\R)$
\end{enumerate}
\end{minipage}
\begin{opl}%hier volgt de opl van deze oef
\mbox{ }\newline
\begin{minipage}[t]{.24\textwidth}
\begin{enumerate}
\item 28
\item 9
\item 36
\end{enumerate}
\end{minipage}
\begin{minipage}[t]{.24\textwidth}
\begin{enumerate}\setcounter{enumi}{3}
\item 46
\item 14
\item $6k$
\end{enumerate}
\end{minipage}
\begin{minipage}[t]{.27\textwidth}
\begin{enumerate}\setcounter{enumi}{6}
\item 147
\item 2
\item $5n-22$
\end{enumerate}
\end{minipage}
\begin{minipage}[t]{.25\textwidth}
\begin{enumerate}\setcounter{enumi}{9}
\item 111111,111
\item 54
\item $12k+j$
\end{enumerate}
\end{minipage}
\end{opl}


\end{oefening2}


\begin{oefening2}
Vul aan.
\begin{enumerate}
\item $\ds{\sum^3_{k=1}k^2=\sum^2_{k=1}k^2+\ldots}$
\item $\ds{\sum^5_{m=2}4m=\ldots+\sum^5_{m=3}4m}$
\item $\ds{\sum^3_{k=1}k^2=\sum^{\ldots}_{k=3}\ldots}$
\item $\ds{\sum^5_{m=2}2^m=\sum^{\ldots}_{p=5}2^{\ldots}}$
\end{enumerate}
\end{oefening2}

\begin{oefening2} Bereken of vul aan. ($m,n\in\N$ met $m\neq 0$ en $n<m$)
\begin{enumerate}
\item $8!$\quad(Bereken.)
\item $6\cdot5!\cdot 7$\quad(Schrijf korter en bereken vervolgens.)
\item $\ds{\frac{10!}{7!}}$\quad(Bereken zo kort mogelijk.)
\item $\ds{\frac{(m+4)!}{m!}}$\quad(Schrijf zonder faculteiten.)
\item $\ds{\frac{m!}{(m-1)!}}$\quad(Vereenvoudig.)
\item $\ds{\frac{(m-n)!}{(m-n-1)(m-n)}}$\quad(Vereenvoudig.)
\item $\ds{(m+2)!+(m+3)!=(m+2)!(\ldots)}$\quad(Vul aan.)
\end{enumerate}
\end{oefening2}

\begin{oefening2}Vul aan. ($m\in\N$)
\begin{enumerate}
\item $\ds{\frac{4}{(m+1)!}-\frac{1}{m!}=\frac{\ldots}{(m+1)!}}$
\item $\ds{\sum^{14}_{k=5}k^2=\sum^{10}_{l=1}\ldots}$
\end{enumerate}
\end{oefening2}

\begin{oefening2}Juist of fout? Verklaar!
\begin{enumerate}
\item $\ds{\frac{k!}{(k+2)!(k-l-1)!}=\frac{k!(k-l)}{(k+2)!(k-l)!}}$\quad voor alle $k,l\in\N$ met $k>l$.\hfill juist/fout
\item $\ds{\sum^{n}_{k=1}(2+k)=2n+\sum^{n}_{k=1}k}$\quad voor alle $n\in\Nnul$.\hfill juist/fout
\item $\ds{{5\choose1}={5\choose4}}$\hfill juist/fout
\item $\ds{{2\choose0}={2\choose2}}$\hfill juist/fout
\item $\ds{(p+4)!(p+3)=(p+3)!}$\quad voor alle $p\in\N$.\hfill juist/fout
\item $\ds{{k\choose l}={k\choose{k-l}}}$\quad voor alle $k,l\in\N$ met $k\geqslant l$.\hfill juist/fout
\item $\ds{\sum^{n}_{k=1}\left(ak^2+b\right)=a\sum^{n}_{k=1}k^{2}+b}$\quad voor alle $a,b\in\R$ en alle $n\in\Nnul$.\hfill juist/fout
\item $\ds{\sum^{n+1}_{l=1}(l+1)^{2}=\sum^{n}_{l=1}(l+1)^{2}+(n+1)^{2}}$ \quad voor alle $n\in\Nnul$.\hfill juist/fout
\item $\ds{\sum^{n}_{k=1}a_k^{2}=\left(\sum^{n}_{k=1}a_k\right)^{2}}$ \quad voor alle $n\in\Nnul$ en alle $a_1,\ldots,a_n\in\R$.\hfill juist/fout
\end{enumerate}
\end{oefening2}


\Closesolutionfile{ans}
\section{Oplossingen}
\setlength{\parskip}{0pt}
\input{ansM02}

\clearpage{\pagestyle{empty}\cleardoublepage}
\end{document}
