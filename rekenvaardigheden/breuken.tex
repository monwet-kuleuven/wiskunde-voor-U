\documentclass[numbers]{ximera}

%
% Opties (enkel te gebruiken voor lokale testen; NOOIT committen met opties)
%
%\pdfOnly{\providecommand\showtodonotes{}}
%\newcommand\nouitweiding{}

%
% preamble: ALLE usepackage moeten daarin !
%
%
% copied from https://github.com/mooculus/calculus
%
\usepackage[utf8]{inputenc}


\graphicspath{
	{./}
	{goniometrie/}
}


%\usepackage{todonotes}
%\usepackage{mathtools} %% Required for wide table Curl and Greens
%\usepackage{cuted} %% Required for wide table Curl and Greens
\newcommand{\todo}{}

% Font niet (correct?) geinstalleerd in MikTeX?
%\usepackage{esint} % for \oiint
%\ifxake%%https://math.meta.stackexchange.com/questions/9973/how-do-you-render-a-closed-surface-double-integral
%\renewcommand{\oiint}{{\large\bigcirc}\kern-1.56em\iint}
%\fi


\newcommand{\mooculus}{\textsf{\textbf{MOOC}\textnormal{\textsf{ULUS}}}}

\usepackage{tkz-euclide}\usepackage{tikz}
\usepackage{tikz-cd}
\usetikzlibrary{arrows}
\tikzset{>=stealth,commutative diagrams/.cd,
  arrow style=tikz,diagrams={>=stealth}} %% cool arrow head
\tikzset{shorten <>/.style={ shorten >=#1, shorten <=#1 } } %% allows shorter vectors

\usetikzlibrary{backgrounds} %% for boxes around graphs
\usetikzlibrary{shapes,positioning}  %% Clouds and stars
\usetikzlibrary{matrix} %% for matrix
\usepgfplotslibrary{polar} %% for polar plots
\usepgfplotslibrary{fillbetween} %% to shade area between curves in TikZ
\usetkzobj{all}
\usepackage[makeroom]{cancel} %% for strike outs
%\usepackage{mathtools} %% for pretty underbrace % Breaks Ximera
%\usepackage{multicol}
\usepackage{pgffor} %% required for integral for loops



%% http://tex.stackexchange.com/questions/66490/drawing-a-tikz-arc-specifying-the-center
%% Draws beach ball
\tikzset{pics/carc/.style args={#1:#2:#3}{code={\draw[pic actions] (#1:#3) arc(#1:#2:#3);}}}



\usepackage{array}
\setlength{\extrarowheight}{+.1cm}
\newdimen\digitwidth
\settowidth\digitwidth{9}
\def\divrule#1#2{
\noalign{\moveright#1\digitwidth
\vbox{\hrule width#2\digitwidth}}}





\newcommand{\RR}{\mathbb R}
\newcommand{\R}{\mathbb R}
\newcommand{\N}{\mathbb N}
\newcommand{\Z}{\mathbb Z}

\newcommand{\sagemath}{\textsf{SageMath}}


%\renewcommand{\d}{\,d\!}
\renewcommand{\d}{\mathop{}\!d}
\newcommand{\dd}[2][]{\frac{\d #1}{\d #2}}
\newcommand{\pp}[2][]{\frac{\partial #1}{\partial #2}}
\renewcommand{\l}{\ell}
\newcommand{\ddx}{\frac{d}{\d x}}

\newcommand{\zeroOverZero}{\ensuremath{\boldsymbol{\tfrac{0}{0}}}}
\newcommand{\inftyOverInfty}{\ensuremath{\boldsymbol{\tfrac{\infty}{\infty}}}}
\newcommand{\zeroOverInfty}{\ensuremath{\boldsymbol{\tfrac{0}{\infty}}}}
\newcommand{\zeroTimesInfty}{\ensuremath{\small\boldsymbol{0\cdot \infty}}}
\newcommand{\inftyMinusInfty}{\ensuremath{\small\boldsymbol{\infty - \infty}}}
\newcommand{\oneToInfty}{\ensuremath{\boldsymbol{1^\infty}}}
\newcommand{\zeroToZero}{\ensuremath{\boldsymbol{0^0}}}
\newcommand{\inftyToZero}{\ensuremath{\boldsymbol{\infty^0}}}



\newcommand{\numOverZero}{\ensuremath{\boldsymbol{\tfrac{\#}{0}}}}
\newcommand{\dfn}{\textbf}
%\newcommand{\unit}{\,\mathrm}
\newcommand{\unit}{\mathop{}\!\mathrm}
\newcommand{\eval}[1]{\bigg[ #1 \bigg]}
\newcommand{\seq}[1]{\left( #1 \right)}
\renewcommand{\epsilon}{\varepsilon}
\renewcommand{\phi}{\varphi}


\renewcommand{\iff}{\Leftrightarrow}

\DeclareMathOperator{\arccot}{arccot}
\DeclareMathOperator{\arcsec}{arcsec}
\DeclareMathOperator{\arccsc}{arccsc}
\DeclareMathOperator{\si}{Si}
\DeclareMathOperator{\scal}{scal}
\DeclareMathOperator{\sign}{sign}


%% \newcommand{\tightoverset}[2]{% for arrow vec
%%   \mathop{#2}\limits^{\vbox to -.5ex{\kern-0.75ex\hbox{$#1$}\vss}}}
\newcommand{\arrowvec}[1]{{\overset{\rightharpoonup}{#1}}}
%\renewcommand{\vec}[1]{\arrowvec{\mathbf{#1}}}
\renewcommand{\vec}[1]{{\overset{\boldsymbol{\rightharpoonup}}{\mathbf{#1}}}\hspace{0in}}

\newcommand{\point}[1]{\left(#1\right)} %this allows \vector{ to be changed to \vector{ with a quick find and replace
\newcommand{\pt}[1]{\mathbf{#1}} %this allows \vec{ to be changed to \vec{ with a quick find and replace
\newcommand{\Lim}[2]{\lim_{\point{#1} \to \point{#2}}} %Bart, I changed this to point since I want to use it.  It runs through both of the exercise and exerciseE files in limits section, which is why it was in each document to start with.

\DeclareMathOperator{\proj}{\mathbf{proj}}
\newcommand{\veci}{{\boldsymbol{\hat{\imath}}}}
\newcommand{\vecj}{{\boldsymbol{\hat{\jmath}}}}
\newcommand{\veck}{{\boldsymbol{\hat{k}}}}
\newcommand{\vecl}{\vec{\boldsymbol{\l}}}
\newcommand{\uvec}[1]{\mathbf{\hat{#1}}}
\newcommand{\utan}{\mathbf{\hat{t}}}
\newcommand{\unormal}{\mathbf{\hat{n}}}
\newcommand{\ubinormal}{\mathbf{\hat{b}}}

\newcommand{\dotp}{\bullet}
\newcommand{\cross}{\boldsymbol\times}
\newcommand{\grad}{\boldsymbol\nabla}
\newcommand{\divergence}{\grad\dotp}
\newcommand{\curl}{\grad\cross}
%\DeclareMathOperator{\divergence}{divergence}
%\DeclareMathOperator{\curl}[1]{\grad\cross #1}
\newcommand{\lto}{\mathop{\longrightarrow\,}\limits}

\renewcommand{\bar}{\overline}

\colorlet{textColor}{black}
\colorlet{background}{white}
\colorlet{penColor}{blue!50!black} % Color of a curve in a plot
\colorlet{penColor2}{red!50!black}% Color of a curve in a plot
\colorlet{penColor3}{red!50!blue} % Color of a curve in a plot
\colorlet{penColor4}{green!50!black} % Color of a curve in a plot
\colorlet{penColor5}{orange!80!black} % Color of a curve in a plot
\colorlet{penColor6}{yellow!70!black} % Color of a curve in a plot
\colorlet{fill1}{penColor!20} % Color of fill in a plot
\colorlet{fill2}{penColor2!20} % Color of fill in a plot
\colorlet{fillp}{fill1} % Color of positive area
\colorlet{filln}{penColor2!20} % Color of negative area
\colorlet{fill3}{penColor3!20} % Fill
\colorlet{fill4}{penColor4!20} % Fill
\colorlet{fill5}{penColor5!20} % Fill
\colorlet{gridColor}{gray!50} % Color of grid in a plot

\newcommand{\surfaceColor}{violet}
\newcommand{\surfaceColorTwo}{redyellow}
\newcommand{\sliceColor}{greenyellow}




\pgfmathdeclarefunction{gauss}{2}{% gives gaussian
  \pgfmathparse{1/(#2*sqrt(2*pi))*exp(-((x-#1)^2)/(2*#2^2))}%
}


%%%%%%%%%%%%%
%% Vectors
%%%%%%%%%%%%%

%% Simple horiz vectors
\renewcommand{\vector}[1]{\left\langle #1\right\rangle}


%% %% Complex Horiz Vectors with angle brackets
%% \makeatletter
%% \renewcommand{\vector}[2][ , ]{\left\langle%
%%   \def\nextitem{\def\nextitem{#1}}%
%%   \@for \el:=#2\do{\nextitem\el}\right\rangle%
%% }
%% \makeatother

%% %% Vertical Vectors
%% \def\vector#1{\begin{bmatrix}\vecListA#1,,\end{bmatrix}}
%% \def\vecListA#1,{\if,#1,\else #1\cr \expandafter \vecListA \fi}

%%%%%%%%%%%%%
%% End of vectors
%%%%%%%%%%%%%

%\newcommand{\fullwidth}{}
%\newcommand{\normalwidth}{}



%% makes a snazzy t-chart for evaluating functions
%\newenvironment{tchart}{\rowcolors{2}{}{background!90!textColor}\array}{\endarray}

%%This is to help with formatting on future title pages.
\newenvironment{sectionOutcomes}{}{}



%% Flowchart stuff
%\tikzstyle{startstop} = [rectangle, rounded corners, minimum width=3cm, minimum height=1cm,text centered, draw=black]
%\tikzstyle{question} = [rectangle, minimum width=3cm, minimum height=1cm, text centered, draw=black]
%\tikzstyle{decision} = [trapezium, trapezium left angle=70, trapezium right angle=110, minimum width=3cm, minimum height=1cm, text centered, draw=black]
%\tikzstyle{question} = [rectangle, rounded corners, minimum width=3cm, minimum height=1cm,text centered, draw=black]
%\tikzstyle{process} = [rectangle, minimum width=3cm, minimum height=1cm, text centered, draw=black]
%\tikzstyle{decision} = [trapezium, trapezium left angle=70, trapezium right angle=110, minimum width=3cm, minimum height=1cm, text centered, draw=black]



% Specifieke settings voor dit deel:
\author{Zomercursus KU Leuven}
\outcome{Algebraïsch kunnen rekenen met breuken.}
\title[Rekenvaardigheden:]{Breuken}

%
% Begin document
%
\begin{document}
\begin{abstract}
	Gebroken verhoudingen.
\end{abstract}
\maketitle

In dit deel herhalen we heel bondig de definities en rekenregels voor
breuken. Deze zijn heel eenvoudig, maar de uitdaging is deze rekenregels automatisch, dus zonder nadenken te kunnen toepassen in eenvoudige situaties. In complexere situaties moet je dat kunnen met een minimale inspanning, en in alle gevallen zonder fouten (of toch in \textit{bijna} alle gevallen \Smiley).

\subsection{Definities en rekenregels voor breuken}
We definiëren een \textit{breuk} als een uitdrukking van de vorm $\frac{a}{b}$ met $a,b\in\R$, met $b\neq0$. 

%\\We noemen $a$ de \textit{teller}, en $b$ de \textit{noemer} van de breuk, en als $a\neq0$ is $\frac ba$ de \textit{omgekeerde} van $\frac ab$ 

Merk op dat uitdrukkingen van de vorm $\frac a0$ niet gedefinieerd zijn (en dus \textit{niet bestaan}).  

Met elke breuk $\frac ab$ associëren we een getal, namelijk het resultaat van de deling van $a$ door $b$.

Bijvoorbeeld: $\frac42=\frac21=2$; \quad $\frac12=0,5$; \quad $\frac13=0,33\dots$ en $\frac{\pi}{2}=1,57\dots$. 

Getallen die we zo bekomen met $a,b\in\Z$ noemen we \textit{rationale getallen}, en de verzameling van alle rationale getallen noteren we met $\Q$.


Voor breuken definiëren we volgende regels voor gelijkheid, optelling en vermenigvuldiging: 

\begin{definition} (Definitie gelijkheid en basisbewerkingen breuken)
	
Zij $a,b,c,d\in \R$. 

Een uitdrukking van de vorm $\frac ab$ met $b\neq0$ noemen we een \textit{breuk} met \textit{teller} $a$ en \textit{noemer} $b$. Als ook $a\neq0$ noemen we $\frac ba$ de \textit{omgekeerde} van $\frac ab$. 

Zodra de noemers verschillend zijn van $0$, geldt:

\begin{align*}
		\important{\frac{a}{b} = \frac{c}{d}}  &\iff \important{ad = bc} 
			& \textrm{gelijkheid van breuken} \\
		\frac{a}{b}+\frac{c}{d} &\perdef \frac{ad+cb}{bd} 
			& \textrm{ optelling (op gelijke noemer brengen)} \\
		\frac{a}{b} \cdot \frac{c}{d} &\perdef \frac{a\cdot c}{b\cdot d} 
			& \textrm{vermenigvuldiging (teller $\times$ teller, noemer $\times$ noemer) }\\%(product tellers $/$ product noemers)} \\
		\frac{a}{b} : \frac{c}{d} &\perdef \frac ab \cdot \frac dc = \frac{a\cdot d}{b\cdot c} 
		& \textrm{deling (maal omgekeerde)} \\
\end{align*}

\end{definition}

\begin{uitweiding} (Subtiele subtiliteiten over breuken)
\begin{expandable}

		In deze definitie van breuken zitten allerlei subtiliteiten verstopt, die in vele cursussen terecht of ten onrechte worden genegeerd. 
		
		Zo is er een noodzakelijk maar subtiel onderscheid tussen een \textit{breuk} en een \textit{getal}. Inderdaad, als we willen spreken van 'de teller van een breuk', dan moeten noodzakelijk de breuken $\frac 12$ en $\frac 24$ verschillend zijn, want ze hebben verschillende tellers! Wie toch zou durven beweren dat \textit{gelijke} dingen (de 'breuken' $\frac 12$ en $\frac 24$)  \textit{verschillende} eigenschappen kunnen hebben (namelijk de teller van $\frac 12$ is 1 maar de teller van 'dezelfde breuk' $\frac24$ zou 2 zijn) verdrinkt onmiddellijk in een diep filosofisch moeras. Als \textit{getallen} zijn de twee breuken natuurlijk wel gelijk (namelijk gelijk aan het getal $0,5$).
		
		In bovenstaande definitie zit dus een onnauwkeurigheid: de \textit{breuken} $\frac ab$ en $\frac cd$ zijn eigenlijk verschillend van zodra $a\neq c$ of $b\neq d$, maar de \textit{getallen} die overeenkomen met de twee breuken zijn gelijk van zodra $ad = bc$.
		
		Dit is een voorbeeld van een typisch wiskundige situatie: als iets wordt gedefinieerd is het van cruciaal belang om niet alleen vast te leggen wat dat iets precies is, maar ook wanneer twee van die 'ietsen' aan elkaar gelijk zijn. En als men het begrip 'de teller van een breuk' een goede betekenis wil geven, dan \textit{moet} men onderscheid maken tussen $\frac 12$ en $\frac 24$, en dan kunnen ze dus \textit{niet} aan elkaar gelijk zijn. 
		
		De oplossing van dergelijke subtiliteiten is meestal dat we via 'misbruik van notatie' geen onderscheid meer maken tussen de breuk $\frac 12$ en het (unieke) getal $\frac 12 = 0.5$ dat met deze breuk overeenkomt. En natuurlijk levert dat in de praktijk (bijna...) nooit problemen op. Integendeel, telkens wel het onderscheid maken zou de zaken erg compliceren. 
		
		Filosofische conclusie: misschien zijn sommige wiskundige definities en eigenschappen eigenlijk enkel begrijpbaar voor twee soorten mensen: enerzijds diegenen die er (dikwijls terecht!) snel overlezen, inzien waarover het gaat en zich geen verdere vragen stellen. Anderzijds heb je diegenen die alles in uiterste detail willen begrijpen, en na verloop van tijd alle subtiliteiten doorhebben. Maar misschien vallen vele studerenden in de tussencategorie: ze proberen de definities zo goed mogelijk te begrijpen maar lopen tegen één of andere moeilijkheid aan: hoe kan het dat $\frac 12$ en $\frac 24$ \textit{hetzelfde} zijn, terwijl ze toch een \textit{verschillende} teller hebben. En, in de handboeken die ze gebruiken worden dat soort moeilijkheden niet uitgelegd...
\end{expandable}
\end{uitweiding}

%\begin{example} (Triviale voorbeelden)
	
\begin{example}$\frac23 + \frac13 = \answer[given]{1}$   
			\begin{feedback} want $\frac23 + \frac13 = \frac{1+2}{3} = \frac33 = 1$ \end{feedback}\end{example}
\begin{example}$\frac35 + \frac12 = \answer[given]{1.1}$
			\begin{feedback} want $\frac35 + \frac12 = \frac{3\cdot 2+1\cdot 5}{5\cdot 2} = \frac{11}{10}=1,1$ \end{feedback}\end{example}
\begin{example}$\frac{1+1}{1} = \answer[given]{2}$
			\begin{feedback} want $\frac{1+1}{1} = \frac21 = 2, \text{ of ook }  \frac{1+1}{1} = \frac11+\frac11 = 2$ \end{feedback}\end{example}		
\begin{example}$\frac{1}{1+1} = \answer[given]{0.5}$
			\begin{feedback} en \textit{niet} $\frac{1}{1+1} = \frac11 + \frac 11 = 1+ 1 = 2$ \end{feedback}\end{example}
%\end{example}


\begin{proposition} (Eenvoudige maar belangrijke gevolgen)
	
	Zij $a,b,c,d\in \R$. Zodra de noemers verschillend zijn van $0$, geldt:
{
\savebox\strutbox{$\vphantom{\dfrac{1^2}{1^2}^n}$}   % hack!
\begin{align*}
		\frac{a\cdot c}{a\cdot d} &= \frac{c}{d} 
			& \text{breuken vereenvoudigen} \hfill\text{(factor $a$ wegdelen)} \\
		a\cdot \frac{c}{d}   &= \frac{a\cdot c}{d} 
			& \text{getal maal breuk}\hfill \text{(maal in teller)} \\
		\frac{a}{c}+\frac{b}{c}  &= \frac{a+b}{c} 
			& \text{breuken optellen}\hfill\text{(geval gelijke noemers)} \\
		\frac{a+b}{c} &= \frac{a}{c}+\frac{b}{c}  
			& \text{breuken splitsen} \hfill\text{(zelfde formule als vorige!)}\\
		\frac{\frac{a}{b}}{\ c\ } &= \frac ab \cdot \frac 1c = \frac{a}{b\cdot c} 
		& \text{breuk gedeeld door getal}\hfill\text{(maal in noemer)}\\
		\frac{a}{\ \frac{b}{c}\ } &= a\cdot \frac cb = \frac{a\cdot c}{b}
		& \text{getal gedeeld door breuk}\hfill\text{(maal omgekeerde)}\\
		\frac{\frac{a}{b}}{\ \frac{c}{d}\ } &= \frac ab \cdot \frac dc =  \frac{a\cdot d}{b\cdot c}
		& \text{breuk gedeeld door breuk}\hfill\text{(maal omgekeerde)}\\
		\frac{\frac{a}{d}}{\ \frac{c}{d}\ } &= \frac ad \cdot \frac dc =  \frac{a}{c}
		& \text{gelijke noemers wegwerken}\hfill\text{($\frac 1d$ wegdelen)}\\
		\end{align*}
}
\end{proposition}


Al deze eigenschappen zijn eenvoudige gevolgen van de vorige definitie. Het is erg belangrijk de evidentie van deze eigenschappen in te zien, en ze vlot te kunnen toepassen op zowel numerieke als symbolische uitdrukkingen.

Het heeft geen zin deze eigenschappen \textit{van buiten} te leren, je moet ze \textit{van binnen} kennen, dus door en door kennen. Dat is allicht al het geval, maar indien niet moet je hierover (erg veel) oefeningen maken.


\begin{uitweiding}
%\begin{example}
	Waarom is $\frac{a}{c}+\frac{b}{c} = \frac{a+b}{c}$ ?
	\begin{expandable}
	\begin{explanation} \ 
		
		De meest eenvoudige uitleg: het is gewoon het optellen van breuken die dezelfde noemer hebben: dat is makkelijk, waarom moet daar nog verder aandacht aan worden besteed.
		
		Een opmerking over die meest eenvoudige uitleg: dat is een correct, en zeer terecht inzicht. Maar, het is enkel correct en terecht omdat je al vertrouwd \textit{was} met breuken. Als dit de eerste keer zou zijn geweest dat je het begrip breuk was tegengekomen, zou deze uitleg misschien \textit{niet} hebben volstaan.		
		
		Een meer wiskundig geformuleerd \textit{bewijs}: per definitie van optelling van breuken geldt dat 
		$$ \frac{a}{c}+\frac{b}{c} = \frac{ac+bc}{c\cdot c}$$
		en dat is wegens distributiviteit en de regel over het vereenvoudigen van breuken
		$$ \frac{ac+bc}{c\cdot c} = \frac{\cancel{c}(a+b)}{\cancel{c} \cdot c} = \frac{a+b}{c}$$
		Dit bewijst dat $\frac{a}{c}+\frac{b}{c} = \frac{a+b}{c}$.
	\end{explanation}
	
\end{expandable}
\end{uitweiding}
%\end{example}


\begin{example} (Numeriek rekenen met breuken)
	
	\begin{enumerate}
		\item Breuk splitsen:\ \ $\ds{\frac{120+74}{2}=\frac{120}{2}+\frac{74}{2}=60+37=97}$
		\item Breuken optellen:\ \ $\ds{\frac{1}{8}+\frac{5}{6}=\frac{1 \cdot 6+5 \cdot 8}{48}=\frac{46}{48}}$
		\item Breuk vereenvoudigen:\ \ $\ds{\frac{125}{50}=\frac{25 \cdot 5}{25 \cdot 2}=\frac{5}{2}}$
		\item Rekenen met meerdere breukstrepen:\ \ $\ds{\frac{\frac{80}{2}}{10}=\frac{80}{2\cdot10}=\frac{80}{20}=4}$
	\end{enumerate}
\end{example}

\begin{example} (Symbolisch rekenen met breuken)

%	($x,y,p,q\in\R$)  % niet echt nodig: x,y,p,k mogen ook gewoon 'letters' zijn
%    (hoewel we natuurlijk geen breuken-met-letters gedefinieerd hebben )
\todo{ Aanpassen en/of oplossingen voorzien}
	\begin{enumerate}
		\item Rekenen met meerdere breukstrepen:\\
		Stel $x,y \neq 0$, dan is $\ds{\frac{x+y}{\frac{x}{y}}=\frac{(x+y)y}{x}}$
		\item Rekenen met meerdere breukstrepen:\\
		Stel $x,y \neq 0$, dan is $\ds{\frac{\frac{x+y}{x}}{\ y\ }=\frac{x+y}{xy}}$
		\item Breuk vereenvoudigen --- verschil van twee kwadraten:\\
		Stel $p-q \neq 0$, dan is $\ds{\frac{p^{2}-q^{2}}{p-q}=\frac{(p+q)(p-q)}{p-q}=p+q}$
		\item Breuk vereenvoudigen, rekenen met meerdere breukstrepen --- factor afzonderen:\\
		Stel $p,q \neq 0$, dan is $\ds{\frac{\frac{p+qp}{pq^{2}}}{\frac{p^{2}}{q}}=\frac{\frac{p(1+q)}{pq^{2}}}{\frac{p^{2}}{q}}=
			\frac{\frac{1+q}{q^{2}}}{\frac{p^{2}}{q}}=\frac{(1+q)q}{q^{2}p^2}=\frac{1+q}{qp^{2}}}$
		\item Breuken splitsen, rekenen met meerdere breukstrepen:\\
		Stel $x \neq 0$, dan is $\ds{\frac{-1}{1-\frac{x+1}{x}}=\frac{-1}{1-\left(1+\frac{1}{x}\right)}=\frac{-1}{1-1-\frac{1}{x}}=\frac{-1}{\frac{-1}{x}}=\frac{-1x}{-1}=x}$
	\end{enumerate}
\end{example}

\begin{exercise} (Basisoefeningen breuken)\ 
	
    Oefeningen toe te voegen ...!
    \todo{Breuken: oefeningen toevoegen}
\end{exercise}

\begin{exercise} (Uitdagingsoefeningen breuken)\ 
	
	Oefeningen toe te voegen ...! (Uitdaging is mss een verkeerd woord...?)
\end{exercise}


\subsection{Samenvatting}

\todo{Samenvatting breuken nakijken}
\begin{proposition}[Samenvatting breuken]\label{breukensamenvatting} \ 
	
	Zij $a,b,c,d\in \R$. Er geldt:
	\begin{align*}
		\frac{a}{b}+\frac{c}{d}& =\frac{ad+cb}{bd} & \text{indien } b,d \neq 0 \\ \\
		\dfrac{\frac{a}{b}}{\ c\ }&=\frac{a}{bc}  & \text{indien }b,c \neq 0 \\ \\
		\frac{a}{\ \frac{b}{c}\ }&=\frac{ac}{b}  & \text{indien } b,c \neq 0 \\ \\
		%\item $\ds{\frac{\frac{a}{b}}{\ \frac{c}{d}\ }}=\ds{\frac{ad}{bc}}$ \hspace{3mm}   indien $b,c,d \neq 0$
	\end{align*}
\end{proposition}


\end{document}
