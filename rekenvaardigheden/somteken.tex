%%
% ximera activiteit
% Copyrigth
%
\documentclass{ximera}
%
% Opties (enkel te gebruiken voor lokale testen; NOOIT committen met opties)
%
%\pdfOnly{\providecommand\showtodonotes{}}
%\newcommand\xmnouitweiding{}


%
% copied from https://github.com/mooculus/calculus
%
\usepackage[utf8]{inputenc}


\graphicspath{
	{./}
	{goniometrie/}
}


%\usepackage{todonotes}
%\usepackage{mathtools} %% Required for wide table Curl and Greens
%\usepackage{cuted} %% Required for wide table Curl and Greens
\newcommand{\todo}{}

% Font niet (correct?) geinstalleerd in MikTeX?
%\usepackage{esint} % for \oiint
%\ifxake%%https://math.meta.stackexchange.com/questions/9973/how-do-you-render-a-closed-surface-double-integral
%\renewcommand{\oiint}{{\large\bigcirc}\kern-1.56em\iint}
%\fi


\newcommand{\mooculus}{\textsf{\textbf{MOOC}\textnormal{\textsf{ULUS}}}}

\usepackage{tkz-euclide}\usepackage{tikz}
\usepackage{tikz-cd}
\usetikzlibrary{arrows}
\tikzset{>=stealth,commutative diagrams/.cd,
  arrow style=tikz,diagrams={>=stealth}} %% cool arrow head
\tikzset{shorten <>/.style={ shorten >=#1, shorten <=#1 } } %% allows shorter vectors

\usetikzlibrary{backgrounds} %% for boxes around graphs
\usetikzlibrary{shapes,positioning}  %% Clouds and stars
\usetikzlibrary{matrix} %% for matrix
\usepgfplotslibrary{polar} %% for polar plots
\usepgfplotslibrary{fillbetween} %% to shade area between curves in TikZ
\usetkzobj{all}
\usepackage[makeroom]{cancel} %% for strike outs
%\usepackage{mathtools} %% for pretty underbrace % Breaks Ximera
%\usepackage{multicol}
\usepackage{pgffor} %% required for integral for loops



%% http://tex.stackexchange.com/questions/66490/drawing-a-tikz-arc-specifying-the-center
%% Draws beach ball
\tikzset{pics/carc/.style args={#1:#2:#3}{code={\draw[pic actions] (#1:#3) arc(#1:#2:#3);}}}



\usepackage{array}
\setlength{\extrarowheight}{+.1cm}
\newdimen\digitwidth
\settowidth\digitwidth{9}
\def\divrule#1#2{
\noalign{\moveright#1\digitwidth
\vbox{\hrule width#2\digitwidth}}}





\newcommand{\RR}{\mathbb R}
\newcommand{\R}{\mathbb R}
\newcommand{\N}{\mathbb N}
\newcommand{\Z}{\mathbb Z}

\newcommand{\sagemath}{\textsf{SageMath}}


%\renewcommand{\d}{\,d\!}
\renewcommand{\d}{\mathop{}\!d}
\newcommand{\dd}[2][]{\frac{\d #1}{\d #2}}
\newcommand{\pp}[2][]{\frac{\partial #1}{\partial #2}}
\renewcommand{\l}{\ell}
\newcommand{\ddx}{\frac{d}{\d x}}

\newcommand{\zeroOverZero}{\ensuremath{\boldsymbol{\tfrac{0}{0}}}}
\newcommand{\inftyOverInfty}{\ensuremath{\boldsymbol{\tfrac{\infty}{\infty}}}}
\newcommand{\zeroOverInfty}{\ensuremath{\boldsymbol{\tfrac{0}{\infty}}}}
\newcommand{\zeroTimesInfty}{\ensuremath{\small\boldsymbol{0\cdot \infty}}}
\newcommand{\inftyMinusInfty}{\ensuremath{\small\boldsymbol{\infty - \infty}}}
\newcommand{\oneToInfty}{\ensuremath{\boldsymbol{1^\infty}}}
\newcommand{\zeroToZero}{\ensuremath{\boldsymbol{0^0}}}
\newcommand{\inftyToZero}{\ensuremath{\boldsymbol{\infty^0}}}



\newcommand{\numOverZero}{\ensuremath{\boldsymbol{\tfrac{\#}{0}}}}
\newcommand{\dfn}{\textbf}
%\newcommand{\unit}{\,\mathrm}
\newcommand{\unit}{\mathop{}\!\mathrm}
\newcommand{\eval}[1]{\bigg[ #1 \bigg]}
\newcommand{\seq}[1]{\left( #1 \right)}
\renewcommand{\epsilon}{\varepsilon}
\renewcommand{\phi}{\varphi}


\renewcommand{\iff}{\Leftrightarrow}

\DeclareMathOperator{\arccot}{arccot}
\DeclareMathOperator{\arcsec}{arcsec}
\DeclareMathOperator{\arccsc}{arccsc}
\DeclareMathOperator{\si}{Si}
\DeclareMathOperator{\scal}{scal}
\DeclareMathOperator{\sign}{sign}


%% \newcommand{\tightoverset}[2]{% for arrow vec
%%   \mathop{#2}\limits^{\vbox to -.5ex{\kern-0.75ex\hbox{$#1$}\vss}}}
\newcommand{\arrowvec}[1]{{\overset{\rightharpoonup}{#1}}}
%\renewcommand{\vec}[1]{\arrowvec{\mathbf{#1}}}
\renewcommand{\vec}[1]{{\overset{\boldsymbol{\rightharpoonup}}{\mathbf{#1}}}\hspace{0in}}

\newcommand{\point}[1]{\left(#1\right)} %this allows \vector{ to be changed to \vector{ with a quick find and replace
\newcommand{\pt}[1]{\mathbf{#1}} %this allows \vec{ to be changed to \vec{ with a quick find and replace
\newcommand{\Lim}[2]{\lim_{\point{#1} \to \point{#2}}} %Bart, I changed this to point since I want to use it.  It runs through both of the exercise and exerciseE files in limits section, which is why it was in each document to start with.

\DeclareMathOperator{\proj}{\mathbf{proj}}
\newcommand{\veci}{{\boldsymbol{\hat{\imath}}}}
\newcommand{\vecj}{{\boldsymbol{\hat{\jmath}}}}
\newcommand{\veck}{{\boldsymbol{\hat{k}}}}
\newcommand{\vecl}{\vec{\boldsymbol{\l}}}
\newcommand{\uvec}[1]{\mathbf{\hat{#1}}}
\newcommand{\utan}{\mathbf{\hat{t}}}
\newcommand{\unormal}{\mathbf{\hat{n}}}
\newcommand{\ubinormal}{\mathbf{\hat{b}}}

\newcommand{\dotp}{\bullet}
\newcommand{\cross}{\boldsymbol\times}
\newcommand{\grad}{\boldsymbol\nabla}
\newcommand{\divergence}{\grad\dotp}
\newcommand{\curl}{\grad\cross}
%\DeclareMathOperator{\divergence}{divergence}
%\DeclareMathOperator{\curl}[1]{\grad\cross #1}
\newcommand{\lto}{\mathop{\longrightarrow\,}\limits}

\renewcommand{\bar}{\overline}

\colorlet{textColor}{black}
\colorlet{background}{white}
\colorlet{penColor}{blue!50!black} % Color of a curve in a plot
\colorlet{penColor2}{red!50!black}% Color of a curve in a plot
\colorlet{penColor3}{red!50!blue} % Color of a curve in a plot
\colorlet{penColor4}{green!50!black} % Color of a curve in a plot
\colorlet{penColor5}{orange!80!black} % Color of a curve in a plot
\colorlet{penColor6}{yellow!70!black} % Color of a curve in a plot
\colorlet{fill1}{penColor!20} % Color of fill in a plot
\colorlet{fill2}{penColor2!20} % Color of fill in a plot
\colorlet{fillp}{fill1} % Color of positive area
\colorlet{filln}{penColor2!20} % Color of negative area
\colorlet{fill3}{penColor3!20} % Fill
\colorlet{fill4}{penColor4!20} % Fill
\colorlet{fill5}{penColor5!20} % Fill
\colorlet{gridColor}{gray!50} % Color of grid in a plot

\newcommand{\surfaceColor}{violet}
\newcommand{\surfaceColorTwo}{redyellow}
\newcommand{\sliceColor}{greenyellow}




\pgfmathdeclarefunction{gauss}{2}{% gives gaussian
  \pgfmathparse{1/(#2*sqrt(2*pi))*exp(-((x-#1)^2)/(2*#2^2))}%
}


%%%%%%%%%%%%%
%% Vectors
%%%%%%%%%%%%%

%% Simple horiz vectors
\renewcommand{\vector}[1]{\left\langle #1\right\rangle}


%% %% Complex Horiz Vectors with angle brackets
%% \makeatletter
%% \renewcommand{\vector}[2][ , ]{\left\langle%
%%   \def\nextitem{\def\nextitem{#1}}%
%%   \@for \el:=#2\do{\nextitem\el}\right\rangle%
%% }
%% \makeatother

%% %% Vertical Vectors
%% \def\vector#1{\begin{bmatrix}\vecListA#1,,\end{bmatrix}}
%% \def\vecListA#1,{\if,#1,\else #1\cr \expandafter \vecListA \fi}

%%%%%%%%%%%%%
%% End of vectors
%%%%%%%%%%%%%

%\newcommand{\fullwidth}{}
%\newcommand{\normalwidth}{}



%% makes a snazzy t-chart for evaluating functions
%\newenvironment{tchart}{\rowcolors{2}{}{background!90!textColor}\array}{\endarray}

%%This is to help with formatting on future title pages.
\newenvironment{sectionOutcomes}{}{}



%% Flowchart stuff
%\tikzstyle{startstop} = [rectangle, rounded corners, minimum width=3cm, minimum height=1cm,text centered, draw=black]
%\tikzstyle{question} = [rectangle, minimum width=3cm, minimum height=1cm, text centered, draw=black]
%\tikzstyle{decision} = [trapezium, trapezium left angle=70, trapezium right angle=110, minimum width=3cm, minimum height=1cm, text centered, draw=black]
%\tikzstyle{question} = [rectangle, rounded corners, minimum width=3cm, minimum height=1cm,text centered, draw=black]
%\tikzstyle{process} = [rectangle, minimum width=3cm, minimum height=1cm, text centered, draw=black]
%\tikzstyle{decision} = [trapezium, trapezium left angle=70, trapezium right angle=110, minimum width=3cm, minimum height=1cm, text centered, draw=black]


\begin{document}
    % Start specifieke settings:    
    \author{Zomercursus KU Leuven}
%    \outcome{}
    \xmtitle{Het somteken $\sum$}{}
    % Start inhoud ximera 


%\title {Het sommatieteken en de faculteit\\{\large(met als belangrijke toepassing binomiaalco\"effici\"enten)}}
%{\qrslidesA{https://set.kuleuven.be/zomercursussen/wiskunde/lesmateriaal/slides-sommatieteken-faculteit.pdf}}{}

\section*{Inleiding}
In deze module zullen we het gebruik van het
sommatieteken en de faculteitsoperatie herhalen en bespreken
a.d.h.v.~voorbeeldoefeningen en enkele toepassingen.

Het \emph{sommatieteken} \lq$\sum$\rq\ wordt veel gebruikt in de
context van rijen en reeksen. Ook in de statistiek, de
kansrekening en de discrete wiskunde is het sommatieteken niet weg
te denken. In deze takken van de wiskunde wordt veel gewerkt met
lange sommen van termen die \lq op een gelijkaardige manier zijn
opgebouwd\rq\footnote{Dit wordt verder uitgelegd.}\ en is er
behoefte aan een elegante schrijfwijze voor deze sommen en een
manier om er gemakkelijk mee te kunnen rekenen. Beschouw
bijvoorbeeld $100$ re\"ele getallen die we aanduiden met
$a_1,a_2,\ldots,a_{100}$ en stel dat we de som van deze getallen
nodig hebben. Deze som kunnen we aanduiden
met\[a_1+a_2+a_3+\cdots+a_{99}+a_{100},\]maar als we deze som vaak
nodig hebben en ermee willen rekenen, is dit geen handige notatie.
Het sommatieteken biedt een uitweg, met behulp van dit
sommatieteken wordt bovenstaande som verkort geschreven
als\[a_1+a_2+a_3+\cdots+a_{99}+a_{100}\
\overset{\textrm{notatie}}{=}\ \sum_{k=1}^{100}a_k.\]We lezen dit
als \lq de som van de termen $a_k$ waarbij (de index) $k$ loopt
van $1$ tot $100$\rq. Dit is een korte en duidelijke (zonder de
vage \lq$\ldots$\rq) schrijfwijze die alle noodzakelijke
informatie bevat. Bovendien laat deze schrijfwijze toe gemakkelijk
met de sommen te rekenen. Hiervoor bestaan de nodige rekenregels
(zie Sectie~\ref{M02_rekenregels}). Een som schrijven m.b.v.~het
sommatieteken werkt wel enkel als de verschillende termen een
gemeenschappelijke vorm hebben (in het voorbeeld hierboven zijn
alle termen van de vorm \lq$a_k$\rq). Deze gemeenschappelijke vorm
kan van velerlei aard zijn, zoals zal blijken in de voorbeelden en
oefeningen. De som $5+283-7/3+\pi/2+18-205-11$ kunnen we
bijvoorbeeld niet korter opschrijven, tenzij natuurlijk dat we de
termen een naam geven: $a_1=5$, $a_2=283$, $a_3=-7/3$, \ldots

De \emph{faculteitsoperatie} is een operatie op natuurlijke
getallen die vaak opduikt in telproblemen, kansberekeningen en
statistiek, maar ook in de analyse (taylorreeksen,
differentiaal- en integraalrekening). Als inleidend voorbeeld
bekijken we volgend telprobleem. Stel er wordt een renwedstrijd
gehouden met twaalf paarden. Bij de \emph{tierc\'{e}} kunnen
gokkers dan voorspellen in welke volgorde de paarden over de
eindstreep komen. Hoeveel mogelijke voorspellingen zijn er? Voor
het winnende paard zijn er precies $12$ mogelijkheden. Eens het
winnende paard vastligt, blijven er nog $11$ paarden over die als
tweede kunnen eindigen, er zijn dus nog $11$ mogelijkheden om het
tweede paard te \lq kiezen\rq. Voor het paard dat als derde over
de eindmeet zal komen, blijven er dan nog $10$ mogelijkheden over
enzovoort\ldots\ Het aantal mogelijke pronostieken voor de
tierc\'e is bijgevolg gelijk aan
$12\cdot11\cdot10\cdot9\cdot8\cdot7\cdot6\cdot5\cdot4\cdot3\cdot2\cdot1=479001600$.
Veralgemenen we dit concept tot een paardenwedstrijd met $n$
paarden ($n\in\Nnul$), bekomen we als aantal mogelijke
pronostieken
\begin{formula}\label{M02_nfac}
\[n(n-1)(n-2)(n-3)\cdots3\cdot2\cdot1.\]
\end{formula}
Een uitdrukking van de vorm (\ref{M02_nfac}) komt heel vaak voor in de
wiskunde en i.h.b.~in de hogergenoemde disciplines. Een verkorte
schrijfwijze is dus aangewezen: het product (\ref{M02_nfac}) noteren
we met $n!$ (lees: $n$-faculteit).

In de toepassingen bij dit pakket behandelen we  nog \emph{de
driehoek van Pascal} en het \emph{binomium van Newton}.
%
%Voor meer oefenmateriaal kan je ook terecht op
%\texttt{www.usolv-it.be} bij de topic \textcolor{red}{(?)}. Meer
%uitleg kan je eveneens vinden in volgende handboeken:
%\begin{itemize}
%\item Delta 5/6, Kansrekenen (6/8u)
%\item Van basis tot limiet 3/4, Statistiek
%\item Van basis tot limiet 5/6, Combinatoriek en Kansrekenen.
%\end{itemize}

\section{Het sommatieteken}
\subsection{Eenvoudige voorbeelden}
Het sommatieteken \lq$\sum$\rq\ is, zoals in de inleiding reeds
gezegd, een verkorte notatie voor een (lange) som van termen die
een gemeenschappelijke vorm hebben. In deze sectie bekijken we
enkele eenvoudige voorbeelden.
\begin{example}
Beschouw de som $x_0+x_1+x_2+x_3+x_4+x_5+x_6$. Dit is de som van
de termen~$x_k$, waarbij $k$ loopt van $0$ tot en met $6$ en kan
dus verkort genoteerd worden met
\[\sum^{6}_{k=0}x_k=x_0+x_1+x_2+x_3+x_4+x_5+x_6.\]
\end{example}
\begin{example}
Beschouw de som $2+3+4+5+6+7+8$. Dit is de som van de termen~$k$, waarbij $k$ loopt van $2$ tot en met
$8$ en kan dus verkort genoteerd worden met\[\sum^{8}_{k=2}k=
2+3+4+5+6+7+8=35.\] Deze schrijfwijze is natuurlijk niet uniek,
men kan dezelfde som ook noteren met bijvoorbeeld
\begin{eqnarray*}
   &&\sum^{7}_{k=1}(k+1)=2+3+4+5+6+7+8=35  \\
   \textrm{of met}&&\sum^{9}_{k=3}(k-1)=
2+3+4+5+6+7+8=35.
\end{eqnarray*}
\end{example}
\begin{example}
Beschouw de som $9+16+25+36+49+64+81$. Wat is de
gemeenschappelijke vorm van deze termen? Het zijn allemaal
kwadraten. We kunnen elke term dus schrijven als $k^2$, waarbij
$k$ de waarden $3$ t.e.m.~$9$ aanneemt. De som wordt bijgevolg
genoteerd met\[\sum^{9}_{k=3}k^2=9+16+25+36+49+64+81=280.\]
\end{example}

\subsection{Algemene definitie}
\begin{notation}[sommatieteken]
Zij $m,n\in\N$ met $m\leqslant n$ en $(a_k)_{k=m}^n$ een eindige
rij re\"{e}le getallen:
\begin{align*}
&a_m,a_{m+1},a_{m+2},\ldots,a_{k-1},a_k,a_{k+1},\ldots,a_{n-1},a_n\\
&\qquad\qquad\qquad\qquad\qquad\textrm{met}\ a_k\in\R\
\textrm{voor alle}\ k\in\{m,m+1,m+2,\ldots,n-1,n\}\subset\N.
\end{align*}
We voeren volgende notatie in:
\begin{equation}\label{M02_def}
\sum_{k=m}^n a_k\ =\ a_m+a_{m+1}+a_{m+2}+\cdots +a_{n-1}+a_n.
\end{equation}
Lees: de som van $a_k$ voor $k$ gaande van $m$ t.e.m.~$n$. Men
noemt $a_k$ de algemene term, $k$ de index, $m$ de ondergrens en
$n$ de bovengrens van de sommatie. Merk op dat
$\sum\limits_{k=m}^n a_k$ onafhankelijk is van de sommatie-index
$k$. De veranderlijke $k$ mag dan ook vervangen worden door eender
welke andere veranderlijke zonder dat de betekenis van (\ref{M02_def})
wijzigt.
\end{notation}

\subsection{Nog meer voorbeelden} We starten met drie voorbeelden
waarin het de bedoeling is een som, geschreven met een
sommatieteken, uit te schrijven en indien mogelijk te berekenen.
\begin{example} \begin{eqnarray*}
\sum_{i=0}^3\left(\frac{2^i}{3}+5\right)&=&\left(\frac{2^0}{3}+5\right)+\left(\frac{2^1}{3}+5\right)+\left(\frac{2^2}{3}+5\right)+\left(\frac{2^3}{3}+5\right)\\
&=&\frac{16}{3}+\frac{17}{3}+\frac{19}{3}+\frac{23}{3}=25.
\end{eqnarray*} \end{example} \begin{example}
\[\sum_{j=6}^92ju_{2j}^2=2\cdot6u_{2\cdot6}^2+2\cdot7u_{2\cdot7}^2+2\cdot8u_{2\cdot8}^2+2\cdot9u_{2\cdot9}^2=12u_{12}^2+14u_{14}^2+16u_{16}^2+18u_{18}^2.\]
\end{example} \begin{example}\footnotemark \begin{eqnarray*}
\sum_{k=0}^2(-1)^k\frac{x^{2k+1}}{(2k+1)!}&=&(-1)^0\frac{x^{2\cdot0+1}}{(2\cdot0+1)!}+(-1)^1\frac{x^{2\cdot1+1}}{(2\cdot1+1)!}+(-1)^2\frac{x^{2\cdot2+1}}{(2\cdot2+1)!}\\
&=&x-\frac{1}{6}x^3+\frac{1}{120}x^5. \end{eqnarray*}
\end{example} \footnotetext{Over de faculteitsoperatie \lq$!$\rq\
verder meer.}

In de volgende voorbeelden trachten we een gegeven som te
schrijven met behulp van het sommatieteken. Hiervoor moeten we in
de eerste plaats op zoek gaan naar de gemeenschappelijke vorm van
de termen, dit resulteert in het formuleren van een algemene term,
zeg $a_k$. Deze algemene term mag afhangen van een
zelfgekozen\footnote{Je moet er wel voor zorgen dat de
veranderlijke die je kiest op die plaats nog geen andere betekenis
heeft. Bijvoorbeeld,
$k+k^2+k^3+k^4\neq\sum\limits_{k=1}^4k^k=1^1+2^2+3^3+4^4$, wel
$k+k^2+k^3+k^4=\sum\limits_{i=1}^4k^i$.}\ veranderlijke, zeg $k$,
die als sommatie-index zal dienen en dit z\'o dat wanneer $k$
opeenvolgende natuurlijke getallen doorloopt, de algemene term
$a_k$ opeenvolgende termen van de som doorloopt. Vervolgens zoek
je onder- en bovengrens, dit zijn respectievelijk de eerste en de
laatste waarde die $k$ moet aannemen opdat $a_k$ precies alle
termen van de som doorloopt.
\begin{example}
Beschouw de som\[\sin x+\cos 2x+\sin 3x+\cos 4x+\sin 5x+\cos
6x+\sin 7x+\cos 8x+\sin 9x+\cos 10x.\] De algemene term is hier
$a_k=\sin(2k-1)x+\cos2kx$. De ondergrens is in dit geval~$1$ en de
bovengrens $5$. Bovenstaande som kunnen we dus korter opschrijven
als\[\sum_{k=1}^5\left(\sin(2k-1)x+\cos2kx\right).\]
\end{example}
\begin{example}
Beschouw de
som\[\frac{1}{2}x+\frac{\sqrt{2}}{2}x^2+\frac{\sqrt{3}}{2}x^3+x^4+\frac{\sqrt{5}}{2}x^5+\frac{\sqrt{6}}{2}x^6+\frac{\sqrt{7}}{2}x^7+\sqrt{2}x^8+\frac{3}{2}x^9+\frac{\sqrt{10}}{2}x^{10}+\frac{\sqrt{11}}{2}x^{11}+\sqrt{3}x^{12}.\]
De algemene term is hier $\ds{a_k=\frac{\sqrt{k}}{2}x^k}$. De
ondergrens is in dat geval~$1$ en de bovengrens $12$. Bovenstaande
som kunnen we dus korter opschrijven
als\[\sum_{k=1}^{12}\frac{\sqrt{k}}{2}x^k.\]
\end{example}
\begin{example}
Beschouw de
som\[u_0-2u_1+3u_2-4u_3+5u_4-6u_5+7u_6-8u_7+\cdots+129u_{128}-130u_{129}.\]
De algemene term is hier $a_k=(-1)^k(k+1)u_k$. De ondergrens is in
dit geval~$0$ en de bovengrens $129$. Bovenstaande som kunnen we
dus korter opschrijven als\[\sum_{k=0}^{129}(-1)^k(k+1)u_k.\]
\end{example}

\subsection{Rekenregels}\label{M02_rekenregels} \newcounter{rr}
\setcounter{rr}{1} In deze paragraaf behandelen we enkele
rekenregels in verband met het sommatieteken. We beginnen echter met
een opmerking over het bereik van een sommatieteken.
\begin{remark}[bereik van een sommatieteken] Een sommatieteken
bindt zoals een product. Hiermee bedoelen we dat\[\sum_{k=m}^n a_k
b_k=\sum_{k=m}^n \left(a_k b_k\right)\qquad\textrm{en
dat}\qquad\sum_{k=m}^n a_k+b_k=\left(\sum_{k=m}^n
a_k\right)+b_k=\sum_{i=m}^n a_i+b_k.\] M.a.w.~in het laatste
voorbeeld valt $b_k$ buiten het bereik van het sommatieteken, deze
term wordt dus niet gesommeerd en de $k$ in $b_k$ heeft niets te
maken met de sommatie-index $k$. Bedoelen we toch $\sum_{k=m}^n
\left(a_k+b_k\right)$, dan mogen de haakjes niet worden weggelaten.
\end{remark} \begin{rekenregel}[afzonderen van de eerste of de
laatste term] \begin{eqnarray*}
\sum_{k=m}^na_k&=&\sum_{k=m}^{n-1}a_k+a_n\\
&=&a_m+\sum_{k=m+1}^na_k. \end{eqnarray*} \end{rekenregel}
\begin{rekenregel}[lineariteit van het sommatieteken] Zij
$\alpha,\beta\in\R$, dan hebben we\[\sum_{k=m}^n\left(\alpha
a_k+\beta
b_k\right)=\alpha\sum_{k=m}^{n}a_k+\beta\sum_{k=m}^{n}b_k.\]
\end{rekenregel} \begin{proof} \begin{eqnarray*}
\sum_{k=m}^n\left(\alpha a_k+\beta b_k\right)&=&\left(\alpha
a_m+\beta b_m\right)+\left(\alpha a_{m+1}+\beta
b_{m+1}\right)+\cdots+\left(\alpha a_n+\beta b_n\right)\\
&=&\left(\alpha a_m+\alpha a_{m+1}+\cdots+\alpha
a_n\right)+\left(\beta b_m+\beta b_{m+1}+\cdots+\beta b_n\right)\\
&=&\alpha\left(a_m+a_{m+1}+\cdots+a_n\right)+\beta\left(b_m+b_{m+1}+\cdots+b_n\right)\\
&=&\alpha\sum_{k=m}^{n}a_k+\beta\sum_{k=m}^{n}b_k. \end{eqnarray*}
\end{proof} \begin{rekenregel}[sommatie van een constante] Zij
$\alpha,\beta\in\R$, dan hebben
we\[\sum_{k=m}^n\alpha=\underbrace{\alpha+\alpha+\cdots+\alpha}_{n-m+1\
\textrm{keer}}=(n-m+1)\alpha\textrm{, i.h.b.
}\sum_{k=1}^n\alpha=\underbrace{\alpha+\alpha+\cdots+\alpha}_{n\
\textrm{keer}}=n\alpha.\] Zo ook\[\sum_{k=m}^n\left(\alpha
a_k+\beta\right)=\alpha\sum_{k=m}^na_k+(n-m+1)\beta.\]
\end{rekenregel} \begin{rekenregel}[verandering (substitutie) van
indexveranderlijke{\normalfont\footnotemark}] \footnotetext{Men kan
deze rekenregel goed vergelijken met de substitutieregel in de
integraalrekening. Als men in een bepaalde integraal een substitutie
van de integratieveranderlijke doorvoert, dient men namelijk ook de
integratiegrenzen van de integraal consequent aan te passen.} We
kunnen in een som steeds overgaan op een nieuwe indexveranderlijke.
Zij $r\in\Z$ met $r\geqslant-m$, dan hebben we\[\sum_{k=m}^na_k\
\overset{j\,\perdef\,k+r}{=} \sum_{j=m+r}^{n+r}a_{j-r} \left(=\
\sum_{k=m+r}^{n+r}a_{k-r}\right).\] \end{rekenregel} \begin{proof}
\begin{eqnarray*}
\sum_{j=m+r}^{n+r}a_{j-r}&=&a_{(m+r)-r}+a_{(m+r+1)-r}+a_{(m+r+2)-r}+\cdots+a_{(n+r-1)-r}+a_{(n+r)-r}\\
&=&a_m+a_{m+1}+a_{m+2}+\cdots+a_{n-1}+a_n\\ &=&\sum_{k=m}^na_k.
\end{eqnarray*} \end{proof}

\startletternummering
\subsection{Voorbeeldoefeningen}
In deze paragraaf worden enkele voorbeeldoefeningen uitgewerkt om
te illustreren hoe bovenstaande definitie en rekenregels worden
toegepast.
\begin{voorbeeldoefening}
Reken onderstaande sommen uit.
\begin{enumerate}
\item $\ds{\sum^{5}_{k=2}k=2+3+4+5=14}$
\item $\ds{\sum^{3}_{k=1}10=10+10+10=30}$
\item $\ds{\sum^{5}_{i=2}(2i+1)^{2}=5^{2}+7^{2}+9^{2}+11^{2}=276}$
\item $\ds{\sum^{6}_{l=3}4=4+4+4+4=16}$
\item $\ds{\sum^{4}_{m=2}\frac{20}{m}=\frac{20}{2}+\frac{20}{3}+\frac{20}{4}=\frac{130}{6}}$
\item $\ds{\sum^{5}_{j=0}2^k=2^k+2^k+2^k+2^k+2^k+2^k=6\cdot2^k}$
\item
$\ds{\sum^{21}_{k=17}5+\sqrt[k]{1+k}=\left(\sum^{21}_{k=17}5\right)+\sqrt[k]{1+k}=5+5+5+5+5+\sqrt[k]{1+k}=25+\sqrt[k]{1+k}}$\\[-1ex]
\mbox{\ }
\end{enumerate}
\end{voorbeeldoefening}
\begin{voorbeeldoefening}
Beschouw de som
\begin{equation}\label{M02_somoef}
\sum_{k=1}^{3}2^{k}=2^{1}+2^{2}+2^{3}=2+4+8=14.
\end{equation}
Zonderen we hierin de eerste term af, dan bekomen
we\[\sum_{k=1}^{3}2^{k}=2+\sum_{k=2}^{3}2^{k},\] de laatste term
afzonderen geeft\[\sum_{k=1}^{3}2^{k}=\sum_{k=1}^{2}2^{k}+8.\]
Stel $p=k+6$. Hoe schrijven we (\ref{M02_somoef}) als een som over de
index $p$ i.p.v.~over de index~$k$?
\[\sum_{k=1}^{3}2^{k}=\sum_{p=\ldots}^{\ldots}2^{\ldots}\] Als $p=k+6$,
dan is $k=p-6$. Als $k=1$, is $p=1+6=7$ en als $k=3$, is $p=9$. Zo
bekomen we
\[\sum_{k=1}^{3}2^{k}=\sum_{p=7}^{9}2^{p-6}=2^{1}+2^{2}+2^{3}=14.\]
Wanneer de sommatie over $k$-waarden van $1$ t.e.m.~$3$ vervangen
wordt door een sommatie met ondergrens $0$ en bovengrens $2$,
wordt hetzelfde resultaat bekomen, mits de juiste aanpassing in de
algemene term:
\[\sum_{k=1}^{3}2^{k}\
\overset{l\,\perdef\,k-1}{=}\
\sum_{l=0}^{2}2^{l+1}=\sum_{k=0}^{2}2^{k+1}=2^{1}+2^{2}+2^{3}=14.\]
\end{voorbeeldoefening}
\begin{voorbeeldoefening}
Reken onderstaande sommen uit als je weet dat
\[\sum^{8}_{m=4}x_{m}=10,\qquad\qquad\quad\sum^{8}_{m=4}y_{m}=15\qquad\ \ \textrm{en}\ \ \qquad x_{3}=2.\]
\begin{enumerate}
\item $\ds{\sum^{8}_{m=4}\left(y_{m}+3\right)=\sum^{8}_{m=4}y_m+\sum^{8}_{m=4}3=\sum^{8}_{m=4}y_{m}+5\cdot 3=15+15=30}$
\item $\ds{\sum^{8}_{m=4}\left(8x_{m}+2\right)=8\sum^{8}_{m=4}x_{m}+5\cdot 2=8\cdot 10+10=90}$
\item $\ds{\sum^{8}_{m=3}\frac{x_{m}}{4}=\frac{x_{3}}{4}+\sum^{8}_{m=4}\frac{x_{m}}{4}=\frac{2}{4}+\frac{1}{4}\sum^{8}_{m=4}x_{m}=\frac{1}{2}+\frac{10}{4}=3}$
\item $\ds{\sum^{6}_{k=1}x_{k+2}\overset{m\perdef k+2}{=}\sum^8_{m=3}x_m=x_3+\sum^8_{m=4}x_k=2+10=12}$
\end{enumerate}
\end{voorbeeldoefening}
\begin{voorbeeldoefening}
Vul aan.
\begin{enumerate}
\item $\ds{\sum^{4}_{i=1}3i^{2}=\sum^{\ldots}_{j=5}\ldots}$
\begin{procedure} Als $i=1$ moet $j=5$. Stel daarom $j=i+4$. Als dan $i=4$, is
$j=8$. Omdat verder $i=j-4$, bekomen we
\[\sum^{4}_{i=1}3i^{2}=\sum^{8}_{j=5}3(j-4)^2.\]
\end{procedure}
\item $\ds{\sum^{5}_{i=3}\left(2i^2+3\right)=\sum^{4}_{i=3}\left(2i^2+3\right)+\ldots}$
\begin{procedure} Hier moet de laatste term ($i=5$) worden afgezonderd. Deze
term is gelijk aan $2\cdot5^2+3=53$, dus volgt
\[\sum^5_{i=3}\left(2i^2+3\right)=\sum^4_{i=3}\left(2i^2+3\right)+53.\]
\end{procedure}
\item $\ds{\sum^{n+1}_{k=1}(k+1)^2=\sum^n_{k=1}(k+1)^2+\ldots}$
\begin{procedure} Ook hier moet de laatste term worden afgezonderd.
\[\sum^{n+1}_{k=1}(k+1)^{2}=\sum^n_{k=1}(k+1)^2+\left((n+1)+1\right)^2=\sum^n_{k=1}(k+1)^{2}+(n+2)^2\]
\end{procedure}~
\end{enumerate}
\end{voorbeeldoefening}
\oefverwijzing{Probeer nu zelf op analoge wijze de Oefeningen 1 en
2 van Sectie~\ref{M02_oef} op te lossen.}
\eindeletternummering

\subsection{Toepassing: Gemiddelde en standaardafwijking in de statistiek}
Laat $x_1, x_2, x_3, \ldots , x_n$ de gemeten waarden zijn voor
een onbekende grootheid $X$. Een eerste voorbeeld hiervan is de
meting van geboortegewichten van alle pasgeboren baby's in een
bepaald regionaal ziekenhuis gedurende \'{e}\'{e}n week. Een
tweede voorbeeld is een experiment uit de fysica waarin de periode
van een slinger herhaaldelijk wordt gemeten met behulp van een
chronometer. Door het vroeg- of laattijdig afdrukken van de
chronometer zal telkens een andere periode gemeten worden. In dit
geval zal de re\"{e}le waarde van de te meten periode het best
benaderd worden door het gemiddelde te nemen van de verschillende
meetresultaten. Daarbij zal het gemiddelde van 10 metingen
betrouwbaarder zijn dan het gemiddelde van 3 metingen.

Het \emph{rekenkundig gemiddelde} van al de geboortegewichten of
van alle gemeten periodes is gedefinieerd als de som van alle
metingen gedeeld door het aantal meetresultaten. In de statistiek
wordt dit rekenkundig gemiddelde genoteerd met
\[\bar{x}\perdef\frac{1}{n}\sum_{i=1}^{n} x_{i}.\]

Een maat voor de spreiding van de metingen (voor hoe ver de
verschillende meetresultaten uit elkaar liggen) is bijvoorbeeld
het rekenkundig gemiddelde van de gekwadrateerde afwijkingen van
de metingen tot hun gemiddelde. Deze spreidingsmaat noemt men de
\emph{(steekproef)variantie} en wordt genoteerd met
\[s^2\perdef\frac{1}{n}\sum^{n}_{i=1}(x_i-\bar{x})^2.\]
Een andere veel gebruikte spreidingsmaat is de
\emph{standaarddeviatie}, hetgeen eenvoudigweg de vierkantswortel
is uit de variantie:
\[s\perdef\sqrt{s^2}=\sqrt{\frac{1}{n}\sum^{n}_{i=1}(x_i-\bar{x})^2}. \]


\end{document}
