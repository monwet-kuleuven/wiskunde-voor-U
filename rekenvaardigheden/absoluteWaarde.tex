\documentclass[numbers,wordchoicegiven]{ximera}
% use of options here to be investigated!!!
% -> in pdf: just pint the correct
% -> in HTML: provide options
% BUT: in exercises: also in pdf provide options!!!

\author{Zomercursus KU Leuven}
\outcome{Algebraïsch kunnen rekenen met absolute waarde.}
\outcome{Het verband begrijpen tussen afstand en absolute waarde.}

%
% copied from https://github.com/mooculus/calculus
%
\usepackage[utf8]{inputenc}


\graphicspath{
	{./}
	{goniometrie/}
}


%\usepackage{todonotes}
%\usepackage{mathtools} %% Required for wide table Curl and Greens
%\usepackage{cuted} %% Required for wide table Curl and Greens
\newcommand{\todo}{}

% Font niet (correct?) geinstalleerd in MikTeX?
%\usepackage{esint} % for \oiint
%\ifxake%%https://math.meta.stackexchange.com/questions/9973/how-do-you-render-a-closed-surface-double-integral
%\renewcommand{\oiint}{{\large\bigcirc}\kern-1.56em\iint}
%\fi


\newcommand{\mooculus}{\textsf{\textbf{MOOC}\textnormal{\textsf{ULUS}}}}

\usepackage{tkz-euclide}\usepackage{tikz}
\usepackage{tikz-cd}
\usetikzlibrary{arrows}
\tikzset{>=stealth,commutative diagrams/.cd,
  arrow style=tikz,diagrams={>=stealth}} %% cool arrow head
\tikzset{shorten <>/.style={ shorten >=#1, shorten <=#1 } } %% allows shorter vectors

\usetikzlibrary{backgrounds} %% for boxes around graphs
\usetikzlibrary{shapes,positioning}  %% Clouds and stars
\usetikzlibrary{matrix} %% for matrix
\usepgfplotslibrary{polar} %% for polar plots
\usepgfplotslibrary{fillbetween} %% to shade area between curves in TikZ
\usetkzobj{all}
\usepackage[makeroom]{cancel} %% for strike outs
%\usepackage{mathtools} %% for pretty underbrace % Breaks Ximera
%\usepackage{multicol}
\usepackage{pgffor} %% required for integral for loops



%% http://tex.stackexchange.com/questions/66490/drawing-a-tikz-arc-specifying-the-center
%% Draws beach ball
\tikzset{pics/carc/.style args={#1:#2:#3}{code={\draw[pic actions] (#1:#3) arc(#1:#2:#3);}}}



\usepackage{array}
\setlength{\extrarowheight}{+.1cm}
\newdimen\digitwidth
\settowidth\digitwidth{9}
\def\divrule#1#2{
\noalign{\moveright#1\digitwidth
\vbox{\hrule width#2\digitwidth}}}





\newcommand{\RR}{\mathbb R}
\newcommand{\R}{\mathbb R}
\newcommand{\N}{\mathbb N}
\newcommand{\Z}{\mathbb Z}

\newcommand{\sagemath}{\textsf{SageMath}}


%\renewcommand{\d}{\,d\!}
\renewcommand{\d}{\mathop{}\!d}
\newcommand{\dd}[2][]{\frac{\d #1}{\d #2}}
\newcommand{\pp}[2][]{\frac{\partial #1}{\partial #2}}
\renewcommand{\l}{\ell}
\newcommand{\ddx}{\frac{d}{\d x}}

\newcommand{\zeroOverZero}{\ensuremath{\boldsymbol{\tfrac{0}{0}}}}
\newcommand{\inftyOverInfty}{\ensuremath{\boldsymbol{\tfrac{\infty}{\infty}}}}
\newcommand{\zeroOverInfty}{\ensuremath{\boldsymbol{\tfrac{0}{\infty}}}}
\newcommand{\zeroTimesInfty}{\ensuremath{\small\boldsymbol{0\cdot \infty}}}
\newcommand{\inftyMinusInfty}{\ensuremath{\small\boldsymbol{\infty - \infty}}}
\newcommand{\oneToInfty}{\ensuremath{\boldsymbol{1^\infty}}}
\newcommand{\zeroToZero}{\ensuremath{\boldsymbol{0^0}}}
\newcommand{\inftyToZero}{\ensuremath{\boldsymbol{\infty^0}}}



\newcommand{\numOverZero}{\ensuremath{\boldsymbol{\tfrac{\#}{0}}}}
\newcommand{\dfn}{\textbf}
%\newcommand{\unit}{\,\mathrm}
\newcommand{\unit}{\mathop{}\!\mathrm}
\newcommand{\eval}[1]{\bigg[ #1 \bigg]}
\newcommand{\seq}[1]{\left( #1 \right)}
\renewcommand{\epsilon}{\varepsilon}
\renewcommand{\phi}{\varphi}


\renewcommand{\iff}{\Leftrightarrow}

\DeclareMathOperator{\arccot}{arccot}
\DeclareMathOperator{\arcsec}{arcsec}
\DeclareMathOperator{\arccsc}{arccsc}
\DeclareMathOperator{\si}{Si}
\DeclareMathOperator{\scal}{scal}
\DeclareMathOperator{\sign}{sign}


%% \newcommand{\tightoverset}[2]{% for arrow vec
%%   \mathop{#2}\limits^{\vbox to -.5ex{\kern-0.75ex\hbox{$#1$}\vss}}}
\newcommand{\arrowvec}[1]{{\overset{\rightharpoonup}{#1}}}
%\renewcommand{\vec}[1]{\arrowvec{\mathbf{#1}}}
\renewcommand{\vec}[1]{{\overset{\boldsymbol{\rightharpoonup}}{\mathbf{#1}}}\hspace{0in}}

\newcommand{\point}[1]{\left(#1\right)} %this allows \vector{ to be changed to \vector{ with a quick find and replace
\newcommand{\pt}[1]{\mathbf{#1}} %this allows \vec{ to be changed to \vec{ with a quick find and replace
\newcommand{\Lim}[2]{\lim_{\point{#1} \to \point{#2}}} %Bart, I changed this to point since I want to use it.  It runs through both of the exercise and exerciseE files in limits section, which is why it was in each document to start with.

\DeclareMathOperator{\proj}{\mathbf{proj}}
\newcommand{\veci}{{\boldsymbol{\hat{\imath}}}}
\newcommand{\vecj}{{\boldsymbol{\hat{\jmath}}}}
\newcommand{\veck}{{\boldsymbol{\hat{k}}}}
\newcommand{\vecl}{\vec{\boldsymbol{\l}}}
\newcommand{\uvec}[1]{\mathbf{\hat{#1}}}
\newcommand{\utan}{\mathbf{\hat{t}}}
\newcommand{\unormal}{\mathbf{\hat{n}}}
\newcommand{\ubinormal}{\mathbf{\hat{b}}}

\newcommand{\dotp}{\bullet}
\newcommand{\cross}{\boldsymbol\times}
\newcommand{\grad}{\boldsymbol\nabla}
\newcommand{\divergence}{\grad\dotp}
\newcommand{\curl}{\grad\cross}
%\DeclareMathOperator{\divergence}{divergence}
%\DeclareMathOperator{\curl}[1]{\grad\cross #1}
\newcommand{\lto}{\mathop{\longrightarrow\,}\limits}

\renewcommand{\bar}{\overline}

\colorlet{textColor}{black}
\colorlet{background}{white}
\colorlet{penColor}{blue!50!black} % Color of a curve in a plot
\colorlet{penColor2}{red!50!black}% Color of a curve in a plot
\colorlet{penColor3}{red!50!blue} % Color of a curve in a plot
\colorlet{penColor4}{green!50!black} % Color of a curve in a plot
\colorlet{penColor5}{orange!80!black} % Color of a curve in a plot
\colorlet{penColor6}{yellow!70!black} % Color of a curve in a plot
\colorlet{fill1}{penColor!20} % Color of fill in a plot
\colorlet{fill2}{penColor2!20} % Color of fill in a plot
\colorlet{fillp}{fill1} % Color of positive area
\colorlet{filln}{penColor2!20} % Color of negative area
\colorlet{fill3}{penColor3!20} % Fill
\colorlet{fill4}{penColor4!20} % Fill
\colorlet{fill5}{penColor5!20} % Fill
\colorlet{gridColor}{gray!50} % Color of grid in a plot

\newcommand{\surfaceColor}{violet}
\newcommand{\surfaceColorTwo}{redyellow}
\newcommand{\sliceColor}{greenyellow}




\pgfmathdeclarefunction{gauss}{2}{% gives gaussian
  \pgfmathparse{1/(#2*sqrt(2*pi))*exp(-((x-#1)^2)/(2*#2^2))}%
}


%%%%%%%%%%%%%
%% Vectors
%%%%%%%%%%%%%

%% Simple horiz vectors
\renewcommand{\vector}[1]{\left\langle #1\right\rangle}


%% %% Complex Horiz Vectors with angle brackets
%% \makeatletter
%% \renewcommand{\vector}[2][ , ]{\left\langle%
%%   \def\nextitem{\def\nextitem{#1}}%
%%   \@for \el:=#2\do{\nextitem\el}\right\rangle%
%% }
%% \makeatother

%% %% Vertical Vectors
%% \def\vector#1{\begin{bmatrix}\vecListA#1,,\end{bmatrix}}
%% \def\vecListA#1,{\if,#1,\else #1\cr \expandafter \vecListA \fi}

%%%%%%%%%%%%%
%% End of vectors
%%%%%%%%%%%%%

%\newcommand{\fullwidth}{}
%\newcommand{\normalwidth}{}



%% makes a snazzy t-chart for evaluating functions
%\newenvironment{tchart}{\rowcolors{2}{}{background!90!textColor}\array}{\endarray}

%%This is to help with formatting on future title pages.
\newenvironment{sectionOutcomes}{}{}



%% Flowchart stuff
%\tikzstyle{startstop} = [rectangle, rounded corners, minimum width=3cm, minimum height=1cm,text centered, draw=black]
%\tikzstyle{question} = [rectangle, minimum width=3cm, minimum height=1cm, text centered, draw=black]
%\tikzstyle{decision} = [trapezium, trapezium left angle=70, trapezium right angle=110, minimum width=3cm, minimum height=1cm, text centered, draw=black]
%\tikzstyle{question} = [rectangle, rounded corners, minimum width=3cm, minimum height=1cm,text centered, draw=black]
%\tikzstyle{process} = [rectangle, minimum width=3cm, minimum height=1cm, text centered, draw=black]
%\tikzstyle{decision} = [trapezium, trapezium left angle=70, trapezium right angle=110, minimum width=3cm, minimum height=1cm, text centered, draw=black]



\title[Rekenvaardigheden:]{Absolute waarde}

\begin{document}
\begin{abstract}
	Bekijk alles positief!
\end{abstract}
\maketitle

\section{Definitie en verband met afstand}\label{basisaw}

\begin{definition}
	Voor een reëel getal $a\in\R$ definiëren we de \textit{absolute waarde} van $a$, genoteerd $|a|$, als
	\[
		|a| \overset{def}{=}\displaystyle\ 
		          \left\{
			\begin{array}{rll  } 
				a  & \mbox{als} & a \geq 0 \\
				-a & \mbox{als} & a<0.
			\end{array}\right.
	\]
\end{definition}

\begin{example} Eenvoudige voorbeelden van absolute waardes
	
\pdfOnly{\begin{multicols}{2}}
		\begin{enumerate}
			\item $|5|=5$ en $|-5|=5$
			\item $|\sqrt{2}-1|=\sqrt{2} - 1$
			\item $|1-\sqrt{2}| = $\wordChoice{\choice[correct]{$\sqrt{2} - 1$}\choice{$1-\sqrt{2}$}}
			\item $|2-\sqrt{2}| = 2 - \sqrt{2}$
			\item $|2 + 1| = |2| + | 1| (=3)$
			\item $|2 + (-1)| \neq |2| + | -1|$
		\end{enumerate}
\pdfOnly{\end{multicols}}
\end{example}
\begin{example} Voorbeelden van absolute waardes (met $a\in\R$)
		\begin{enumerate}[resume]
			\item $|a^2| = a^2$ (want $a^2 \geq 0$)
			\item $|-a|= ....$ \\ (er is GEEN eenvoudige algemene formule! \\En ZEER ZEKER NIET $|-a|=a$: fout zodra $a<0$ !!!)
			\item $|a^2 + 1| = a^2 + 1$
			\item $|a^2 - 1| = ....$ \qquad(er is GEEN eenvoudige algemene formule!)
\end{enumerate}
\end{example}

Als we het getal $a$ voorstellen op de getallenas, zien we onmiddelijk volgende
\begin{proposition} 
	De absolute waarde van een getal is zijn afstand tot de oorsprong (op de getallenas).
\end{proposition}
\begin{enumerate}
\item als $a \geq 0$
%
% Gecopieerd van Zomercursus: to be done in Tikz !!!
\begin{picture}(240,30)
\put(100,0){\vector(1,0){240}}
\put(340,-10){\makebox(0,0){$\R$}}
\put(220,0){\makebox(0,0){$\bullet$}}
\put(320,0){\makebox(0,0){$\bullet$}}
\put(220,10){\makebox(0,0){$0$}} \put(320,10){\makebox(0,0){$a$}}
\put(270,-20){\makebox(0,0){$|a|=a$}}
\put(250,-20){\vector(-1,0){30}}
\put(290,-20){\vector(1,0){30}}
\end{picture}
\\
\item als $a<0$
\begin{picture}(240,30)
\put(100,0){\vector(1,0){240}}
\put(340,-10){\makebox(0,0){$\R$}}
\put(120,0){\makebox(0,0){$\bullet$}}
\put(220,0){\makebox(0,0){$\bullet$}}
\put(120,10){\makebox(0,0){$a$}} 
\put(220,10){\makebox(0,0){$0$}}
\put(170,-20){\makebox(0,0){$|a|=-a$}}
\put(147,-20){\vector(-1,0){28}}
\put(193,-20){\vector(1,0){28}}
\end{picture}
\\
\end{enumerate}
Deze (triviale) eigenschap, die op het eerste zicht is toegevoegd door overijverige wiskundigen die hun cursussen wat langer wilden maken, leidt nochtans onmiddellijk tot volgend (echt!) interessant inzicht: 
\begin{proposition}
	Zij $a,b\in\R$ Dan geldt\\
	$|x-y|$ is de afstand tussen $x$ en $y$ (op de getallenas).
\end{proposition}
Het is ERG BELANGRIJK om bij 'de absolute waarde van een verschil' ALTIJD minstens even te denken aan 'afstand tussen de twee termen'. Merk op dat je elke som gemakkelijk kan schrijven als een verschil, en dat dus ook $|x+y| = |x+(-y)|$ gelijk is aan de afstand tussen $x$ en $-y$. Dit is een ERG BELANGRIJK inzicht om de meeste van volgende oefeningen snel en eenvoudig te kunnen oplossen.

\section{Eigenschappen}
De betekenis en geldigheid van volgende veel gebruikte eigenschappen kunnen eenvoudig worden nagegaan op de getallenas:
\begin{proposition}\label{basisaw}
	Als $x,a \in \R$ en $r\in\R^+_0$, dan is 
%	\begin{enumerate} 
%		\item \label{aw<r}		$|x|<r\qquad	\Leftrightarrow \qquad -r<x<r$ 
%		\item \label{aw>r} 		$|x|>r\qquad	\Leftrightarrow \qquad x<-r\;$ of $\;r<x$ 
%		\item \label{aw	x-a<r}	$|x-a|<r\qquad 	\Leftrightarrow \qquad a-r<x<a+r$ 
%		\item \label{aw x-a>r}	$|x-a|>r\qquad 	\Leftrightarrow \qquad x<a-r\;$ of $\;a+r<x.$
%	\end{enumerate}
%	\begin{flalign} 
%	 |x|<r   &\Leftrightarrow &-r<x<r   &\text{$x$ kleiner dan $r$} &  \label{aaw<r}	          \\
%	 |x|>r   &\Leftrightarrow &x<-r \text{of} r<x &\text{$x$ groter dan $r$} &  \\
%	\end{flalign}

	\begin{tabular}{lrclrrrr} 
	1)&$|x|<r$   &$\Leftrightarrow$ &$-r<x<r$  & $\Leftrightarrow$ & $x\in]-r,r[$     & $\Leftrightarrow$&$x$ ligt dichter dan $r$ bij $0$    \\
	2)&$|x|>r$   &$\Leftrightarrow$ &$x<-r$ of $r<x$ & $\Leftrightarrow$ & $x\notin]-r,r[$   & $\Leftrightarrow$ &$x$ ligt verder dan $r$ van $0$    \\
	3)&$|x-a|<r$ & $\Leftrightarrow$ & $ a-r<x<a+r$  & $\Leftrightarrow$ & $x\in]a-r,a+r[$  & $\Leftrightarrow$ & $x$ ligt dichter dan $r$ bij $a$           \\
	4)&$|x-a|>r$ & $\Leftrightarrow$ & $ x<a-r\;$ of $\;a+r<x.$  & $\Leftrightarrow$ & $x\notin]a-r,a+r[$  & $\Leftrightarrow$ & $x$ ligt verder dan $r$ van $a$\\
	\end{tabular}        
\end{proposition}

\todo{TODO: Toegevoegen tekening!}

\begin{remark} \  %forceer newline; betere oplossing? (te veel whitespace...)
%	\begin{enumerate}
%		\item 
		Met deze eigenschap kan je dus één ongelijkheid \textit{met} een absolute waarde omzetten in een \textit{dubbele} ongelijkheid \textit{zonder} absolute waarde. Het is ERG BELANGRIJK zo snel mogelijk deze equivalentie als volledig evident en natuurlijk te beschouwen. Je kent deze formules voldoende goed van zodra je niet meer verschil ziet tussen de formules $|x|<r$ en $-r<x<r$ en $x\in]-r,r[$ als tussen de woorden aap, Aap en \texttt{Aap}.
 %	\end{enumerate}
\end{remark}
% Uit Zomercursus

Voor verdere referentie volgen hier nog enkele eigenschappen van de absolute waarde
(zonder bewijs):
\begin{proposition}\label{eig: aw}
	Als $x,y,z\in \R$, dan is
	\begin{enumerate}
		\item \label{aw tss}$-|x| \leqslant x \leqslant |x|$ 
		\item \label{aw 0}$|x|=0 \quad\Leftrightarrow \quad x=0$ 
		\item\label{aw x -x} $|x|=|-x|$
		\item \label{aw kwadr}$|x|^2=x^2=|x^2|$
		\item \label{wortel	kwadr}$\sqrt{x^2}=|x|$
		\item \label{kwadr wortel}$(\sqrt{x})^2=x$ \hfill(hier is $x \geqslant 0$
			want anders bestaat de wortel niet!)
		\item \label{aw product}$|x\;y|=|x||y|,$ \hfill(de absolute waarde van het
			product is het product van de absolute waarden.)
		\item \label{aw	quotient}$\displaystyle \left|\frac{x}{y}\right|=\frac{|x|}{|y|}$
		met $y\neq 0,$ \hfill(de absolute waarde van het quoti\"ent is
			het quoti\"ent van de absolute waarden.)
			
		\todo{\item TODO:Volgende eigenschappen horen eerder bij ongelijkheden ...?}
		\item \label{aw1edrieh}$|x+y| \leqslant|x|+|y|$ \hfill ({\bf eerste
			drie\-hoeks\-on\-ge\-lijk\-heid})
		\item \label{aw1e drieh2}$|x-y|
		\leqslant|x-z|+|z-y|\;\;$ \hfill ({\bf eerste
			drie\-hoeks\-on\-ge\-lijk\-heid;  \textit{equivalente vorm}})
		\item \label{aw 2e drieh} $\ds |\;|x|-|y|\;| \leqslant|x-y|$ \hfill
		(\bf tweede driehoeksongelijkheid)
	\end{enumerate}
\end{proposition}

\begin{remark}
Uit \ref{aw1edrieh} volgt ook onmiddellijk dat $|x-y| \leq |x|+|y|$ (want $x-y=x+(-y)$ en $|y|=|-y|$), maar natuurlijk geldt NIET dat $\cancel{|x-y| \leq |x| - |y|}$. We weten dan weer WEL dat $|\;|x|-|y|\;|\leq|x-y|$ (dat is \ref{aw 2e drieh}).
\end{remark}


\begin{problem} Onderzoek welke $x\in \R$ voldoen aan $|x-5|<9.$
		\begin{oplossing} 
			\begin{eqnarray*}
				|x-5|&<&9 \qquad\qquad \textit{de afstand tussen $x$ en $5$ is kleiner dan $9$}\\
				&\Updownarrow &\textit{eig. \ref{basisaw}(\ref{aw x-a<r})}\\
				5-9<&x&<5+9\\
				&\Updownarrow &\\
				-4<&x&<14
			\end{eqnarray*}
			De oplossingsverzameling is dus $V=]-4,14\,[.$
		\end{oplossing}
	\end{problem}
\begin{problem}	Onderzoek welke $x\in \R$ voldoen aan $|x+3|>5$
		\begin{oplossing} 
			\begin{eqnarray*}
				|x+3|&>&5 \qquad \textit{de afstand tussen $x$ en $-3$ is groter dan $5$}\\
				&\Updownarrow &\textit{eig. \ref{basisaw}(\ref{aw x-a>r})}
				\\x<-3-5&\mbox{of}&-3+5<x\\
				&\Updownarrow &\\
				x<-8&\mbox{of}&2<x
			\end{eqnarray*}
			De oplossingsverzameling is dus $V=\left]-\infty,-8\right[
			\;\cup\;\left]2, +\infty\right[.$
		\end{oplossing}
\end{problem}
\begin{problem}	Onderzoek welke $x\in \R$ voldoen aan $|x^2-3|<9$.
		\begin{oplossing}
			\begin{eqnarray*}|x^2-3|&<&9 \qquad \textit{de afstand tussen $x^2$ en $3$ is
					kleiner dan $9$}\\&\Updownarrow &\textit{eig.
					\ref{basisaw}(\ref{aw x-a<r})}
				\\3-9<&x^2&<3+9\\&\Updownarrow
				&\\-6<&x^2&<12\\&\Updownarrow&\textit{$-6<x^2$ is automatisch
					voldaan en is dus geen
					voorwaarde}\\x^2&<&12\\&\Updownarrow&\textit{beide leden pos.,
					worteltrekken mag, behoudt de ongelijkheid (eig. \ref{eig:
						ongelijkheden}
					(\ref{wortel}))}\\\sqrt{x^2}&<&\sqrt{12}=2\sqrt{3}\\&\Updownarrow
				&\qquad \sqrt{x^2}=|x| \;\textit{eig. ~\ref{eig: aw}(\ref{wortel
						kwadr})}\\|x|&<&2\sqrt{3}
				\\&\Updownarrow &\textit{eig. \ref{basisaw}(\ref{aw x-a<r})}\\-2\sqrt{3}<&x&<2\sqrt{3}.
			\end{eqnarray*}
			De oplossingsverzameling is dus $\displaystyle
			V=\left]-2\sqrt{3},2\sqrt{3}\right[.$
		\end{oplossing}
\end{problem}

\begin{problem} Onderzoek welke $x\in \R$ voldoet aan $\displaystyle \left|\frac{2x}{3}\right|<1$.
		\begin{oplossing}
			\begin{eqnarray*}
				\displaystyle \left|\frac{2x}{3}\right|&<&1 \\&\Updownarrow
				&\textit{eigenschap \ref{eig: aw}(\ref{aw product}) en (\ref{aw
						quotient})} \\ \displaystyle \frac{|2||x|}{|3|}&<&1
				\\&\Updownarrow &\\\displaystyle \frac{2|x|}{3}&<&1\\&\Updownarrow
				&\textit{beide
					leden vermenigvuldigen met de positieve factor $\frac{3}{2}$ (eig.~\ref{eig: ongelijkheden}(\ref{factor}))}\\
				|x|&<&\displaystyle\frac{3}{2}\\&\Updownarrow &\textit{eigenschap
					\ref{basisaw}(\ref{aw<r})}\\\displaystyle-\frac{3}{2}<&x&<\displaystyle\frac{3}{2}
			\end{eqnarray*}
			De oplossingsverzameling is dus $\displaystyle
			V=\left]-\frac{3}{2},\frac{3}{2}\right[$.
		\end{oplossing}
\end{problem}
\begin{problem}	 Als $x,y \in\R$, $\varepsilon\in \R_0^+$ en er geldt dat
		$|x-y|<\varepsilon,$ kan je dan een waarde vinden voor
		$\varepsilon$ waarvoor $\displaystyle \frac{|y|}{2}<|x|$?
\begin{expandable}
		\begin{oplossing} \ 
			\textit{De tweede driehoeksongelijkheid legt een verband tussen
				$|x|,|y|$ en $|x-y|$. We zullen daarmee aan de slag gaan.}
			\\We weten dat $\displaystyle |\;|x|-|y|\;| \leqslant|x-y|$ en verder
			dat $|x-y|<\varepsilon$ zodat door de transitiviteit geldt dat
			\begin{eqnarray*}\displaystyle |\;|x|-|y|\;|&<&\varepsilon\\&\Updownarrow& \textit{eigenschap \ref{basisaw}(\ref{aw<r})}\\-\varepsilon<&|x|-|y|&<\varepsilon\\
				&\Updownarrow&\textit{als een term van lid verandert, verandert
					hij van teken (eig. \ref{eig:
						ongelijkheden}(\ref{opt}))}\\|y|-\varepsilon<&|x|&<|y|+\varepsilon\end{eqnarray*}
			Als we bijvoorbeeld $\ds\varepsilon=\frac{|y|}{2}$ kiezen geeft de
			linkse ongelijkheid het gevraagde $\ds \frac{|y|}{2}<|x|$.
		\end{oplossing}
	\end{expandable}
\end{problem}

%\section{Oefeningen}
%\practice{awbasisrekenen}

\end{document}
