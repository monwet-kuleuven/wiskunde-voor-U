%
% ximera activiteit
% Copyrigth
%
\documentclass{ximera}
%
% Opties (enkel te gebruiken voor lokale testen; NOOIT committen met opties)
%
%\pdfOnly{\providecommand\showtodonotes{}}
%\newcommand\xmnouitweiding{}


%
% copied from https://github.com/mooculus/calculus
%
\usepackage[utf8]{inputenc}


\graphicspath{
	{./}
	{goniometrie/}
}


%\usepackage{todonotes}
%\usepackage{mathtools} %% Required for wide table Curl and Greens
%\usepackage{cuted} %% Required for wide table Curl and Greens
\newcommand{\todo}{}

% Font niet (correct?) geinstalleerd in MikTeX?
%\usepackage{esint} % for \oiint
%\ifxake%%https://math.meta.stackexchange.com/questions/9973/how-do-you-render-a-closed-surface-double-integral
%\renewcommand{\oiint}{{\large\bigcirc}\kern-1.56em\iint}
%\fi


\newcommand{\mooculus}{\textsf{\textbf{MOOC}\textnormal{\textsf{ULUS}}}}

\usepackage{tkz-euclide}\usepackage{tikz}
\usepackage{tikz-cd}
\usetikzlibrary{arrows}
\tikzset{>=stealth,commutative diagrams/.cd,
  arrow style=tikz,diagrams={>=stealth}} %% cool arrow head
\tikzset{shorten <>/.style={ shorten >=#1, shorten <=#1 } } %% allows shorter vectors

\usetikzlibrary{backgrounds} %% for boxes around graphs
\usetikzlibrary{shapes,positioning}  %% Clouds and stars
\usetikzlibrary{matrix} %% for matrix
\usepgfplotslibrary{polar} %% for polar plots
\usepgfplotslibrary{fillbetween} %% to shade area between curves in TikZ
\usetkzobj{all}
\usepackage[makeroom]{cancel} %% for strike outs
%\usepackage{mathtools} %% for pretty underbrace % Breaks Ximera
%\usepackage{multicol}
\usepackage{pgffor} %% required for integral for loops



%% http://tex.stackexchange.com/questions/66490/drawing-a-tikz-arc-specifying-the-center
%% Draws beach ball
\tikzset{pics/carc/.style args={#1:#2:#3}{code={\draw[pic actions] (#1:#3) arc(#1:#2:#3);}}}



\usepackage{array}
\setlength{\extrarowheight}{+.1cm}
\newdimen\digitwidth
\settowidth\digitwidth{9}
\def\divrule#1#2{
\noalign{\moveright#1\digitwidth
\vbox{\hrule width#2\digitwidth}}}





\newcommand{\RR}{\mathbb R}
\newcommand{\R}{\mathbb R}
\newcommand{\N}{\mathbb N}
\newcommand{\Z}{\mathbb Z}

\newcommand{\sagemath}{\textsf{SageMath}}


%\renewcommand{\d}{\,d\!}
\renewcommand{\d}{\mathop{}\!d}
\newcommand{\dd}[2][]{\frac{\d #1}{\d #2}}
\newcommand{\pp}[2][]{\frac{\partial #1}{\partial #2}}
\renewcommand{\l}{\ell}
\newcommand{\ddx}{\frac{d}{\d x}}

\newcommand{\zeroOverZero}{\ensuremath{\boldsymbol{\tfrac{0}{0}}}}
\newcommand{\inftyOverInfty}{\ensuremath{\boldsymbol{\tfrac{\infty}{\infty}}}}
\newcommand{\zeroOverInfty}{\ensuremath{\boldsymbol{\tfrac{0}{\infty}}}}
\newcommand{\zeroTimesInfty}{\ensuremath{\small\boldsymbol{0\cdot \infty}}}
\newcommand{\inftyMinusInfty}{\ensuremath{\small\boldsymbol{\infty - \infty}}}
\newcommand{\oneToInfty}{\ensuremath{\boldsymbol{1^\infty}}}
\newcommand{\zeroToZero}{\ensuremath{\boldsymbol{0^0}}}
\newcommand{\inftyToZero}{\ensuremath{\boldsymbol{\infty^0}}}



\newcommand{\numOverZero}{\ensuremath{\boldsymbol{\tfrac{\#}{0}}}}
\newcommand{\dfn}{\textbf}
%\newcommand{\unit}{\,\mathrm}
\newcommand{\unit}{\mathop{}\!\mathrm}
\newcommand{\eval}[1]{\bigg[ #1 \bigg]}
\newcommand{\seq}[1]{\left( #1 \right)}
\renewcommand{\epsilon}{\varepsilon}
\renewcommand{\phi}{\varphi}


\renewcommand{\iff}{\Leftrightarrow}

\DeclareMathOperator{\arccot}{arccot}
\DeclareMathOperator{\arcsec}{arcsec}
\DeclareMathOperator{\arccsc}{arccsc}
\DeclareMathOperator{\si}{Si}
\DeclareMathOperator{\scal}{scal}
\DeclareMathOperator{\sign}{sign}


%% \newcommand{\tightoverset}[2]{% for arrow vec
%%   \mathop{#2}\limits^{\vbox to -.5ex{\kern-0.75ex\hbox{$#1$}\vss}}}
\newcommand{\arrowvec}[1]{{\overset{\rightharpoonup}{#1}}}
%\renewcommand{\vec}[1]{\arrowvec{\mathbf{#1}}}
\renewcommand{\vec}[1]{{\overset{\boldsymbol{\rightharpoonup}}{\mathbf{#1}}}\hspace{0in}}

\newcommand{\point}[1]{\left(#1\right)} %this allows \vector{ to be changed to \vector{ with a quick find and replace
\newcommand{\pt}[1]{\mathbf{#1}} %this allows \vec{ to be changed to \vec{ with a quick find and replace
\newcommand{\Lim}[2]{\lim_{\point{#1} \to \point{#2}}} %Bart, I changed this to point since I want to use it.  It runs through both of the exercise and exerciseE files in limits section, which is why it was in each document to start with.

\DeclareMathOperator{\proj}{\mathbf{proj}}
\newcommand{\veci}{{\boldsymbol{\hat{\imath}}}}
\newcommand{\vecj}{{\boldsymbol{\hat{\jmath}}}}
\newcommand{\veck}{{\boldsymbol{\hat{k}}}}
\newcommand{\vecl}{\vec{\boldsymbol{\l}}}
\newcommand{\uvec}[1]{\mathbf{\hat{#1}}}
\newcommand{\utan}{\mathbf{\hat{t}}}
\newcommand{\unormal}{\mathbf{\hat{n}}}
\newcommand{\ubinormal}{\mathbf{\hat{b}}}

\newcommand{\dotp}{\bullet}
\newcommand{\cross}{\boldsymbol\times}
\newcommand{\grad}{\boldsymbol\nabla}
\newcommand{\divergence}{\grad\dotp}
\newcommand{\curl}{\grad\cross}
%\DeclareMathOperator{\divergence}{divergence}
%\DeclareMathOperator{\curl}[1]{\grad\cross #1}
\newcommand{\lto}{\mathop{\longrightarrow\,}\limits}

\renewcommand{\bar}{\overline}

\colorlet{textColor}{black}
\colorlet{background}{white}
\colorlet{penColor}{blue!50!black} % Color of a curve in a plot
\colorlet{penColor2}{red!50!black}% Color of a curve in a plot
\colorlet{penColor3}{red!50!blue} % Color of a curve in a plot
\colorlet{penColor4}{green!50!black} % Color of a curve in a plot
\colorlet{penColor5}{orange!80!black} % Color of a curve in a plot
\colorlet{penColor6}{yellow!70!black} % Color of a curve in a plot
\colorlet{fill1}{penColor!20} % Color of fill in a plot
\colorlet{fill2}{penColor2!20} % Color of fill in a plot
\colorlet{fillp}{fill1} % Color of positive area
\colorlet{filln}{penColor2!20} % Color of negative area
\colorlet{fill3}{penColor3!20} % Fill
\colorlet{fill4}{penColor4!20} % Fill
\colorlet{fill5}{penColor5!20} % Fill
\colorlet{gridColor}{gray!50} % Color of grid in a plot

\newcommand{\surfaceColor}{violet}
\newcommand{\surfaceColorTwo}{redyellow}
\newcommand{\sliceColor}{greenyellow}




\pgfmathdeclarefunction{gauss}{2}{% gives gaussian
  \pgfmathparse{1/(#2*sqrt(2*pi))*exp(-((x-#1)^2)/(2*#2^2))}%
}


%%%%%%%%%%%%%
%% Vectors
%%%%%%%%%%%%%

%% Simple horiz vectors
\renewcommand{\vector}[1]{\left\langle #1\right\rangle}


%% %% Complex Horiz Vectors with angle brackets
%% \makeatletter
%% \renewcommand{\vector}[2][ , ]{\left\langle%
%%   \def\nextitem{\def\nextitem{#1}}%
%%   \@for \el:=#2\do{\nextitem\el}\right\rangle%
%% }
%% \makeatother

%% %% Vertical Vectors
%% \def\vector#1{\begin{bmatrix}\vecListA#1,,\end{bmatrix}}
%% \def\vecListA#1,{\if,#1,\else #1\cr \expandafter \vecListA \fi}

%%%%%%%%%%%%%
%% End of vectors
%%%%%%%%%%%%%

%\newcommand{\fullwidth}{}
%\newcommand{\normalwidth}{}



%% makes a snazzy t-chart for evaluating functions
%\newenvironment{tchart}{\rowcolors{2}{}{background!90!textColor}\array}{\endarray}

%%This is to help with formatting on future title pages.
\newenvironment{sectionOutcomes}{}{}



%% Flowchart stuff
%\tikzstyle{startstop} = [rectangle, rounded corners, minimum width=3cm, minimum height=1cm,text centered, draw=black]
%\tikzstyle{question} = [rectangle, minimum width=3cm, minimum height=1cm, text centered, draw=black]
%\tikzstyle{decision} = [trapezium, trapezium left angle=70, trapezium right angle=110, minimum width=3cm, minimum height=1cm, text centered, draw=black]
%\tikzstyle{question} = [rectangle, rounded corners, minimum width=3cm, minimum height=1cm,text centered, draw=black]
%\tikzstyle{process} = [rectangle, minimum width=3cm, minimum height=1cm, text centered, draw=black]
%\tikzstyle{decision} = [trapezium, trapezium left angle=70, trapezium right angle=110, minimum width=3cm, minimum height=1cm, text centered, draw=black]


\begin{document}
    % Start specifieke settings:    
    \author{Zomercursus KU Leuven}
    \outcome{Algebra\"isch kunnen rekenen met haakjes.}
    \xmtitle[Haakwerk in ontbindig.]{Haakjes en ontbinden in factoren}{Volgorde van bewerkingen, uitwerken van formules met haakjes, ontbinden in factoren.}
    % Start inhoud ximera 

In dit deel herhalen we de belangrijkste rekenregels voor het uitwerken van haakjes en het ontbinden in factoren. 
De voorbeelden geven de kans om deze rekenregels verder in te oefenen.


\begin{proposition} (Rekenregels haakjes)\label{eig: rekenregels haakjes}
    
    
Zij $a,b,c,d\in \R$. Dan geldt:


Uitwerken van haakjes:
\todo{derdemachten en binomium is eerder uitbreiding?}
\begin{align*}
%
    (a+b)c     &=  ac+bc = c(a+b)     &\tag{distributiviteit van $\cdot$ t.o.v.~$+$}\\
    (a+b)(c+d) &=  ac+ad+bc+bd        &\\
%
    -(a+b)     &=  -a-b               &\tag{minteken voor de haken binnenbrengen}\\
    -(a-b)     &= -a+b                &\\
    -(-a+b)    &=  a-b                &\\
%
    (a+b)^2    &= a^2+2ab+b^2         &\tag{kwadraat van een tweeterm}\\
%
    (a+b)^3    &= a^3+3a^2b+3ab^2+b^3 &\tag{derdemacht van een tweeterm}\\
%
    (a+b)^n    &= \ds{\sum^n_{k=0}{n\choose k} a^{n-k}b^k}
                                      &\tag{Binomium Van Newton}\footnotemark
%
\end{align*}
Ontbinden in factoren\footnotemark / merkwaardige producten 
\begin{align*}
   ab+ac    &= a(b+c)=(b+c)a      &\tag{afzonderen van een gemeenschappelijke factor}\\
%
   a^2-b^2  &=  (a+b)(a-b)        &\tag{verschil van twee kwadraten}\\
% 
    a^3+b^3 &=  (a+b)(a^2-ab+b^2) & \tag{som van twee derdemachten}\\
% 
    a^3-b^3 &= (a-b)(a^2+ab+b^2)  & \tag{verschil van twee derdemachten}\\
% 
    a^n-b^n &= (a-b)(a^{n-1}+ba^{n-2}+b^2a^{n-3}+\cdots+b^{n-2}a+b^{n-1})
                                  &\tag{verschil van twee $n^{\text{e}}$ machten}\\
% 
    a^n+b^n &=  (a+b)(a^{n-1}-ba^{n-2}+b^2a^{n-3}- \cdots - b^{n-2}a+b^{n-1})
                                  & \tag{som v. twee $n^{\text{e}}$ machten, $n$ oneven}\\
\end{align*}
\end{proposition}

\todo{Voetnoten aanpassen}
\addtocounter{footnote}{-1} \footnotetext{Zie pakket Sommatieteken
	en faculteit.} \addtocounter{footnote}{1}
\footnotetext{Rekenregels $(6)$,$(7)$ en $(8)$ zijn, wanneer men
	ze in de omgekeerde richting toepast, tevens rekenregels voor het
	ontbinden in factoren.}
    
\subsubsection*{Voorbeelden}
\begin{example} Werk de haakjes uit en vereenvoudig. ($x,y,p,q\in\R$)
\begin{enumerate}
\item $1-(x-y-1)=$\xmopl{$1-x+y+1=-x+y+2$}
\item $1-(p-q)+p=$\xmopl{$1-p+q+p=q+1$}
\item $p-(p+q)+2q=$\xmopl{$p-p-q+2q=q$}
\end{enumerate}
\end{example}

\begin{example} Breng zoveel mogelijk factoren buiten haakjes. ($a,b,c,d,x,y,p,q\in\R$)
\begin{enumerate}
\item $18x+24y+30p=$\xmopl{$6\cdot3x+6\cdot4y+6\cdot5p=6\,(3x+4y+5p)$}
\item $\ds{20ab^3+30bc+25b^2c^3\sqrt{d}}=$\xmopl{$5b\cdot4ab^2+5b\cdot6c+5b\cdot5bc^3\sqrt{d}=5b\left(4ab^2+6c+5bc^3\sqrt{d}\right)$}
\item $-9x^2y-3xy=$\xmopl{$-3xy\cdot3x-3xy\cdot1=-3xy\,(3x+1)$}
\item $\ds{\frac{p^2q^3}{8}-\frac{p^2q}{2}+pq}=$\xmopl{$\frac{pq}{2}\cdot\frac{pq^2}{4}-\frac{pq}{2}\cdot p+\frac{pq}{2}\cdot 2=\frac{pq}{2}\left(\frac{pq^2}{4}-p+2\right)$
\\ of ook  $\ds{\frac{p^2q^3}{8}-\frac{p^2q}{2}+pq=pq\cdot\frac{pq^2}{8}-pq\cdot\frac{p}{2}+pq\cdot 1=pq\left(\frac{pq^2}{8}-\frac{p}{2}+1\right)}$}
\end{enumerate}
\end{example}

%\begin{example} 
    
Vul aan door de voorgestelde factor buiten haken te brengen. ($p,r,s,x\in\R$)
    \begin{example}

$\ds{\left(\frac{p+1}{2}\right)^{2}+(p+1)^{3}=\left(\frac{p+1}{2}\right)^{2}(\ldots)}$
\begin{feedback}
\underline{Oplossing:}
\xmopl{
\begin{eqnarray*}
\left(\frac{p+1}{2}\right)^{2}+(p+1)^{3}
   &=&\left(\frac{p+1}{2}\right)^{2}\cdot1
      +\left(\frac{p+1}{2}\right)^{2}\frac{\left(p+1\right)^{3}}{\left(\frac{p+1}{2}\right)^{2}}\\
   &=&\left(\frac{p+1}{2}\right)^{2} \left(1+\frac{(p+1)^{3}}{\left(\frac{p+1}{2}\right)^{2}}\right)\\
   &=&\left(\frac{p+1}{2}\right)^{2} \left(1+\frac{4(p+1)^{3}}{(p+1)^{2}}\right)\\
   &=&\left(\frac{p+1}{2}\right)^{2} \left(1+4(p+1)\right)\\
   &=&\left(\frac{p+1}{2}\right)^{2} \left(1+4p+4\right)\\
   &=&\left(\frac{p+1}{2}\right)^{2} \left(4p+5\right).
\end{eqnarray*}
}
\end{feedback}
\end{example}
\begin{example}

$\ds{(3p+2)\frac{4r}{3}-4r(p-1)=\frac{4r}{3}(\ldots)}$

\begin{feedback}
\underline{Oplossing:}
\xmopl{
\begin{eqnarray*}
(3p+2)\frac{4r}{3}-4r(p-1)
   &=&\frac{4r}{3}(3p+2)-\frac{4r}{3}\ \frac{4r(p-1)}{\frac{4r}{3}}\\
   &=&\frac{4r}{3}(3p+2)-\frac{4r}{3}\ \frac{4r(p-1)\cdot3}{4r}\\
   &=&\frac{4r}{3}(3p+2-(p-1)\cdot3)\\
   &=&\frac{4r}{3}(3p+2-3p+3)\\
   &=&\frac{4r}{3}\cdot5.
\end{eqnarray*}
}
\end{feedback}


\begin{example}
$\ds{\frac{xs}{8}+\frac{xs^{2}}{4}+3x^{3}s^{3}=\frac{xs}{8}(\ldots)}$

\begin{feedback}
\underline{Oplossing:}
\xmopl{
\begin{eqnarray*}
\frac{xs}{8}+\frac{xs^{2}}{4}+3x^{3}s^{3}
   &=&\frac{xs}{8}\cdot1
       +\frac{xs}{8}\ \frac{\frac{xs^{2}}{4}}{\frac{xs}{8}}
       +\frac{xs}{8}\ \frac{3x^{3}s^{3}}{\frac{xs}{8}}\\
   &=&\frac{xs}{8}\left(1+\frac{8xs^{2}}{4xs}+\frac{24x^{3}s^{3}}{xs}\right)\\
   &=&\frac{xs}{8}(1+2s+24x^{2}s^{2}).
\end{eqnarray*}
}
\end{feedback}
\end{example}
\end{example}
%$\hookrightarrow$ \textit{Maak nu Oefening 1 van Paragraaf~\ref{M01_oef}}.


\begin{example} Ontbind in factoren: $81x^{2}y^{2}-25=$
 \begin{feedback}
     \underline{Oplossing:}
 \xmopl{$81x^{2}y^{2}-25=(9xy)^{2}-5^{2}=(9xy+5)(9xy-5)$\hfill(verschil van twee kwadraten)}
 \end{feedback}

\end{example}
    
\begin{example} Ontbind in factoren: $9r^{2 }-24r+16=$
     \begin{feedback}
         \underline{Oplossing:}
         \xmopl{$9r^{2 }-24r+16=(3r)^{2}-(2\cdot3r\cdot4)+4^{2}=(3r-4)^{2}$\hfill(kwadraat van een verschil)}
        \end{feedback}
    \end{example}

Opmerking: als je hiermee vlot kan rekenen hoef je natuurlijk niet telkens alle tussenstappen op te schrijven.
% $\hookrightarrow$ \textit{Maak nu Oefening 2 van Paragraaf~\ref{M01_oef}}.
% oefeningen in aparte ximera
\end{document}
