%macro's voor zomercursus wiskunde - Groep Wetenschap & Technologie - 29 januari 2008
%Bart Bories en Riet Callens
%aangepast door Riet Callens op 23/06/2009
%aangepast door Riet Callens op 9/7/2009 (newtheorem oefening toegevoegd)
%aangepast door Riet Callens op 27/07/2012 voor pdf-tex
%aangepast door Riet Callens op 22/08/2012 voor vragen ijkingstoets (numopl}
%aangepast door Riet Callens op 1/08/2019 voor implementatie QR-codes - qrslidesA,qrslidesB,qrclip,qrapplet


\documentclass[a4paper,12pt,twoside]{article}
\usepackage{qrcode}
\usepackage{etex}
%standaard pakketten
%\usepackage{a4wide}
\usepackage[dutch]{babel}
\usepackage{hyphenat}
\usepackage{xcolor}
\definecolor{KULblauw}{RGB}{26,67,121}
\usepackage[urlbordercolor=white,linkbordercolor=white ]{hyperref}
\usepackage{amsmath}
\usepackage{amssymb}
\usepackage{amsfonts}
\usepackage{graphicx}
\usepackage{wrapfig}
%\usepackage{pst-plot}
%\usepackage{pst-pdf}
\usepackage{enumerate}
\usepackage{tikz}
\usepackage{pgfplots}
\usetikzlibrary{through,calc,intersections}
\pgfplotsset{axis lines = center}
\pgfplotsset{axis equal image}
\pgfplotsset{xlabel = x}
\pgfplotsset{ylabel = y}
\pgfplotsset{xlabel style={below right}}
\pgfplotsset{ylabel style={above left}}

%\usepackage{color}
\usepackage[thmmarks,framed,amsmath]{ntheorem}
\usepackage{answers}

%definities nieuwe omgevingen
\Newassociation{opl}{Oplossing}{ans}
\Newassociation{numopl}{NumOplossing}{numans}
\theoremstyle{break}
\theoremheaderfont{\normalfont\bfseries}
\theoremsymbol{}

\theorembodyfont{\upshape}
\newtheorem{oefening}{Oefening}[section]
\newtheorem{oefening2}{Oefening}
\newtheorem{soefening}{Samengestelde oefening}
\newtheorem{vraag}[oefening2] {Vraag}
\newtheorem{oefeningen}[oefening]{Oefeningen}
\newtheorem{taak}{Taak}
\newtheorem{definitie}{Definitie}[section]
\newtheorem{voorbeeld}[definitie]{Voorbeeld}

%nieuwe commando's
\newcommand{\qrslidesA}[1]{Lesmateriaal gebruikt in het A-programma: \vspace{3mm}\\  \qrcode{#1} \href{#1}{\tiny{#1}}\\\vspace{3mm}}
\newcommand{\qrslidesB}[1]{Lesmateriaal gebruikt in het B-programma: \vspace{3mm}\\  \qrcode{#1} \href{#1}{\tiny{#1}}\\\vspace{3mm}}
\newcommand{\qrclip}[1]{\qrcode{#1} clip: \href{#1}{\tiny{#1}}}
\newcommand{\qrapplet}[1]{\qrcode{#1} applet: \href{#1}{\tiny{#1}}}

%bladschikking
\voffset=-1in
\topmargin=1.5 cm
\headheight=2cm
\headsep=1cm
\textheight=22cm
\footskip=1.2cm
\hoffset=-1in
\oddsidemargin=2.5cm
\evensidemargin=2.5cm
\textwidth=15.5cm
\marginparsep=0cm
\marginparwidth=0cm
\setlength{\parindent}{0pt}
\setlength{\parskip}{5 pt}

\newcounter{module}



%hoofding +kop- en voettekst
\usepackage{fancyhdr}
\pagestyle{fancy}
\renewcommand{\sectionmark}[1]{\markboth{\small{\textsc{\thesection.\ #1}}}{}}
\def\hoofding #1#2#3{\newpage
							\thispagestyle{empty}

             \hrule
             \begin{center}
             {\large {\textsc{Zomercursus Wiskunde}}}

             \bigskip
%             \begin{minipage}[c]{2cm}\includegraphics[width=2cm]{./../macros/sedes.jpg}\end{minipage}

             \bigskip
             {\textsc{KU Leuven}}\\
						 {\textsc{Groep Wetenschap \& Technologie}}
						
						 \bigskip
             {\textsc{September 2019} }
             \bigskip

             {\large \textsc {Module \arabic{module}\\ \bigskip #1}} \\
             (\small{versie \today})
             \bigskip

             \end{center}

             \hrule

             \bigskip
#2

\bigskip

#3

             
						 \newpage
						 \lfoot{\fancyplain{\textsc{\small Zomercursus Wiskunde\\Groep Wetenschap \& Technologie, KU Leuven}}
					        {\textsc{\small Zomercursus Wiskunde\\Groep Wetenschap \& Technologie, KU Leuven}}}%tekst linksonder
					   \cfoot{\fancyplain{}{}}
			       \rfoot{}
			           {}
			            
			       \fancyhead{}
             \fancyhead[L]{\small{\textsc{Module \arabic{module}: #1}}}

  					 \clearpage{\pagestyle{empty}\cleardoublepage}
						 \tableofcontents
  					 \clearpage{\pagestyle{empty}\cleardoublepage}
			\pagestyle{fancy}
			\newpage
			\setcounter{page}{1}



%\lhead[\fancyplain{}{\thepage}]{\fancyplain{}{\small{\leftmark}}}
%\rhead[\fancyplain{}{\small{\textsc{#1}}}]{\fancyplain{}{\thepage}}

%\fancyhead[LE,RO]{\thepage}
%\fancyhead[LO]{\small{\textsc{Module \arabic{module}\\ #1}}}
%\fancyhead[RE]{\leftmark}
\fancyhead{}
\fancyhead[RO]{\textsc{\arabic{module} - \thepage} \\ \small{\textsc{ Module \arabic{module}: #1}} }
\fancyhead[LE]{\textsc{\arabic{module} - \thepage} \\ \small{\textsc{Module \arabic{module}: #1}}}



}

%definities nieuwe commando's - afkortingen veel gebruikte symbolen
\newcommand{\ds}{\displaystyle}
\newcommand{\R}{\ensuremath{\mathbb{R}}}
\newcommand{\Rnul}{\ensuremath{\mathbb{R}_0}}
\newcommand{\Reen}{\ensuremath{\mathbb{R}\setminus\{1\}}}
\newcommand{\Rnuleen}{\ensuremath{\mathbb{R}\setminus\{0,1\}}}
\newcommand{\Rplus}{\ensuremath{\mathbb{R}^+}}
\newcommand{\Rmin}{\ensuremath{\mathbb{R}^-}}
\newcommand{\Rnulplus}{\ensuremath{\mathbb{R}_0^+}}
\newcommand{\Rnulmin}{\ensuremath{\mathbb{R}_0^-}}
\newcommand{\Rnuleenplus}{\ensuremath{\mathbb{R}^+\setminus\{0,1\}}}
\newcommand{\N}{\ensuremath{\mathbb{N}}}
\newcommand{\Nnul}{\ensuremath{\mathbb{N}_0}}
\newcommand{\Z}{\ensuremath{\mathbb{Z}}}
\newcommand{\Znul}{\ensuremath{\mathbb{Z}_0}}
\newcommand{\Zplus}{\ensuremath{\mathbb{Z}^+}}
\newcommand{\Zmin}{\ensuremath{\mathbb{Z}^-}}
\newcommand{\Znulplus}{\ensuremath{\mathbb{Z}_0^+}}
\newcommand{\Znulmin}{\ensuremath{\mathbb{Z}_0^-}}
\newcommand{\C}{\ensuremath{\mathbb{C}}}
\newcommand{\Cnul}{\ensuremath{\mathbb{C}_0}}
\newcommand{\Cplus}{\ensuremath{\mathbb{C}^+}}
\newcommand{\Cmin}{\ensuremath{\mathbb{C}^-}}
\newcommand{\Cnulplus}{\ensuremath{\mathbb{C}_0^+}}
\newcommand{\Cnulmin}{\ensuremath{\mathbb{C}_0^-}}
\newcommand{\Q}{\ensuremath{\mathbb{Q}}}
\newcommand{\Qnul}{\ensuremath{\mathbb{Q}_0}}
\newcommand{\Qplus}{\ensuremath{\mathbb{Q}^+}}
\newcommand{\Qmin}{\ensuremath{\mathbb{Q}^-}}
\newcommand{\Qnulplus}{\ensuremath{\mathbb{Q}_0^+}}
\newcommand{\Qnulmin}{\ensuremath{\mathbb{Q}_0^-}}
\newcommand{\perdef}{\overset{\mathrm{def}}{=}}
\newcommand{\pernot}{\overset{\mathrm{not}}{=}}
\newcommand{\bgsin}{\mathrm{bgsin}\,}
\newcommand{\bgcos}{\mathrm{bgcos}\,}
\newcommand{\bgtan}{\mathrm{bgtan}\,}
\newcommand{\bgcot}{\mathrm{bgcot}\,}
\newcommand{\bgsinh}{\mathrm{bgsinh}\,}
\newcommand{\bgcosh}{\mathrm{bgcosh}\,}
\newcommand{\bgtanh}{\mathrm{bgtanh}\,}
\newcommand{\bgcoth}{\mathrm{bgcoth}\,}
\newcommand{\Bgsin}{\mathrm{Bgsin}\,}
\newcommand{\Bgcos}{\mathrm{Bgcos}\,}
\newcommand{\Bgtan}{\mathrm{Bgtan}\,}
\newcommand{\Bgcot}{\mathrm{Bgcot}\,}
\newcommand{\Bgsinh}{\mathrm{Bgsinh}\,}
\newcommand{\Bgcosh}{\mathrm{Bgcosh}\,}
\newcommand{\Bgtanh}{\mathrm{Bgtanh}\,}
\newcommand{\Bgcoth}{\mathrm{Bgcoth}\,}
\newcommand{\cosec}{\mathrm{cosec}\,}
\newcommand{\dom}{\mathrm{dom}\,}
\newcommand{\bld}{\mathrm{bld}\,}
\newcommand{\graf}{\mathrm{graf}\,}
\newcommand{\rc}{\mathrm{rc}\,}
\newcommand{\co}{\mathrm{co}\,}
\newcommand{\oefverwijzing}[1]{\ensuremath{\hookrightarrow}\ \textsl{#1}}
\newcommand{\startletternummering}{\renewcommand{\labelenumi}{(\alph{enumi})}}
\newcommand{\eindeletternummering}{\renewcommand{\labelenumi}{\arabic{enumi}.}}
\newcommand{\bron}[1]{\begin{scriptsize} \emph{#1} \end{scriptsize}} 

\usepackage{framed}
\usepackage{array}
\usepackage{booktabs}
\usepackage{multirow}
\usepackage{color}





\newtheorem{toepassing}[definitie]{Toepassing}
\newtheorem{opmerking}[definitie]{Opmerking}
\newtheorem{notatie}[definitie]{Notatie}
\newtheorem{voorbeelden}[definitie]{Voorbeelden}
\newtheorem{toepassingen}[definitie]{Toepassingen}
\newtheorem{opmerkingen}[definitie]{Opmerkingen}
\newtheorem{notaties}[definitie]{Notaties}
\newtheorem{voorbeeldoefening}[definitie]{Voorbeeldoefening}
\newtheorem{voorbeeldoefeningen}[definitie]{Voorbeeldoefeningen}
\newtheorem{opdrn}[definitie]{Opdrachten}
\newframedtheorem{kaderdefinitie}[definitie]{Definitie}
\newframedtheorem{kadernotatie}[definitie]{Notatie}
\newframedtheorem{kadernotaties}[definitie]{Notaties}


\theorembodyfont{\itshape}
\newtheorem{stelling}[definitie]{Stelling}
\newtheorem{eigenschap}[definitie]{Eigenschap}
\newtheorem{resultaat}[definitie]{Resultaat}
\newtheorem{lemma}[definitie]{Lemma}
\newtheorem{propositie}[definitie]{Propositie}
\newtheorem{rekenregel}[definitie]{Rekenregel}
\newtheorem{bijzondergeval}[definitie]{Bijzonder geval}
\newtheorem{eigenschappen}[definitie]{Eigenschappen}
\newtheorem{rekenregels}[definitie]{Rekenregels}
\newtheorem{bijzonderegevallen}[definitie]{Bijzondere gevallen}
\newframedtheorem{kaderstelling}[definitie]{Stelling}
\newframedtheorem{kaderbijzonderegevallen}[definitie]{Bijzondere gevallen}
\newframedtheorem{kadereigenschap}[definitie]{Eigenschap}
\newframedtheorem{kaderresultaat}[definitie]{Resultaat}
\newframedtheorem{kaderlemma}[definitie]{Lemma}
\newframedtheorem{kaderpropositie}[definitie]{Propositie}
\newframedtheorem{kaderrekenregel}[definitie]{Rekenregel}
\newframedtheorem{kaderbijzondergeval}[definitie]{Bijzonder geval}
\newframedtheorem{kadereigenschappen}[definitie]{Eigenschappen}
\newframedtheorem{kaderrekenregels}[definitie]{Rekenregels}

\theoremstyle{nonumberbreak}
\theorembodyfont{\upshape}
\theoremindent\parindent
\newtheorem{oplossing}{Oplossing}
\newtheorem{uitwerking}{Uitwerking}
\newtheorem{werkwijze}{Werkwijze}
\theoremsymbol{\ensuremath{_\blacksquare}}
\newtheorem{bewijs}{Bewijs}
\theoremindent0cm
\theoremsymbol{}
\newtheorem{oplossingen}[definitie]{Oplossingen}

\theoremstyle{nonumberplain}
\theorembodyfont{\normalfont}
\theoremseparator{\hspace{-1ex}}
\newframedtheorem{kader}{}
%\newshadedtheorem{grijs}{} 
