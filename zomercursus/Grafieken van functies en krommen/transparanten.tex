\include{./../macros/macros_zc_basic}%macro's laden
\include{./../macros/macros_zc_extra}

 
\hyphenation{}%foutief afgebroken woorden, gesplitst en door spaties gescheiden, opgeven


\usepackage{pstricks}

\begin{document}

\parindent 0mm
\newcommand{\be}{\begin{equation}}
\newcommand{\ee}{\end{equation}}
\newcommand{\ba}{\begin{array}}
\newcommand{\ea}{\end{array}}
\newcommand{\bea}{\begin{eqnarray}}
\newcommand{\eea}{\end{eqnarray}}
\newcommand{\lba}{\left[\begin{array}}
\newcommand{\ear}{\end{array}\right]}
%\newcommand{\R}{\mathbb{R}}
%\newcommand{\C}{\mathbb{C}}
%\newcommand{\Z}{\mathbb{Z}}
\newcommand{\PP}{\mathbb{P}}
\newcommand{\col}{\mbox{Col }}
\newcommand{\Col}{\mbox{Col }}
\newcommand{\row}{\mbox{Row }}
\newcommand{\Row}{\mbox{Row }}
\newcommand{\nul}{\mbox{Nul }}
\newcommand{\Nul}{\mbox{Nul }}
\newcommand{\spn}{\mbox{Span}}
\newcommand{\rank}{\mbox{rank }}
\newcommand{\ddim}{\mbox{dim }}
\newcommand{\vgl}{\mbox{vgl}}
\newcommand{\ddet}{\mbox{det }}
\renewcommand{\det}{\mbox{det }}
\newcommand{\tr}{\mbox{tr }}
\newcommand{\ssp}{\mbox{sp }}
\newcommand{\proj}{\mbox{proj}}
\newcommand{\spieg}{\mbox{spieg}}
\newcommand{\ora}{\overrightarrow}
\newcommand{\mb}{\mathbf}

\theoremstyle{break}
\theoremheaderfont{\normalfont\bfseries}
\theoremsymbol{}
\theorembodyfont{\upshape}
\newtheorem{probleem}[definitie]{Probleem}
\newtheorem{interpretatie}[definitie]{Interpretatie}
\newtheorem{algoritme}[definitie]{Algoritme}

\newcommand{\spil}[1]{\mbox{\put(7,0){\makebox(0,0)[b]{$#1$}}
    \put(7,4){\circle{14}} \hspace{14 pt}}} 
\newcommand{\spill}[1]{\mbox{\put(8,0){\makebox(0,0)[b]{$#1$}}
    \put(8,4){\circle{16}} \hspace{16 pt}}} %eigen definities
%\maketitle\thispagestyle{fancy}%drukt titel af en zorgt voor voettekst op titelpagina

\[
f:[0,1]\rightarrow \R:x\mapsto x^2
\]
\[
f:\R \rightarrow \R:x\mapsto 1+x^2-\frac{1}{2}x^4
\]

\[
f:\R^2 \rightarrow \R:(x,y)\mapsto x^2+y^2
\]

\[
f:\R \rightarrow \R^2:t\mapsto (t,t)
\]

\[
\ba{l}
f\\
g\\
f+g\\
f-g\\
f\cdot g\\
f/g
\ea
\]

\vspace{5mm}
\centerline{\hspace{3cm}\input{fig/pres/pres_gridgon.pst}}
\centerline{\hspace{3cm}\input{fig/pres/pres_x__cosx.pst}}
\centerline{\hspace{3cm}\input{fig/pres/pres_x+cosx.pst}}
\centerline{\hspace{3cm}\input{fig/pres/pres_xcosx.pst}}
\centerline{\hspace{3cm}\input{fig/pres/pres_cosx.pst}}
\centerline{\hspace{3cm}\input{fig/pres/pres_cosx_x.pst}}
\vspace{50mm}


%%%%%%%%%%%%%%%%%%%%%%%%%

Bepaal $f\circ g$ en $g\circ f$ met $ f:x\mapsto \sin(x+2) $ en $ g:x\mapsto x+1$

Bepaal $f\circ g$ en $g\circ f$ met $ f:x\mapsto x+e^x $ en $ g:x\mapsto 2x+1$

Schets $f\circ g$ en $g\circ f$ met $ f:x\mapsto x+2 $

Schets $f\circ g$ en $g\circ f$ met $ f:x\mapsto 2x $

Schets $f\circ g$ en $g\circ f$ met $ f:x\mapsto -x $

Schets de krommen met  parametrisatie $ \phi: [0,2\pi] \rightarrow \R^2: t\mapsto (2\cos(t),4\sin(t))$

Schets de krommen met  parametrisatie $ \phi: \R \rightarrow \R^2: t\mapsto (t,t^2)$

\centerline{\hspace{3cm}\input{fig/pres/pres05.pst}}

\centerline{\hspace{3cm}\input{fig/pres/pres_grid14.pst}}
\centerline{\hspace{3cm}\input{fig/pres/pres_grid05.pst}}

Schets de kromme met vergelijking in poolco\"ordinaten $r=1$

Schets de kromme met vergelijking in poolco\"ordinaten $2r=\theta$


Schets de grafiek van de volgende functie en zijn inverse. 

Bepaal het voorschrift van de inverse.

$ h: [0,4] \rightarrow \R: x\mapsto
 \left\{\ba{ll} 2 x^2 & \mbox{ als } x \in [0,1]\\
               x+1   & \mbox{ als } x \in [1,4]
       \ea\right.
$

\end{document}
