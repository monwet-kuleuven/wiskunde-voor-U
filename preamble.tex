% 201908: PAS OP: babel lijkt problemen te veroorzaken met online math
% (wegens syntax error met toegevoegde \newcommands ...)
%\usepackage[dutch]{babel}

%\usepackage[a4paper]{geometry}
%\usepackage{geometry}   % probleem met .sty files 


\usepackage[utf8]{inputenc}

\usepackage{morewrites}   % nav zomercursus (answer...?)

\usepackage{multicol}
\usepackage{tikzsymbols}
\usepackage{ifthen}
\usepackage{animate}

\usepackage{ulem} % \sout voor strike-out


\usetikzlibrary{calc,trees,positioning,arrows,fit,shapes,calc}
\usetikzlibrary{decorations.markings}

% Met "\newcommand\showtodonotes{}" kan je todonotes tonen (in pdf/online)
% 201908: online werkt het niet (goed)
\providecommand\showtodonotes{disable}
\usepackage[\showtodonotes]{todonotes}
%\usepackage{todonotes}

%
% \xmxxx commands: Extra functionaliteit van, boven of naast Ximera
%
% (Met een minimale ximera.cls en preamble.tex zou een bruikbare .pdf moeten kunnen worden gemaakt van eender welke ximera)
%
% Adhoc command om oplossingen blauw te printen. Nog uit te breiden zodat ze ook optioneel kunnen worden getoond (cfr handout mode)
\newcommand{\xmopl}[1]{{\color{blue}\;#1}}
%
% Usage: \xmtitle[Mijn korte abstract]{Mijn titel}{Mijn abstract}
% Eerste command na \begin{document}:
%  -> definieert de \title
%  -> definieert de abstract
%  -> doet \maketitle ( dus: print de hoofding als 'chapter' of 'sectie')
% Optionele parameter geeft eenn kort abstract (die met de globale setting \xmshortabstract{} al dan niet kan worden geprint.
% De optionele korte abstract kan worden gebruikt voor pseudo-grappige abtsarts, dus dus globaal al dan niet kunnen worden gebuikt...
% Globale settings:
%  de (optionele) 'korte abstract' wordt enkele getoond als \xmshortabstract is gezet
\providecommand\xmshortabstract{} % default: print (only!) short abstract if present
\newcommand{\xmtitle}[3][]{
\title{#2}
\begin{abstract}
\ifdefined\xmshortabstract
\ifstrempty{#1}{%
#3
}{%
#1
}%
\else
#3
\fi
\end{abstract}
\maketitle
}
    
\graphicspath{
    {./}
    {goniometrie/}
    {limieten/}
    {zomercursus/M09/}
}


% we willen (bijna) altijd \geqslant ipv \geq ...!
\newcommand{\geqnoslant}{\geq}
\renewcommand{\geq}{\geqslant}
\newcommand{\leqnoslant}{\leq}
\renewcommand{\leq}{\leqslant}

% Shortcuts voor limieten
% MERK OP: hier kan dus ook de notatie voor linker/rechterlimiet worden gekozen !!!
% Usage: \limx geeft lim voor x-> 0;  \limx[a^2]  geeft lim voor x-> a^2 en \limxi geeft lim voor x -> \infy \limxmi -> -\infty 

% Mmm, zonder de \ifblank lijkt het niet te werken in htlatex ...?
%\newcommand{\limx}[1][]{\lim_{x \rightarrow \ifblank{#1}{0}{#1}}}

\newcommand{\limxg}[1]{\lim_{x \rightarrow #1}}    % g  voor Generic ...
\newcommand{\llimxg}[1]{\lim_{x \rightarrow {#1}^-}}
\newcommand{\rlimxg}[1]{\lim_{x \rightarrow {#1}^+}}
%\newcommand{\llimxg}[1]{\lim_{x \underset{<}{\rightarrow} {#1}}}  % werkt niet in html?
%\newcommand{\rlimxc}{\lim_{x \underset > \rightarrow c}}  % werkt niet in html?
% < gave illegal macro definition
%\text{\textless} showed "\textless" in html
%\DeclareUnicodeCharacter{227A}{\xgt}
%\DeclareUnicodeCharacter{227B}{\xlt}
%\newcommand{\llimx}{\lim_{x \underset{\char8826} \rightarrow 0}}
%\newcommand{\rlimx}{\lim_{x \underset{\char8826} \rightarrow 0}}

\newcommand{\limx}{\limxg{0}}     % limieten naar 0
\newcommand{\llimx}{\llimxg{0}}
\newcommand{\rlimx}{\rlimxg{0}}

\newcommand{\limxi}{\limxg{+\infty}}  % I voor \Infty
\newcommand{\limxmi}{\limxg{-\infty}} % MI voor Min \Infty
\newcommand{\limxc}{\limxg{c}}     % is/was handig in de module over limieten ...
\newcommand{\llimxc}{\llimxg{c}}
\newcommand{\rlimxc}{\rlimxg{c}}

%werkt in math en text mode om MATH met grijze achtergond te tonen (ook \important{\text{blabla}} lijkt te werken)
\newcommand{\important}[1]{\ensuremath{\colorbox{lightgray}{$#1$}}}


%
% Poging tot aanpassen layout
%
%%%%%%%\usepackage{mdframed}  % error incompatible options
\usepackage[framemethod=TikZ]{mdframed}

\usepackage{tcolorbox}
\tcbuselibrary{theorems}


\ifdefined\nouitweiding  
	\excludecomment{uitweiding}
\else
	\theoremstyle{definition}
	\newmdtheoremenv[backgroundcolor=lightgray]{uitweiding}{Uitweiding}
	\AtBeginEnvironment{uitweiding}{\small}
\fi

\ifdefined\xmnouitweiding  
\excludecomment{xmuitweiding}
\else
\theoremstyle{definition}
\newmdtheoremenv[backgroundcolor=lightgray]{xmuitweiding}{Uitweiding}
\AtBeginEnvironment{xmuitweiding}{\small\begin{expandable}}
\AtEndEnvironment{xmuitweiding}{\small\end{expandable}}    
\fi

\newcommand{\xmgrapje}[1]{{\small#1{\reversemarginpar\marginpar{\Smiley}}}}
%\newcommand{\xmgrapje}[1]{}
\newcommand{\grapje}[1]{\xmgrapje{#1}}    % obsolete: use \xmgrapje



%\theoremstyle{definition}
%\newmdtheoremenv[backgroundcolor=lightgray]{grapje}{}
%\AtBeginEnvironment{uitwijding}{\small\Smiley}

% Herdefinieer enkele omgevingen (PAS OP: enkel voor PDF, voor html: zie css..!!!)
% remove italics def
\makeatletter   % because of the @ below: make @ a (normal) letter!!
\let\definition\relax
\let\c@definition\relax
\let\enddefinition\relax
\theoremstyle{definition}
%\newtheorem*{definition}{Definitie}
\newmdtheoremenv{definition}{Definitie}
%\newtcbtheorem[number within=section]{definition}{Definitie}{colback=blue!5,colframe=blue!35!black,fonttitle=\bfseries}{th}
%\newtcbtheorem{definition}{Definitie}{colback=blue!5,colframe=blue!35!black,fonttitle=\bfseries}{th}



% remove italics def
\let\example\relax
\let\c@example\relax
\let\endexample\relax
\theoremstyle{definition}
\newtheorem{example}{Voorbeeld}

% remove italics def
\let\explanation\relax
\let\c@explanation\relax
\let\endexplanation\relax
\theoremstyle{definition}
\newtheorem*{explanation}{Uitleg}

% remove italics defprovidecommand
\let\remark\relax
\let\c@remark\relax
\let\endremark\relax
\theoremstyle{definition}
\newtheorem{remark}{Opmerking}

% remove italics def
\let\proposition\relax
\let\c@proposition\relax
\let\endproposition\relax
%\theoremstyle{proposition}
\newmdtheoremenv{proposition}{Eigenschap}

% remove italics def
\let\problem\relax
\let\c@problem\relax
\let\endproblem\relax
%\theoremstyle{problem}
\newtheorem{problem}{Probleem}

% remove italics def
\let\exercise\relax
\let\c@exercise\relax
\let\endexercise\relax
%\theoremstyle{problem}
\newtheorem{exercise}{Oefening}

% remove italics def
\let\question\relax
\let\c@question\relax
\let\endquestion\relax
%\theoremstyle{problem}
\newtheorem{question}{Vraag}

% remove italics def
\let\notation\relax
\let\c@notation\relax
\let\endnotation\relax
%\theoremstyle{problem}
\newmdtheoremenv{notation}{Notatie}

%% cfr infra %%%\newtheorem*{oplossing}{Oplossing}
\makeatother

%
% Zomercursus
%
\newcounter{module}

%
% Zomercursus environments
%
\newcommand\newframedtheorem\newtheorem
%
\newtheorem{oefening}{Oefening}[section]
\newtheorem{oefening2}{Oefening}
\newtheorem{soefening}{Samengestelde oefening}
\newtheorem{vraag}[oefening2] {Vraag}
\newtheorem{oefeningen}[oefening]{Oefeningen}
\newtheorem{taak}{Taak}
\newtheorem{definitie}{Definitie}[section]
\newtheorem{voorbeeld}[definitie]{Voorbeeld}

\newtheorem{toepassing}[definitie]{Toepassing}
\newtheorem{opmerking}[definitie]{Opmerking}
\newtheorem{notatie}[definitie]{Notatie}
\newtheorem{voorbeelden}[definitie]{Voorbeelden}
\newtheorem{toepassingen}[definitie]{Toepassingen}
\newtheorem{opmerkingen}[definitie]{Opmerkingen}
\newtheorem{notaties}[definitie]{Notaties}
\newtheorem{voorbeeldoefening}[definitie]{Voorbeeldoefening}
\newtheorem{voorbeeldoefeningen}[definitie]{Voorbeeldoefeningen}
\newtheorem{opdrn}[definitie]{Opdrachten}

\newframedtheorem{kaderdefinitie}[definitie]{Definitie}
\newframedtheorem{kadernotatie}[definitie]{Notatie}
\newframedtheorem{kadernotaties}[definitie]{Notaties}


%\theorembodyfont{\itshape}
\newtheorem{stelling}[definitie]{Stelling}
\newtheorem{eigenschap}[definitie]{Eigenschap}
\newtheorem{resultaat}[definitie]{Resultaat}
%\newtheorem{lemma}[definitie]{Lemma}
\newtheorem{propositie}[definitie]{Propositie}
\newtheorem{rekenregel}[definitie]{Rekenregel}
\newtheorem{bijzondergeval}[definitie]{Bijzonder geval}
\newtheorem{eigenschappen}[definitie]{Eigenschappen}
\newtheorem{rekenregels}[definitie]{Rekenregels}
\newtheorem{bijzonderegevallen}[definitie]{Bijzondere gevallen}
\newframedtheorem{kaderstelling}[definitie]{Stelling}
\newframedtheorem{kaderbijzonderegevallen}[definitie]{Bijzondere gevallen}
\newframedtheorem{kadereigenschap}[definitie]{Eigenschap}
\newframedtheorem{kaderresultaat}[definitie]{Resultaat}
\newframedtheorem{kaderlemma}[definitie]{Lemma}
\newframedtheorem{kaderpropositie}[definitie]{Propositie}
\newframedtheorem{kaderrekenregel}[definitie]{Rekenregel}
\newframedtheorem{kaderbijzondergeval}[definitie]{Bijzonder geval}
\newframedtheorem{kadereigenschappen}[definitie]{Eigenschappen}
\newframedtheorem{kaderrekenregels}[definitie]{Rekenregels}

%\theoremstyle{nonumberbreak}
%\theorembodyfont{\upshape}
%\theoremindent\parindent
\newtheorem{oplossing}{Oplossing}
\newtheorem{uitwerking}{Uitwerking}
\newtheorem{werkwijze}{Werkwijze}
%\theoremsymbol{\ensuremath{_\blacksquare}}
\newtheorem{bewijs}{Bewijs}
%\theoremindent0cm
%\theoremsymbol{}
\newtheorem{oplossingen}[definitie]{Oplossingen}

%\theoremstyle{nonumberplain}
%\theorembodyfont{\normalfont}
%\theoremseparator{\hspace{-1ex}}
\newframedtheorem{kader}{}
%\newshadedtheorem{grijs}{} 


%
% Special packages with setup
%
\usepackage{answers}

%definities nieuwe omgevingen
\Newassociation{opl}{Oplossing}{ans}
\Newassociation{numopl}{NumOplossing}{numans}
%\theoremstyle{break}
%\theoremheaderfont{\normalfont\bfseries}
%\theoremsymbol{}


%definities nieuwe commando's - afkortingen veel gebruikte symbolen
\newcommand{\ds}{\displaystyle}
\newcommand{\R}{\ensuremath{\mathbb{R}}}
\newcommand{\Rnul}{\ensuremath{\mathbb{R}_0}}
\newcommand{\Reen}{\ensuremath{\mathbb{R}\setminus\{1\}}}
\newcommand{\Rnuleen}{\ensuremath{\mathbb{R}\setminus\{0,1\}}}
\newcommand{\Rplus}{\ensuremath{\mathbb{R}^+}}
\newcommand{\Rmin}{\ensuremath{\mathbb{R}^-}}
\newcommand{\Rnulplus}{\ensuremath{\mathbb{R}_0^+}}
\newcommand{\Rnulmin}{\ensuremath{\mathbb{R}_0^-}}
\newcommand{\Rnuleenplus}{\ensuremath{\mathbb{R}^+\setminus\{0,1\}}}
\newcommand{\N}{\ensuremath{\mathbb{N}}}
\newcommand{\Nnul}{\ensuremath{\mathbb{N}_0}}
\newcommand{\Z}{\ensuremath{\mathbb{Z}}}
\newcommand{\Znul}{\ensuremath{\mathbb{Z}_0}}
\newcommand{\Zplus}{\ensuremath{\mathbb{Z}^+}}
\newcommand{\Zmin}{\ensuremath{\mathbb{Z}^-}}
\newcommand{\Znulplus}{\ensuremath{\mathbb{Z}_0^+}}
\newcommand{\Znulmin}{\ensuremath{\mathbb{Z}_0^-}}
\newcommand{\C}{\ensuremath{\mathbb{C}}}
\newcommand{\Cnul}{\ensuremath{\mathbb{C}_0}}
\newcommand{\Cplus}{\ensuremath{\mathbb{C}^+}}
\newcommand{\Cmin}{\ensuremath{\mathbb{C}^-}}
\newcommand{\Cnulplus}{\ensuremath{\mathbb{C}_0^+}}
\newcommand{\Cnulmin}{\ensuremath{\mathbb{C}_0^-}}
\newcommand{\Q}{\ensuremath{\mathbb{Q}}}
\newcommand{\Qnul}{\ensuremath{\mathbb{Q}_0}}
\newcommand{\Qplus}{\ensuremath{\mathbb{Q}^+}}
\newcommand{\Qmin}{\ensuremath{\mathbb{Q}^-}}
\newcommand{\Qnulplus}{\ensuremath{\mathbb{Q}_0^+}}
\newcommand{\Qnulmin}{\ensuremath{\mathbb{Q}_0^-}}
\newcommand{\perdef}{\overset{\mathrm{def}}{=}}
\newcommand{\pernot}{\overset{\mathrm{not}}{=}}
\newcommand\perinderdaad{\overset{!}{=}}     % voorlopig gebruikt in limietenrekenregels
\newcommand\perhaps{\overset{?}{=}}          % voorlopig gebruikt in limietenrekenregels
\newcommand{\bgsin}{\mathrm{bgsin}\,}
\newcommand{\bgcos}{\mathrm{bgcos}\,}
\newcommand{\bgtan}{\mathrm{bgtan}\,}
\newcommand{\bgcot}{\mathrm{bgcot}\,}
\newcommand{\bgsinh}{\mathrm{bgsinh}\,}
\newcommand{\bgcosh}{\mathrm{bgcosh}\,}
\newcommand{\bgtanh}{\mathrm{bgtanh}\,}
\newcommand{\bgcoth}{\mathrm{bgcoth}\,}
\newcommand{\Bgsin}{\mathrm{Bgsin}\,}
\newcommand{\Bgcos}{\mathrm{Bgcos}\,}
\newcommand{\Bgtan}{\mathrm{Bgtan}\,}
\newcommand{\Bgcot}{\mathrm{Bgcot}\,}
\newcommand{\Bgsinh}{\mathrm{Bgsinh}\,}
\newcommand{\Bgcosh}{\mathrm{Bgcosh}\,}
\newcommand{\Bgtanh}{\mathrm{Bgtanh}\,}
\newcommand{\Bgcoth}{\mathrm{Bgcoth}\,}
\newcommand{\cosec}{\mathrm{cosec}\,}
\newcommand{\dom}{\mathrm{dom}\,}
\newcommand{\codom}{\mathrm{codom}\,}
\newcommand{\bld}{\mathrm{bld}\,}
\newcommand{\graf}{\mathrm{graf}\,}
\newcommand{\rc}{\mathrm{rc}\,}
\newcommand{\co}{\mathrm{co}\,}
\newcommand{\oefverwijzing}[1]{\ensuremath{\hookrightarrow}\ \textsl{#1}}
\newcommand{\startletternummering}{\renewcommand{\labelenumi}{(\alph{enumi})}}
\newcommand{\eindeletternummering}{\renewcommand{\labelenumi}{\arabic{enumi}.}}
\newcommand{\bron}[1]{\begin{scriptsize} \emph{#1} \end{scriptsize}} 

%
% Uit zcdef.tex; willen we dit wel ...????
%
\newcommand{\be}{\begin{equation}}
\newcommand{\ee}{\end{equation}}
\newcommand{\ba}{\begin{array}}
\newcommand{\ea}{\end{array}}
\newcommand{\bea}{\begin{eqnarray}}
\newcommand{\eea}{\end{eqnarray}}
\newcommand{\lba}{\left[\begin{array}}
\newcommand{\ear}{\end{array}\right]}

%
% commando's voor QR codes
%
\newcommand{\qrslidesA}[1]{Lesmateriaal gebruikt in het A-programma: \vspace{3mm}\\  \qrcode{#1} \href{#1}{\tiny{#1}}\\\vspace{3mm}}
\newcommand{\qrslidesB}[1]{Lesmateriaal gebruikt in het B-programma: \vspace{3mm}\\  \qrcode{#1} \href{#1}{\tiny{#1}}\\\vspace{3mm}}
\newcommand{\qrclip}[1]{\qrcode{#1} clip: \href{#1}{\tiny{#1}}}
\newcommand{\qrapplet}[1]{\qrcode{#1} applet: \href{#1}{\tiny{#1}}}

%
% bladschikking
%
\pdfOnly{ %%% Mmm, werkt niet; issue if used in beamer !
\voffset=-1in
\topmargin=1.5 cm
\headheight=2cm
\headsep=1cm
\textheight=22cm
\footskip=1.2cm
\hoffset=-1in
\oddsidemargin=2.5cm
\evensidemargin=2.5cm
\textwidth=15.5cm
\marginparsep=0cm
\marginparwidth=0cm
\setlength{\parindent}{0pt}
\setlength{\parskip}{5 pt}

\def\hoofding #1#2#3{\maketitle}
}

%
%   Gecopieerd calculus preamble
%
%%This is to help with formatting on future title pages.
\newenvironment{sectionOutcomes}{}{}

%
% Aangepast van ximera.cls
%
%  (hoort allicht in een .sty thuis !!!)
%
% Toon feedback zonder label in PDF's
% 
\ifhandout% toon niets
\renewenvironment{feedback}
{%
    \setbox0\vbox\bgroup
}
{%
    \egroup
}
\else  % toon zonder meer (dus zonder 'Feedback(optie) zoals is  ximera.cls
\renewenvironment{feedback}[1][]{
    
%    \def\PH@Command{#1}% Use PH@Command to hold the content and be a target for "\expandafter" to expand once.
    
%    \begin{trivlist}% Begin the trivlist to use formating of the "Feedback" label.
%        \item[\hskip \labelsep\small\slshape\bfseries Feedback% Format the "Feedback" label. Don't forget the space.
%        (\texttt{\detokenize\expandafter{\PH@Command}}):% Format (and detokenize) the condition for feedback to trigger
%        \hspace{2ex}]\small\slshape% Insert some space before the actual feedback given.
    }{
%    \end{trivlist}
}

\fi

 
