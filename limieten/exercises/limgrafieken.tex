\documentclass{ximera}

%
% copied from https://github.com/mooculus/calculus
%
\usepackage[utf8]{inputenc}


\graphicspath{
	{./}
	{goniometrie/}
}


%\usepackage{todonotes}
%\usepackage{mathtools} %% Required for wide table Curl and Greens
%\usepackage{cuted} %% Required for wide table Curl and Greens
\newcommand{\todo}{}

% Font niet (correct?) geinstalleerd in MikTeX?
%\usepackage{esint} % for \oiint
%\ifxake%%https://math.meta.stackexchange.com/questions/9973/how-do-you-render-a-closed-surface-double-integral
%\renewcommand{\oiint}{{\large\bigcirc}\kern-1.56em\iint}
%\fi


\newcommand{\mooculus}{\textsf{\textbf{MOOC}\textnormal{\textsf{ULUS}}}}

\usepackage{tkz-euclide}\usepackage{tikz}
\usepackage{tikz-cd}
\usetikzlibrary{arrows}
\tikzset{>=stealth,commutative diagrams/.cd,
  arrow style=tikz,diagrams={>=stealth}} %% cool arrow head
\tikzset{shorten <>/.style={ shorten >=#1, shorten <=#1 } } %% allows shorter vectors

\usetikzlibrary{backgrounds} %% for boxes around graphs
\usetikzlibrary{shapes,positioning}  %% Clouds and stars
\usetikzlibrary{matrix} %% for matrix
\usepgfplotslibrary{polar} %% for polar plots
\usepgfplotslibrary{fillbetween} %% to shade area between curves in TikZ
\usetkzobj{all}
\usepackage[makeroom]{cancel} %% for strike outs
%\usepackage{mathtools} %% for pretty underbrace % Breaks Ximera
%\usepackage{multicol}
\usepackage{pgffor} %% required for integral for loops



%% http://tex.stackexchange.com/questions/66490/drawing-a-tikz-arc-specifying-the-center
%% Draws beach ball
\tikzset{pics/carc/.style args={#1:#2:#3}{code={\draw[pic actions] (#1:#3) arc(#1:#2:#3);}}}



\usepackage{array}
\setlength{\extrarowheight}{+.1cm}
\newdimen\digitwidth
\settowidth\digitwidth{9}
\def\divrule#1#2{
\noalign{\moveright#1\digitwidth
\vbox{\hrule width#2\digitwidth}}}





\newcommand{\RR}{\mathbb R}
\newcommand{\R}{\mathbb R}
\newcommand{\N}{\mathbb N}
\newcommand{\Z}{\mathbb Z}

\newcommand{\sagemath}{\textsf{SageMath}}


%\renewcommand{\d}{\,d\!}
\renewcommand{\d}{\mathop{}\!d}
\newcommand{\dd}[2][]{\frac{\d #1}{\d #2}}
\newcommand{\pp}[2][]{\frac{\partial #1}{\partial #2}}
\renewcommand{\l}{\ell}
\newcommand{\ddx}{\frac{d}{\d x}}

\newcommand{\zeroOverZero}{\ensuremath{\boldsymbol{\tfrac{0}{0}}}}
\newcommand{\inftyOverInfty}{\ensuremath{\boldsymbol{\tfrac{\infty}{\infty}}}}
\newcommand{\zeroOverInfty}{\ensuremath{\boldsymbol{\tfrac{0}{\infty}}}}
\newcommand{\zeroTimesInfty}{\ensuremath{\small\boldsymbol{0\cdot \infty}}}
\newcommand{\inftyMinusInfty}{\ensuremath{\small\boldsymbol{\infty - \infty}}}
\newcommand{\oneToInfty}{\ensuremath{\boldsymbol{1^\infty}}}
\newcommand{\zeroToZero}{\ensuremath{\boldsymbol{0^0}}}
\newcommand{\inftyToZero}{\ensuremath{\boldsymbol{\infty^0}}}



\newcommand{\numOverZero}{\ensuremath{\boldsymbol{\tfrac{\#}{0}}}}
\newcommand{\dfn}{\textbf}
%\newcommand{\unit}{\,\mathrm}
\newcommand{\unit}{\mathop{}\!\mathrm}
\newcommand{\eval}[1]{\bigg[ #1 \bigg]}
\newcommand{\seq}[1]{\left( #1 \right)}
\renewcommand{\epsilon}{\varepsilon}
\renewcommand{\phi}{\varphi}


\renewcommand{\iff}{\Leftrightarrow}

\DeclareMathOperator{\arccot}{arccot}
\DeclareMathOperator{\arcsec}{arcsec}
\DeclareMathOperator{\arccsc}{arccsc}
\DeclareMathOperator{\si}{Si}
\DeclareMathOperator{\scal}{scal}
\DeclareMathOperator{\sign}{sign}


%% \newcommand{\tightoverset}[2]{% for arrow vec
%%   \mathop{#2}\limits^{\vbox to -.5ex{\kern-0.75ex\hbox{$#1$}\vss}}}
\newcommand{\arrowvec}[1]{{\overset{\rightharpoonup}{#1}}}
%\renewcommand{\vec}[1]{\arrowvec{\mathbf{#1}}}
\renewcommand{\vec}[1]{{\overset{\boldsymbol{\rightharpoonup}}{\mathbf{#1}}}\hspace{0in}}

\newcommand{\point}[1]{\left(#1\right)} %this allows \vector{ to be changed to \vector{ with a quick find and replace
\newcommand{\pt}[1]{\mathbf{#1}} %this allows \vec{ to be changed to \vec{ with a quick find and replace
\newcommand{\Lim}[2]{\lim_{\point{#1} \to \point{#2}}} %Bart, I changed this to point since I want to use it.  It runs through both of the exercise and exerciseE files in limits section, which is why it was in each document to start with.

\DeclareMathOperator{\proj}{\mathbf{proj}}
\newcommand{\veci}{{\boldsymbol{\hat{\imath}}}}
\newcommand{\vecj}{{\boldsymbol{\hat{\jmath}}}}
\newcommand{\veck}{{\boldsymbol{\hat{k}}}}
\newcommand{\vecl}{\vec{\boldsymbol{\l}}}
\newcommand{\uvec}[1]{\mathbf{\hat{#1}}}
\newcommand{\utan}{\mathbf{\hat{t}}}
\newcommand{\unormal}{\mathbf{\hat{n}}}
\newcommand{\ubinormal}{\mathbf{\hat{b}}}

\newcommand{\dotp}{\bullet}
\newcommand{\cross}{\boldsymbol\times}
\newcommand{\grad}{\boldsymbol\nabla}
\newcommand{\divergence}{\grad\dotp}
\newcommand{\curl}{\grad\cross}
%\DeclareMathOperator{\divergence}{divergence}
%\DeclareMathOperator{\curl}[1]{\grad\cross #1}
\newcommand{\lto}{\mathop{\longrightarrow\,}\limits}

\renewcommand{\bar}{\overline}

\colorlet{textColor}{black}
\colorlet{background}{white}
\colorlet{penColor}{blue!50!black} % Color of a curve in a plot
\colorlet{penColor2}{red!50!black}% Color of a curve in a plot
\colorlet{penColor3}{red!50!blue} % Color of a curve in a plot
\colorlet{penColor4}{green!50!black} % Color of a curve in a plot
\colorlet{penColor5}{orange!80!black} % Color of a curve in a plot
\colorlet{penColor6}{yellow!70!black} % Color of a curve in a plot
\colorlet{fill1}{penColor!20} % Color of fill in a plot
\colorlet{fill2}{penColor2!20} % Color of fill in a plot
\colorlet{fillp}{fill1} % Color of positive area
\colorlet{filln}{penColor2!20} % Color of negative area
\colorlet{fill3}{penColor3!20} % Fill
\colorlet{fill4}{penColor4!20} % Fill
\colorlet{fill5}{penColor5!20} % Fill
\colorlet{gridColor}{gray!50} % Color of grid in a plot

\newcommand{\surfaceColor}{violet}
\newcommand{\surfaceColorTwo}{redyellow}
\newcommand{\sliceColor}{greenyellow}




\pgfmathdeclarefunction{gauss}{2}{% gives gaussian
  \pgfmathparse{1/(#2*sqrt(2*pi))*exp(-((x-#1)^2)/(2*#2^2))}%
}


%%%%%%%%%%%%%
%% Vectors
%%%%%%%%%%%%%

%% Simple horiz vectors
\renewcommand{\vector}[1]{\left\langle #1\right\rangle}


%% %% Complex Horiz Vectors with angle brackets
%% \makeatletter
%% \renewcommand{\vector}[2][ , ]{\left\langle%
%%   \def\nextitem{\def\nextitem{#1}}%
%%   \@for \el:=#2\do{\nextitem\el}\right\rangle%
%% }
%% \makeatother

%% %% Vertical Vectors
%% \def\vector#1{\begin{bmatrix}\vecListA#1,,\end{bmatrix}}
%% \def\vecListA#1,{\if,#1,\else #1\cr \expandafter \vecListA \fi}

%%%%%%%%%%%%%
%% End of vectors
%%%%%%%%%%%%%

%\newcommand{\fullwidth}{}
%\newcommand{\normalwidth}{}



%% makes a snazzy t-chart for evaluating functions
%\newenvironment{tchart}{\rowcolors{2}{}{background!90!textColor}\array}{\endarray}

%%This is to help with formatting on future title pages.
\newenvironment{sectionOutcomes}{}{}



%% Flowchart stuff
%\tikzstyle{startstop} = [rectangle, rounded corners, minimum width=3cm, minimum height=1cm,text centered, draw=black]
%\tikzstyle{question} = [rectangle, minimum width=3cm, minimum height=1cm, text centered, draw=black]
%\tikzstyle{decision} = [trapezium, trapezium left angle=70, trapezium right angle=110, minimum width=3cm, minimum height=1cm, text centered, draw=black]
%\tikzstyle{question} = [rectangle, rounded corners, minimum width=3cm, minimum height=1cm,text centered, draw=black]
%\tikzstyle{process} = [rectangle, minimum width=3cm, minimum height=1cm, text centered, draw=black]
%\tikzstyle{decision} = [trapezium, trapezium left angle=70, trapezium right angle=110, minimum width=3cm, minimum height=1cm, text centered, draw=black]


\author{Wim Obbels}
\license{Creative Commons 3.0 By-bC}


\outcome{}


\begin{document}
\begin{exercise} \ 
	Bepaal de gevraagde limieten op basis van een grafiek.

% todo: find out how image is (supposed to be) working
\begin{image}[\textwidth]
%\begin{figure}[width=\textwidth]
		\begin{tikzpicture}[scale=0.5]
		\begin{axis}
		[
		axis equal,
		ymax=5,ymin=-5,
		samples=200,
		axis lines=center,
		extra y ticks={0},
		restrict y to domain=-10:10,
		]
		\addplot[domain=-5:5,color=blue] {1/x};
		\legend{$1/x$}
		\end{axis}
		\end{tikzpicture}
\quad
		\begin{tikzpicture}[scale=0.5]
			\begin{axis}
			[
			axis equal,
			ymax=5,ymin=-5,
			samples=200,
			axis lines=center,
			extra y ticks={0},
			restrict y to domain=-10:10,
			]
			\addplot[domain=-5:5,color=blue] {1/x^2};
			\legend{$1/x^2$}
			\end{axis}
		\end{tikzpicture}
		\quad
		\begin{tikzpicture}[scale=0.5]
		\begin{axis}
		[
		axis equal,
		ymax=5,ymin=-5,
		samples=200,
		axis lines=center,
		extra y ticks={0},
		restrict y to domain=-10:10,
		]
		\addplot[domain=-5:5,thick,color=blue] {exp(x)};
		\addplot[domain=0.001:5,thick,color=blue,dashed] {ln(x)};
		\legend{$e^x$, $\ln x$}
		\end{axis}
		\end{tikzpicture}
\end{image}
%\end{figure}

	
    % probeersel om gemakkelijk oefeningen te kunnen maken; to be done properly, want dit is (nog) geknoei ... !
	\def\isA{}
	\def\isB{}
	\def\isC{}
	\def\isD{}
	\def\isE{}
	\def\isF{}

	\newcommand{\localoefoptions}{\hfill\wordChoice{\choice[\isA]{$+\infty$}\choice[\isB]{$-\infty$}\choice[\isC]{0}\choice[\isD]{1}\choice[\isE]{bestaat niet}\choice[\isF]{andere oplossing}}}
		
	% todo: limiet van sin x: bestaat niet
	% todo: limiet van sin(1/x): bestaat ook niet


		Deze oefeningen zouden zeer eenvoudig moeten zijn:
		
		\begin{question} 
			{\reversemarginpar\marginpar{\Smiley}}
			\def\isC{correct}   $\limxi    \frac{1}{x^2}= \;$   \localoefoptions 
			\begin{feedback}[correct]Inderdaad, voor grote $x$  gaat $1/x^2$ naar $0$\end{feedback}	
			\begin{feedback}[false]Mmm, wat gebeurt er met $1/x^2$ als $x$ groot wordt? We zoeken de limiet in $+\infty$!\end{feedback}
		\end{question}
		\begin{question} \def\isA{correct}   $\limx     \frac{1}{x^2}= \;$   \localoefoptions 
			\begin{feedback}[correct]Inderdaad, voor (zowel positieve als negatieve!) kleine $x$ wordt $1/x^2$ groot \end{feedback}	
			\begin{feedback}[1==0]Mmm, wat gebeurt er met $1/x^2$ als $x$ klein wordt? We zoeken de limiet in $0$!\end{feedback}
		\end{question}

		\begin{question} \def\isA{correct}   $\limxi    e^x = \;$            \localoefoptions \end{question}
		\begin{question} \def\isB{correct}   $\limxmi   e^x = \;$            \localoefoptions \end{question}
		\begin{question} \def\isB{correct}   $\rlimx    \ln x = \;$          \localoefoptions \end{question}
			{\reversemarginpar\marginpar{\Smiley}}
		\begin{question} \def\isE{correct}   $\llimx    \ln x = \;$          \localoefoptions \end{question}
		\begin{question} \def\isE{correct}   $\limxz    \ln x = \;$          \localoefoptions \end{question}
		De volgende oefeningen vragen net iets meer nadenken, maar het zou toch niet al te moeilijk moeten zijn om het gedrag van bijvoorbeeld $e^{\frac1x}$ voor $x\to\infty$ te vinden als je het gedrag van de functies $e^x$ en $\frac1x$ kent.				
		\begin{question} \def\isD{correct}   $\limxi    e^{\frac 1x} = \;$   \localoefoptions \end{question}
		\begin{question} \def\isC{correct}   $\limxmi   e^{\frac 1x} = \;$   \localoefoptions \end{question}
		\begin{question} \def\isE{correct}   $\limxz     e^{\frac 1x} = \;$   \localoefoptions \end{question}
		\begin{question} \def\isA{correct}   $\rlimx    e^{\frac 1x} = \;$   \localoefoptions \end{question}
		\begin{question} \def\isC{correct}   $\llimx    e^{\frac 1x} = \;$   \localoefoptions \end{question}
		\begin{question} \def\isC{correct}   $\limxi    \ln(1+ \frac 1x) = \;$ \localoefoptions \end{question}
		
\end{exercise}

\end{document}
