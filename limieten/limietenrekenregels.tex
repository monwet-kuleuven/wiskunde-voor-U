\documentclass{ximera}
%\documentclass[handout,numbers]{ximera}
% todo: proper use of options of documentclass to be investigated!!!


\author{Zomercursus KU Leuven}

\outcome{Limieten kunnen berekenen}
\outcome{Onbepaaldheden met oneindig kunnen herkennen}

%
% copied from https://github.com/mooculus/calculus
%
\usepackage[utf8]{inputenc}


\graphicspath{
	{./}
	{goniometrie/}
}


%\usepackage{todonotes}
%\usepackage{mathtools} %% Required for wide table Curl and Greens
%\usepackage{cuted} %% Required for wide table Curl and Greens
\newcommand{\todo}{}

% Font niet (correct?) geinstalleerd in MikTeX?
%\usepackage{esint} % for \oiint
%\ifxake%%https://math.meta.stackexchange.com/questions/9973/how-do-you-render-a-closed-surface-double-integral
%\renewcommand{\oiint}{{\large\bigcirc}\kern-1.56em\iint}
%\fi


\newcommand{\mooculus}{\textsf{\textbf{MOOC}\textnormal{\textsf{ULUS}}}}

\usepackage{tkz-euclide}\usepackage{tikz}
\usepackage{tikz-cd}
\usetikzlibrary{arrows}
\tikzset{>=stealth,commutative diagrams/.cd,
  arrow style=tikz,diagrams={>=stealth}} %% cool arrow head
\tikzset{shorten <>/.style={ shorten >=#1, shorten <=#1 } } %% allows shorter vectors

\usetikzlibrary{backgrounds} %% for boxes around graphs
\usetikzlibrary{shapes,positioning}  %% Clouds and stars
\usetikzlibrary{matrix} %% for matrix
\usepgfplotslibrary{polar} %% for polar plots
\usepgfplotslibrary{fillbetween} %% to shade area between curves in TikZ
\usetkzobj{all}
\usepackage[makeroom]{cancel} %% for strike outs
%\usepackage{mathtools} %% for pretty underbrace % Breaks Ximera
%\usepackage{multicol}
\usepackage{pgffor} %% required for integral for loops



%% http://tex.stackexchange.com/questions/66490/drawing-a-tikz-arc-specifying-the-center
%% Draws beach ball
\tikzset{pics/carc/.style args={#1:#2:#3}{code={\draw[pic actions] (#1:#3) arc(#1:#2:#3);}}}



\usepackage{array}
\setlength{\extrarowheight}{+.1cm}
\newdimen\digitwidth
\settowidth\digitwidth{9}
\def\divrule#1#2{
\noalign{\moveright#1\digitwidth
\vbox{\hrule width#2\digitwidth}}}





\newcommand{\RR}{\mathbb R}
\newcommand{\R}{\mathbb R}
\newcommand{\N}{\mathbb N}
\newcommand{\Z}{\mathbb Z}

\newcommand{\sagemath}{\textsf{SageMath}}


%\renewcommand{\d}{\,d\!}
\renewcommand{\d}{\mathop{}\!d}
\newcommand{\dd}[2][]{\frac{\d #1}{\d #2}}
\newcommand{\pp}[2][]{\frac{\partial #1}{\partial #2}}
\renewcommand{\l}{\ell}
\newcommand{\ddx}{\frac{d}{\d x}}

\newcommand{\zeroOverZero}{\ensuremath{\boldsymbol{\tfrac{0}{0}}}}
\newcommand{\inftyOverInfty}{\ensuremath{\boldsymbol{\tfrac{\infty}{\infty}}}}
\newcommand{\zeroOverInfty}{\ensuremath{\boldsymbol{\tfrac{0}{\infty}}}}
\newcommand{\zeroTimesInfty}{\ensuremath{\small\boldsymbol{0\cdot \infty}}}
\newcommand{\inftyMinusInfty}{\ensuremath{\small\boldsymbol{\infty - \infty}}}
\newcommand{\oneToInfty}{\ensuremath{\boldsymbol{1^\infty}}}
\newcommand{\zeroToZero}{\ensuremath{\boldsymbol{0^0}}}
\newcommand{\inftyToZero}{\ensuremath{\boldsymbol{\infty^0}}}



\newcommand{\numOverZero}{\ensuremath{\boldsymbol{\tfrac{\#}{0}}}}
\newcommand{\dfn}{\textbf}
%\newcommand{\unit}{\,\mathrm}
\newcommand{\unit}{\mathop{}\!\mathrm}
\newcommand{\eval}[1]{\bigg[ #1 \bigg]}
\newcommand{\seq}[1]{\left( #1 \right)}
\renewcommand{\epsilon}{\varepsilon}
\renewcommand{\phi}{\varphi}


\renewcommand{\iff}{\Leftrightarrow}

\DeclareMathOperator{\arccot}{arccot}
\DeclareMathOperator{\arcsec}{arcsec}
\DeclareMathOperator{\arccsc}{arccsc}
\DeclareMathOperator{\si}{Si}
\DeclareMathOperator{\scal}{scal}
\DeclareMathOperator{\sign}{sign}


%% \newcommand{\tightoverset}[2]{% for arrow vec
%%   \mathop{#2}\limits^{\vbox to -.5ex{\kern-0.75ex\hbox{$#1$}\vss}}}
\newcommand{\arrowvec}[1]{{\overset{\rightharpoonup}{#1}}}
%\renewcommand{\vec}[1]{\arrowvec{\mathbf{#1}}}
\renewcommand{\vec}[1]{{\overset{\boldsymbol{\rightharpoonup}}{\mathbf{#1}}}\hspace{0in}}

\newcommand{\point}[1]{\left(#1\right)} %this allows \vector{ to be changed to \vector{ with a quick find and replace
\newcommand{\pt}[1]{\mathbf{#1}} %this allows \vec{ to be changed to \vec{ with a quick find and replace
\newcommand{\Lim}[2]{\lim_{\point{#1} \to \point{#2}}} %Bart, I changed this to point since I want to use it.  It runs through both of the exercise and exerciseE files in limits section, which is why it was in each document to start with.

\DeclareMathOperator{\proj}{\mathbf{proj}}
\newcommand{\veci}{{\boldsymbol{\hat{\imath}}}}
\newcommand{\vecj}{{\boldsymbol{\hat{\jmath}}}}
\newcommand{\veck}{{\boldsymbol{\hat{k}}}}
\newcommand{\vecl}{\vec{\boldsymbol{\l}}}
\newcommand{\uvec}[1]{\mathbf{\hat{#1}}}
\newcommand{\utan}{\mathbf{\hat{t}}}
\newcommand{\unormal}{\mathbf{\hat{n}}}
\newcommand{\ubinormal}{\mathbf{\hat{b}}}

\newcommand{\dotp}{\bullet}
\newcommand{\cross}{\boldsymbol\times}
\newcommand{\grad}{\boldsymbol\nabla}
\newcommand{\divergence}{\grad\dotp}
\newcommand{\curl}{\grad\cross}
%\DeclareMathOperator{\divergence}{divergence}
%\DeclareMathOperator{\curl}[1]{\grad\cross #1}
\newcommand{\lto}{\mathop{\longrightarrow\,}\limits}

\renewcommand{\bar}{\overline}

\colorlet{textColor}{black}
\colorlet{background}{white}
\colorlet{penColor}{blue!50!black} % Color of a curve in a plot
\colorlet{penColor2}{red!50!black}% Color of a curve in a plot
\colorlet{penColor3}{red!50!blue} % Color of a curve in a plot
\colorlet{penColor4}{green!50!black} % Color of a curve in a plot
\colorlet{penColor5}{orange!80!black} % Color of a curve in a plot
\colorlet{penColor6}{yellow!70!black} % Color of a curve in a plot
\colorlet{fill1}{penColor!20} % Color of fill in a plot
\colorlet{fill2}{penColor2!20} % Color of fill in a plot
\colorlet{fillp}{fill1} % Color of positive area
\colorlet{filln}{penColor2!20} % Color of negative area
\colorlet{fill3}{penColor3!20} % Fill
\colorlet{fill4}{penColor4!20} % Fill
\colorlet{fill5}{penColor5!20} % Fill
\colorlet{gridColor}{gray!50} % Color of grid in a plot

\newcommand{\surfaceColor}{violet}
\newcommand{\surfaceColorTwo}{redyellow}
\newcommand{\sliceColor}{greenyellow}




\pgfmathdeclarefunction{gauss}{2}{% gives gaussian
  \pgfmathparse{1/(#2*sqrt(2*pi))*exp(-((x-#1)^2)/(2*#2^2))}%
}


%%%%%%%%%%%%%
%% Vectors
%%%%%%%%%%%%%

%% Simple horiz vectors
\renewcommand{\vector}[1]{\left\langle #1\right\rangle}


%% %% Complex Horiz Vectors with angle brackets
%% \makeatletter
%% \renewcommand{\vector}[2][ , ]{\left\langle%
%%   \def\nextitem{\def\nextitem{#1}}%
%%   \@for \el:=#2\do{\nextitem\el}\right\rangle%
%% }
%% \makeatother

%% %% Vertical Vectors
%% \def\vector#1{\begin{bmatrix}\vecListA#1,,\end{bmatrix}}
%% \def\vecListA#1,{\if,#1,\else #1\cr \expandafter \vecListA \fi}

%%%%%%%%%%%%%
%% End of vectors
%%%%%%%%%%%%%

%\newcommand{\fullwidth}{}
%\newcommand{\normalwidth}{}



%% makes a snazzy t-chart for evaluating functions
%\newenvironment{tchart}{\rowcolors{2}{}{background!90!textColor}\array}{\endarray}

%%This is to help with formatting on future title pages.
\newenvironment{sectionOutcomes}{}{}



%% Flowchart stuff
%\tikzstyle{startstop} = [rectangle, rounded corners, minimum width=3cm, minimum height=1cm,text centered, draw=black]
%\tikzstyle{question} = [rectangle, minimum width=3cm, minimum height=1cm, text centered, draw=black]
%\tikzstyle{decision} = [trapezium, trapezium left angle=70, trapezium right angle=110, minimum width=3cm, minimum height=1cm, text centered, draw=black]
%\tikzstyle{question} = [rectangle, rounded corners, minimum width=3cm, minimum height=1cm,text centered, draw=black]
%\tikzstyle{process} = [rectangle, minimum width=3cm, minimum height=1cm, text centered, draw=black]
%\tikzstyle{decision} = [trapezium, trapezium left angle=70, trapezium right angle=110, minimum width=3cm, minimum height=1cm, text centered, draw=black]



\title{Limieten: Rekenregels}

\begin{document}
\begin{abstract}
	lim(som) is soms som(lims) 
\end{abstract}
\maketitle

\subsection{Rekenregels voor het berekenen van limieten}

Om efficiënt te rekenen  met limieten is het erg belangrijk om over handige rekenregels te beschikken.

\begin{exandable}
\begin{uitweiding}[Waarom rekenregels ...?] \ 
	
Zoals wel meer gebeurt in de wiskunde (en daarbuiten), proberen we eerst iets simpel en willen we dat daarna zo eenvoudig mogelijk uitbreiden naar meer ingewikkelde situaties. We kunnen nu limieten aflezen van (eenvoudige) grafieken van functies. Maar het zou wel erg onhandig zijn om bijvoorbeeld voor de limieten $\limx \frac{8}{x^2}, \limx (4+\frac{1}{x^2}), \limx ((x+2)^2 + \frac{8}{x^2})$ telkens grafieken te moeten tekenen! %\grapje{Er zijn grenzen aan de waanzin, zelfs voor wiskundigen.}

Dus moeten we hopen dat er (zo eenvoudig mogelijke) regeltjes zijn om meer ingewikkelde limieten te kunnen berekenen op basis van eenvoudige(re) limieten. In een ideale wereld is de limiet van een som gewoon de som van de limieten, en hetzelfde voor verschillen, producten, quotiënten, machten enzovoort. De wereld is niet ideaal, maar het kon slechter. In vele gevallen is rekenen met limieten  zo eenvoudig als het redelijkerwijs kan zijn. Maar, jammer genoeg zijn er enkele venijnige uitzonderingsgevallen. We bekijken eerst enkele voorbeelden.
\end{uitweiding}
\end{expandable}

\begin{example} (Limieten van $x^2 \pm e^x$)
	
	We kunnen de functie $x^2+e^x$ bekijken als de som van de functies $x\mapsto x^2$ en $x\mapsto e^x$. 
	Dat ziet er als volgt uit:
\begin{image}[\width]
	\begin{tikzpicture}[scale=1]
	\begin{axis}
	[
	axis equal,
	ymax=5,ymin=-5,
	samples=200,
	axis lines=center,
	extra y ticks={0},
	restrict y to domain=-10:10,
	]
	\addplot[domain=-5:5,color=blue,dotted] {x^2};
	\addplot[domain=-5:5,color=blue,dotted] {exp(x)};
%	\addplot[domain=-5:5,color=blue,dotted] {2};
	\addplot[domain=-5:5,color=red,thick] {x^2+exp(x)};
	\legend{$x^2 + e^x$}
	\end{axis}
	\end{tikzpicture}	
	\quad
		\begin{tikzpicture}[scale=1]
	\begin{axis}
	[
	axis equal,
	ymax=5,ymin=-5,
	samples=200,
	axis lines=center,
	extra y ticks={0},
	restrict y to domain=-10:10,
	]
	\addplot[domain=-5:5,color=blue,dotted] {x^2};
	\addplot[domain=-5:5,color=blue,dotted] {-exp(x)};
	%	\addplot[domain=-5:5,color=blue,dotted] {2};
	\addplot[domain=-5:5,color=red,thick] {x^2-exp(x)};
	\legend{$x^2 + (-e^x))$}
	\end{axis}
	\end{tikzpicture}	
\end{image}
	
	Op de grafiek is duidelijk dat de limiet van de som $x^2+e^x$ altijd gelijk is aan de som van de limieten van de termen $x^2$ en $e^x$:
	\begin{alignat*}{6}
	% Mmm, alignment nog niet wat het moet zijn ...!
	& \limx &(x^2+e^x)   & = \limx x^2   &+ \limx e^x   & = 0 + 1 = 1 \\
	& \limxi &(x^2+e^x)  & = \limxi x^2  &+ \limxi e^x  & = \infty + \infty = +\infty \\
	& \limxmi &(x^2+e^x) & = \limxmi x^2 &+ \limxmi e^x & = \infty + 0 = \infty \\
	\end{alignat*}
	Mooi en gemakkelijk!
	
	Maar, voor $x^2 - e^x$ duiken er mogelijke problemen op:
	\begin{alignat*}{6}
		% Mmm, alignment nog niet wat het moet zijn ...!
		& \limx &(x^2-e^x)   & = \limx x^2   &- \limx e^x   & = 0 - 1 = -1 \\
		& \limxi &(x^2-e^x)  & = \limxi x^2  &- \limxi e^x  & = \infty - \infty = -\infty (???) \\
		& \limxmi &(x^2-e^x) & = \limxmi x^2 &- \limxmi e^x & = \infty - 0 = \infty \\
	\end{alignat*}
	Het is wat verdacht dat $\infty - \infty = -\infty$ (wat blijkbaar nodig is om $\limxi (x^2-e^x) = \limxi x^2  - \limxi e^x$ te doen kloppen.  En het geeft inderdaad een onoverkomelijk probleem: als we ook willen dat $\limxi (x^2  - e^x) = \limxi x^2  - \limxi e^x  = \infty - \infty$  dan moet $\infty - \infty = 0$, en voor $\limxi ((x+a) - x) = \limxi (x+a) - \limxi x$ moet zelfs $\infty - \infty = a$, met $a$ een willekeurig reëel getal. Miserie miserie ...  

\end{example}
\begin{example}(Extra: Limieten van $x - (a-x)$)
	
	Zij $a\in\R$, en beschouw de functies $f(x)=x$ en $g(x)=a-x$. Dan geldt (evident) dat $f(x)+g(x)=x+(a-x)=a$ en dat 
	\begin{align*}
	\limxi f(x) & = \limxi x = +\infty \\
	\limxi g(x) & = \limxi (a-x) = -\infty \\
	\limxi (g(x)+f(x)) & = \limxi a = a \\
	\end{align*}
	In dit geval impliceert dus $\limxc \big(f(x) +g(x)\big) = \limxc f(x) + \limxc g(x)$ dat $(+\infty) + (-\infty)$ gelijk zou moeten zijn aan $a$. Maar $a$ is willekeurig (of \textit{onbepaald}): we kunnen voor $a$ even goed $0$ kiezen als $2$ of $2\pi/3$ \dots! Dus: de rekenregels voor de som van limieten impliceren dat $(+\infty)+(-\infty)$ elke waarde kan aannemen, en dus noodzakelijkerwijze \textit{onbepaald} moet blijven. Jammer maar helaas ...
	
	
\end{example}

\todo{Wordt voorbeeld, met uitleg ...}
\begin{problem}
	Zoek een gelijkaardig eenvoudig (tegen)voorbeeld voor $0\cdot\infty$ en voor $\dfrac00$ en $\dfrac\infty\infty$.
\end{problem}



We zijn dus verplicht om voldoende voorzichtig rekenregels vast te leggen voor de symbolen $+\infty$ en $-\infty$. Men kan bewijzen dat je zonder problemen kan rekenen met $+\infty$ en $-\infty$  op voorwaarde dat je zogenaamde \textit{onbepaalde vormen} vermijdt.

Merk op: we schrijven verder meestal gewoon $\infty$ voor $+\infty$ (net zoals we $5$ schrijven voor $+5$). \\
En $\pm\infty$ betekent $+\infty$ of $-\infty$ (in die volgorde, zodat we ook kunnen schrijven dat $-(\pm\infty) = \mp\infty$!)

\begin{notation} (Afspraken in verband met oneindig)	
	\begin{align*}
	\infty & = +\infty   \\
	\pm\infty &\text{ staat voor} +\infty \text{ of } -\infty \\
	\end{align*}
\end{notation}


Eerst enkele regels die wel gewoon werken zoals bij de reële getallen:
\todo{Uitleg over symbool X ...?}
\todo{Uitzoeken: intertext online ..}


\begin{proposition} (Rekenregels voor de symbolen $+\infty$ en $-\infty$)

	Zij $a\in\R$. Dan geldt:	
	\begin{align*}
	-(+\infty) & = -\infty \\
	-(-\infty) & = +\infty  \\
	-(\pm\infty) &= \mp\infty \\
	-(\mp\infty) &= \pm\infty & \text{($+\infty$ en $-\infty$ zijn elkaars tegengestelde)}\\ 
	\\
	a+\infty & = \infty+a = \infty - a = -a +\infty \\
	& = \colorbox{lightgray}{$\infty + \infty = \infty$} & \text{($\infty + X =\infty$ behalve als $X=-\infty$)} \\
	\\
	a+(-\infty) & = a-\infty = -\infty+a =-\infty - a = -a - \infty \\
	& = \colorbox{lightgray}{$-\infty - \infty = -\infty+(-\infty) = -\infty$} & \text{($-\infty+X=-\infty$ behalve als $X=+\infty$)} \\
%	\pdfOnly{\intertext{Als $a>0$ ($a$ strikt positief)}}
	\mbox{ Als $a>0$ }\\% ($a$ strikt positief) }\\	
%	\begin{onlineOnly}\mbox{ Als $a>0$ ($a$ strikt positief) }\end{onlineOnly}
	a\cdot\infty & = \infty\cdot a \\
	& = \colorbox{lightgray}{$\infty \cdot \infty = (-\infty) \cdot (-\infty) = \infty$} & \text{($X\cdot\pm\infty=\pm\infty$ als $X>0$)} \\
	\mbox{Als $a<0$ } \\% ($a$ strikt negatief)} \\
%	\pdfOnly{\intertext{Als $a<0$ ($a$ strikt negatief)}}
%	\begin{onlineOnly}\text{Als $a<0$ ($a$ strikt negatief)}\\ \end{onlineOnly}%
		a\cdot\infty & = \infty\cdot a \\
	& = \colorbox{lightgray}{$\infty \cdot (-\infty) = (-\infty)\cdot\infty = -\infty$} & \text{($X\cdot(\pm\infty) = \mp\infty$ als $X<0$)} \\
	\end{align*}	
\end{proposition}  

Al deze regels zijn erg natuurlijk, en stellen op zich geen probleem. \\
Maar er zit dus een enigszins venijnig addertje onder het gras: er ontbreken enkele combinaties en die \textit{gelden niet} (altijd). Op examens durven studenten zich wel eens laten bijten door zo'n addertje. Dat is steeds erg pijnlijk. Je probeert het dus best te vermijden.

\begin{proposition} (\textbf{Onbepaalde vormen} met $\pm\infty$)
	
	Volgende uitdrukkingen zijn onbepaald. Dat betekent dat ze \textit{geen betekenis} hebben.	
	
	\begin{tabular}{cccr}
		\colorbox{lightgray}{$\infty - \infty$} &$(+\infty) + (-\infty)$  & 	$(-\infty) + (+\infty) $  & oneindig min oneindig \\
		\colorbox{lightgray}{$0\cdot\infty$} & $0 \cdot (\pm\infty)$   & 	$(\pm\infty)\cdot 0$   & nul maal oneindig \\
		\\
		\colorbox{lightgray}{$\dfrac 00$} & \colorbox{lightgray}{$\dfrac\infty\infty$} & $\dfrac{\pm\infty}{\mp\infty}$ & \begin{tabular}{r} nul gedeeld door nul \\ oneindig gedeeld door oneindig\end{tabular}
	\end{tabular}
\end{proposition}
Merk op: ook $0^0$, $\infty^0$ en $1^\infty$ zijn onbepaald, maar we zullen ze in deze cursus niet tegenkomen.


De rekenregels voor $\pm\infty$ stellen ons in staat volgende interessante eigenschap te formuleren voor het berekenen van limieten:

\begin{proposition} (Rekenregels limieten)
	
	Zij $A\subseteq\R, f,g:A\to\R$ twee continue functies, en $c\in A$ of $c$ ligt 'op de rand van $A$'. \\
	In het bijzonder kan eventueel $c=+\infty$ of $c=-\infty$. 
	
%	Veronderstel dat $\limxc f(x) = a$ en $\limxc g(x) = b$ (met $a,b\in \R\cap \{\pm\infty\}$).
	Veronderstel dat $\limxc f(x)$ en $\limxc g(x)$ bestaan%, en \textbf{beide {\large eindig} zijn}.
	
	Dan gelden volgende (reken-)regels  \textbf{\color{red} tenzij ze een {\large onbepaaldheid} geven}: 
	
	
	\begin{align}
		 \limxc \big(f(x) +g(x)\big) & = \limxc f(x) + \limxc g(x)  
		     & \text{[lim(som) = som(lims)]} \label{elim-som} \\
		 \limxc \big(f(x) -g(x)\big) & = \limxc f(x) - \limxc g(x)  
		     & \text{[lim(verschil) = verschil(lims)])} \label{elim-verschil}\\
		 \limxc \big(f(x)\cdot g(x)\big) & = \limxc f(x) \cdot \limxc g(x)  
		     & \text{[lim(product) =  product(lims)]} \label{elim-product} \\
		 %
		 \mbox{Als $\limxc g(x) \neq 0:$} \notag\\
		 \limxc \dfrac{f(x)}{g(x)} & = \dfrac{\limxc f(x)}{\limxc g(x)}  
		     & \text{[lim(quotiënt) = quotiënt(lims)]} \label{elim-quotient} \\
		 %
		 \mbox{Als $c\in A$} \notag\\ % (i.e. $c$ behoort tot dom $f$)}
		 \limxc f(x) & = f(c) 
		     & \text{[lim\_in\_c($f$) = $f$(c) als $f$(c) bestaat en $f$ continu]} \label{elim-domein} \\
		 %
%		 \intertext{Als $f(g(x))$ bestaat voor $x$ dicht bij $c$, $f(\limxc(x))$ bestaat, $f$ is continu (TBV!):} 
		 \text{Als alle vw voldaan:} \notag \\ %& \mbox{\small Als LL en RL bestaan:} \\ 
		 \limxc f(g(x)) & = f(\limxc g(x)) 
		     & \text{[lim($f$(iets)) =  $f$(lim(iets)) \textit{als alle vw voldaan!}]} \label{elim-continu} 
	\end{align}
\end{proposition}

\begin{remark} \ 
	
\begin{enumerate}
	\item Als je bij een berekening een onbepaalde vorm uitkomt, is er iets mis. Je zal een andere berekeningswijze moeten vinden, die de onbepaalde vorm vermijdt. Gelukkig bestaan er een aantal standaard trukjes (zie verder).
	\item De lijn lim(som) = som(lims) leest als 'de limiet van een som is de som van de limieten'. Dit is een voorbeeld van een in de wiskunde erg populair \link[mantra]{https://nl.wikipedia.org/wiki/Mantra} dat zeer de moeite waard is om vertrouwd mee te worden. Het mantra is steeds van de vorm "de X van de Y's is de Y van de X'en", met X en Y twee concepten of uitdrukkingen. De distributiviteit van het product ten opzichte van de som wordt dan 'het product van een som is de som van de producten', en dat is inderdaad precies wat de distributiviteit zegt: $a\cdot(b+c) = a\cdot b+a\cdot c$: het linkerlid is het product van ($a$ met) een som (namelijk $b+c$)en dat is gelijk aan het rechterlid, namelijk een som van (twee) producten (namelijk $a\cdot b$ en $a\cdot c$). Een groot deel van de wiskunde kan worden geformuleerd in dit soort mantra's, en het is van belang te weten wanneer ze opgaan, en wanneer niet. Zo werkt het bijvoorbeeld \textit{niet} voor de sinus en de som: de sinus van een som is \textit{niet} gelijk aan de som van de sinussen: $\sin(\alpha+\beta) \neq \sin\alpha+\sin\beta$. 
%	\item Deze rekenregels maken het leven zo eenvoudig als überhaupt mogelijk: we kunnen 'gewoon' rekenen met limieten, \textsc{tot we op onbepaaldheden uitkomen}. Als dat het geval is, moeten we nadenken, en een trukje proberen vinden om de onbepaaldheden te  vermijden. Gelukkig bestaan er een aantal standaard trukjes (zie verder).
	\item Voor continue functies, en onder bepaalde bestaansvoorwaarden, zegt eigenschap (\ref{elim-continu})  dat de limiet en de functieoproep van plaats mogen worden gewisseld, of nog dat de limiet 'door' de functie gaat: de limiet van de functie (van iets) is de functie van de limiet (van dat iets). Bijvoorbeeld: $\limxg{1}(\sin(x^2-1)) = \sin(\limxg{1}(x^2-1)) = \limxg{1}\sin(0) = 0$.

	\grapje{\item In verband met eigenschap (\ref{elim-continu}) zijn er voorlopig nog geen nuttige toepassingen bekend van de formules lim($f$(iets)) =  $f$(lim(iets)) in de rijwielsector, hoewel kwatongen beweren dat er daar misschien toch mogelijkheden zouden kunnen zijn.}

\end{enumerate}
\end{remark}



\subsection{Limieten in $c=\pm\infty$ }


\todo{uitleg toevoegen}
\begin{exercise}
Probeer met de rekenregels voor limieten en voor $\pm\infty$ om $\limxi (3x^2-4x+2)$ te berekenen.



\end{exercise}

\begin{proposition} (Limiet veelterm in $c=\pm\infty$)

De limiet van een veelterm in $\pm\infty$ is de limiet van de hoogstegraadsterm (en dus gelijk aan $+\infty$ of $-\infty$ naargelang het teken van de hoogstegraadscoëfficiënt en het al dan niet even zijn van de graad van de veelterm).

De limiet van een rationale functie in $\pm\infty$ wordt bepaald door de hoogstegraadstermen van teller en noemer (en is gelijk aan $0$, $\pm\infty$ of het quotiënt van de hoogstegraadscoëfficiënten).
\end{proposition}


\begin{exercise}
	\textit{Bereken met behulp van voorgaande rekenregels}
	\begin{enumerate}
		\renewcommand{\labelenumi}{(\alph{enumi})}
		\item $\displaystyle{\lim_{x \rightarrow + \infty}} (-4x^5 + 7x^4 -
		11)$
		\item $\displaystyle{\lim_{x \rightarrow - \infty}} \frac{-x^2 + 3x
			- 2}{3x-7}$
		\item $\displaystyle{\lim_{x \rightarrow + \infty}} \frac{2x^2 -
			3}{x^3 - x + 2}$
		\item $\displaystyle{\lim_{x \rightarrow - \infty}} \frac{x^3 -
			1}{x - 1}$
		\item $\displaystyle{\lim_{x \rightarrow + \infty}} \frac{x^6 +
			5x^2 + 1}{x^4 - x^6}$
		\item $\displaystyle{\lim_{x \rightarrow + \infty}} (\sqrt{4x^2 - 7x} + 3x)$
		\item $\displaystyle{\lim_{x \rightarrow + \infty}} (\sqrt{2 x^2 + 3} - x)$
		\item $\displaystyle{\lim_{x \rightarrow - \infty}} (\sqrt{2 x^2 + 3} - x)$
		\item $\displaystyle{\lim_{x \rightarrow + \infty}} \frac{\sqrt{x^2 + 5} + 2x}{3x-1}$
		\item $\displaystyle{\lim_{x \rightarrow - \infty}} \frac{\sqrt{x^2 + 5} + 2x}{3x-1}$
		\item $\displaystyle{\lim_{x \rightarrow - \infty}} \frac{\sqrt{x^2+1} + ax}{\sqrt{x^2+2}}$
		\item $\displaystyle{\lim_{x \rightarrow - \infty}} (\sqrt{4x^2 + 7x} +
		2x)$  
		\item $\displaystyle{\lim_{x \rightarrow - \infty}} (\sqrt{4x^2 + 7x} +
		ax)$  
		\item $\displaystyle{\lim_{x \rightarrow + \infty}} x(\sqrt{2+ \frac{3}{x^2}} - \sqrt{2- \frac{3}{x^2}})$
		\item $\displaystyle{\lim_{x \rightarrow - \infty}} \frac{ax + b}{7x- \sqrt{x^2 + 10}}$
	\end{enumerate}
\begin{comment}  % remove for now: to be integrated with \anwer / \choice \ \... ?
	\begin{oplossing}
		\begin{enumerate}
			\renewcommand{\labelenumi}{(\alph{enumi})}
			\item $\displaystyle{\lim_{x \rightarrow + \infty}} (-4x^5 + 7x^4 -	11) = - \infty$
			\item $\displaystyle{\lim_{x \rightarrow - \infty}} \frac{-x^2 + 3x	- 2}{3x-7} = + \infty$
			\item $\displaystyle{\lim_{x \rightarrow + \infty}} \frac{2x^2 - 3}{x^3 - x + 2} = 0$
			\item $\displaystyle{\lim_{x \rightarrow - \infty}} \frac{x^3 -	1}{x - 1} = + \infty$
			\item $\displaystyle{\lim_{x \rightarrow + \infty}} \frac{x^6 + 5x^2 + 1}{x^4 - x^6}= -1$
			\item $\displaystyle{\lim_{x \rightarrow + \infty}} (\sqrt{4x^2 - 7x} + 3x)= + \infty$
			\item $\displaystyle{\lim_{x \rightarrow + \infty}} (\sqrt{2 x^2 + 3} - x)= + \infty$
			\item $\displaystyle{\lim_{x \rightarrow - \infty}} (\sqrt{2 x^2 + 3} - x)= + \infty$
			\item $\displaystyle{\lim_{x \rightarrow + \infty}} \frac{\sqrt{x^2 + 5} + 2x}{3x-1} = 1$
			\item $\displaystyle{\lim_{x \rightarrow - \infty}} \frac{\sqrt{x^2 + 5} + 2x}{3x-1} = \frac13$
			\item $\displaystyle{\lim_{x \rightarrow - \infty}} \frac{\sqrt{x^2+1} + ax}{\sqrt{x^2+2}} = 1-a$
			\item $\displaystyle{\lim_{x \rightarrow - \infty}} (\sqrt{4x^2 + 7x} + 2x)= - \frac74$
			\item $\displaystyle{\lim_{x \rightarrow - \infty}} (\sqrt{4x^2 + 7x} +	ax)$  
			\begin{itemize}
				\item[] voor $a<2: + \infty$
				\item[] voor $a>2: - \infty$
				\item[] voor $a=2:$ zie vorige oefening 
			\end{itemize}
			\item $\displaystyle{\lim_{x \rightarrow + \infty}} x(\sqrt{2+ \frac{3}{x^2}} - \sqrt{2- \frac{3}{x^2}}) = 0$
			\item $\displaystyle{\lim_{x \rightarrow - \infty}} \frac{ax + b}{7x- \sqrt{x^2 + 10}}= \frac{a}{8}$
		\end{enumerate}
	\end{oplossing}
\end{comment}
\end{exercise}

\subsection{Limieten in nulpunten van de noemer}

\todo{(wijzen op) herhaling samenstelling functies (f(1/x) / 1/f(x) / f(x+a) etc)}

\subsubsection{Limieten van $\dfrac{1}{g(x)}$ met $g(x)$ een veelterm}

\begin{example}
	De functie $f_1(x)=\dfrac{1}{x}$ heeft geen limiet in $x=0$, 
	omdat $\llimx f_1(x) = -\infty$ terwijl $\rlimx f_1(x) = +\infty$. 
	Maar, $f_2(x)=\dfrac{1}{x^2} = +\infty$ omdat $x^2$ (en dus ook $\frac{1}{x^2}$) altijd positief is 
	(in de buurt van $0$ zou voldoende zijn, maar $x^2$ is natuurlijk overal positief). 
	
	Om in het algemeen het limietgedrag van $\dfrac{1}{g(x)}$ te kennen, is het dus (nodig en) voldoende om het tekenverloop te kennen van $g(x)$ in de buurt van zijn nulpunten. 
	
	Op de onderstaande tekening is $g(x)=(x+5)(x-5)(x-1)^2/50$ getekend, samen met $1/g(x)$. We zien dat de limiet in $x=1$ gelijk is aan $-\infty$ (omdat $g(x)$ er een dubbel nulpunt heeft en negatief is in de omgeving van $-1$, terwijl er in $x=\pm5$ telkens enkelvoudige nulpunten zijn, waar $g(x)$ dus van teken verandert, en $1/g(x)$ dus geen limiet heeft (maar enkel een linker- en een rechterlimiet).
	

\begin{image}[\width]
%\begin{figure}
	\begin{tikzpicture}
	 [
	declare function=
	{
		t(\x) = (\x-5)*(\x-1)^2*(x+5)/50 ;
	}
	]
	\begin{axis}
	[ samples=1000, axis lines=center,
	  axis equal,
	  scale=2,
	  restrict y to domain=-10:10,
	  ymax=8,ymin=-8,
      extra y ticks={0},
      legend style={at={(0.15,0.45)},anchor=north east},
	]
	\addplot[domain=-8.001:8,semithick,dashed,color=blue] {t(x)};
	\addplot[domain=-8.001:8,thick,color=red] {1/t(x)};
	\legend{$g(x)$, $1/g(x)$}
	\end{axis}
	\end{tikzpicture}
%	\captionof{figure}{Test}
%\end{figure}
\end{image}

\end{example}

Voor het vervolg is het erg belangrijk ten diepste overtuigd te zijn van de waarheid en trivialiteit van volgende twee eigenschappen (die op de vorige tekening duidelijk worden geïllustreerd):

\begin{proposition} (Triviale eigenschap veeltermen en rationale functies)
% \todo: er is meer vertikale whitespace tussen de titel en de enumerate dan tussen de titel en de \paragraph infra!
\begin{enumerate}
	\item Een veelterm $f(x)$ kan alleen van teken veranderen in een nulpunt. 
	\item Een veelterm verandert niet van teken in elk nulpunt.
	\item Een rationale functie $f(x)$ kan alleen van teken veranderen in een nulpunt van teller of noemer. 
	\item Een rationale functie  verandert niet van teken in elk nulpunt van teller of noemer.
\end{enumerate}
\end{proposition}

\begin{proposition} (Triviale eigenschap omgekeerde functie)
	
Zij $f$ een reële functie, $c\in\R$. Dan geldt dat
\begin{enumerate}
\item	$\dfrac{1}{f(c)}$ heeft altijd hetzelfde teken als $f(c)$  \quad(tenzij $f(c)=0$\dots)

\item    $\dfrac{1}{f(x)}$ heeft \textit{mogelijk} een oneindige limiet in $x=c \quad\iff\quad f(c)=0$ 

\item    $\dfrac{1}{f(x)}$ heeft \textit{echt} een oneindige limiet in $x=c \quad\iff\quad f(c)=0$ en $f(x)$ verandert niet van teken in de buurt van $c$ 
\end{enumerate}
\end{proposition}

\subsubsection{Limieten van rationale functies $f(x)/g(x)$}

\begin{verbatim}
Er zijn volgende mogelijkheden met $m_teller$ en $m_noemer$ de multipliciteiten van $c$:

* de limiet is een getal verschillend van 0  
                           (als $m_{noemer} = m_{teller}$)
* de limiet is 0           (als $m_{noemer} > m_{teller}$)
* de limiet is $\pm\infty$ (als $m_{noemer} < m_{teller}$ en llim = rlim)
* de limiet bestaat niet   (als $m_{noemer} < m_{teller}$ en llim \neq rlim)

\end{verbatim}



\end{document}
