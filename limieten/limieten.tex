\documentclass[numbers]{ximera}
% todo: proper use of options of documentclass to be investigated!!!
% eg with '\wordchoice'
% -> in pdf: just pint the correct options
% -> in HTML: provide all options
% BUT: in exercises: also in pdf provide all options!!!

% todo: make todo's (optionally) printable in the text !

\author{Zomercursus KU Leuven}
% todo: nadenken over outcomes (formulering/gebruik/...)
\outcome{Limieten kunnen berekenen}
\outcome{Onbepaaldheden met oneindig kunnen herkennen}

%
% copied from https://github.com/mooculus/calculus
%
\usepackage[utf8]{inputenc}


\graphicspath{
	{./}
	{goniometrie/}
}


%\usepackage{todonotes}
%\usepackage{mathtools} %% Required for wide table Curl and Greens
%\usepackage{cuted} %% Required for wide table Curl and Greens
\newcommand{\todo}{}

% Font niet (correct?) geinstalleerd in MikTeX?
%\usepackage{esint} % for \oiint
%\ifxake%%https://math.meta.stackexchange.com/questions/9973/how-do-you-render-a-closed-surface-double-integral
%\renewcommand{\oiint}{{\large\bigcirc}\kern-1.56em\iint}
%\fi


\newcommand{\mooculus}{\textsf{\textbf{MOOC}\textnormal{\textsf{ULUS}}}}

\usepackage{tkz-euclide}\usepackage{tikz}
\usepackage{tikz-cd}
\usetikzlibrary{arrows}
\tikzset{>=stealth,commutative diagrams/.cd,
  arrow style=tikz,diagrams={>=stealth}} %% cool arrow head
\tikzset{shorten <>/.style={ shorten >=#1, shorten <=#1 } } %% allows shorter vectors

\usetikzlibrary{backgrounds} %% for boxes around graphs
\usetikzlibrary{shapes,positioning}  %% Clouds and stars
\usetikzlibrary{matrix} %% for matrix
\usepgfplotslibrary{polar} %% for polar plots
\usepgfplotslibrary{fillbetween} %% to shade area between curves in TikZ
\usetkzobj{all}
\usepackage[makeroom]{cancel} %% for strike outs
%\usepackage{mathtools} %% for pretty underbrace % Breaks Ximera
%\usepackage{multicol}
\usepackage{pgffor} %% required for integral for loops



%% http://tex.stackexchange.com/questions/66490/drawing-a-tikz-arc-specifying-the-center
%% Draws beach ball
\tikzset{pics/carc/.style args={#1:#2:#3}{code={\draw[pic actions] (#1:#3) arc(#1:#2:#3);}}}



\usepackage{array}
\setlength{\extrarowheight}{+.1cm}
\newdimen\digitwidth
\settowidth\digitwidth{9}
\def\divrule#1#2{
\noalign{\moveright#1\digitwidth
\vbox{\hrule width#2\digitwidth}}}





\newcommand{\RR}{\mathbb R}
\newcommand{\R}{\mathbb R}
\newcommand{\N}{\mathbb N}
\newcommand{\Z}{\mathbb Z}

\newcommand{\sagemath}{\textsf{SageMath}}


%\renewcommand{\d}{\,d\!}
\renewcommand{\d}{\mathop{}\!d}
\newcommand{\dd}[2][]{\frac{\d #1}{\d #2}}
\newcommand{\pp}[2][]{\frac{\partial #1}{\partial #2}}
\renewcommand{\l}{\ell}
\newcommand{\ddx}{\frac{d}{\d x}}

\newcommand{\zeroOverZero}{\ensuremath{\boldsymbol{\tfrac{0}{0}}}}
\newcommand{\inftyOverInfty}{\ensuremath{\boldsymbol{\tfrac{\infty}{\infty}}}}
\newcommand{\zeroOverInfty}{\ensuremath{\boldsymbol{\tfrac{0}{\infty}}}}
\newcommand{\zeroTimesInfty}{\ensuremath{\small\boldsymbol{0\cdot \infty}}}
\newcommand{\inftyMinusInfty}{\ensuremath{\small\boldsymbol{\infty - \infty}}}
\newcommand{\oneToInfty}{\ensuremath{\boldsymbol{1^\infty}}}
\newcommand{\zeroToZero}{\ensuremath{\boldsymbol{0^0}}}
\newcommand{\inftyToZero}{\ensuremath{\boldsymbol{\infty^0}}}



\newcommand{\numOverZero}{\ensuremath{\boldsymbol{\tfrac{\#}{0}}}}
\newcommand{\dfn}{\textbf}
%\newcommand{\unit}{\,\mathrm}
\newcommand{\unit}{\mathop{}\!\mathrm}
\newcommand{\eval}[1]{\bigg[ #1 \bigg]}
\newcommand{\seq}[1]{\left( #1 \right)}
\renewcommand{\epsilon}{\varepsilon}
\renewcommand{\phi}{\varphi}


\renewcommand{\iff}{\Leftrightarrow}

\DeclareMathOperator{\arccot}{arccot}
\DeclareMathOperator{\arcsec}{arcsec}
\DeclareMathOperator{\arccsc}{arccsc}
\DeclareMathOperator{\si}{Si}
\DeclareMathOperator{\scal}{scal}
\DeclareMathOperator{\sign}{sign}


%% \newcommand{\tightoverset}[2]{% for arrow vec
%%   \mathop{#2}\limits^{\vbox to -.5ex{\kern-0.75ex\hbox{$#1$}\vss}}}
\newcommand{\arrowvec}[1]{{\overset{\rightharpoonup}{#1}}}
%\renewcommand{\vec}[1]{\arrowvec{\mathbf{#1}}}
\renewcommand{\vec}[1]{{\overset{\boldsymbol{\rightharpoonup}}{\mathbf{#1}}}\hspace{0in}}

\newcommand{\point}[1]{\left(#1\right)} %this allows \vector{ to be changed to \vector{ with a quick find and replace
\newcommand{\pt}[1]{\mathbf{#1}} %this allows \vec{ to be changed to \vec{ with a quick find and replace
\newcommand{\Lim}[2]{\lim_{\point{#1} \to \point{#2}}} %Bart, I changed this to point since I want to use it.  It runs through both of the exercise and exerciseE files in limits section, which is why it was in each document to start with.

\DeclareMathOperator{\proj}{\mathbf{proj}}
\newcommand{\veci}{{\boldsymbol{\hat{\imath}}}}
\newcommand{\vecj}{{\boldsymbol{\hat{\jmath}}}}
\newcommand{\veck}{{\boldsymbol{\hat{k}}}}
\newcommand{\vecl}{\vec{\boldsymbol{\l}}}
\newcommand{\uvec}[1]{\mathbf{\hat{#1}}}
\newcommand{\utan}{\mathbf{\hat{t}}}
\newcommand{\unormal}{\mathbf{\hat{n}}}
\newcommand{\ubinormal}{\mathbf{\hat{b}}}

\newcommand{\dotp}{\bullet}
\newcommand{\cross}{\boldsymbol\times}
\newcommand{\grad}{\boldsymbol\nabla}
\newcommand{\divergence}{\grad\dotp}
\newcommand{\curl}{\grad\cross}
%\DeclareMathOperator{\divergence}{divergence}
%\DeclareMathOperator{\curl}[1]{\grad\cross #1}
\newcommand{\lto}{\mathop{\longrightarrow\,}\limits}

\renewcommand{\bar}{\overline}

\colorlet{textColor}{black}
\colorlet{background}{white}
\colorlet{penColor}{blue!50!black} % Color of a curve in a plot
\colorlet{penColor2}{red!50!black}% Color of a curve in a plot
\colorlet{penColor3}{red!50!blue} % Color of a curve in a plot
\colorlet{penColor4}{green!50!black} % Color of a curve in a plot
\colorlet{penColor5}{orange!80!black} % Color of a curve in a plot
\colorlet{penColor6}{yellow!70!black} % Color of a curve in a plot
\colorlet{fill1}{penColor!20} % Color of fill in a plot
\colorlet{fill2}{penColor2!20} % Color of fill in a plot
\colorlet{fillp}{fill1} % Color of positive area
\colorlet{filln}{penColor2!20} % Color of negative area
\colorlet{fill3}{penColor3!20} % Fill
\colorlet{fill4}{penColor4!20} % Fill
\colorlet{fill5}{penColor5!20} % Fill
\colorlet{gridColor}{gray!50} % Color of grid in a plot

\newcommand{\surfaceColor}{violet}
\newcommand{\surfaceColorTwo}{redyellow}
\newcommand{\sliceColor}{greenyellow}




\pgfmathdeclarefunction{gauss}{2}{% gives gaussian
  \pgfmathparse{1/(#2*sqrt(2*pi))*exp(-((x-#1)^2)/(2*#2^2))}%
}


%%%%%%%%%%%%%
%% Vectors
%%%%%%%%%%%%%

%% Simple horiz vectors
\renewcommand{\vector}[1]{\left\langle #1\right\rangle}


%% %% Complex Horiz Vectors with angle brackets
%% \makeatletter
%% \renewcommand{\vector}[2][ , ]{\left\langle%
%%   \def\nextitem{\def\nextitem{#1}}%
%%   \@for \el:=#2\do{\nextitem\el}\right\rangle%
%% }
%% \makeatother

%% %% Vertical Vectors
%% \def\vector#1{\begin{bmatrix}\vecListA#1,,\end{bmatrix}}
%% \def\vecListA#1,{\if,#1,\else #1\cr \expandafter \vecListA \fi}

%%%%%%%%%%%%%
%% End of vectors
%%%%%%%%%%%%%

%\newcommand{\fullwidth}{}
%\newcommand{\normalwidth}{}



%% makes a snazzy t-chart for evaluating functions
%\newenvironment{tchart}{\rowcolors{2}{}{background!90!textColor}\array}{\endarray}

%%This is to help with formatting on future title pages.
\newenvironment{sectionOutcomes}{}{}



%% Flowchart stuff
%\tikzstyle{startstop} = [rectangle, rounded corners, minimum width=3cm, minimum height=1cm,text centered, draw=black]
%\tikzstyle{question} = [rectangle, minimum width=3cm, minimum height=1cm, text centered, draw=black]
%\tikzstyle{decision} = [trapezium, trapezium left angle=70, trapezium right angle=110, minimum width=3cm, minimum height=1cm, text centered, draw=black]
%\tikzstyle{question} = [rectangle, rounded corners, minimum width=3cm, minimum height=1cm,text centered, draw=black]
%\tikzstyle{process} = [rectangle, minimum width=3cm, minimum height=1cm, text centered, draw=black]
%\tikzstyle{decision} = [trapezium, trapezium left angle=70, trapezium right angle=110, minimum width=3cm, minimum height=1cm, text centered, draw=black]


% todo: captions en image/figure/tikzpicture to be investigated
\usepackage{caption}  

\title{Limieten}

\begin{document}
\begin{abstract}
	\cancel{The sky }$\infty$ is the limit
\end{abstract}
\maketitle

\subsection{Inleiding}
%todo: nadenken over 'jij' vorm of 'we' vormin dit soort teksten !!!
In dit deel bestudeer je \textit{limieten} en leer je rekenen met het symbool $\infty$ ('\textit{oneindig}').

\begin{expandable}
Voor wie (nog) niet overtuigd is dat dit een mogelijk nuttige bezigheid is, hebben we volgende% een uitwijding over het wat en waarom van limieten.

% todo: layout zodat duidelijk is dat dit een 'uitwijding' is (dus: expandable aanpassen/vervangen) (voorlopig adhoc tcolorbox) 
% todo: nagaan hoe exandable precies werkt

\begin{tcolorbox}[title=Uitwijding: achtergrondinformatie over het hoe en waarom van limieten]
	\footnotesize
(Merk op: dit is enkel inleidende achtergrondinformatie, en geen leerstof.)	

Dikwijls is men geïnteresseerd in wat er gebeurt 'net op de rand van wat al gekend is'. \todo{TODO: goed voorbeeldje vinden}. Wiskundig kunnen we dit soort situaties (in gunstige gevallen) beschrijven (en soms zelfs begrijpen) in termen van \textit{limieten}. 

In het eerder eenvoudige geval van rijen is het intuïtief aannemelijk dat de rij getallen
\begin{align*}
% geen (afschrikwekkende?) formules ... ?
%(a_n) = \Big(\frac{1}{n^2}\Big) & = 1, \frac{1}{4}, \frac{1}{9}, \frac{1}{16},\frac{1}{25},\dots\\
(a_n) & = 1, \;\frac{1}{4}, \;\frac{1}{9}, \;\frac{1}{16},\;\frac{1}{25},\;\dots\\
                                & =1,\; 0.25,\;  0.1111\dots,\; 0.0625,\; 0.04,\; 0.0277\dots,\; \dots
\end{align*}

'in de limiet' gelijk wordt aan $0$ (hoewel er natuurlijk geen enkele van de termen $a_n$ echt $0$ is!). \\
We zeggen dat 'de limiet van de rij $(a_n)$ gelijk is aan $0$' of dat 'de $a_n$ in de limiet $0$ worden', en schrijven $\lim_{n\to\infty}a_n = 0$.

Voor de rij 
\[
%(b_n) = ((-1)^{n+1}) = 1,-1,1,-1,1,-1,\dots
(b_n) = \;1,\;-1,\;1,\;-1,\;1,\;-1,\dots
\]
lijkt er geen limiet te bestaan,
terwijl voor de rij 
\[
%(c_n) = (n^2) = 1,4,9,16,25,\dots
(c_n) = 1,\;4,\;9,\;16,\;25,\;\dots
\]
de natuurlijke limiet 'oneindig' zou kunnen zijn.

Een bijzonder belangrijke, maar erg subtiele, bezigheid bestaat erin te bestuderen \textit{wanneer} dergelijke limiet precies bestaat, en vooral \textit{welke} eigenschappen van dingen onder \textit{welke} voorwaarden ook \textit{in de limiet} bewaard blijven. 

Dat dit erg subtiel is blijkt uit het volgende voorbeeld voor de rij $(a_n)$ van hierboven:

De eigenschap 'is groter dan of gelijk aan nul', dus '$a_n\geq0$' geldt voor elke term $a_n$, en ze geldt ook \textit{in de limiet} (we zeggen ook 'voor de limiet') want  $0\geq0$. Dus: deze eigenschap $a_n\geq 0$ die geldt voor elke term, blijft ook gelden in de limiet. Dat is handig en mooi. 

Maar, de erg gelijkaardige eigenschap 'is groter dan nul', dus '$a_n>0$', geldt ook voor alle $a_n$, maar ze geldt niet meer in de limiet, want $0\not>0$. Dat is erg jammer en erg vervelend. Het is de oorzaak van vele extra pagina's leerstof in veel wiskundecursussen (en van veel fouten op examens, maar dit geheel terzijde). \\


In dit hoofdstuk krijg je een kleine inleiding in de problematiek van limieten van \textit{functies} (en dus niet van \textit{rijen} zoals in dit voorbeeld). Later gebruiken we die limieten om asymptoten te bekijken, afgeleiden en integralen te definiëren en nog veel meer. 

Merk op: een meer formele en exacte behandeling van limieten wordt snel technisch (bijvoorbeeld met de onder (niet-)kenners erg beruchte $\epsilon$'s en $\delta$'s) \todo{TODO voetnoot?} en dat valt -jammer voor ons schrijvers van cursusteksten, maar gelukkig voor u die het resultaat van onze arbeid moeten studeren- buiten het bestek van deze cursus. Sommigen van u kennen het al, en voor anderen komt het nog. Maar, velen onder u zullen aan de technische details ontsnappen.
\end{tcolorbox}
\end{expandable}


\subsection{Voorbeeld en definitie}

We geven als kennismaking enkele voorbeelden van limieten van functies, om dan tot een soort van intuïtieve definitie te komen. In  deze cursus geven we geen formeel-wiskundige definitie van limiten, we beperken ons het intuïtief aflezen van limieten op grafieken van functies. In een volgende paragfraaf geven we wel wiskundig exacte rekenregels voor limieten.

%\todo{
% We kiezen sin(x)/x als eerste voorbeeld, omdat daarbij \infty geen enkele rol speelt, 
% en dus enkel het begrip 'limiet' wordt gebruikt
% 1/x is 'complexer' omdat er een mix is tussen de (in principe onafhankelijke) begrippen 'limiet' en 'oneindig'
% nadeel is dat sin(x)/x misschien ook als'complex' wordt ervaren, hoewelodat eenvoudig kan worden uitgelegd op de tekening
%
% todo: maak hiervan 'instructornotes?
% todo: uitleggen waarom de rode lijn uinderdaad sin(x)/x is ...?
%}

\begin{example} (Limieten van $\frac{\sin x}{x}$)
	
Beschouw de functie $f:x\mapsto \dfrac{\sin x}{x}$. Dat is de gekende sinusfunctie die we delen door $x$. Voor grote $x$ zal die functie zeker erg dicht bij $0$ komen te liggen (want we delen de $\sin x$, wat tussen $-1$ en $1$ ligt, door een grote $x$, dus het resultaat wordt erg klein. 

De situatie is wat moeilijker direct te begrijpen voor erg kleine $x$: als we delen door kleine $x$, wordt het resultaat in principe steeds groter, maar hier wordt tegelijk ook de $sin x$ in de teller kleiner. We delen dus iets kleins door iets kleins, en dat is niet noodzakelijk altijd klein (de meeste mensen vinden bijvoorbeeld $10^{-90}$ klein, maar $10^{-90}/10^{-100} = 10^{10}$ eerder groot ...!). 

Merk ook op dat we niet zomaar $x=0$ kunnen invullen in de formule, want dan krijgen we $\frac{\sin 0}{0} = \frac{0}{0}$, en 'dat mag niet'. \todo{TODO: verwijzing naar onbepaaldheid $0/0$ evt in voetnoot???}. 

Een rekenmachine leert voor $x=0,01$ dat $\frac{\sin(0,01)}{0,01}=0,999983\dots$, en we kunnen dus vermoeden dat $\frac{\sin x}{x}$ steeds dichter bij $1$ zal liggen naarmate we $x$ kleiner kiezen. Men kan ook wiskundig aantonen dat dat inderdaad zo is.

Ook uit de grafiek van de functie  $\dfrac{\sin x}{x}$ blijkt trouwens dat voor $x$ dicht bij $0$ de waarde van $\dfrac{\sin x}{x}$ erg dicht ligt bij $1$. 

We zeggen dan ook dat de functie (of de uitdrukking) $\dfrac{\sin x}{x}$ gelijk is (of ook gelijk \textit{wordt}) aan $1$ \textit{als $x$ naar $0$ gaat}, en we schrijven $\limx \dfrac{\sin x}{x} = 1$.

Over het gedrag voor grote $x$ zeggen we dat $\dfrac{\sin x}{x}$  gelijk wordt aan $0$ \textit{als $x$ naar oneindig gaat}, en we schrijven  $\limxi \dfrac{\sin x}{x} = 0$. 

\begin{image}
	\begin{tikzpicture}[scale=2]
	\begin{axis}
	[
	samples=200,
	axis lines=center,
	axis equal,
	ymax=3, ymin=-1,	
    restrict y to domain=-2:3,
	extra y ticks={0},
	]
	%pas op:sin werkt in DEGREE, en niet in RADIAAL;dus sin(deg(x))...!)
	\addplot[domain=0.001:10,semithick,dashed,color=blue] {sin((deg(x)))};
	\addplot[domain=0.001:10,semithick,dotted,color=blue] {x};
	\addplot[domain=0.001:10,ultra thick,color=red] {sin((deg(x)))/x}; 
	\legend{$y=sin(x)$,$y=x$,$y=\frac{\sin x}{x}$};
	% todo legende nog niet goed geformateerd ...?
	\end{axis}
	\end{tikzpicture}
\end{image}
\end{example}

\begin{example} (Limieten van $\frac{1}{x^2}$)
	
	We doen hetzelfde als hierboven voor $f: x\mapsto  1/x$.
	
	Voor $x$ heel groot, wordt natuurlijk $1/x$ heel klein, en dus kunnen we redelijkerwijs zeggen dat 'als  $x$  naar oneindig gaat, $1/x$ nul wordt.' We schrijven $\limxi\;\frac1x=0$.
	
	Voor $x$ heel klein (in de betekenis van heel dicht bij $0$), lijkt het ook eenvoudig: als $x$ naar $0$ gaat, dat wordt $1/x$ immers oneindig groot. Maar, er is jammer genoeg een complicatie: als $x$ negatief is en dicht bij $0$ ligt, dan wordt $x$ niet heel groot, maar net heel klein (in de betekenis van heel erg negatief). Dus: voor $x$ dicht bij $0$, kan $1/x$ ofwel heel groot zijn (als $x>0$, ofwel net heel erg negatief (als $x<0$). Dus: het hangt er maar van af 'langs welke kant' we $x$ naar $0$ laten gaan, of we in $+\infty$ dan wel in $-\infty$ uitkomen. 
	
	Het heeft geen zin om in dit geval te spreken van een limiet, maar het is in een aantal gevallen (in het bijzonder in deze cursus) erg handig om toch een begrip en een notatie te hebben in deze situatie. We spreken van \textit{linkerlimiet} als we $0$ langs links benaderen, dus met $x<$ (of $0>x$, dat is natuurlijk hetzelfde!), en \textit{rechterlimiet} voor langs rechts (dus $x>0$, of ook $0<x$). En we hebben dus enkel een (echte) \textit{limiet} als de linkerlimiet gelijk is aan de rechterlimiet!
	
	De situatie zou volledig duidelijk moeten zijn op volgende tekening:
	\begin{image}
		\begin{tikzpicture}[scale=4]
		\begin{axis}
		[
		axis equal,
		ymax=5,ymin=-5,
		samples=200,
		axis lines=center,
		extra y ticks={0},
		restrict y to domain=-10:10,
		]
		\addplot[domain=-6:6,color=blue] {1/x};
		\addlegendentry{$y=\frac1x$};	 
		\addlegendentry{};   
		\node[anchor=east, font=\small] at  (axis cs: -0.2,4.4) {$\rlimx[0] \frac1x =+\infty$};
		\node[anchor=west] at  (axis cs: 0.2,-4) {$\llimx[0] \frac1x =-\infty$};
		\node[anchor=south west] at (axis cs: -6,0.2) {$\limxmi \frac1x =0$};
		\node[anchor=north east] at  (axis cs: 6,-0.4) {$\limxi \frac1x =0$};

		\end{axis}
%		\node[draw,text width=4cm] at (0.1,-4) {$\llimx[0] 1/0 =1\infty$};
		\end{tikzpicture}
	\end{image}
\end{example}

\todo{TODO: tekst aanpassen!} Het begrip \textit{limiet} heeft dus niet noodzakelijk een direct verband met het begrip \textit{oneindig} (tenzij in de betekenis van 'ergens oneindig dicht bij liggen'). Maar we zullen verder toch erg veel met oneindig te maken krijgen: enerzijds zullen we zeggen dat een 'limiet gelijk is aan oneindig' als de functiewaarden steeds groter worden, en anderzijds zullen we ook 'limieten in oneindig' bekijken, namelijk wat er gebeurt als $x$ steeds groter wordt.

%We bestuderen enkele eigenschappen van limieten (bij veeltermen zoals $x^5-2x^2+7$, rationale functies zoals$\frac{5x}{x-5}$ en irrationale functies zoals $\frac{\sqrt{x-1}}{x-1}$). 

We gaan zoals vermeld in deze module niet in op de exacte wiskundige definitie van een limiet. We gebruiken enkel volgende notatie en (pseudo-)definities:

\begin{definition} (Intuïtieve pseudo-definitie van limiet)
	
	Zij $f$ een continue functie, en $a$ en $c$ telkens ofwel een reëel getal, ofwel één van de symbolen $+\infty$ of $-\infty$. 
	Dan zeggen en schrijven we dat 

\begin{align*}
	\limx[c]  f(x) = a & \quad\iff\quad \text{de limiet van $f$ als $x$ naar $c$ gaat is (gelijk aan) $a$} \\
	                   & \quad\iff\quad \text{als }x\to c\text{, dan } 									f(x) \to a \\
	                   \\
	\llimx[c] f(x) = a & \quad\iff\quad \text{de linkerlimiet van $f$ als $x$ naar $c$ gaat is (gelijk aan) $a$} \\
	                   & \quad\iff\quad \text{de limiet van $f$ als $x$ langs links naar $c$ gaat is (gelijk aan) $a$} \\
	                   & \quad\iff\quad \text{als }x\to c\text{\textbf{ langs links} (dus $x<c$), dan } f(x) \to a \\
	                   \\
	\rlimx[c] f(x) = a & \quad\iff\quad \text{de rechterlimiet van $f$ als $x$ naar $c$ gaat is (gelijk aan) $a$} \\
					   & \quad\iff\quad \text{de limiet van $f$ als $x$ langs rechts naar $c$ gaat is (gelijk aan) $a$} \\
					   & \quad\iff\quad \text{als }x\to c\text{\textbf{ langs rechts} (dus $x>c$), dan }f(x) \to a 
\end{align*}
Merk op: $(x<c\iff c>x)$ en $(x>c \iff c<x)$. Let dus zowel op de tekens $<,>$ \textit{en} op de volgorde van de letters of cijfers!  
\end{definition}

Merk op: als we  $x\to c$ lezen als '$x$ gaat naar $c$' of '$x$ ligt (erg) dicht bij $c$', dan heeft dat een zeker dynamisch aspect: $x$ wordt verondersteld te 'bewegen' naar iets. Het is belangrijk te beseffen dat het resultaat, of de 'uitkomst' van zo'n limiet, dus datgene wat met het symbool $\limx[c] f(x)$ wordt aangeduid, gewoon een bepaald getal is (of eventueel het symbool $\pm\infty$: zie daarvoor verder).

%(overbodig) Merk op: hierbij staan zowel de letters $a$ als $c$ dus voor reële getallen \textit{of} voor één van de symbolen $+\infty$ of $-\infty$. Zie verder voor de rekenregels voor $\pm\infty$.

Merk op: de definitie zegt niet (wiskundig) nauwkeurig wat een limiet \textit{precies is}. Het volstaat -voor deze module- om op basis van dit soort definitie een intuïtief begrip op te bouwen over wat limieten zijn, en hoe ze kunnen worden berekend. Met dit soort van intuïtieve definities kunnen van limieten natuurlijk geen eigenschappen worden \textit{bewezen}, of echte berekeningen worden gemaakt. We kunnen op deze basis limieten (enkel) bepalen op volgende manieren:
\begin{enumerate}
	\item gebruik maken van betrouwbare rekenregels (zie verder)
	\item aflezen op een grafiek  
	\\(enigszins betrouwbaar als de grafiek betrouwbaar is!)
	\item ons baseren op berekeningen 'in de buurt' en/of intuïtie 
	\\(werkt dikwijls: bv $\frac{\sin(0,01)}{0,01}=0,999983\dots$, en dus geldt 'allicht' $\limx \frac{\sin x}{x} = 1$)
\end{enumerate}
Het is duidelijk dat opties 2 en  3 \textsc{strikt wiskundig niet geldig zijn}. Maar, de definitie van limiet is dat ook niet, dus tot nader order is het -mits de nodige voorzichtigheid- niet verboden, en dus toegelaten, om er toch gebruik van te maken. De 'tot nader order' hangt af van welke studierichtingen en cursussen u verder nog zal volgen! De voorlopige methode die hier wordt uiteengezet bestaat uit het toepassen van onderstaande rekenregels op gekend veronderstelde basislimieten (die bijvoorbeeld zijn bepaald via hun grafiek). 
% (is boven al gezegd) Afhankelijk van u vooropleiding kent u er eigenlijk al meer van, en afhankelijk van uw huidige opleiding, zal u er misschien in de nabije toekomst veel meer over moeten weten! 

\begin{exercise} \ 
	Bereken op basis van volgende welbekende grafieken onderstaande limieten

\begin{image}
% todo: scale werkt niet ...?
	\begin{tikzpicture}[scale=4]
\begin{axis}
[
axis equal,
ymax=5,ymin=-5,
samples=200,
axis lines=center,
extra y ticks={0},
  restrict y to domain=-10:10,
]
\addplot[domain=-5:5,color=blue] {1/x^2};
\legend{$1/x^2$}
\end{axis}
\end{tikzpicture}
\quad
	\begin{tikzpicture}[scale=4]
	\begin{axis}
	[
	axis equal,
	ymax=5,ymin=-5,
	samples=200,
	axis lines=center,
	extra y ticks={0},
    restrict y to domain=-10:10,
	]
	\addplot[domain=-5:5,color=blue] {1/x};
	\legend{$1/x$}
	\end{axis}
	\end{tikzpicture}
	\quad
	\begin{tikzpicture}[scale=4]
	\begin{axis}
	[
    axis equal,
    ymax=5,ymin=-5,
	samples=200,
	axis lines=center,
	extra y ticks={0},
    restrict y to domain=-10:10,
	]
	\addplot[domain=-5:5,color=blue] {exp(x)};
	\addplot[domain=0.001:5,dashed,color=red] {ln(x)};
	\legend{$e^x$, $\ln x$}
	\end{axis}
	\end{tikzpicture}
\end{image}
	
	\newcommand{\iscorrect}{}
	\newcommand{\localoefoptions}{\hfill\wordChoice{\choice{$+\infty$}\choice{$-\infty$}\choice{0}\choice{1}\choice[\iscorrect]{bestaat niet}\choice{andere oplossing}}}
	
	%\todo{adjust alignment over all items (https://tex.stackexchange.com/questions/29119 ?)}
	%\todo{vind iets omde juiste oplossing door te geven aan localoefoptions!!!} 

	% todo: limiet van sin x: bestaat niet
	% todo: limiet van sin(1/x): bestaat ook niet
	
	\begin{enumerate}
%		\item $\limx    \frac2x=\;      $\localoefoptions
		\item $\limx    \frac{1}{x^2}=\;$\let\i\localoefoptions
		\item $\limxi   e^x = \;$        \localoefoptions
		\item $\limxmi  e^x = \;$        \localoefoptions
		\item $\rlimx   e^{\frac 1x} = $
		\item $\llimx   e^{\frac 1x} = $
		\item $\limxi   \ln x = $
		\item $\rlimx   \ln x = $
		\item $\limxmi  \ln x = $
		\item $\limxi   \ln(1+ \frac 1x) = $
%		\item \[\lim_{x\to{1}}\dfrac{x^{2} + 12 \, x - 13}{x - 1}=\answer{14}\]
% from https://github.com/XronosUF/MAC2311
	\end{enumerate}
\end{exercise}


\subsection{Rekenregels berekenen van limieten}

Zoals wel meer gebeurt in de wiskunde (en daarbuiten), proberen we eerst iets simpel en willen we dat daarna zo eenvoudig mogelijk uitbreiden naar meer ingewikkelde situaties. We kunnen nu limieten aflezen van (eenvoudige) grafieken van functies. Maar het zou wel erg onhandig zijn om bijvoorbeeld voor de limieten $\limx \frac{8}{x^2}, \limx (4+\frac{1}{x^2}), \limx ((x+2)^2 + \frac{8}{x^2})$ telkens grafieken te moeten tekenen! Er zijn grenzen aan de waanzin, zelfs voor wiskundigen.

Dus moeten we hopen dat er (zo eenvoudig mogelijke) regeltjes zijn om meer ingewikkelde limieten te kunnen berekenen op basis van eenvoudige(re) limieten. In een ideale wereld is de limiet van een som gewoon de som van de limieten, en hetzelfde voor verschillen, producten, quotiënten, machten enzovoort. De wereld is niet ideaal, maar het kon slechter. In vele gevallen is rekenen met limieten  zo eenvoudig als het redelijkerwijs kan zijn. Maar, jammer genoeg zijn er enkele venijnige uitzonderingsgevallen. We bekijken eerst enkele voorbeelden.

\begin{example} (Limieten van $x^2 \pm e^x$)
	
	Dit is een veelterm en we kunnen die bekijken als de som van de functies $x\mapsto3x^2, x\mapsto-4x$ en $x\mapsto2$. 
	Dat ziet er als volgt uit:
\begin{image}
	% todo: scale werkt niet ...?
	\begin{tikzpicture}[scale=4]
	\begin{axis}
	[
	axis equal,
	ymax=5,ymin=-5,
	samples=200,
	axis lines=center,
	extra y ticks={0},
	restrict y to domain=-10:10,
	]
	\addplot[domain=-5:5,color=blue,dotted] {x^2};
	\addplot[domain=-5:5,color=blue,dotted] {exp(x)};
%	\addplot[domain=-5:5,color=blue,dotted] {2};
	\addplot[domain=-5:5,color=red,thick] {x^2+exp(x)};
%	\legend{$1/x^2$}
	\end{axis}
	\end{tikzpicture}	
\end{image}
	
	To Be Completed
\end{example}

\begin{proposition} (Rekenregels \textit{eindige} limieten)
	
	Zij $A\subseteq\R, f,g:A\to\R$ twee continue functies, en $c\in A$ of $c$ ligt 'op de rand van $A$'. \\
	In het bijzonder kan eventueel $c=+\infty$ of $c=-\infty$. 
	
%	Veronderstel dat $\limx[c] f(x) = a$ en $\limx[c] g(x) = b$ (met $a,b\in \R\cap \{\pm\infty\}$).
	Veronderstel dat $\limx[c] f(x)$ en $\limx[c] g(x)$ bestaan%, en \textbf{beide {\large eindig} zijn}.
	
	Dan gelden volgende (reken-)regels  \textbf{\color{red} tenzij ze een {\large onbepaaldheid} geven}: 
	
	
	\begin{align}
		 \limx[c] \big(f(x) +g(x)\big) & = \limx[c] f(x) + \limx[c] g(x)  
		     & \text{[lim(som) = som(lims)]} \label{elim-som} \\
		 \limx[c] \big(f(x) -g(x)\big) & = \limx[c] f(x) - \limx[c] g(x)  
		     & \text{[lim(verschil) = verschil(lims)])} \label{elim-verschil}\\
		 \limx[c] \big(f(x)\cdot g(x)\big) & = \limx[c] f(x) \cdot \limx[c] g(x)  
		     & \text{[lim(product) =  product(lims)]} \label{elim-product} \\
		 %
		 \intertext{Als $\limx[c] g(x) \neq 0:$}
		 \limx[c] \dfrac{f(x)}{g(x)} & = \dfrac{\limx[c] f(x)}{\limx[c] g(x)}  
		     & \text{[lim(quotiënt) = quotiënt(lims)]} \label{elim-quotient} \\
		 %
		 \intertext{Als $c\in A$ (dus $c$ behoort tot dom $f$)} %(en dus $c$ tot het domein van $f$ behoort):} 
		 \limx[c] f(x) & = f(c) 
		     & \text{[lim\_in\_c($f$) = $f$(c) als $f$(c) bestaat en $f$ continu]} \label{elim-domein} \\
		 %
		 \intertext{Als $f(g(x))$ bestaat voor $x$ dicht bij $c$, $f(\limx[c]g(x))$ bestaat, $f$ is continu (TBV!):} 
		 \limx[c] f(g(x)) & = f(\limx[c] g(x)) 
		     & \text{[lim($f$(iets)) =  $f$(lim(iets)) \textit{als} alle vw voldaan!]} \label{elim-continu} 
	\end{align}
\end{proposition}

Uitwijding: merk op dat:
\begin{enumerate}
	\item je de lijn lim(som) = som(\textit{eindige} lims) leest als 'de limiet van een som is de som van de (eindige) limieten'. Dit is een voorbeeld van een in de wiskunde erg populair \link[mantra]{https://nl.wikipedia.org/wiki/Mantra} dat zeer de moeite waard is om vertrouwd mee te worden. Het mantra is steeds van de vorm "de X van de Y's is de Y van de X'en", met X en Y twee concepten of uitdrukkingen. De distributiviteit van het product ten opzichte van de som wordt dan 'het product van een som is de som van de producten', en dat is inderdaad precies wat de distributiviteit zegt: $a\cdot(b+c) = a\cdot b+a\cdot c$: het linkerlid is het product van ($a$ met) een som (namelijk $b+c$)en dat is gelijk aan het rechterlid, namelijk een som van (twee) producten (namelijk $a\cdot b$ en $a\cdot c$). Een groot deel van de wiskunde kan worden geformuleerd in dit soort mantra's, en het is van belang te weten wanneer ze opgaan, en wanneer niet.Zowerkt hetr bijvoorbeeld \textit{niet} voor de sinus en de som: de sinus van een som is \textit{niet} gelijk aan de som van de sinussen: $\sin(\alpha+\beta) \neq \sin\alpha+\sin\beta$). 
	\item deze rekenregels het leven erg eenvoudig maken: alles is zo eenvoudig als überhaupt mogelijk! We vinden het zelfs niet zinvol om voor deze eigenschap erg uitgebreide sets oefeningen te voorzien! Jammer genoeg gelden deze regels \textsc{niet} meer altijd van zodra oneindige limieten $\pm\infty$ voorkomen. In dat geval wordt het leven veel ingewikkelder, zoals uit verdere voorbeelden zal blijken.
	\item voor continue functies, en onder bepaalde bestaansvoorwaarden, eigenschap (\ref{elim-continu}) zegt dat de limiet en de functieoproep van plaats mogen worden gewisseld, of nog dat de limiet 'door' de functie gaat: de limiet van de functie (van iets) is de functie van de limiet (van dat iets). Bijvoorbeeld: $\lim(\sin(x^2-1)) = \sin(\lim(x^2-1))$.
	\item er in verband met eigenschap (\ref{elim-continu}) voorlopig nog geen nuttige toepassingen bekend zijn van de formules lim($f$(iets)) =  $f$(lim(iets)) in de rijwielsector, hoewel kwatongen beweren dat er daar misschien toch mogelijkheden  zouden kunnen zijn. 
\end{enumerate}
\begin{example}
	Zij $a\in\R$, en beschouw de functies $f(x)=x$ en $g(x)=a-x$. Dan geldt (evident) dat $f(x)+g(x)=x+(a-x)=a$ en dat 
	\begin{align*}
	\limxi f(x) & = \limxi x = +\infty \\
	\limxi g(x) & = \limxi (a-x) = -\infty \\
	\limxi (g(x)+f(x)) & = \limxi a = a \\
	\end{align*}
    In dit geval impliceert dus $\limx[c] \big(f(x) +g(x)\big) = \limx[c] f(x) + \limx[c] g(x)$ dat $(+\infty) + (-\infty)$ gelijk zou moeten zijn aan $a$. Maar $a$ is willekeurig (of \textit{onbepaald}): we kunnen voor $a$ even goed $0$ kiezen als $2$ of $2\pi/3$ \dots! Dus: de rekenregels voor de som van limieten impliceren dat $(+\infty)+(-\infty)$ elke waarde kan aannemen, en dus noodzakelijkerwijze \textit{onbepaald} moet blijven. Jammer maar helaas ...
    
    
\end{example}
\begin{example}
	
Zoek een gelijkaardig eenvoudig (tegen)voorbeeld voor $0\cdot\infty$ en voor $\dfrac00$ en $\dfrac\infty\infty$.
\end{example}



We zijn dus verplicht om voldoende voorzichtig rekenregels vast te leggen voor de symbolen $+\infty$ en $-\infty$. We kunnen rekenen met $+\infty$ en $-\infty$  op voorwaarde dat we zogenaamde \textit{onbepaalde vormen} vermijden.

Merk op: we schrijven verder meestal gewoon $\infty$ voor $+\infty$. \\
En $\pm\infty$ betekent $+\infty$ of $-\infty$ (in die volgorde, zodat we ook kunnen schrijven dat $-(\pm\infty) = \mp\infty$!)

Eerst enkele regels die wel gewoon werken zoals het hoort:
\begin{proposition} (Rekenregels voor de symbolen $+\infty$ en $-\infty$)
	
% todo: Uitleg over symbool X ...?
Zij $a\in\R$. Dan geldt:	
	\begin{align*}
	-(+\infty) & = -\infty \\
	-(-\infty) & = +\infty  & \text{($+\infty$ en $-\infty$ zijn elkaars tegengestelde)}\\ 
	\\
		\infty & = +\infty   \\
		\pm\infty &\text{ staat voor} +\infty \text{ of } -\infty \\
		-(\pm\infty) &= \mp\infty &\text{(voor de hand liggende afspraken)} \\
	\\
	a+\infty & = \infty+a = \infty - a = -a +\infty \\
	         & = \infty + \infty = \infty & \text{($\infty + X =\infty$ behalve als $X=-\infty$)} \\
	\\
	a+(-\infty) & = a-\infty = -\infty+a =-\infty - a = -a - \infty \\
	         & = -\infty - \infty = -\infty+(-\infty) = -\infty & \text{($-\infty+X=-\infty$ behalve als $X=+\infty$)} \\
\intertext{Als $a>0$ ($a$ strikt positief)}	         
	a\cdot\infty & = \infty\cdot a \\
& = \infty \cdot \infty = (-\infty) \cdot (-\infty) = \infty & \text{($X\cdot\pm\infty=\pm\infty$ als $X>0$)} \\
\intertext{Als $a<0$ ($a$ strikt negatief)}	         
a\cdot\infty & = \infty\cdot a \\
& = \infty \cdot (-\infty) = (-\infty)\cdot\infty = -\infty & \text{($X\cdot(\pm\infty) = \mp\infty$ als $X<0$)} \\
	\end{align*}	
\end{proposition}  

Ook al deze regels zijn erg natuurlijk, en stellen op zich geen probleem. \\
Maar er zit dus een enigszins venijnig addertje onder het gras: er ontbreken enkele combinaties en die \textit{gelden niet} (altijd). Op examens durven studenten zich wel eens laten bijten door zo'n addertje. Dat is steeds erg pijnlijk. Je probeert het dus best te vermijden.

\begin{proposition} (\textbf{Onbepaalde vormen} met $\pm\infty$)
	
	Volgende uitdrukkingen zijn onbepaald. Dat betekent dat ze \textit{geen betekenis} hebben.	
	
	\begin{tabular}{cccr}
	$\infty - \infty$ &$(+\infty) + (-\infty)$  & 	$(-\infty) + (+\infty) $  & oneindig min oneindig \\
	$0\cdot\infty$ & $0 \cdot (\pm\infty)$   & 	$(\pm\infty)\cdot 0$   & nul maal oneindig \\
	\\
	% Mmm, shortstack werkt niet (goed) in html (wordt een afbeelding!)
	$\dfrac 00$ & $\dfrac\infty\infty$ & $\dfrac{\pm\infty}{\mp\infty}$ & \shortstack{\hfill nul gedeeld door nul \\ oneindig gedeeld door oneindig}
	\end{tabular}
\end{proposition}
Merk op: ook $0^0$, $\infty^0$ en $1^\infty$ zijn onbepaald, maar we zullen ze in deze cursus niet tegenkomen.

Merk op: als je bij een berekening een onbepaalde vorm uitkomt, is er iets mis. Je zal een andere berekeningswijze moeten vinden, die de onbepaalde vorm vermijdt.

\subsection{Limieten in $c=\pm\infty$ }


\begin{exercise}
Probeer met de rekenregels voor limieten en voor $\pm\infty$ om $\limxi (3x^2-4x+2)$ te berekenen.

\todo{TODO: uitleg toevoegen}

\end{exercise}

\begin{proposition} (Limiet veelterm in $c=\pm\infty$)

De limiet van een veelterm in $\pm\infty$ is de limiet van de hoogstegraadsterm (en dus gelijk aan $+\infty$ of $-\infty$ naargelang het teken van de hoogstegraadscoëfficiënt en het al dan niet even zijn van de graad van de veelterm).

De limiet van een rationale functie in $\pm\infty$ wordt bepaald door de hoogstegraadstermen van teller en noemer (en is gelijk aan $0$, $\pm\infty$ of het quotiënt van de hoogstegraadscoëfficiënten).
\end{proposition}


\begin{exercise}
	\textit{Bereken met behulp van voorgaande rekenregels}
	\begin{enumerate}
		\renewcommand{\labelenumi}{(\alph{enumi})}
		\item $\displaystyle{\lim_{x \rightarrow + \infty}} (-4x^5 + 7x^4 -
		11)$
		\item $\displaystyle{\lim_{x \rightarrow - \infty}} \frac{-x^2 + 3x
			- 2}{3x-7}$
		\item $\displaystyle{\lim_{x \rightarrow + \infty}} \frac{2x^2 -
			3}{x^3 - x + 2}$
		\item $\displaystyle{\lim_{x \rightarrow - \infty}} \frac{x^3 -
			1}{x - 1}$
		\item $\displaystyle{\lim_{x \rightarrow + \infty}} \frac{x^6 +
			5x^2 + 1}{x^4 - x^6}$
		\item $\displaystyle{\lim_{x \rightarrow + \infty}} (\sqrt{4x^2 - 7x} + 3x)$
		\item $\displaystyle{\lim_{x \rightarrow + \infty}} (\sqrt{2 x^2 + 3} - x)$
		\item $\displaystyle{\lim_{x \rightarrow - \infty}} (\sqrt{2 x^2 + 3} - x)$
		\item $\displaystyle{\lim_{x \rightarrow + \infty}} \frac{\sqrt{x^2 + 5} + 2x}{3x-1}$
		\item $\displaystyle{\lim_{x \rightarrow - \infty}} \frac{\sqrt{x^2 + 5} + 2x}{3x-1}$
		\item $\displaystyle{\lim_{x \rightarrow - \infty}} \frac{\sqrt{x^2+1} + ax}{\sqrt{x^2+2}}$
		\item $\displaystyle{\lim_{x \rightarrow - \infty}} (\sqrt{4x^2 + 7x} +
		2x)$  
		\item $\displaystyle{\lim_{x \rightarrow - \infty}} (\sqrt{4x^2 + 7x} +
		ax)$  
		\item $\displaystyle{\lim_{x \rightarrow + \infty}} x(\sqrt{2+ \frac{3}{x^2}} - \sqrt{2- \frac{3}{x^2}})$
		\item $\displaystyle{\lim_{x \rightarrow - \infty}} \frac{ax + b}{7x- \sqrt{x^2 + 10}}$
	\end{enumerate}
\begin{comment}  % remove for now: to be integrated with \anwer / \choice \ \... ?
	\begin{oplossing}
		\begin{enumerate}
			\renewcommand{\labelenumi}{(\alph{enumi})}
			\item $\displaystyle{\lim_{x \rightarrow + \infty}} (-4x^5 + 7x^4 -	11) = - \infty$
			\item $\displaystyle{\lim_{x \rightarrow - \infty}} \frac{-x^2 + 3x	- 2}{3x-7} = + \infty$
			\item $\displaystyle{\lim_{x \rightarrow + \infty}} \frac{2x^2 - 3}{x^3 - x + 2} = 0$
			\item $\displaystyle{\lim_{x \rightarrow - \infty}} \frac{x^3 -	1}{x - 1} = + \infty$
			\item $\displaystyle{\lim_{x \rightarrow + \infty}} \frac{x^6 + 5x^2 + 1}{x^4 - x^6}= -1$
			\item $\displaystyle{\lim_{x \rightarrow + \infty}} (\sqrt{4x^2 - 7x} + 3x)= + \infty$
			\item $\displaystyle{\lim_{x \rightarrow + \infty}} (\sqrt{2 x^2 + 3} - x)= + \infty$
			\item $\displaystyle{\lim_{x \rightarrow - \infty}} (\sqrt{2 x^2 + 3} - x)= + \infty$
			\item $\displaystyle{\lim_{x \rightarrow + \infty}} \frac{\sqrt{x^2 + 5} + 2x}{3x-1} = 1$
			\item $\displaystyle{\lim_{x \rightarrow - \infty}} \frac{\sqrt{x^2 + 5} + 2x}{3x-1} = \frac13$
			\item $\displaystyle{\lim_{x \rightarrow - \infty}} \frac{\sqrt{x^2+1} + ax}{\sqrt{x^2+2}} = 1-a$
			\item $\displaystyle{\lim_{x \rightarrow - \infty}} (\sqrt{4x^2 + 7x} + 2x)= - \frac74$
			\item $\displaystyle{\lim_{x \rightarrow - \infty}} (\sqrt{4x^2 + 7x} +	ax)$  
			\begin{itemize}
				\item[] voor $a<2: + \infty$
				\item[] voor $a>2: - \infty$
				\item[] voor $a=2:$ zie vorige oefening 
			\end{itemize}
			\item $\displaystyle{\lim_{x \rightarrow + \infty}} x(\sqrt{2+ \frac{3}{x^2}} - \sqrt{2- \frac{3}{x^2}}) = 0$
			\item $\displaystyle{\lim_{x \rightarrow - \infty}} \frac{ax + b}{7x- \sqrt{x^2 + 10}}= \frac{a}{8}$
		\end{enumerate}
	\end{oplossing}
\end{comment}
\end{exercise}

\subsection{Limieten in nulpunten van de noemer}

\todo{TODO: wijzen op herhaling samenstelling functies (f(1/x) / 1/f(x) / f(x+a) etc)}

\subsubsection{Limieten van $\dfrac{1}{g(x)}$ met $g(x)$ een veelterm}

\begin{example}
	De functie $f_1(x)=\dfrac{1}{x}$ heeft geen limiet in $x=0$, 
	omdat $\llimx f_1(x) = -\infty$ terwijl $\rlimx f_1(x) = +\infty$. 
	Maar, $f_2(x)=\dfrac{1}{x^2} = +\infty$ omdat $x^2$ (en dus ook $\frac{1}{x^2}$) altijd positief is 
	(in de buurt van $0$ zou voldoende zijn, maar $x^2$ is natuurlijk overal positief). 
	
	Om in het algemeen het limietgedrag van $\dfrac{1}{g(x)}$ te kennen, is het dus (nodig en) voldoende om het tekenverloop te kennen van $g(x)$ in de buurt van zijn nulpunten. 
	
	Op de onderstaande tekening is $g(x)=(x+5)(x-5)(x-1)^2/50$ getekend, samen met $1/g(x)$. We zien dat de limiet in $x=1$ gelijk is aan $-\infty$ (omdat $g(x)$ er een dubbel nulpunt heeft en negatief is in de omgeving van $-1$, terwijl er in $x=\pm5$ telkens enkelvoudige nulpunten zijn, waar $g(x)$ dus van teken verandert, en $1/g(x)$ dus geen limiet heeft (maar enkel een linker- en een rechterlimiet).
	

\begin{image}
%\begin{figure}
	\begin{tikzpicture}
	 [
	declare function=
	{
		t(\x) = (\x-5)*(\x-1)^2*(x+5)/50 ;
	}
	]
	\begin{axis}
	[ samples=1000, axis lines=center,
	  axis equal,
	  scale=2,
	  restrict y to domain=-10:10,
	  ymax=8,ymin=-8,
      extra y ticks={0},
      legend style={at={(0.15,0.45)},anchor=north east},
	]
	\addplot[domain=-8.001:8,semithick,dashed,color=blue] {t(x)};
	\addplot[domain=-8.001:8,thick,color=red] {1/t(x)};
	\legend{$g(x)$, $1/g(x)$}
	\end{axis}
	\end{tikzpicture}
%	\captionof{figure}{Test}
%\end{figure}
\end{image}

\end{example}

Voor het vervolg is het erg belangrijk ten diepste overtuigd te zijn van de waarheid en trivialiteit van volgende twee eigenschappen (die op de vorige tekening duidelijk worden geïllustreerd):

\begin{proposition} (Triviale eigenschap veeltermen en rationale functies)
% \todo: er is meer vertikale whitespace tussen de titel en de enumerate dan tussen de titel en de \paragraph infra!
\begin{enumerate}
	\item Een veelterm $f(x)$ kan alleen van teken veranderen in een nulpunt. 
	\item Een veelterm verandert niet van teken in elk nulpunt.
	\item Een rationale functie $f(x)$ kan alleen van teken veranderen in een nulpunt van teller of noemer. 
	\item Een rationale functie  verandert niet van teken in elk nulpunt van teller of noemer.
\end{enumerate}
\end{proposition}

\begin{proposition} (Triviale eigenschap omgekeerde functie)
	
Zij $f$ een reële functie, $c\in\R$. Dan geldt dat
\begin{enumerate}
\item	$\dfrac{1}{f(c)}$ heeft altijd hetzelfde teken als $f(c)$  \quad(tenzij $f(c)=0$\dots)

\item    $\dfrac{1}{f(x)}$ heeft \textit{mogelijk} een oneindige limiet in $x=c \quad\iff\quad f(c)=0$ 

\item    $\dfrac{1}{f(x)}$ heeft \textit{echt} een oneindige limiet in $x=c \quad\iff\quad f(c)=0$ en $f(x)$ verandert niet van teken in de buurt van $c$ 
\end{enumerate}
\end{proposition}

\subsubsection{Limieten van rationale functies $f(x)/g(x)$}

\begin{verbatim}
Er zijn volgende mogelijkheden met $m_teller$ en $m_noemer$ de multipliciteiten van $c$:

* de limiet is een getal verschillend van 0  
                           (als $m_{noemer} = m_{teller}$)
* de limiet is 0           (als $m_{noemer} > m_{teller}$)
* de limiet is $\pm\infty$ (als $m_{noemer} < m_{teller}$ en llim = rlim)
* de limiet bestaat niet   (als $m_{noemer} < m_{teller}$ en llim \neq rlim)

\end{verbatim}



\end{document}
