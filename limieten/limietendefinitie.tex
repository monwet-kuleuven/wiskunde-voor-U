\documentclass{ximera}
%\documentclass[numbers,wordchoicegiven]{ximera}

% todo: proper use of options of documentclass to be investigated!!!
% eg with '\wordchoice'
% -> in pdf: just pint the correct options
% -> in HTML: provide all options
% BUT: in exercises: also in pdf provide all options!!!

%\providecommand\showtodonotes{}

\usepackage[a4paper]{geometry}

%\usepackage[utf8]{inputenc}
\usepackage{multicol}


\graphicspath{
	{./}
	{goniometrie/}
}

% we willen (bijna) altijd \geqslant ipv \geq ...!
\newcommand{\geqnoslant}{\geq}
\renewcommand{\geq}{\geqslant}
\newcommand{\leqnoslant}{\leq}
\renewcommand{\leq}{\leqslant}

%overkill? Gebuikt in module limieten
\newcommand{\naar}{\rightarrow}

% Shortcuts voor limieten
% MERK OP: hier kan dus ook de notatie voor linker/rechterlimiet worden gekozen !!!
% Usage: \limx geeft lim voor x-> 0;  \limx[a^2]  geeft lim voor x-> a^2 en \limxi geeft lim voor x -> \infy \limxmi -> -\infty 
% Mmm, zonder de \ifblank lijkt het niet te werken in htlatex ...?
\newcommand{\limx}[1][]{\lim_{x \rightarrow \ifblank{#1}{0}{#1}}}
\newcommand{\llimx}[1][]{\lim_{x \underset < \rightarrow \ifblank{#1}{0}{#1}}}
\newcommand{\rlimx}[1][]{\lim_{x \underset > \rightarrow \ifblank{#1}{0}{#1}}}

\newcommand{\limxi}{\limx[+\infty]}  % I voor \Infty
\newcommand{\limxmi}{\limx[-\infty]} % MI voor Min \Infty

% bestaan niet ...!
%\newcommand{\rlimxi}{\rlimx[+\infty]}
%\newcommand{\rlimxmi}{\rlimx[-\infty]}
%
%\newcommand{\llimxi}{\llimx[+\infty]}
%\newcommand{\llimxmi}{\llimx[-\infty]}

%
% Poging tot aanpassen layout
%
\usepackage{mdframed}
\usepackage{tcolorbox}
\tcbuselibrary{theorems}

% Herdefinieer enkele omgevingen (PAS OP: enkel voor PDF, voor html: zie css..!!!)
% remove italics def
\makeatletter   % because of the @ below: make @ a (normal) letter!!
\let\definition\relax
\let\c@definition\relax
\let\enddefinition\relax
\theoremstyle{definition}
%\newtheorem*{definition}{Definitie}
\newmdtheoremenv{definition}{Definitie}
%\newtcbtheorem[number within=section]{definition}{Definitie}{colback=blue!5,colframe=blue!35!black,fonttitle=\bfseries}{th}
%\newtcbtheorem{definition}{Definitie}{colback=blue!5,colframe=blue!35!black,fonttitle=\bfseries}{th}



% remove italics def
\let\example\relax
\let\c@example\relax
\let\endexample\relax
\theoremstyle{definition}
\newtheorem{example}{Voorbeeld}

% remove italics def
\let\remark\relax
\let\c@remark\relax
\let\endremark\relax
\theoremstyle{definition}
\newtheorem{remark}{Opmerking}

% remove italics def
\let\proposition\relax
\let\c@proposition\relax
\let\endproposition\relax
%\theoremstyle{proposition}
\newmdtheoremenv{proposition}{Eigenschap}

% remove italics def
\let\problem\relax
\let\c@problem\relax
\let\endproblem\relax
%\theoremstyle{problem}
\newtheorem{problem}{Voorbeeld oefening}

% remove italics def
\let\exercise\relax
\let\c@exercise\relax
\let\endexercise\relax
%\theoremstyle{problem}
\newtheorem{exercise}{Oef.}

\newtheorem*{oplossing}{Oplossing}
%\newtheorem{oplossing}[definition]{Oplossing}

\makeatother




%definities nieuwe commando's - afkortingen veel gebruikte symbolen
\newcommand{\ds}{\displaystyle}
\newcommand{\R}{\ensuremath{\mathbb{R}}}
\newcommand{\Rnul}{\ensuremath{\mathbb{R}_0}}
\newcommand{\Reen}{\ensuremath{\mathbb{R}\setminus\{1\}}}
\newcommand{\Rnuleen}{\ensuremath{\mathbb{R}\setminus\{0,1\}}}
\newcommand{\Rplus}{\ensuremath{\mathbb{R}^+}}
\newcommand{\Rmin}{\ensuremath{\mathbb{R}^-}}
\newcommand{\Rnulplus}{\ensuremath{\mathbb{R}_0^+}}
\newcommand{\Rnulmin}{\ensuremath{\mathbb{R}_0^-}}
\newcommand{\Rnuleenplus}{\ensuremath{\mathbb{R}^+\setminus\{0,1\}}}
\newcommand{\N}{\ensuremath{\mathbb{N}}}
\newcommand{\Nnul}{\ensuremath{\mathbb{N}_0}}
\newcommand{\Z}{\ensuremath{\mathbb{Z}}}
\newcommand{\Znul}{\ensuremath{\mathbb{Z}_0}}
\newcommand{\Zplus}{\ensuremath{\mathbb{Z}^+}}
\newcommand{\Zmin}{\ensuremath{\mathbb{Z}^-}}
\newcommand{\Znulplus}{\ensuremath{\mathbb{Z}_0^+}}
\newcommand{\Znulmin}{\ensuremath{\mathbb{Z}_0^-}}
\newcommand{\C}{\ensuremath{\mathbb{C}}}
\newcommand{\Cnul}{\ensuremath{\mathbb{C}_0}}
\newcommand{\Cplus}{\ensuremath{\mathbb{C}^+}}
\newcommand{\Cmin}{\ensuremath{\mathbb{C}^-}}
\newcommand{\Cnulplus}{\ensuremath{\mathbb{C}_0^+}}
\newcommand{\Cnulmin}{\ensuremath{\mathbb{C}_0^-}}
\newcommand{\Q}{\ensuremath{\mathbb{Q}}}
\newcommand{\Qnul}{\ensuremath{\mathbb{Q}_0}}
\newcommand{\Qplus}{\ensuremath{\mathbb{Q}^+}}
\newcommand{\Qmin}{\ensuremath{\mathbb{Q}^-}}
\newcommand{\Qnulplus}{\ensuremath{\mathbb{Q}_0^+}}
\newcommand{\Qnulmin}{\ensuremath{\mathbb{Q}_0^-}}
\newcommand{\perdef}{\overset{\mathrm{def}}{=}}
\newcommand{\pernot}{\overset{\mathrm{not}}{=}}
\newcommand{\bgsin}{\mathrm{bgsin}\,}
\newcommand{\bgcos}{\mathrm{bgcos}\,}
\newcommand{\bgtan}{\mathrm{bgtan}\,}
\newcommand{\bgcot}{\mathrm{bgcot}\,}
\newcommand{\bgsinh}{\mathrm{bgsinh}\,}
\newcommand{\bgcosh}{\mathrm{bgcosh}\,}
\newcommand{\bgtanh}{\mathrm{bgtanh}\,}
\newcommand{\bgcoth}{\mathrm{bgcoth}\,}
\newcommand{\Bgsin}{\mathrm{Bgsin}\,}
\newcommand{\Bgcos}{\mathrm{Bgcos}\,}
\newcommand{\Bgtan}{\mathrm{Bgtan}\,}
\newcommand{\Bgcot}{\mathrm{Bgcot}\,}
\newcommand{\Bgsinh}{\mathrm{Bgsinh}\,}
\newcommand{\Bgcosh}{\mathrm{Bgcosh}\,}
\newcommand{\Bgtanh}{\mathrm{Bgtanh}\,}
\newcommand{\Bgcoth}{\mathrm{Bgcoth}\,}
\newcommand{\cosec}{\mathrm{cosec}\,}
\newcommand{\dom}{\mathrm{dom}\,}
\newcommand{\bld}{\mathrm{bld}\,}
\newcommand{\graf}{\mathrm{graf}\,}
\newcommand{\rc}{\mathrm{rc}\,}
\newcommand{\co}{\mathrm{co}\,}
\newcommand{\oefverwijzing}[1]{\ensuremath{\hookrightarrow}\ \textsl{#1}}
\newcommand{\startletternummering}{\renewcommand{\labelenumi}{(\alph{enumi})}}
\newcommand{\eindeletternummering}{\renewcommand{\labelenumi}{\arabic{enumi}.}}
\newcommand{\bron}[1]{\begin{scriptsize} \emph{#1} \end{scriptsize}} 


%
% copied from https://github.com/mooculus/calculus
%

%\usepackage{todonotes}
%\usepackage{mathtools} %% Required for wide table Curl and Greens
%\usepackage{cuted} %% Required for wide table Curl and Greens
\newcommand{\todo}{}

% Font niet (correct?) geinstalleerd in MikTeX?
%\usepackage{esint} % for \oiint
%\ifxake%%https://math.meta.stackexchange.com/questions/9973/how-do-you-render-a-closed-surface-double-integral
%\renewcommand{\oiint}{{\large\bigcirc}\kern-1.56em\iint}
%\fi


\newcommand{\mooculus}{\textsf{\textbf{MOOC}\textnormal{\textsf{ULUS}}}}

\usepackage{tkz-euclide}\usepackage{tikz}
\usepackage{tikz-cd}
\usetikzlibrary{arrows}
\tikzset{>=stealth,commutative diagrams/.cd,
  arrow style=tikz,diagrams={>=stealth}} %% cool arrow head
\tikzset{shorten <>/.style={ shorten >=#1, shorten <=#1 } } %% allows shorter vectors

\usetikzlibrary{backgrounds} %% for boxes around graphs
\usetikzlibrary{shapes,positioning}  %% Clouds and stars
\usetikzlibrary{matrix} %% for matrix
\usepgfplotslibrary{polar} %% for polar plots
\usepgfplotslibrary{fillbetween} %% to shade area between curves in TikZ
\usetkzobj{all}
\usepackage[makeroom]{cancel} %% for strike outs
%\usepackage{mathtools} %% for pretty underbrace % Breaks Ximera
%\usepackage{multicol}
\usepackage{pgffor} %% required for integral for loops



%% http://tex.stackexchange.com/questions/66490/drawing-a-tikz-arc-specifying-the-center
%% Draws beach ball
\tikzset{pics/carc/.style args={#1:#2:#3}{code={\draw[pic actions] (#1:#3) arc(#1:#2:#3);}}}



\usepackage{array}
\setlength{\extrarowheight}{+.1cm}
\newdimen\digitwidth
\settowidth\digitwidth{9}
\def\divrule#1#2{
\noalign{\moveright#1\digitwidth
\vbox{\hrule width#2\digitwidth}}}





\newcommand{\RR}{\mathbb R}
%\newcommand{\R}{\mathbb R}
%\newcommand{\N}{\mathbb N}
%\newcommand{\Z}{\mathbb Z}

\newcommand{\sagemath}{\textsf{SageMath}}


%\renewcommand{\d}{\,d\!}
\renewcommand{\d}{\mathop{}\!d}
\newcommand{\dd}[2][]{\frac{\d #1}{\d #2}}
\newcommand{\pp}[2][]{\frac{\partial #1}{\partial #2}}
\renewcommand{\l}{\ell}
\newcommand{\ddx}{\frac{d}{\d x}}

\newcommand{\zeroOverZero}{\ensuremath{\boldsymbol{\tfrac{0}{0}}}}
\newcommand{\inftyOverInfty}{\ensuremath{\boldsymbol{\tfrac{\infty}{\infty}}}}
\newcommand{\zeroOverInfty}{\ensuremath{\boldsymbol{\tfrac{0}{\infty}}}}
\newcommand{\zeroTimesInfty}{\ensuremath{\small\boldsymbol{0\cdot \infty}}}
\newcommand{\inftyMinusInfty}{\ensuremath{\small\boldsymbol{\infty - \infty}}}
\newcommand{\oneToInfty}{\ensuremath{\boldsymbol{1^\infty}}}
\newcommand{\zeroToZero}{\ensuremath{\boldsymbol{0^0}}}
\newcommand{\inftyToZero}{\ensuremath{\boldsymbol{\infty^0}}}



\newcommand{\numOverZero}{\ensuremath{\boldsymbol{\tfrac{\#}{0}}}}
\newcommand{\dfn}{\textbf}
%\newcommand{\unit}{\,\mathrm}
\newcommand{\unit}{\mathop{}\!\mathrm}
\newcommand{\eval}[1]{\bigg[ #1 \bigg]}
\newcommand{\seq}[1]{\left( #1 \right)}
\renewcommand{\epsilon}{\varepsilon}
\renewcommand{\phi}{\varphi}


\renewcommand{\iff}{\Leftrightarrow}

\DeclareMathOperator{\arccot}{arccot}
\DeclareMathOperator{\arcsec}{arcsec}
\DeclareMathOperator{\arccsc}{arccsc}
\DeclareMathOperator{\si}{Si}
\DeclareMathOperator{\scal}{scal}
\DeclareMathOperator{\sign}{sign}


%% \newcommand{\tightoverset}[2]{% for arrow vec
%%   \mathop{#2}\limits^{\vbox to -.5ex{\kern-0.75ex\hbox{$#1$}\vss}}}
\newcommand{\arrowvec}[1]{{\overset{\rightharpoonup}{#1}}}
%\renewcommand{\vec}[1]{\arrowvec{\mathbf{#1}}}
\renewcommand{\vec}[1]{{\overset{\boldsymbol{\rightharpoonup}}{\mathbf{#1}}}\hspace{0in}}

\newcommand{\point}[1]{\left(#1\right)} %this allows \vector{ to be changed to \vector{ with a quick find and replace
\newcommand{\pt}[1]{\mathbf{#1}} %this allows \vec{ to be changed to \vec{ with a quick find and replace
\newcommand{\Lim}[2]{\lim_{\point{#1} \to \point{#2}}} %Bart, I changed this to point since I want to use it.  It runs through both of the exercise and exerciseE files in limits section, which is why it was in each document to start with.

\DeclareMathOperator{\proj}{\mathbf{proj}}
\newcommand{\veci}{{\boldsymbol{\hat{\imath}}}}
\newcommand{\vecj}{{\boldsymbol{\hat{\jmath}}}}
\newcommand{\veck}{{\boldsymbol{\hat{k}}}}
\newcommand{\vecl}{\vec{\boldsymbol{\l}}}
\newcommand{\uvec}[1]{\mathbf{\hat{#1}}}
\newcommand{\utan}{\mathbf{\hat{t}}}
\newcommand{\unormal}{\mathbf{\hat{n}}}
\newcommand{\ubinormal}{\mathbf{\hat{b}}}

\newcommand{\dotp}{\bullet}
\newcommand{\cross}{\boldsymbol\times}
\newcommand{\grad}{\boldsymbol\nabla}
\newcommand{\divergence}{\grad\dotp}
\newcommand{\curl}{\grad\cross}
%\DeclareMathOperator{\divergence}{divergence}
%\DeclareMathOperator{\curl}[1]{\grad\cross #1}
\newcommand{\lto}{\mathop{\longrightarrow\,}\limits}

\renewcommand{\bar}{\overline}

\colorlet{textColor}{black}
\colorlet{background}{white}
\colorlet{penColor}{blue!50!black} % Color of a curve in a plot
\colorlet{penColor2}{red!50!black}% Color of a curve in a plot
\colorlet{penColor3}{red!50!blue} % Color of a curve in a plot
\colorlet{penColor4}{green!50!black} % Color of a curve in a plot
\colorlet{penColor5}{orange!80!black} % Color of a curve in a plot
\colorlet{penColor6}{yellow!70!black} % Color of a curve in a plot
\colorlet{fill1}{penColor!20} % Color of fill in a plot
\colorlet{fill2}{penColor2!20} % Color of fill in a plot
\colorlet{fillp}{fill1} % Color of positive area
\colorlet{filln}{penColor2!20} % Color of negative area
\colorlet{fill3}{penColor3!20} % Fill
\colorlet{fill4}{penColor4!20} % Fill
\colorlet{fill5}{penColor5!20} % Fill
\colorlet{gridColor}{gray!50} % Color of grid in a plot

\newcommand{\surfaceColor}{violet}
\newcommand{\surfaceColorTwo}{redyellow}
\newcommand{\sliceColor}{greenyellow}




\pgfmathdeclarefunction{gauss}{2}{% gives gaussian
  \pgfmathparse{1/(#2*sqrt(2*pi))*exp(-((x-#1)^2)/(2*#2^2))}%
}


%%%%%%%%%%%%%
%% Vectors
%%%%%%%%%%%%%

%% Simple horiz vectors
%\renewcommand{\vector}[1]{\left\langle #1\right\rangle}


%% %% Complex Horiz Vectors with angle brackets
%% \makeatletter
%% \renewcommand{\vector}[2][ , ]{\left\langle%
%%   \def\nextitem{\def\nextitem{#1}}%
%%   \@for \el:=#2\do{\nextitem\el}\right\rangle%
%% }
%% \makeatother

%% %% Vertical Vectors
%% \def\vector#1{\begin{bmatrix}\vecListA#1,,\end{bmatrix}}
%% \def\vecListA#1,{\if,#1,\else #1\cr \expandafter \vecListA \fi}

%%%%%%%%%%%%%
%% End of vectors
%%%%%%%%%%%%%

%\newcommand{\fullwidth}{}
%\newcommand{\normalwidth}{}



%% makes a snazzy t-chart for evaluating functions
%\newenvironment{tchart}{\rowcolors{2}{}{background!90!textColor}\array}{\endarray}

%%This is to help with formatting on future title pages.
\newenvironment{sectionOutcomes}{}{}



%% Flowchart stuff
%\tikzstyle{startstop} = [rectangle, rounded corners, minimum width=3cm, minimum height=1cm,text centered, draw=black]
%\tikzstyle{question} = [rectangle, minimum width=3cm, minimum height=1cm, text centered, draw=black]
%\tikzstyle{decision} = [trapezium, trapezium left angle=70, trapezium right angle=110, minimum width=3cm, minimum height=1cm, text centered, draw=black]
%\tikzstyle{question} = [rectangle, rounded corners, minimum width=3cm, minimum height=1cm,text centered, draw=black]
%\tikzstyle{process} = [rectangle, minimum width=3cm, minimum height=1cm, text centered, draw=black]
%\tikzstyle{decision} = [trapezium, trapezium left angle=70, trapezium right angle=110, minimum width=3cm, minimum height=1cm, text centered, draw=black]


\author{Zomercursus KU Leuven}

\outcome{Limieten kunnen berekenen}
\outcome{Onbepaaldheden met oneindig kunnen herkennen}


\title{Limieten: Definitie}

\begin{document}
\begin{abstract}
	\cancel{The sky }$\infty$ is the limit
\end{abstract}
\maketitle

%\listoftodos

\todo[inline]{nadenken over outcomes (formulering/gebruik/...)}
\todo[inline]{nadenken over 'jij' vorm of 'we' vorm in dit soort teksten !!!}
	

\subsection{Inleiding}

In dit deel bestuderen we \textit{limieten} en leren we rekenen met het symbool $\infty$ ('\textit{oneindig}').

\todo[inline]{layout/functionaliteit 'uitwijding' (dus: expandable aanpassen/vervangen; voorlopig adhoc tcolorbox) }
\todo[inline]{Uitzoeken: hoe werkt expandable precies}
\todo[inline] {limieten: goed voorbeeldje vinden 'interesse net op de rand'}


\begin{expandable}
Voor wie (nog) niet overtuigd is dat dit een mogelijk nuttige bezigheid is, hebben we volgende% een uitwijding over het wat en waarom van limieten.


\begin{uitweiding}[Wat en waarom van limieten] \ 
	
(Merk op: dit is enkel inleidende achtergrondinformatie, en geen leerstof.)	

Dikwijls is men geïnteresseerd in wat er gebeurt 'net op de rand van wat al gekend is'. Wiskundig kunnen we dit soort situaties - in gunstige gevallen-  beschrijven en soms zelfs begrijpen in termen van \textit{limieten}. 

In het eerder eenvoudige geval van rijen is het intuïtief aannemelijk dat de rij getallen
%\todo{Notatie $(a_n)$ uitleggen (of weglaten, of verwijzen naar ...)}
\begin{align*}
% geen (afschrikwekkende?) formules ... ?
%(a_n) = \Big(\frac{1}{n^2}\Big) & = 1, \frac{1}{4}, \frac{1}{9}, \frac{1}{16},\frac{1}{25},\dots\\
(a_n) & = 1, \;\frac{1}{4}, \;\frac{1}{9}, \;\frac{1}{16},\;\frac{1}{25},\;\dots\\
                                & =1,\; 0.25,\;  0.1111\dots,\; 0.0625,\; 0.04,\; 0.0277\dots,\; \dots
\end{align*}

'in de limiet' gelijk wordt aan $0$ (hoewel er natuurlijk geen enkele van de termen $a_n$ echt $0$ is!). \\
We zeggen dat 'de limiet van de rij $(a_n)$ gelijk is aan $0$' of dat 'de $a_n$ in de limiet $0$ worden', en schrijven $\lim_{n\to\infty}a_n = 0$.

Voor de rij 
\[
%(b_n) = ((-1)^{n+1}) = 1,-1,1,-1,1,-1,\dots
(b_n) = \;1,\;-1,\;1,\;-1,\;1,\;-1,\dots
\]
lijkt er geen limiet te bestaan, 
%\todo{uitleggen waarom $b_n$ geen limiet heeft}
terwijl voor de rij 
\[
%(c_n) = (n^2) = 1,4,9,16,25,\dots
(c_n) = 1,\;4,\;9,\;16,\;25,\;\dots
\]
de natuurlijke limiet 'oneindig' zou kunnen zijn.

Een bijzonder belangrijke, maar erg subtiele, bezigheid bestaat erin te bestuderen \textit{wanneer} dergelijke limiet precies bestaat, en vooral \textit{welke} eigenschappen van dingen onder \textit{welke} voorwaarden ook \textit{in de limiet} bewaard blijven. 

Dat dit erg subtiel is blijkt uit het volgende voorbeeld voor de rij $(a_n)$ van hierboven:

De eigenschap 'is groter dan of gelijk aan nul', dus '$a_n\geq0$' geldt voor elke term $a_n$, en ze geldt ook \textit{in de limiet} (we zeggen ook 'voor de limiet') want  $0\geq0$. Dus: deze eigenschap $a_n\geq 0$ die geldt voor elke term, blijft ook gelden in de limiet. Dat is handig en mooi. 

Maar, de erg gelijkaardige eigenschap 'is groter dan nul', dus '$a_n>0$', geldt ook voor alle $a_n$, maar ze geldt niet meer in de limiet, want $0\not>0$. Dat is erg jammer en erg vervelend. Het is de oorzaak van vele extra pagina's leerstof in veel wiskundecursussen (en van veel fouten op examens, maar dit geheel terzijde). \\


In dit hoofdstuk krijg je een kleine inleiding in de problematiek van limieten van \textit{functies} (en dus niet van \textit{rijen} zoals in dit voorbeeld). Later gebruiken we die limieten om asymptoten te bekijken, afgeleiden en integralen te definiëren en nog veel meer. 

Merk op: een meer formele en exacte behandeling van limieten wordt snel technisch (bijvoorbeeld met de onder (niet-)kenners erg beruchte $\epsilon$'s en $\delta$'s) en dat valt -jammer voor ons schrijvers van cursusteksten, maar gelukkig voor u die het resultaat van onze arbeid moeten studeren- buiten het bestek van deze cursus. Sommigen van u kennen het al, en voor anderen komt het nog. Maar, velen onder u zullen aan de technische details ontsnappen.
\end{uitweiding}

\end{expandable}
\todo{voetnoot ivm epsilons/deltas ?}


\subsection{Voorbeeld en definitie}

We geven als kennismaking enkele voorbeelden van limieten van functies, om dan tot een soort van intuïtieve definitie te komen. In  deze cursus geven we geen formeel-wiskundige definitie van limieten, maar we geven voorbeelden die het begrip duidelijk moeten maken op grafieken van functies. In een volgende paragraaf geven we wel wiskundig exacte rekenregels voor limieten.

\todo{Uitzoeken: hoe werkt instructorNotes precies}
% \begin{instructorNotes}
% We kiezen sin(x)/x als eerste voorbeeld, omdat daarbij \infty geen enkele rol speelt, 
% en dus enkel het begrip 'limiet' wordt gebruikt
% 1/x is 'complexer' omdat er een mix is tussen de (in principe onafhankelijke) begrippen 'limiet' en 'oneindig'
% nadeel is dat sin(x)/x misschien ook als'complex' wordt ervaren, hoewelodat eenvoudig kan worden uitgelegd op de tekening
% \end{instructorNotes}
%}

\begin{example} (Limieten van $\frac{\sin x}{x}$)
	
Beschouw de functie $f:x\mapsto \dfrac{\sin x}{x}$. Dat is de gekende sinusfunctie die we delen door $x$. Voor grote $x$ zal die functie zeker erg dicht bij $0$ komen te liggen (want we delen de $\sin x$, wat tussen $-1$ en $1$ ligt, door een grote $x$, dus het resultaat wordt erg klein. 

De situatie is wat moeilijker direct te begrijpen voor erg kleine $x$: als we delen door kleine $x$, wordt het resultaat in principe steeds groter, maar hier wordt tegelijk ook de $\sin x$ in de teller kleiner. We delen dus iets kleins door iets kleins, en dat is niet noodzakelijk altijd klein (de meeste mensen vinden bijvoorbeeld $10^{-90}$ klein, maar $10^{-90}/10^{-100} = 10^{10}$ eerder groot ...!). 

Merk ook op dat we niet zomaar $x=0$ kunnen invullen in de formule, want dan krijgen we $\frac{\sin 0}{0} = \frac{0}{0}$, en 'dat mag niet'. 
\todo{verwijzing naar onbepaaldheid $0/0$ (evt in voetnoot)???}. 

Een rekenmachine leert ons dat voor $x=0,1$ de functiewaarde gelijk is aan $\frac{\sin(0,1)}{0,1}=0,998341\dots$, en voor  $x=0,01$ wordt dat $\frac{\sin(0,01)}{0,01}=0,999983\dots$. We  kunnen dus vermoeden dat $\frac{\sin x}{x}$ steeds dichter bij $1$ zal liggen naarmate we $x$ kleiner kiezen. Men kan ook wiskundig aantonen dat dat inderdaad zo is.

Ook uit de grafiek van de functie  $\dfrac{\sin x}{x}$ blijkt trouwens dat voor $x$ dicht bij $0$ de waarde van $\dfrac{\sin x}{x}$ erg dicht ligt bij $1$. 

We zeggen dan ook dat de functie (of de uitdrukking) $\dfrac{\sin x}{x}$ gelijk is (of ook gelijk \textit{wordt}) aan $1$ \textit{als $x$ naar $0$ gaat}, en we schrijven $\limx \dfrac{\sin x}{x} = 1$.

Over het gedrag voor grote $x$ zeggen we dat $\dfrac{\sin x}{x}$  gelijk wordt aan $0$ \textit{als $x$ naar oneindig gaat}, en we schrijven  $\limxi \dfrac{\sin x}{x} = 0$. 

\todo{ legende nog niet goed geformateerd ...}
\todo{ Uitzoeken: functionaliteit van width in image environment (online vs pdf)}
\todo{ Uitleggen waarom de rode lijn inderdaad sin(x)/x is ...?}


\begin{image}[\width]
	\begin{tikzpicture}[scale=1]
	\begin{axis}
	[
	samples=200,
	axis lines=center,
	axis equal,
	ymax=3, ymin=-1,	
    restrict y to domain=-2:3,
	extra y ticks={0},
	]
	%pas op:sin werkt in DEGREE, en niet in RADIAAL;dus sin(deg(x))...!)
	\addplot[domain=0.001:10,semithick,dashed,color=blue] {sin((deg(x)))};
	\addplot[domain=0.001:10,semithick,dotted,color=blue] {x};
	\addplot[domain=0.001:10,ultra thick,color=red] {sin((deg(x)))/x}; 
	\legend{$y=sin(x)$,$y=x$,$y=\frac{\sin x}{x}$};
	\end{axis}
	\end{tikzpicture}
\end{image}
\end{example}

\begin{example} (Limieten van $\frac{1}{x}$)
	
	We doen hetzelfde als hierboven voor $f: x\mapsto  1/x$.
	
	Voor $x$ heel groot, wordt natuurlijk $1/x$ heel \wordChoice{\choice{groot}\choice[correct]{klein}}, en dus kunnen we redelijkerwijs zeggen dat 'als  $x$  naar oneindig gaat, $1/x$ nul wordt.' We schrijven $\limxi\;\frac1x=0$.
	
	Voor $x$ heel klein (in de betekenis van heel dicht bij $0$), lijkt het ook eenvoudig: als $x$ naar $0$ gaat, dat wordt $1/x$ immers oneindig groot. Maar, er is jammer genoeg een complicatie: als $x$ negatief is en dicht bij $0$ ligt, dan wordt $x$ niet heel groot, maar net heel klein (in de betekenis van heel erg negatief). Dus: voor $x$ dicht bij $0$, kan $1/x$ ofwel heel groot zijn (als $x>0$), ofwel net heel erg negatief (als $x<0$). Dus: het hangt er maar van af 'langs welke kant' we $x$ naar $0$ laten gaan, of we in $+\infty$ dan wel in $-\infty$ uitkomen. 
	
	Het heeft geen zin om in dit geval te spreken van een limiet, maar het is in een aantal gevallen erg handig om toch een begrip en een notatie te hebben in deze situatie. We spreken van \textit{linkerlimiet} als we $0$ langs links benaderen, dus met $x<0$ (of $0>x$, dat is natuurlijk hetzelfde!), en \textit{rechterlimiet} voor langs rechts (dus $x>0$, of ook $0<x$). En we hebben dus enkel een echte \textit{limiet} als de linkerlimiet gelijk is aan de rechterlimiet!
	
	De situatie zou volledig duidelijk moeten zijn op volgende tekening:
	\begin{image}[\width]
		\begin{tikzpicture}[scale=1]
		\begin{axis}
		[
		axis equal,
		ymax=5,ymin=-5,
		samples=200,
		axis lines=center,
		extra y ticks={0},
		restrict y to domain=-10:10,
		]
		\addplot[domain=-6:6,color=blue] {1/x};
		\addlegendentry{$y=\frac1x$};	 
		\addlegendentry{};   
		\node[anchor=east, font=\small] at  (axis cs: -0.2,4.4) {$\rlimx \frac1x =+\infty$};
		\node[anchor=west] at  (axis cs: 0.2,-4) {$\llimx \frac1x =-\infty$};
		\node[anchor=south west] at (axis cs: -6,0.2) {$\limxmi \frac1x =0$};
		\node[anchor=north east] at  (axis cs: 6,-0.4) {$\limxi \frac1x =0$};

		\end{axis}
%		\node[draw,text width=4cm] at (0.1,-4) {$\llimx[0] 1/0 =1\infty$};
		\end{tikzpicture}
	\end{image}
\end{example}

\todo{tekst aanpassen: oneindig vs limiet} 
Het begrip \textit{limiet} heeft dus niet noodzakelijk een direct verband met het begrip \textit{oneindig} (tenzij in de betekenis van 'ergens oneindig dicht bij liggen'). Maar we zullen verder toch erg veel met oneindig te maken krijgen: enerzijds zullen we zeggen dat een 'limiet gelijk is aan oneindig' als de functiewaarden steeds groter worden, en anderzijds zullen we ook 'limieten in oneindig' bekijken, namelijk wat er gebeurt als $x$ steeds groter wordt.

%We bestuderen enkele eigenschappen van limieten (bij veeltermen zoals $x^5-2x^2+7$, rationale functies zoals$\frac{5x}{x-5}$ en irrationale functies zoals $\frac{\sqrt{x-1}}{x-1}$). 

We gaan zoals vermeld in deze module niet in op de exacte wiskundige definitie van een limiet. We gebruiken enkel volgende notatie en (pseudo-)definities:

\begin{definition} (Intuïtieve pseudo-definitie van limiet)
	
	Zij $f$ een continue functie, en $a$ en $c$ telkens ofwel een reëel getal, ofwel één van de symbolen $+\infty$ of $-\infty$. 
	Dan zeggen en schrijven we dat 

\begin{align*}
	\limxc  f(x) = a & \quad\iff\quad \text{de limiet van $f$ als $x$ naar $c$ gaat is (gelijk aan) $a$} \\
	                   & \quad\iff\quad \text{als }x\to c\text{, dan } 									f(x) \to a \\
	                   \\
	\llimxc f(x) = a & \quad\iff\quad \text{de linkerlimiet van $f$ als $x$ naar $c$ gaat is (gelijk aan) $a$} \\
	                   & \quad\iff\quad \text{de limiet van $f$ als $x$ langs links naar $c$ gaat is (gelijk aan) $a$} \\
	                   & \quad\iff\quad \text{als }x\to c\text{\textbf{ langs links} (dus $x<c$), dan } f(x) \to a \\
	                   \\
	\rlimxc f(x) = a & \quad\iff\quad \text{de rechterlimiet van $f$ als $x$ naar $c$ gaat is (gelijk aan) $a$} \\
					   & \quad\iff\quad \text{de limiet van $f$ als $x$ langs rechts naar $c$ gaat is (gelijk aan) $a$} \\
					   & \quad\iff\quad \text{als }x\to c\text{\textbf{ langs rechts} (dus $x>c$), dan }f(x) \to a 
\end{align*}

\end{definition}

\begin{remark} \ 

\begin{itemize}	
\item Let op: $x<c$ is hetzelfde als  $c>x$, net zoals  $x>c$ hetzelfde betekent als $c<x$. Let dus zowel op de tekens $<,>$ \textit{en} op de volgorde van de letters of cijfers!  

\item Als we  $x\to c$ lezen als '$x$ gaat naar $c$' of '$x$ ligt (erg) dicht bij $c$', dan heeft dat een zeker dynamisch aspect: $x$ wordt verondersteld te 'bewegen' naar iets. Het is belangrijk te beseffen dat het resultaat, of de 'uitkomst' van zo'n limiet, dus datgene wat met het symbool $\limxc f(x)$ wordt aangeduid, gewoon een bepaald getal is of het symbool $\pm\infty$. Daar is dus niets 'bewegends' of 'mysterieus' aan!
\end{itemize}

\end{remark}

%(overbodig) Merk op: hierbij staan zowel de letters $a$ als $c$ dus voor reële getallen \textit{of} voor één van de symbolen $+\infty$ of $-\infty$. Zie verder voor de rekenregels voor $\pm\infty$.

Deze definitie zegt dus \textit{niet} (wiskundig) nauwkeurig wat een limiet \textit{precies is}. Het volstaat -voor deze module- om op basis van dit soort definitie een intuïtief begrip op te bouwen over wat limieten zijn, en hoe ze kunnen worden berekend. Met dit soort van intuïtieve definities kunnen van limieten natuurlijk geen eigenschappen worden \textit{bewezen}, of echte berekeningen worden gemaakt. We kunnen op deze basis limieten (enkel) bepalen op volgende manieren:
\begin{enumerate}
	\item gebruik maken van betrouwbare rekenregels (zie verder)
	\item aflezen op een grafiek  
	\\(enigszins betrouwbaar als de grafiek betrouwbaar is!)
	\item ons baseren op berekeningen 'in de buurt' en/of intuïtie 
	\\(werkt dikwijls: bv $\frac{\sin(0,01)}{0,01}=0,999983\dots$, en dus geldt 'allicht' $\limx \frac{\sin x}{x} = 1$)
\end{enumerate}
Het is duidelijk dat opties 2 en  3 \textsc{strikt wiskundig niet geldig zijn}. Maar, de definitie van limiet is dat ook niet, dus tot nader order is het -mits de nodige voorzichtigheid- niet verboden, en dus toegelaten, om er toch gebruik van te maken. De 'tot nader order' hangt af van welke studierichtingen en cursussen u verder nog zal volgen! De voorlopige methode die hier wordt uiteengezet bestaat uit het toepassen van onderstaande rekenregels op gekend veronderstelde basislimieten (die bijvoorbeeld zijn bepaald via hun grafiek). 
% (is boven al gezegd) Afhankelijk van u vooropleiding kent u er eigenlijk al meer van, en afhankelijk van uw huidige opleiding, zal u er misschien in de nabije toekomst veel meer over moeten weten! 

\begin{comment}
% Zie exercices ...!

\begin{exercise} \ 
	Bereken op basis van volgende welbekende grafieken onderstaande limieten

\begin{image}[\textwidth]
	\begin{tikzpicture}
		\begin{axis}
		[
		axis equal,
		ymax=5,ymin=-5,
		samples=200,
		axis lines=center,
		extra y ticks={0},
		  restrict y to domain=-10:10,
		]
		\addplot[domain=-5:5,color=blue] {1/x^2};
		\legend{$1/x^2$}
		\end{axis}
	\end{tikzpicture}
\quad
	\begin{tikzpicture}
	\begin{axis}
	[
	axis equal,
	ymax=5,ymin=-5,
	samples=200,
	axis lines=center,
	extra y ticks={0},
    restrict y to domain=-10:10,
	]
	\addplot[domain=-5:5,color=blue] {1/x};
	\legend{$1/x$}
	\end{axis}
	\end{tikzpicture}
\quad
	\begin{tikzpicture}
	\begin{axis}
	[
    axis equal,
    ymax=5,ymin=-5,
	samples=200,
	axis lines=center,
	extra y ticks={0},
    restrict y to domain=-10:10,
	]
	\addplot[domain=-5:5,color=blue] {exp(x)};
	\addplot[domain=0.001:5,color=blue,dashed] {ln(x)};
	\legend{$e^x$, $\ln x$}
	\end{axis}
	\end{tikzpicture}
\end{image}
	
	\newcommand{\iscorrect}{}
	\newcommand{\localoefoptions}{\hfill\wordChoice{\choice{$+\infty$}\choice{$-\infty$}\choice{0}\choice{1}\choice[\iscorrect]{bestaat niet}\choice{andere oplossing}}}
	
	%\todo{adjust alignment over all items (https://tex.stackexchange.com/questions/29119 ?)}
	%\todo{vind iets omde juiste oplossing door te geven aan localoefoptions!!!} 

	% todo: limiet van sin x: bestaat niet
	% todo: limiet van sin(1/x): bestaat ook niet
	
	\begin{enumerate}
%		\item $\limx    \frac2x=\;      $\localoefoptions
		\item $\limx    \frac{1}{x^2}=\;$\let\i\localoefoptions
		\item $\limxi   e^x = \;$        \localoefoptions
		\item $\limxmi  e^x = \;$        \localoefoptions
		\item $\rlimx   e^{\frac 1x} = $
		\item $\llimx   e^{\frac 1x} = $
		\item $\limxi   \ln x = $
		\item $\rlimx   \ln x = $
		\item $\limxmi  \ln x = $
		\item $\limxi   \ln(1+ \frac 1x) = $
%		\item \[\lim_{x\to{1}}\dfrac{x^{2} + 12 \, x - 13}{x - 1}=\answer{14}\]
% from https://github.com/XronosUF/MAC2311
	\end{enumerate}
\end{exercise}
\end{comment}

\end{document}
