\documentclass{ximera}
%\documentclass[handout,numbers]{ximera}

\author{Zomercursus KU Leuven}

%\outcome{Limieten kunnen berekenen}

%
% copied from https://github.com/mooculus/calculus
%
\usepackage[utf8]{inputenc}


\graphicspath{
	{./}
	{goniometrie/}
}


%\usepackage{todonotes}
%\usepackage{mathtools} %% Required for wide table Curl and Greens
%\usepackage{cuted} %% Required for wide table Curl and Greens
\newcommand{\todo}{}

% Font niet (correct?) geinstalleerd in MikTeX?
%\usepackage{esint} % for \oiint
%\ifxake%%https://math.meta.stackexchange.com/questions/9973/how-do-you-render-a-closed-surface-double-integral
%\renewcommand{\oiint}{{\large\bigcirc}\kern-1.56em\iint}
%\fi


\newcommand{\mooculus}{\textsf{\textbf{MOOC}\textnormal{\textsf{ULUS}}}}

\usepackage{tkz-euclide}\usepackage{tikz}
\usepackage{tikz-cd}
\usetikzlibrary{arrows}
\tikzset{>=stealth,commutative diagrams/.cd,
  arrow style=tikz,diagrams={>=stealth}} %% cool arrow head
\tikzset{shorten <>/.style={ shorten >=#1, shorten <=#1 } } %% allows shorter vectors

\usetikzlibrary{backgrounds} %% for boxes around graphs
\usetikzlibrary{shapes,positioning}  %% Clouds and stars
\usetikzlibrary{matrix} %% for matrix
\usepgfplotslibrary{polar} %% for polar plots
\usepgfplotslibrary{fillbetween} %% to shade area between curves in TikZ
\usetkzobj{all}
\usepackage[makeroom]{cancel} %% for strike outs
%\usepackage{mathtools} %% for pretty underbrace % Breaks Ximera
%\usepackage{multicol}
\usepackage{pgffor} %% required for integral for loops



%% http://tex.stackexchange.com/questions/66490/drawing-a-tikz-arc-specifying-the-center
%% Draws beach ball
\tikzset{pics/carc/.style args={#1:#2:#3}{code={\draw[pic actions] (#1:#3) arc(#1:#2:#3);}}}



\usepackage{array}
\setlength{\extrarowheight}{+.1cm}
\newdimen\digitwidth
\settowidth\digitwidth{9}
\def\divrule#1#2{
\noalign{\moveright#1\digitwidth
\vbox{\hrule width#2\digitwidth}}}





\newcommand{\RR}{\mathbb R}
\newcommand{\R}{\mathbb R}
\newcommand{\N}{\mathbb N}
\newcommand{\Z}{\mathbb Z}

\newcommand{\sagemath}{\textsf{SageMath}}


%\renewcommand{\d}{\,d\!}
\renewcommand{\d}{\mathop{}\!d}
\newcommand{\dd}[2][]{\frac{\d #1}{\d #2}}
\newcommand{\pp}[2][]{\frac{\partial #1}{\partial #2}}
\renewcommand{\l}{\ell}
\newcommand{\ddx}{\frac{d}{\d x}}

\newcommand{\zeroOverZero}{\ensuremath{\boldsymbol{\tfrac{0}{0}}}}
\newcommand{\inftyOverInfty}{\ensuremath{\boldsymbol{\tfrac{\infty}{\infty}}}}
\newcommand{\zeroOverInfty}{\ensuremath{\boldsymbol{\tfrac{0}{\infty}}}}
\newcommand{\zeroTimesInfty}{\ensuremath{\small\boldsymbol{0\cdot \infty}}}
\newcommand{\inftyMinusInfty}{\ensuremath{\small\boldsymbol{\infty - \infty}}}
\newcommand{\oneToInfty}{\ensuremath{\boldsymbol{1^\infty}}}
\newcommand{\zeroToZero}{\ensuremath{\boldsymbol{0^0}}}
\newcommand{\inftyToZero}{\ensuremath{\boldsymbol{\infty^0}}}



\newcommand{\numOverZero}{\ensuremath{\boldsymbol{\tfrac{\#}{0}}}}
\newcommand{\dfn}{\textbf}
%\newcommand{\unit}{\,\mathrm}
\newcommand{\unit}{\mathop{}\!\mathrm}
\newcommand{\eval}[1]{\bigg[ #1 \bigg]}
\newcommand{\seq}[1]{\left( #1 \right)}
\renewcommand{\epsilon}{\varepsilon}
\renewcommand{\phi}{\varphi}


\renewcommand{\iff}{\Leftrightarrow}

\DeclareMathOperator{\arccot}{arccot}
\DeclareMathOperator{\arcsec}{arcsec}
\DeclareMathOperator{\arccsc}{arccsc}
\DeclareMathOperator{\si}{Si}
\DeclareMathOperator{\scal}{scal}
\DeclareMathOperator{\sign}{sign}


%% \newcommand{\tightoverset}[2]{% for arrow vec
%%   \mathop{#2}\limits^{\vbox to -.5ex{\kern-0.75ex\hbox{$#1$}\vss}}}
\newcommand{\arrowvec}[1]{{\overset{\rightharpoonup}{#1}}}
%\renewcommand{\vec}[1]{\arrowvec{\mathbf{#1}}}
\renewcommand{\vec}[1]{{\overset{\boldsymbol{\rightharpoonup}}{\mathbf{#1}}}\hspace{0in}}

\newcommand{\point}[1]{\left(#1\right)} %this allows \vector{ to be changed to \vector{ with a quick find and replace
\newcommand{\pt}[1]{\mathbf{#1}} %this allows \vec{ to be changed to \vec{ with a quick find and replace
\newcommand{\Lim}[2]{\lim_{\point{#1} \to \point{#2}}} %Bart, I changed this to point since I want to use it.  It runs through both of the exercise and exerciseE files in limits section, which is why it was in each document to start with.

\DeclareMathOperator{\proj}{\mathbf{proj}}
\newcommand{\veci}{{\boldsymbol{\hat{\imath}}}}
\newcommand{\vecj}{{\boldsymbol{\hat{\jmath}}}}
\newcommand{\veck}{{\boldsymbol{\hat{k}}}}
\newcommand{\vecl}{\vec{\boldsymbol{\l}}}
\newcommand{\uvec}[1]{\mathbf{\hat{#1}}}
\newcommand{\utan}{\mathbf{\hat{t}}}
\newcommand{\unormal}{\mathbf{\hat{n}}}
\newcommand{\ubinormal}{\mathbf{\hat{b}}}

\newcommand{\dotp}{\bullet}
\newcommand{\cross}{\boldsymbol\times}
\newcommand{\grad}{\boldsymbol\nabla}
\newcommand{\divergence}{\grad\dotp}
\newcommand{\curl}{\grad\cross}
%\DeclareMathOperator{\divergence}{divergence}
%\DeclareMathOperator{\curl}[1]{\grad\cross #1}
\newcommand{\lto}{\mathop{\longrightarrow\,}\limits}

\renewcommand{\bar}{\overline}

\colorlet{textColor}{black}
\colorlet{background}{white}
\colorlet{penColor}{blue!50!black} % Color of a curve in a plot
\colorlet{penColor2}{red!50!black}% Color of a curve in a plot
\colorlet{penColor3}{red!50!blue} % Color of a curve in a plot
\colorlet{penColor4}{green!50!black} % Color of a curve in a plot
\colorlet{penColor5}{orange!80!black} % Color of a curve in a plot
\colorlet{penColor6}{yellow!70!black} % Color of a curve in a plot
\colorlet{fill1}{penColor!20} % Color of fill in a plot
\colorlet{fill2}{penColor2!20} % Color of fill in a plot
\colorlet{fillp}{fill1} % Color of positive area
\colorlet{filln}{penColor2!20} % Color of negative area
\colorlet{fill3}{penColor3!20} % Fill
\colorlet{fill4}{penColor4!20} % Fill
\colorlet{fill5}{penColor5!20} % Fill
\colorlet{gridColor}{gray!50} % Color of grid in a plot

\newcommand{\surfaceColor}{violet}
\newcommand{\surfaceColorTwo}{redyellow}
\newcommand{\sliceColor}{greenyellow}




\pgfmathdeclarefunction{gauss}{2}{% gives gaussian
  \pgfmathparse{1/(#2*sqrt(2*pi))*exp(-((x-#1)^2)/(2*#2^2))}%
}


%%%%%%%%%%%%%
%% Vectors
%%%%%%%%%%%%%

%% Simple horiz vectors
\renewcommand{\vector}[1]{\left\langle #1\right\rangle}


%% %% Complex Horiz Vectors with angle brackets
%% \makeatletter
%% \renewcommand{\vector}[2][ , ]{\left\langle%
%%   \def\nextitem{\def\nextitem{#1}}%
%%   \@for \el:=#2\do{\nextitem\el}\right\rangle%
%% }
%% \makeatother

%% %% Vertical Vectors
%% \def\vector#1{\begin{bmatrix}\vecListA#1,,\end{bmatrix}}
%% \def\vecListA#1,{\if,#1,\else #1\cr \expandafter \vecListA \fi}

%%%%%%%%%%%%%
%% End of vectors
%%%%%%%%%%%%%

%\newcommand{\fullwidth}{}
%\newcommand{\normalwidth}{}



%% makes a snazzy t-chart for evaluating functions
%\newenvironment{tchart}{\rowcolors{2}{}{background!90!textColor}\array}{\endarray}

%%This is to help with formatting on future title pages.
\newenvironment{sectionOutcomes}{}{}



%% Flowchart stuff
%\tikzstyle{startstop} = [rectangle, rounded corners, minimum width=3cm, minimum height=1cm,text centered, draw=black]
%\tikzstyle{question} = [rectangle, minimum width=3cm, minimum height=1cm, text centered, draw=black]
%\tikzstyle{decision} = [trapezium, trapezium left angle=70, trapezium right angle=110, minimum width=3cm, minimum height=1cm, text centered, draw=black]
%\tikzstyle{question} = [rectangle, rounded corners, minimum width=3cm, minimum height=1cm,text centered, draw=black]
%\tikzstyle{process} = [rectangle, minimum width=3cm, minimum height=1cm, text centered, draw=black]
%\tikzstyle{decision} = [trapezium, trapezium left angle=70, trapezium right angle=110, minimum width=3cm, minimum height=1cm, text centered, draw=black]



\title{Limieten: Limieten in $\infty$}

\begin{document}
\begin{abstract}
	lim(som) is soms som(lims) 
\end{abstract}
\maketitle

\subsection{Limieten in $c=\pm\infty$ }


\todo{uitleg toevoegen}
\begin{exercise}
Probeer met de rekenregels voor limieten en voor $\pm\infty$ om $\limxi (3x^2-4x+2)$ te berekenen.



\end{exercise}

\begin{proposition} (Limiet veelterm in $c=\pm\infty$)

De limiet van een veelterm in $\pm\infty$ is de limiet van de hoogstegraadsterm (en dus gelijk aan $+\infty$ of $-\infty$ naargelang het teken van de hoogstegraadscoëfficiënt en het al dan niet even zijn van de graad van de veelterm).

De limiet van een rationale functie in $\pm\infty$ wordt bepaald door de hoogstegraadstermen van teller en noemer (en is gelijk aan $0$, $\pm\infty$ of het quotiënt van de hoogstegraadscoëfficiënten).
\end{proposition}


\begin{exercise}
	\textit{Bereken met behulp van voorgaande rekenregels}
	\begin{enumerate}
		\renewcommand{\labelenumi}{(\alph{enumi})}
		\item $\displaystyle{\lim_{x \rightarrow + \infty}} (-4x^5 + 7x^4 -
		11)$
		\item $\displaystyle{\lim_{x \rightarrow - \infty}} \frac{-x^2 + 3x
			- 2}{3x-7}$
		\item $\displaystyle{\lim_{x \rightarrow + \infty}} \frac{2x^2 -
			3}{x^3 - x + 2}$
		\item $\displaystyle{\lim_{x \rightarrow - \infty}} \frac{x^3 -
			1}{x - 1}$
		\item $\displaystyle{\lim_{x \rightarrow + \infty}} \frac{x^6 +
			5x^2 + 1}{x^4 - x^6}$
		\item $\displaystyle{\lim_{x \rightarrow + \infty}} (\sqrt{4x^2 - 7x} + 3x)$
		\item $\displaystyle{\lim_{x \rightarrow + \infty}} (\sqrt{2 x^2 + 3} - x)$
		\item $\displaystyle{\lim_{x \rightarrow - \infty}} (\sqrt{2 x^2 + 3} - x)$
		\item $\displaystyle{\lim_{x \rightarrow + \infty}} \frac{\sqrt{x^2 + 5} + 2x}{3x-1}$
		\item $\displaystyle{\lim_{x \rightarrow - \infty}} \frac{\sqrt{x^2 + 5} + 2x}{3x-1}$
		\item $\displaystyle{\lim_{x \rightarrow - \infty}} \frac{\sqrt{x^2+1} + ax}{\sqrt{x^2+2}}$
		\item $\displaystyle{\lim_{x \rightarrow - \infty}} (\sqrt{4x^2 + 7x} +
		2x)$  
		\item $\displaystyle{\lim_{x \rightarrow - \infty}} (\sqrt{4x^2 + 7x} +
		ax)$  
		\item $\displaystyle{\lim_{x \rightarrow + \infty}} x(\sqrt{2+ \frac{3}{x^2}} - \sqrt{2- \frac{3}{x^2}})$
		\item $\displaystyle{\lim_{x \rightarrow - \infty}} \frac{ax + b}{7x- \sqrt{x^2 + 10}}$
	\end{enumerate}
\begin{comment}  % remove for now: to be integrated with \anwer / \choice \ \... ?
	\begin{oplossing}
		\begin{enumerate}
			\renewcommand{\labelenumi}{(\alph{enumi})}
			\item $\displaystyle{\lim_{x \rightarrow + \infty}} (-4x^5 + 7x^4 -	11) = - \infty$
			\item $\displaystyle{\lim_{x \rightarrow - \infty}} \frac{-x^2 + 3x	- 2}{3x-7} = + \infty$
			\item $\displaystyle{\lim_{x \rightarrow + \infty}} \frac{2x^2 - 3}{x^3 - x + 2} = 0$
			\item $\displaystyle{\lim_{x \rightarrow - \infty}} \frac{x^3 -	1}{x - 1} = + \infty$
			\item $\displaystyle{\lim_{x \rightarrow + \infty}} \frac{x^6 + 5x^2 + 1}{x^4 - x^6}= -1$
			\item $\displaystyle{\lim_{x \rightarrow + \infty}} (\sqrt{4x^2 - 7x} + 3x)= + \infty$
			\item $\displaystyle{\lim_{x \rightarrow + \infty}} (\sqrt{2 x^2 + 3} - x)= + \infty$
			\item $\displaystyle{\lim_{x \rightarrow - \infty}} (\sqrt{2 x^2 + 3} - x)= + \infty$
			\item $\displaystyle{\lim_{x \rightarrow + \infty}} \frac{\sqrt{x^2 + 5} + 2x}{3x-1} = 1$
			\item $\displaystyle{\lim_{x \rightarrow - \infty}} \frac{\sqrt{x^2 + 5} + 2x}{3x-1} = \frac13$
			\item $\displaystyle{\lim_{x \rightarrow - \infty}} \frac{\sqrt{x^2+1} + ax}{\sqrt{x^2+2}} = 1-a$
			\item $\displaystyle{\lim_{x \rightarrow - \infty}} (\sqrt{4x^2 + 7x} + 2x)= - \frac74$
			\item $\displaystyle{\lim_{x \rightarrow - \infty}} (\sqrt{4x^2 + 7x} +	ax)$  
			\begin{itemize}
				\item[] voor $a<2: + \infty$
				\item[] voor $a>2: - \infty$
				\item[] voor $a=2:$ zie vorige oefening 
			\end{itemize}
			\item $\displaystyle{\lim_{x \rightarrow + \infty}} x(\sqrt{2+ \frac{3}{x^2}} - \sqrt{2- \frac{3}{x^2}}) = 0$
			\item $\displaystyle{\lim_{x \rightarrow - \infty}} \frac{ax + b}{7x- \sqrt{x^2 + 10}}= \frac{a}{8}$
		\end{enumerate}
	\end{oplossing}
\end{comment}
\end{exercise}

\subsection{Limieten in nulpunten van de noemer}

\todo{(wijzen op) herhaling samenstelling functies (f(1/x) / 1/f(x) / f(x+a) etc)}

\subsubsection{Limieten van $\dfrac{1}{g(x)}$ met $g(x)$ een veelterm}

\begin{example}
	De functie $f_1(x)=\dfrac{1}{x}$ heeft geen limiet in $x=0$, 
	omdat $\llimx f_1(x) = -\infty$ terwijl $\rlimx f_1(x) = +\infty$. 
	Maar, $f_2(x)=\dfrac{1}{x^2} = +\infty$ omdat $x^2$ (en dus ook $\frac{1}{x^2}$) altijd positief is 
	(in de buurt van $0$ zou voldoende zijn, maar $x^2$ is natuurlijk overal positief). 
	
	Om in het algemeen het limietgedrag van $\dfrac{1}{g(x)}$ te kennen, is het dus (nodig en) voldoende om het tekenverloop te kennen van $g(x)$ in de buurt van zijn nulpunten. 
	
	Op de onderstaande tekening is $g(x)=(x+5)(x-5)(x-1)^2/50$ getekend, samen met $1/g(x)$. We zien dat de limiet in $x=1$ gelijk is aan $-\infty$ (omdat $g(x)$ er een dubbel nulpunt heeft en negatief is in de omgeving van $-1$, terwijl er in $x=\pm5$ telkens enkelvoudige nulpunten zijn, waar $g(x)$ dus van teken verandert, en $1/g(x)$ dus geen limiet heeft (maar enkel een linker- en een rechterlimiet).
	

\begin{image}[\width]
%\begin{figure}
	\begin{tikzpicture}
	 [
	declare function=
	{
		t(\x) = (\x-5)*(\x-1)^2*(x+5)/50 ;
	}
	]
	\begin{axis}
	[ samples=1000, axis lines=center,
	  axis equal,
	  scale=2,
	  restrict y to domain=-10:10,
	  ymax=8,ymin=-8,
      extra y ticks={0},
%      legend style={at={(0.15,0.45)},anchor=north east},
      legend pos=south west,
	]
	\addplot[domain=-8.001:8,semithick,dashed,color=blue] {t(x)};
	\addplot[domain=-8.001:8,thick,color=red] {1/t(x)};
	\legend{$g(x)$, $1/g(x)$}
	\end{axis}
	\end{tikzpicture}
%	\captionof{figure}{Test}
%\end{figure}
\end{image}

\end{example}

Voor het vervolg is het erg belangrijk ten diepste overtuigd te zijn van de waarheid en trivialiteit van volgende twee eigenschappen (die op de vorige tekening duidelijk worden geïllustreerd):

\begin{proposition} (Triviale eigenschap veeltermen en rationale functies)
% \todo: er is meer vertikale whitespace tussen de titel en de enumerate dan tussen de titel en de \paragraph infra!
\begin{enumerate}
	\item Een veelterm $f(x)$ kan alleen van teken veranderen in een nulpunt. 
	\item Een veelterm verandert niet van teken in elk nulpunt.
	\item Een rationale functie $f(x)$ kan alleen van teken veranderen in een nulpunt van teller of noemer. 
	\item Een rationale functie  verandert niet van teken in elk nulpunt van teller of noemer.
\end{enumerate}
\end{proposition}

\begin{proposition} (Triviale eigenschap omgekeerde functie)
	
Zij $f$ een reële functie, $c\in\R$. Dan geldt dat
\begin{enumerate}
\item	$\dfrac{1}{f(c)}$ heeft altijd hetzelfde teken als $f(c)$  \quad(tenzij $f(c)=0$\dots)

\item    $\dfrac{1}{f(x)}$ heeft \textit{mogelijk} een oneindige limiet in $x=c \quad\iff\quad f(c)=0$ 

\item    $\dfrac{1}{f(x)}$ heeft \textit{echt} een oneindige limiet in $x=c \quad\iff\quad f(c)=0$ en $f(x)$ verandert niet van teken in de buurt van $c$ 
\end{enumerate}
\end{proposition}

\subsubsection{Limieten van rationale functies $f(x)/g(x)$}

\begin{verbatim}
Er zijn volgende mogelijkheden met $m_teller$ en $m_noemer$ de multipliciteiten van $c$:

* de limiet is een getal verschillend van 0  
                           (als $m_{noemer} = m_{teller}$)
* de limiet is 0           (als $m_{noemer} > m_{teller}$)
* de limiet is $\pm\infty$ (als $m_{noemer} < m_{teller}$ en llim = rlim)
* de limiet bestaat niet   (als $m_{noemer} < m_{teller}$ en llim \neq rlim)

\end{verbatim}



\end{document}
