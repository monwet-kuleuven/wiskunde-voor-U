\documentclass{ximera}
%\documentclass[wordchoicegiven]{ximera}

%
% copied from https://github.com/mooculus/calculus
%
\usepackage[utf8]{inputenc}


\graphicspath{
	{./}
	{goniometrie/}
}


%\usepackage{todonotes}
%\usepackage{mathtools} %% Required for wide table Curl and Greens
%\usepackage{cuted} %% Required for wide table Curl and Greens
\newcommand{\todo}{}

% Font niet (correct?) geinstalleerd in MikTeX?
%\usepackage{esint} % for \oiint
%\ifxake%%https://math.meta.stackexchange.com/questions/9973/how-do-you-render-a-closed-surface-double-integral
%\renewcommand{\oiint}{{\large\bigcirc}\kern-1.56em\iint}
%\fi


\newcommand{\mooculus}{\textsf{\textbf{MOOC}\textnormal{\textsf{ULUS}}}}

\usepackage{tkz-euclide}\usepackage{tikz}
\usepackage{tikz-cd}
\usetikzlibrary{arrows}
\tikzset{>=stealth,commutative diagrams/.cd,
  arrow style=tikz,diagrams={>=stealth}} %% cool arrow head
\tikzset{shorten <>/.style={ shorten >=#1, shorten <=#1 } } %% allows shorter vectors

\usetikzlibrary{backgrounds} %% for boxes around graphs
\usetikzlibrary{shapes,positioning}  %% Clouds and stars
\usetikzlibrary{matrix} %% for matrix
\usepgfplotslibrary{polar} %% for polar plots
\usepgfplotslibrary{fillbetween} %% to shade area between curves in TikZ
\usetkzobj{all}
\usepackage[makeroom]{cancel} %% for strike outs
%\usepackage{mathtools} %% for pretty underbrace % Breaks Ximera
%\usepackage{multicol}
\usepackage{pgffor} %% required for integral for loops



%% http://tex.stackexchange.com/questions/66490/drawing-a-tikz-arc-specifying-the-center
%% Draws beach ball
\tikzset{pics/carc/.style args={#1:#2:#3}{code={\draw[pic actions] (#1:#3) arc(#1:#2:#3);}}}



\usepackage{array}
\setlength{\extrarowheight}{+.1cm}
\newdimen\digitwidth
\settowidth\digitwidth{9}
\def\divrule#1#2{
\noalign{\moveright#1\digitwidth
\vbox{\hrule width#2\digitwidth}}}





\newcommand{\RR}{\mathbb R}
\newcommand{\R}{\mathbb R}
\newcommand{\N}{\mathbb N}
\newcommand{\Z}{\mathbb Z}

\newcommand{\sagemath}{\textsf{SageMath}}


%\renewcommand{\d}{\,d\!}
\renewcommand{\d}{\mathop{}\!d}
\newcommand{\dd}[2][]{\frac{\d #1}{\d #2}}
\newcommand{\pp}[2][]{\frac{\partial #1}{\partial #2}}
\renewcommand{\l}{\ell}
\newcommand{\ddx}{\frac{d}{\d x}}

\newcommand{\zeroOverZero}{\ensuremath{\boldsymbol{\tfrac{0}{0}}}}
\newcommand{\inftyOverInfty}{\ensuremath{\boldsymbol{\tfrac{\infty}{\infty}}}}
\newcommand{\zeroOverInfty}{\ensuremath{\boldsymbol{\tfrac{0}{\infty}}}}
\newcommand{\zeroTimesInfty}{\ensuremath{\small\boldsymbol{0\cdot \infty}}}
\newcommand{\inftyMinusInfty}{\ensuremath{\small\boldsymbol{\infty - \infty}}}
\newcommand{\oneToInfty}{\ensuremath{\boldsymbol{1^\infty}}}
\newcommand{\zeroToZero}{\ensuremath{\boldsymbol{0^0}}}
\newcommand{\inftyToZero}{\ensuremath{\boldsymbol{\infty^0}}}



\newcommand{\numOverZero}{\ensuremath{\boldsymbol{\tfrac{\#}{0}}}}
\newcommand{\dfn}{\textbf}
%\newcommand{\unit}{\,\mathrm}
\newcommand{\unit}{\mathop{}\!\mathrm}
\newcommand{\eval}[1]{\bigg[ #1 \bigg]}
\newcommand{\seq}[1]{\left( #1 \right)}
\renewcommand{\epsilon}{\varepsilon}
\renewcommand{\phi}{\varphi}


\renewcommand{\iff}{\Leftrightarrow}

\DeclareMathOperator{\arccot}{arccot}
\DeclareMathOperator{\arcsec}{arcsec}
\DeclareMathOperator{\arccsc}{arccsc}
\DeclareMathOperator{\si}{Si}
\DeclareMathOperator{\scal}{scal}
\DeclareMathOperator{\sign}{sign}


%% \newcommand{\tightoverset}[2]{% for arrow vec
%%   \mathop{#2}\limits^{\vbox to -.5ex{\kern-0.75ex\hbox{$#1$}\vss}}}
\newcommand{\arrowvec}[1]{{\overset{\rightharpoonup}{#1}}}
%\renewcommand{\vec}[1]{\arrowvec{\mathbf{#1}}}
\renewcommand{\vec}[1]{{\overset{\boldsymbol{\rightharpoonup}}{\mathbf{#1}}}\hspace{0in}}

\newcommand{\point}[1]{\left(#1\right)} %this allows \vector{ to be changed to \vector{ with a quick find and replace
\newcommand{\pt}[1]{\mathbf{#1}} %this allows \vec{ to be changed to \vec{ with a quick find and replace
\newcommand{\Lim}[2]{\lim_{\point{#1} \to \point{#2}}} %Bart, I changed this to point since I want to use it.  It runs through both of the exercise and exerciseE files in limits section, which is why it was in each document to start with.

\DeclareMathOperator{\proj}{\mathbf{proj}}
\newcommand{\veci}{{\boldsymbol{\hat{\imath}}}}
\newcommand{\vecj}{{\boldsymbol{\hat{\jmath}}}}
\newcommand{\veck}{{\boldsymbol{\hat{k}}}}
\newcommand{\vecl}{\vec{\boldsymbol{\l}}}
\newcommand{\uvec}[1]{\mathbf{\hat{#1}}}
\newcommand{\utan}{\mathbf{\hat{t}}}
\newcommand{\unormal}{\mathbf{\hat{n}}}
\newcommand{\ubinormal}{\mathbf{\hat{b}}}

\newcommand{\dotp}{\bullet}
\newcommand{\cross}{\boldsymbol\times}
\newcommand{\grad}{\boldsymbol\nabla}
\newcommand{\divergence}{\grad\dotp}
\newcommand{\curl}{\grad\cross}
%\DeclareMathOperator{\divergence}{divergence}
%\DeclareMathOperator{\curl}[1]{\grad\cross #1}
\newcommand{\lto}{\mathop{\longrightarrow\,}\limits}

\renewcommand{\bar}{\overline}

\colorlet{textColor}{black}
\colorlet{background}{white}
\colorlet{penColor}{blue!50!black} % Color of a curve in a plot
\colorlet{penColor2}{red!50!black}% Color of a curve in a plot
\colorlet{penColor3}{red!50!blue} % Color of a curve in a plot
\colorlet{penColor4}{green!50!black} % Color of a curve in a plot
\colorlet{penColor5}{orange!80!black} % Color of a curve in a plot
\colorlet{penColor6}{yellow!70!black} % Color of a curve in a plot
\colorlet{fill1}{penColor!20} % Color of fill in a plot
\colorlet{fill2}{penColor2!20} % Color of fill in a plot
\colorlet{fillp}{fill1} % Color of positive area
\colorlet{filln}{penColor2!20} % Color of negative area
\colorlet{fill3}{penColor3!20} % Fill
\colorlet{fill4}{penColor4!20} % Fill
\colorlet{fill5}{penColor5!20} % Fill
\colorlet{gridColor}{gray!50} % Color of grid in a plot

\newcommand{\surfaceColor}{violet}
\newcommand{\surfaceColorTwo}{redyellow}
\newcommand{\sliceColor}{greenyellow}




\pgfmathdeclarefunction{gauss}{2}{% gives gaussian
  \pgfmathparse{1/(#2*sqrt(2*pi))*exp(-((x-#1)^2)/(2*#2^2))}%
}


%%%%%%%%%%%%%
%% Vectors
%%%%%%%%%%%%%

%% Simple horiz vectors
\renewcommand{\vector}[1]{\left\langle #1\right\rangle}


%% %% Complex Horiz Vectors with angle brackets
%% \makeatletter
%% \renewcommand{\vector}[2][ , ]{\left\langle%
%%   \def\nextitem{\def\nextitem{#1}}%
%%   \@for \el:=#2\do{\nextitem\el}\right\rangle%
%% }
%% \makeatother

%% %% Vertical Vectors
%% \def\vector#1{\begin{bmatrix}\vecListA#1,,\end{bmatrix}}
%% \def\vecListA#1,{\if,#1,\else #1\cr \expandafter \vecListA \fi}

%%%%%%%%%%%%%
%% End of vectors
%%%%%%%%%%%%%

%\newcommand{\fullwidth}{}
%\newcommand{\normalwidth}{}



%% makes a snazzy t-chart for evaluating functions
%\newenvironment{tchart}{\rowcolors{2}{}{background!90!textColor}\array}{\endarray}

%%This is to help with formatting on future title pages.
\newenvironment{sectionOutcomes}{}{}



%% Flowchart stuff
%\tikzstyle{startstop} = [rectangle, rounded corners, minimum width=3cm, minimum height=1cm,text centered, draw=black]
%\tikzstyle{question} = [rectangle, minimum width=3cm, minimum height=1cm, text centered, draw=black]
%\tikzstyle{decision} = [trapezium, trapezium left angle=70, trapezium right angle=110, minimum width=3cm, minimum height=1cm, text centered, draw=black]
%\tikzstyle{question} = [rectangle, rounded corners, minimum width=3cm, minimum height=1cm,text centered, draw=black]
%\tikzstyle{process} = [rectangle, minimum width=3cm, minimum height=1cm, text centered, draw=black]
%\tikzstyle{decision} = [trapezium, trapezium left angle=70, trapezium right angle=110, minimum width=3cm, minimum height=1cm, text centered, draw=black]


\author{Zomercursus KU Leuven}
\outcome{In deze cursus gebruikte afspraken en notaties kunnen terugvinden}


\title[Inleiding:]{Notaties en afspraken}

\begin{document}
\begin{abstract}
	Leest en gij zult weten.
\end{abstract}
\maketitle


%\subsection{De machtsverheffing: definitie en rekenregels}
\subsection{Afspraken }

\begin{remark} (Specifieke afspraken voor deze cursus)
	
	\begin{enumerate}
		\item Deze cursus is beschikbaar in meerdere vormen:
		\begin{enumerate}
			\item als geprinte PDF
			\item als online PDF (met hyperlinks, en soms zelfs enkele bewegende afbeeldingen, zie bv (ref))
			\item als online applicatie op : met extra functionaliteit voor theorien en oefeningen (met hints, meerkeuze, antwoorden etc.)
		\end{enumerate} 
		(TODO: kort uitleggen wat de verschillen en voor/nadelen zijn van elke vorm)
		
		\pdfOnly{
		MERK OP: U gebruikt de PDF versie. Op \link[Ximera]{https://ximera.osu.edu/wvu/wiskundeVoorkennisLessen} vindt u een online versie, die extra functionaliteiten biedt.
		}
	
	\begin{onlineOnly}
		MERK OP: U gebruikt de ONLINE versie. Op \link[KU Leuven]{https://xxx.be} vindt u een PDF versie, die u kan uitprinten.	
	\end{onlineOnly}

		\item Deze cursus is beschikbaar voor meerdere niveaus:
		\begin{enumerate}
			\item als voorbereiding voor studies Wetenschappen aan de KU Leuven
			\item als voorbereiding voor studies Burgerlijk Ingenieur aan de KU Leuven
			\item als voorbereiding voor studies Wiskunde aan de KU Leuven
		\end{enumerate}
		(TODO: kort uitleggen ( en eerst vastleggen!!!)  wat en hoe de verschillen precies werken)
		\item De secties 'Uitweiding' zijn geen leerstof. Ze geven achtergrondinformatie, motivatie of een kleine uitbreiding. Je kan ze negeren als ze eerder verwarring veroorzaken dan oplossen.
		\item Elke hoofdstuk eindigt met een 'Samenvatting leerstof': als je die sectie voldoende vlot kent, ken je de theorie van dat hoofdstuk. Je kan dus zeker voor de eerste hoofdstukken, direct naar de Samenvatting gaan kijken, om na te gaan of je het hoofdstuk moet lezen.
		\item Elk hoofdstuk eindigt met 'Checklist oefeningen' met \textsc{eenvoudige} oefeningen die de behandelde theorie op elementaire wijze toepassen. Als je die erg vlot kan oplossen, ken je allicht dat hoofdstuk voldoende. 
		
		Het is echter ook van belang om ook de niet-zo-eenvoudige oefeningen te kunnen oplossen! 
		
	\end{enumerate}
\end{remark}

\begin{notation} (Specifieke begrippen en notaties voor deze cursus) 
	
	We gebruiken enkele begrippen en notaties die niet algemeen gebruikt worden in de wiskunde, maar die we hier toch handig vonden.
	\begin{enumerate}
		\item belangrijke definities en eigenschappen staan in een kader.
		
		 \textsc{Zeer belangrijke} formules vallen nog meer op, en die ken je dan ook van binnen en van buiten. Je hoeft daar zelfs geen seconde over na te denken. (Enkele milliseconden mag wel). Bijvoorbeeld: $\important{(a+b)^2 = a^2 + 2ab + b^2}$. In doorlopende tekst gebruiken we ook wel \textsc{small caps} om iets te laten opvallen.
		\item een \textit{pseudo-definitie} is iets dat erg lijkt op een definitie, maar strikt wiskundig geen correcte definitie is. Bijvoorbeeld voor limieten en continuïteit van functies geven we in de hoofdtekst enkel 'pseudo-definities', omdat strikt wiskundige definities ons te ver zouden leiden...
		\item een \textit{niet-eigenschap} (en een niet-voorbeeld) is een (voorbeeld van een) bewering (meestal zo typisch mogelijk) die \textsc{niet} (algemeen) geldig is. Bijvoorbeeld $(1+1)^2 \neq 1^2+1^2$ (want $2^2 = 4 \neq 1 + 1 = 2$) is een niet-voorbeeld van de niet-eigenschap dat de macht van een som gelijk is aan de som van de machten. Het is in het algemeen even belangrijk de eigenschappen te kennen als de niet-eigenschappen! (Uitweiding: omdat in de wiskunde natuurlijk enkel gebruik mag gemaakt worden van de echte eigenschappen, en alles wat daar niet onder valt \textit{niet} mag worden gebruikt, zou men kunnen denken dat het begrip niet-eigenschap overbodig is. Toch stelt men vast dat veel lerenden baat hebben bij expliciete voorbeelden van niet-eigenschappen.)
	\end{enumerate}
\end{notation}

\begin{notation} (Algemeen gebruikte wiskundige notatie) 
	
	We geven een kort overzicht van algemeen geldende afspraken en notaties in de wiskunde, waarmee sommigen misschien toch niet direct vertrouwd zijn.
	\begin{enumerate}
		\item We noteren $x\mapsto f(x)$ om duidelijk te maken dat $f$ een functie is die $x$ afbeeldt op $f(x)$. Als we nauwkeuriger willen zijn, schrijven we ook $f:A\to B, x\mapsto f(x)$ (Zie (ref)).
		\item We gebruiken het symbool $\iff$ om aan te geven dat twee uitdrukkingen equivalent zijn: $A \iff B$ betekent dat $A$ waar is als en slechts als $B$ waar is. (Uitweiding:) Soms schrijft men ook 'asa' voor 'als en slechts als'. In het engels gebruikt met soms 'iff': if and only if.
		\item We noteren $a\in\R$ voor '$a$ is een element van $\R$', dus $a$ is een (willekeurig) reëel getal.
		\item We noteren $A\subset \R$ voor $A$ is een deelverzameling van $\R$.
		\item (Uitweiding) We gebruiken $(a_n)$ voor een rij met elementen $a_1,a_2,a_3,\dots,a_i,\dots$ 
		\item We gebruiken $A\perdef B$ om aan te duiden dat uitdrukking $A$ per definitie gelijk is aan $B$. We veronderstellen dus dat uitdrukking $A$ voordien geen betekenis had, uitdrukking $B$ wel, en dat vanaf nu $A$ hetzelfde betekent als $B$. Voorbeeld: $\sqrt{a}\perdef a^{\frac12}$ definieert het symbool $\sqrt{a}$ als we rationale machten al kennen, maar $a^{\frac12} \perdef \sqrt{a}$ definieert een rationale macht, als we het symbool $\sqrt{a}$ al zouden kennen!
		\item Als we zeggen dat 'men kan bewijzen dat ...', dan bedoelen we inderdaad dat 'men' dat kan bewijzen, maar dat we niet verwachten dat jij dan ook kan bewijzen. Het bewijs is dus \textit{geen} leerstof, en allicht ook niet volledig triviaal. 
		Als het bewijs wel relatief eenvoudig is, zullen we dat soms zeggen 'je kan als oefening bewijzen dat...'.
	\end{enumerate}
\end{notation}



\end{document}
