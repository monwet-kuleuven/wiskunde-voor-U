\documentclass{ximera}

\title[Examples:]{Hoeken meten}

\begin{document}
\begin{abstract}
In graden, maar vooral radialen	
\end{abstract}
\maketitle


\subsection{Achtergrondinformatie}
	In het algemeen betekent 'iets meten' een maat (meestal een getal) associëren aan dat iets. We zullen dat hier doen voor hoeken.
\begin{expandable}
	In sommige gevallen is er een min of meer natuurlijke maateenheid. Om te meten hoe lang het duurt om wiskunde te leren ligt het enigszins voor de hand om het aantal 'dagen' (dus: omwentelingen van de aarde om haar as), 'maanden' (omwentelingen van de maan om de aarde) of 'jaren' (aantal omwentelingen van de aarde om de zon) te gebruiken. Voor kortere tijden hebben we sinds mensenheugnis enigszins arbitraire keuzes gemaakt: 'uren' zijn 1/24ste van een dag, en worden verdeeld in 60 'minuten' die op hun beurt verdeeld zijn in 60 'seconden'. Nog nauwkeuriger rekenen we toch weer decimaal: in tienden of honderdsten van een seconde. Voor afstanden waren er vroeger duimen, voeten, ellen, boogscheuten en dergelijke, maar is er (essentieel sinds Napoleon) de arbitraire, maar handige 'meter' met tiendelige verdelingen en veelvouden (cm, km, ...). Gelijkaardige situaties doen zich voorbij het meten van gewichten, krachten, elektrische ladingen enzovoort.

	Ook voor hoeken stelt zich het probleem van het 'meten' van zo'n hoek. Historisch -en tot vandaag in het dagelijkse leven- gebruikt men een systeem van graden, minuten en seconden. In de wetenschap worden hoeken echter bijna altijd gemeten in radialen. Dat is (veel) handiger.
\end{expandable}


\end{document}
