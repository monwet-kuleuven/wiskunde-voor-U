\documentclass[handout]{ximera}
%\documentclass{ximera}

%
% copied from https://github.com/mooculus/calculus
%
\usepackage[utf8]{inputenc}


\graphicspath{
	{./}
	{goniometrie/}
}


%\usepackage{todonotes}
%\usepackage{mathtools} %% Required for wide table Curl and Greens
%\usepackage{cuted} %% Required for wide table Curl and Greens
\newcommand{\todo}{}

% Font niet (correct?) geinstalleerd in MikTeX?
%\usepackage{esint} % for \oiint
%\ifxake%%https://math.meta.stackexchange.com/questions/9973/how-do-you-render-a-closed-surface-double-integral
%\renewcommand{\oiint}{{\large\bigcirc}\kern-1.56em\iint}
%\fi


\newcommand{\mooculus}{\textsf{\textbf{MOOC}\textnormal{\textsf{ULUS}}}}

\usepackage{tkz-euclide}\usepackage{tikz}
\usepackage{tikz-cd}
\usetikzlibrary{arrows}
\tikzset{>=stealth,commutative diagrams/.cd,
  arrow style=tikz,diagrams={>=stealth}} %% cool arrow head
\tikzset{shorten <>/.style={ shorten >=#1, shorten <=#1 } } %% allows shorter vectors

\usetikzlibrary{backgrounds} %% for boxes around graphs
\usetikzlibrary{shapes,positioning}  %% Clouds and stars
\usetikzlibrary{matrix} %% for matrix
\usepgfplotslibrary{polar} %% for polar plots
\usepgfplotslibrary{fillbetween} %% to shade area between curves in TikZ
\usetkzobj{all}
\usepackage[makeroom]{cancel} %% for strike outs
%\usepackage{mathtools} %% for pretty underbrace % Breaks Ximera
%\usepackage{multicol}
\usepackage{pgffor} %% required for integral for loops



%% http://tex.stackexchange.com/questions/66490/drawing-a-tikz-arc-specifying-the-center
%% Draws beach ball
\tikzset{pics/carc/.style args={#1:#2:#3}{code={\draw[pic actions] (#1:#3) arc(#1:#2:#3);}}}



\usepackage{array}
\setlength{\extrarowheight}{+.1cm}
\newdimen\digitwidth
\settowidth\digitwidth{9}
\def\divrule#1#2{
\noalign{\moveright#1\digitwidth
\vbox{\hrule width#2\digitwidth}}}





\newcommand{\RR}{\mathbb R}
\newcommand{\R}{\mathbb R}
\newcommand{\N}{\mathbb N}
\newcommand{\Z}{\mathbb Z}

\newcommand{\sagemath}{\textsf{SageMath}}


%\renewcommand{\d}{\,d\!}
\renewcommand{\d}{\mathop{}\!d}
\newcommand{\dd}[2][]{\frac{\d #1}{\d #2}}
\newcommand{\pp}[2][]{\frac{\partial #1}{\partial #2}}
\renewcommand{\l}{\ell}
\newcommand{\ddx}{\frac{d}{\d x}}

\newcommand{\zeroOverZero}{\ensuremath{\boldsymbol{\tfrac{0}{0}}}}
\newcommand{\inftyOverInfty}{\ensuremath{\boldsymbol{\tfrac{\infty}{\infty}}}}
\newcommand{\zeroOverInfty}{\ensuremath{\boldsymbol{\tfrac{0}{\infty}}}}
\newcommand{\zeroTimesInfty}{\ensuremath{\small\boldsymbol{0\cdot \infty}}}
\newcommand{\inftyMinusInfty}{\ensuremath{\small\boldsymbol{\infty - \infty}}}
\newcommand{\oneToInfty}{\ensuremath{\boldsymbol{1^\infty}}}
\newcommand{\zeroToZero}{\ensuremath{\boldsymbol{0^0}}}
\newcommand{\inftyToZero}{\ensuremath{\boldsymbol{\infty^0}}}



\newcommand{\numOverZero}{\ensuremath{\boldsymbol{\tfrac{\#}{0}}}}
\newcommand{\dfn}{\textbf}
%\newcommand{\unit}{\,\mathrm}
\newcommand{\unit}{\mathop{}\!\mathrm}
\newcommand{\eval}[1]{\bigg[ #1 \bigg]}
\newcommand{\seq}[1]{\left( #1 \right)}
\renewcommand{\epsilon}{\varepsilon}
\renewcommand{\phi}{\varphi}


\renewcommand{\iff}{\Leftrightarrow}

\DeclareMathOperator{\arccot}{arccot}
\DeclareMathOperator{\arcsec}{arcsec}
\DeclareMathOperator{\arccsc}{arccsc}
\DeclareMathOperator{\si}{Si}
\DeclareMathOperator{\scal}{scal}
\DeclareMathOperator{\sign}{sign}


%% \newcommand{\tightoverset}[2]{% for arrow vec
%%   \mathop{#2}\limits^{\vbox to -.5ex{\kern-0.75ex\hbox{$#1$}\vss}}}
\newcommand{\arrowvec}[1]{{\overset{\rightharpoonup}{#1}}}
%\renewcommand{\vec}[1]{\arrowvec{\mathbf{#1}}}
\renewcommand{\vec}[1]{{\overset{\boldsymbol{\rightharpoonup}}{\mathbf{#1}}}\hspace{0in}}

\newcommand{\point}[1]{\left(#1\right)} %this allows \vector{ to be changed to \vector{ with a quick find and replace
\newcommand{\pt}[1]{\mathbf{#1}} %this allows \vec{ to be changed to \vec{ with a quick find and replace
\newcommand{\Lim}[2]{\lim_{\point{#1} \to \point{#2}}} %Bart, I changed this to point since I want to use it.  It runs through both of the exercise and exerciseE files in limits section, which is why it was in each document to start with.

\DeclareMathOperator{\proj}{\mathbf{proj}}
\newcommand{\veci}{{\boldsymbol{\hat{\imath}}}}
\newcommand{\vecj}{{\boldsymbol{\hat{\jmath}}}}
\newcommand{\veck}{{\boldsymbol{\hat{k}}}}
\newcommand{\vecl}{\vec{\boldsymbol{\l}}}
\newcommand{\uvec}[1]{\mathbf{\hat{#1}}}
\newcommand{\utan}{\mathbf{\hat{t}}}
\newcommand{\unormal}{\mathbf{\hat{n}}}
\newcommand{\ubinormal}{\mathbf{\hat{b}}}

\newcommand{\dotp}{\bullet}
\newcommand{\cross}{\boldsymbol\times}
\newcommand{\grad}{\boldsymbol\nabla}
\newcommand{\divergence}{\grad\dotp}
\newcommand{\curl}{\grad\cross}
%\DeclareMathOperator{\divergence}{divergence}
%\DeclareMathOperator{\curl}[1]{\grad\cross #1}
\newcommand{\lto}{\mathop{\longrightarrow\,}\limits}

\renewcommand{\bar}{\overline}

\colorlet{textColor}{black}
\colorlet{background}{white}
\colorlet{penColor}{blue!50!black} % Color of a curve in a plot
\colorlet{penColor2}{red!50!black}% Color of a curve in a plot
\colorlet{penColor3}{red!50!blue} % Color of a curve in a plot
\colorlet{penColor4}{green!50!black} % Color of a curve in a plot
\colorlet{penColor5}{orange!80!black} % Color of a curve in a plot
\colorlet{penColor6}{yellow!70!black} % Color of a curve in a plot
\colorlet{fill1}{penColor!20} % Color of fill in a plot
\colorlet{fill2}{penColor2!20} % Color of fill in a plot
\colorlet{fillp}{fill1} % Color of positive area
\colorlet{filln}{penColor2!20} % Color of negative area
\colorlet{fill3}{penColor3!20} % Fill
\colorlet{fill4}{penColor4!20} % Fill
\colorlet{fill5}{penColor5!20} % Fill
\colorlet{gridColor}{gray!50} % Color of grid in a plot

\newcommand{\surfaceColor}{violet}
\newcommand{\surfaceColorTwo}{redyellow}
\newcommand{\sliceColor}{greenyellow}




\pgfmathdeclarefunction{gauss}{2}{% gives gaussian
  \pgfmathparse{1/(#2*sqrt(2*pi))*exp(-((x-#1)^2)/(2*#2^2))}%
}


%%%%%%%%%%%%%
%% Vectors
%%%%%%%%%%%%%

%% Simple horiz vectors
\renewcommand{\vector}[1]{\left\langle #1\right\rangle}


%% %% Complex Horiz Vectors with angle brackets
%% \makeatletter
%% \renewcommand{\vector}[2][ , ]{\left\langle%
%%   \def\nextitem{\def\nextitem{#1}}%
%%   \@for \el:=#2\do{\nextitem\el}\right\rangle%
%% }
%% \makeatother

%% %% Vertical Vectors
%% \def\vector#1{\begin{bmatrix}\vecListA#1,,\end{bmatrix}}
%% \def\vecListA#1,{\if,#1,\else #1\cr \expandafter \vecListA \fi}

%%%%%%%%%%%%%
%% End of vectors
%%%%%%%%%%%%%

%\newcommand{\fullwidth}{}
%\newcommand{\normalwidth}{}



%% makes a snazzy t-chart for evaluating functions
%\newenvironment{tchart}{\rowcolors{2}{}{background!90!textColor}\array}{\endarray}

%%This is to help with formatting on future title pages.
\newenvironment{sectionOutcomes}{}{}



%% Flowchart stuff
%\tikzstyle{startstop} = [rectangle, rounded corners, minimum width=3cm, minimum height=1cm,text centered, draw=black]
%\tikzstyle{question} = [rectangle, minimum width=3cm, minimum height=1cm, text centered, draw=black]
%\tikzstyle{decision} = [trapezium, trapezium left angle=70, trapezium right angle=110, minimum width=3cm, minimum height=1cm, text centered, draw=black]
%\tikzstyle{question} = [rectangle, rounded corners, minimum width=3cm, minimum height=1cm,text centered, draw=black]
%\tikzstyle{process} = [rectangle, minimum width=3cm, minimum height=1cm, text centered, draw=black]
%\tikzstyle{decision} = [trapezium, trapezium left angle=70, trapezium right angle=110, minimum width=3cm, minimum height=1cm, text centered, draw=black]


% Allicht enkel in online  versie te includen !!!

\author{Wim Obbels}
\outcome{Elementaire vertrouwdheid met Ximera functionaliteit}
\outcome{(Uitbreiding) Enig inzicht in Ximera architectuur}


\title{Ximera Howto}

\begin{document}
\begin{abstract}
	Ximewatisditnuweer?
\end{abstract}
\maketitle

\subsection{Ximera functionaliteit}
De online module voor deze cursus maakt gebruik van Ximera.

We wijzen u op volgende handige (en soms minder handige) aspecten van dit systeem:


TODO: uitwerken / aanvullen


\begin{itemize}
	\item Je kan iets moeten invullen / uitklappen / updaten / saven/erasen

	\item Iets over engels vs nederlands

	\item Iets over punten en komma's  (1.2 vs 1,2) ?

\end{itemize}

\subsection{(Uitbreiding) Ximera architectuur}

TODO

\subsection{(Uitbreiding) Ximera Showcase}

\pdfOnly{
    \ifhandout
    Je gebruikt de HANDOUT PDF versie van de cursus; 
    
    Er bestaat ook een 'gewone' PDF die antwoorden en hints  bevat.
    \else
    Je gebruikt de STANDAARD PDF versie van de cursus; 

    Er bestaat ook een \textit{handout} PDF zonder de antwoorden!
    \fi
    
    Er bestaat ook een  online versie met extra functionaliteit
    
}

\begin{onlineOnly}
    Je gebruikt de ONLINE versie van de cursus; er bestaan PDF versies.
\end{onlineOnly}

\begin{prompt}  
    BEGIN PROMPT
    
    Deze tekst staat in een zogenaamde 'prompt' omgeving in \LaTeX. Het is de auteur van deze cursus voorlopig enigszins onduidelijk waarvoor dat precies dient.
    
    Volgens de documentatie zou dit 'hidden tekst' moeten zijn.
    (In de handout-versie wordt het inderdaad niet getoond.)
    
    EINDE PROMPT
\end{prompt}

\begin{problem}
    
    Los volgende vragen correct op:
    \begin{enumerate}
        \item $1+1 = \answer{2}$
        \item $1+1 = \answer[given]{2}$   ( Met 'answer[given]')
        \item $\frac{1}{2} =  \answer{\frac{1}{2}}$  (Gebruik een punt voor decimale getallen...)
        \item $\frac{1}{2} =  \answer[tolerance=.2]{\frac{1}{2}}$  (Je mag er 0.2 naast zitten)
    \end{enumerate}
\end{problem}

\begin{problem}
       Los volgende vragen correct op:
    \begin{enumerate}
        \item $1+1 = $\wordChoice{\choice[correct]{$2$}\choice{$3$}\choice{geen van de vorige antwoorden}\choice{het vorige antwoord}}
        
        \item $1+1 = $\begin{multipleChoice} \choice[correct]{$2$}\choice{$3$}\choice{geen van de vorige antwoorden}\choice{het vorige antwoord}\end{multipleChoice}
        
        \item $1+1 = $\begin{selectAll} \choice[correct]{$2$}\choice{$3$}\choice{geen van de vorige antwoorden}\choice{het vorige antwoord}\choice[correct]{$1+1$}\end{selectAll}
    \end{enumerate}
\end{problem}

\begin{problem}
    Los volgende erg moeilijke vragen correct op. 
    
    Paniek is echter niet nodig, want we voorzien uitvoerige hints en feedback!
    
        \begin{question} % (Gebruik de hint!)
          $2+2 = $\wordChoice{\choice{$2$}\choice{$3$}\choice[correct]{$4$}\choice{geen van de vorige antwoorden}\choice{het vorige antwoord}}
          \begin{hint}
              Per definitie is $2 = 1+1$, en de optelling associatief.
           \end{hint}  
           \begin{hint}
              Voor $1+1+1+1$ hadden we een kortere notatie ingevoerd ...
           \end{hint}
           \begin{feedback}[correct] Proficiat, je kan al erg goed rekenen! Doe zo voort.
           \end{feedback}          
        \end{question}
   
        \begin{question}
          $4+4 = \answer[format=integer,id=antw]{8}$
          \begin{hint}[0]
              Probeer eerst 7 (want dan krijg je erg nuttige feedback!)
          \end{hint}
          \begin{hint}[5]
            Herinner u het antwoord van de vorige vraag voor de preciese betekenis van het symbool $4$ !
          \end{hint}
          \begin{hint}[5]
            Gewoon optellen ja ....
          \end{hint}
      
          \begin{feedback}[correct]
              Correct !
          \end{feedback}
          \begin{feedback}[antw<>8]
               Fout !
           \end{feedback}
          \begin{feedback}[attempt]
            De organisatie laat niet na u bij deze vriendelijk te danken voor uw verdienstelijke poging iom deze vraag te beantwoorden.   
           \end{feedback}

 
      
          \begin{feedback}[antw=7]
            Wel, je volgt de richtlijnen erg nauwkeurig. Maar ook voor andere foute antwoorden geven we interessante feedback
          \end{feedback}
          \begin{feedback}[antw=8]
          Proficiat. Je heheerst deze module voldoende. Je bent nu voldoende voorbereid omù verder te gaan naar de fascinerende problematiek van \link[HoTT]{https://github.com/HoTT/HoTT}
          \end{feedback}
          \begin{feedback}[antw<7]
              Mmm, dat is wat weinig. Reken alles nog eens naukeurig na.
            \end{feedback}
          \begin{feedback}[antw>8]
               Mmm, dat is wat veel. Reken alles nog eens naukeurig na.
           \end{feedback}
       \end{question}
\end{problem}

\subsubsection{Ken je Sage al?}

In Sage kan je relatief eenvoudig de parametervergelijkingen van een cirkel bestuderen:

\begin{sageCell}
    var('s t')
    x(t) = 3*cos(t)
    y(t) = 3*sin(t)
    c(t) = (x(t),y(t))
    circle=parametric_plot(c(t),(t,0,2*pi),color="black")
    circle
\end{sageCell}

\pdfOnly{In de onlineversie kan je met deze code experimenteren. 
    
    Zie [één of andere ingewikkelde url] of [een qrcode]
}

\begin{onlineOnly}
    
    Pas de code aan, en druk op Evaluate!
    
    Nice, zoals ze zeggen ...
    
\begin{sageOutput}
    var('s t')
    x(t) = 3*cos(t)
    y(t) = 3*sin(t)
    c(t) = (x(t),y(t))
    circle=parametric_plot(c(t),(t,0,2*pi),color="black")
    circle
\end{sageOutput}
\end{onlineOnly}


\end{document}